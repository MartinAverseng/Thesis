\documentclass{article}
\usepackage{amssymb,ntheorem}
\usepackage[a4paper,hmargin=3.6cm]{geometry}
\usepackage{mathtools}
\usepackage{graphicx}
\usepackage{xcolor}
\usepackage{hyperref}
\usepackage{subcaption}
\usepackage{listings}
\usepackage{polyglossia}
\setdefaultlanguage{french}


\newcommand{\RR}{\mathbb R}
\newcommand{\PP}{\mathbb P}

\title{\textbf{Projet MAP431:}\\
\textit{Écoulements visqueux}}

\author{
    WANG Guillaume\\
    JALLET Wilson
}

\date{Pour le: 4 avril 2018}

\theoremstyle{plain}
\theorembodyfont{\normalfont}
\newtheorem{ques}{Question}
\let\div\undefined
\DeclareMathOperator{\div}{div}
\let\Id\undefined
\DeclareMathOperator{\Id}{Id}

%%% Listings config %%%

\setsansfont{DejaVu Sans}

\lstset{
    basicstyle=\footnotesize\ttfamily,
    breaklines=true,
    frame=L,
    numbers=left,
    numbersep=9pt,
    keywordstyle=\bfseries\color{black},
    identifierstyle=\color{blue},
    stringstyle=\color{orange!50!black},
    commentstyle=\color{purple!40!black},
    tabsize=2
}


\begin{document}

\maketitle

\section{Formulations variationnelles}

\begin{ques}
Définissons le sous-espace vectoriel de $H^1(\Omega,\RR^d)$
\[
    \tilde V \coloneqq \{v\in H^1(\Omega,\RR^d)\,|\, \gamma_0(v)_{|\Gamma_D} = 0 \text{ et } \div(v) = 0 \text{ sur } \Omega\},
\]
où $\gamma_0:H^1(\Omega,\RR^d)\longrightarrow L^2(\partial\Omega)$ est l'application trace canonique. 


Pour toutes fonctions $v \in \tilde{V}$, on a 
\begin{align*}
    -\Delta u\cdot v + \nabla p\cdot v &= f\cdot v \\
    -\div(u)\,w &= 0
\end{align*}
donc en intégrant sur $\Omega$ et en appliquant la formule de Green-Riemann
\begin{align*}
    \int_\Omega \nabla u\,:\nabla v - \int_{\partial\Omega}(\nabla u\, n)\cdot v - \int_\Omega p\div(v) + \int_{\partial\Omega}pn\cdot v
    &= \int_\Omega f\cdot v
\end{align*}

Le second terme du membre de gauche vaut
\[
    -\int_{\Gamma_N} p(x) n(x)\cdot v(x)\,dx
\]
puisque $v\in \tilde{V}$ donc $v = 0$ sur $\Gamma_D$ (au sens de la trace) et la condition de Neumann homogène $\sigma\,n = (\nabla u - pI)\,n = 0$ d'où $\nabla u\, n = pn$. Le dernier terme étant $-\int_\Omega p\div(v)$, on en déduit finalement la formulation variationnelle
\begin{equation}\label{formVar0}
    \int_\Omega \nabla u\,:\nabla v = \int_\Omega f\cdot v \tag{FV}
\end{equation}

La forme bilinéaire $a(u,v)$ sur $\tilde{V}\times \tilde{V}$ définie par le membre de gauche est continue coercive. En effet, on a
\[
    \int_\Omega \nabla u\,:\nabla u = \|\nabla u\|_{L^2(\Omega,\mathcal M_d(\RR))}^2 \leq \|u\|_{H^1}^2
\]
d'une part, et d'autre part l'inégalité de Poincaré permet d'écrire
\[
    \int_\Omega \nabla u\,:\nabla u 
    = \sum_{i=1}^d \|\nabla u_i\|_{L^2(\Omega,\RR^d)}^2
    \geq \frac{1}{C_P}\sum_{i=1}^d \|u_i\|_{L^2} = \frac 1{C_P}\|u\|_{L^2}
\]
puis
\[
    \int_\Omega \nabla u\,:\nabla u \geq K \|u \|_{H^1} \quad\text{où} \; K > 0
    \]

La forme linéaire du membre de droite est continue: l'inégalité de Cauchy-Schwarz garantit $\int_\Omega f\cdot v\leq \|f\|_{L^2}\|v\|_{L^2}$, donc la continuité $L^2$, donc la continuité $H^1$ puisque la norme $H^1$ est plus fine.

Enfin, $\tilde{V}$ est bien un sous-espace de Hilbert de $H^1(\Omega,\RR^d)$ puisque $\gamma_0(v)_{|\Gamma_D}$ et $v \rightarrow \div(v)$ sont des applications linéaires continues.

On peut donc appliquer le théorème de Lax-Milgram à \eqref{formVar0}, et en déduire qu'il existe un unique $u\in \tilde{V}$ vérifiant cette équation.

\end{ques}

\begin{ques}

On prend un maillage de triangles $(K_i)_{1\leq i\leq N}$ dont les barycentres sont les $x_i\in \mathring{K_i}$, et la vitesse $u$ s'écrirait
\[
    u = \sum_{k=1}^N u(x_i)\phi_i
\]
où les $\phi_i$ sont les éléments finis $\PP_1$ sur l'espace $V$, vérifiant $\phi_i(x_i) = 1$, $\phi_i = 0$ en dehors de $K_i$ et $\div(\phi_i) = 0$.

Une description explicite de ces éléments finis sous-entend de résoudre $\div(v) = 0$ avec les contraintes précédentes sur les $\phi_i$, ce qui est un problème difficile.
\end{ques}

\begin{ques}
On a déjà calculé à la question 1 la formulation variationnelle de la première équation :
\[
    \int_\Omega \nabla u\,:\nabla v - \int_\Omega p\div(v) - \int_{\partial\Omega} (\sigma_1 n) \cdot v
    = \int_\Omega f\cdot v
\]

Pour la deuxième équation, on obtient tout simplement :
\[
    -\int_\Omega \div(u) q = 0
\]

En sommant les deux équations (on a enlevé le terme de bord en $ (\sigma_1 n) \cdot v $, nul par hypothèse)
\begin{equation}\label{formVar1}
	\int_\Omega \nabla u\,:\nabla v\,dx -\int_\Omega p \div(v) + q \div(u)\,dx
    = \int_\Omega f\cdot v \tag{FV1}
\end{equation}

Dans cette formulation variationnelle, $(u, p)$ et $(v, q)$ sont dans l'espace produit
$
    V \coloneqq V_0 \times V_1
$
où
\[
    V_0 \coloneqq \{v\in H^1(\Omega,\RR^d)\,|\, \gamma_0(v)_{|\Gamma_D} = 0 \}
    \text{ et }
    V_1 = L^2(\Omega,\RR)
\]
sont des espaces de Hilbert: ainsi $V$ est également un espace de Hilbert.

La forme bilinéaire
\[
    a((u,p),(v,q)) = \int_\Omega \nabla u:\nabla v\,dx 
    -\int_\Omega p\div(v) + q\div(u)\,dx
\]
associée à \eqref{formVar1} est clairement symétrique.


\textbf{Coercivité:} La forme bilinéaire $a$ n'est \underline{pas} coercive dans le cas général. En effet, soient $\Omega = { ]0,1[ }^2$ le carré unité en dimension $d=2$ et $\Gamma_D = \{0\} \times { ]0,1[ } \cup \{1\}\times { ]0,1[ } $ les côtés ouest et est, et posons
\[
    u(x,y) = \left(x(1-y), y(1-x) \right)
\]
de sorte que
\[
    \nabla u(x,y) = \begin{pmatrix}
        1-y & -x \\ -y & 1-x
    \end{pmatrix}
\]
donc
\[
    \int_\Omega \nabla u:\nabla u = 
    \int_\Omega (1-y)^2+y^2+ x^2+(1-x)^2\,dx\,dy
    = 4\int_0^1 t^2\,dt = \frac 43
\]
et \[
    \int_\Omega \div(u) = \int_\Omega 2-x-y\,dx\,dy = 2 - 2\int_0^1 t\,dt = 1
\]

Il suffit donc de choisir pour $p$ la fonction constante égale à $2$ et on obtient
\[
    a((u,2),(u,2)) = \frac 43 - 2 = -\frac 23 < 0.
\]

\end{ques}




\begin{ques}
Calculons $-\sum_{j=1}^d \frac{\partial}{\partial x_j}\sigma_{2, ij}$.

\begin{align*}
    \sigma_2 &= \nabla u + \nabla^T u - p \Id \\
    \sigma_{2, ij} &= \frac{\partial u_i}{\partial x_j} +\frac{\partial u_j}{\partial x_i} - p \delta_{ij} \\
    \frac{\partial}{\partial x_j}\sigma_{2, ij} &= \frac{\partial^2 u_i}{\partial x_j^2} + \frac{\partial^2 u_j}{\partial x_i \partial x_j} - \frac{\partial p}{\partial x_j} \delta_{ij} \\
    -\sum_{j} \frac{\partial}{\partial x_j}\sigma_{2, ij} &= -\Delta u_i - \frac{\partial}{\partial x_i} \div(u) + \frac{\partial p}{\partial x_i} \\
\end{align*}
Comme on a supposé $\div(u)=0$, on trouve bien
\[
    -\sum_{j} \frac{\partial}{\partial x_j}\sigma_{2, ij} = -\Delta u_i + \frac{\partial p}{\partial x_i}
\]
\end{ques}

\begin{ques}
Multiplions scalairement par v et intégrons la première équation. D'après la question précédente et la formule de Green,
\begin{align*}
    \int_\Omega f \cdot v  &= \int_\Omega (-\Delta u + \nabla p) \cdot v
    = \int_\Omega \sum_{i=1}^d \left(-\Delta u_i + \frac{\partial p}{\partial x_i}\right) v_i \\
    &= - \int_\Omega \sum_{i=1}^d \sum_j \left( \frac{\partial}{\partial x_j}\sigma_{2, ij} \right)  v_i
    = - \sum_{i, j} \int_\Omega \left( \frac{\partial}{\partial x_j}\sigma_{2, ij} \right)  v_i \\
    &= \sum_{i, j} \int_\Omega \sigma_{2, ij} \frac{\partial v_i}{\partial x_j} - \int_{\partial\Omega} \sigma_{2, ij} v_i n_j \\
    &= \int_\Omega \sum_{i, j} \sigma_{2, ij} \frac{\partial v_i}{\partial x_j} - \int_{\partial\Omega} \sum_{i, j} \sigma_{2, ij} n_j v_i \\
    &= \int_\Omega \sigma_2 : \nabla v - \int_{\partial\Omega} (\sigma_2 n) \cdot v
    = \int_\Omega \sigma_2 : \nabla v
\end{align*}

Le deuxième terme est nul puisque sur $\Gamma_D$, $v=0$ et sur $\Gamma_N$, $\sigma_2 n=0$. 

Finalement, en explicitant de nouveau $\sigma_2$, l'expression s'écrit :
\begin{align*}
    & \int_\Omega (\nabla u + \nabla^T u) : \nabla v - p \Id : \nabla v \\
    =& \int_\Omega (\nabla u + \nabla^T u) : \nabla v\,dx -\int_\Omega p \div(v)\,dx
\end{align*}

Faisons de même sur la deuxième équation :
\[
    \int_\Omega q \div(u) = 0
\]

En soustrayant, on aboutit à la formulation variationnelle :
\[
    \int_\Omega (\nabla u + \nabla^T u) : \nabla v\,dx 
    - \int_\Omega p \div(v) + q \div(u)\,dx
    = \int_\Omega f \cdot v
\]

La symétrie de la forme bilinéaire associée (le membre de gauche) n'est pas immédiate: il faudrait peut-être trouver une reformulation.

\end{ques}

\begin{ques}
$A$ symétrique et $B$ antisymétrique.
\begin{align*}
    2 \sum_{i, j=1}^d A_{ij} B_{ij}
    &= \sum_{i, j=1}^d A_{ij} B_{ij} + \sum_{j, i=1}^d A_{ji} B_{ji} \\
    &= \sum_{i, j=1}^d A_{ij} B_{ij} - \sum_{j, i=1}^d A_{ij} B_{ij} \\
    &= 0
\end{align*}

Utilisons ceci pour simplifier le premier terme du membre de gauche de la formulation variationnelle trouvée à la question précédente.
(On peut voir $\nabla u$ et $\nabla v$ comme des matrices.)
\begin{align*}
    (\nabla u + \nabla u^T) : \nabla v &= (\nabla u + \nabla^T u) : \frac{(\nabla v + \nabla v^T) + (\nabla v - \nabla v^T)}{2} \\
    &= (\nabla u + \nabla u^T) : \frac{\nabla v + \nabla v^T}{2}
\end{align*}
puisque $\nabla u + \nabla u^T$ est symétrique et $\nabla v - \nabla v^T$ est antisymétrique.

On obtient la formulation variationnelle (strictement équivalente à celle donnée dans la question précédente):
\begin{equation*}
	\int_\Omega \frac{1}{2} (\nabla u + \nabla u^T) : (\nabla v + \nabla v^T) 
	- p \div(v)
	- q \div(u)
    = \int_\Omega f \cdot v
    \tag{FV2}
\end{equation*}
ou encore, avec les notations du cours
\begin{equation}\label{formVar2}
    2\int_\Omega e(u):e(v)\,dx - \int_\Omega q\div(u) + p\div(v)\,dx = \int_\Omega f\cdot v\,dx
\end{equation}


On voit qu'en fait la forme bilinéaire est effectivement symétrique entre $(u, p)$ et $(v, q)$.

On calcule, pour $(u,p)\in V$ :
\[
    a\left((u,p),(u,p)\right) =
    2\int_\Omega e(u):e(u)\,dx - 2\int_\Omega p\div(u)\,dx
\]

\textbf{Coercivité:} Le contre exemple utilisé à la question 3 fonctionne encore. On a ici
\[
    e(u) = \begin{pmatrix}
        1-y & -\frac{x+y}{2} \\ - \frac{x+y}{2} & 1-x
    \end{pmatrix}
\]
donc
\begin{align*}
    2\int_\Omega e(u): e(u) &= 4\int_0^1 t^2\,dt + 2\int_0^1\int_0^1 2\left(\frac{x+y}{2}\right)^2\,dx\,dy \\
    &= \frac 43 + \int_0^1\left[\frac 13(x+y)^3 \right]_0^1\,dy \\
    &= \frac 43 + \frac 13\int_0^1(1+y)^3 \,dy \\ 
    &= \frac 43 + \frac 1{12}\left[(1+y)^4 \right]_0^1 = \frac 43 + \frac{16}{12} = \frac 83 
\end{align*}
et il suffit de considérer $p \coloneqq 3$ pour avoir
\[
    a((u,4),(u,4)) = \frac 83 - 3 = - \frac 13 < 0.
\]

\end{ques}


\begin{ques}
    Dans le cas général, \eqref{formVar1} et \eqref{formVar2} traitent des problèmes avec des conditions aux limites différentes $\sigma \cdot n = \sigma_1\cdot n = 0$ et $\sigma\cdot n = \sigma_2\cdot n = 0$.

    Si $\Gamma_N = \varnothing$, les conditions aux limites de Neumann sont vides et les formulations \eqref{formVar1} et \eqref{formVar2} sont strictement équivalentes et ont les mêmes solutions.
\end{ques}

\section{Implémentation numérique}

\begin{ques}
    Le code source \verb|FreeFem++| se trouve dans le fichier \verb|simulation-projet.edp|.

    \lstinputlisting[firstline=10,lastline=50,language=C++]{simulation-projet.edp}
\end{ques}

\begin{ques}
    
\end{ques}

\begin{ques}
    Pour le terme source la gravité
    \[
        f = \begin{pmatrix}0 \\ -1\end{pmatrix}
    \]
    sur le domaine $\Omega = {]0,1 [}^2$ avec une condition de Dirichlet sur le bord $\Gamma_D = {]0,1[}\times \{0\}$

    \lstinputlisting[firstline=51,lastline=74,language=C++]{simulation-projet.edp}

    On trouvera les résultats pour le champ de pression $p:\Omega\longrightarrow\RR$ à la figure \ref{fig:ques10pressure} et pour le champ de vitesse $u:\Omega\longrightarrow\RR^2$ à la figure \ref{fig:ques10speed}.

    \begin{figure}[h]
        \begin{subfigure}{0.96\textwidth}
            \includegraphics[width=\textwidth]{fv1-pressure.eps}
            \caption{Champ de pression $p$ pour la formulation variationnelle \eqref{formVar1}.}
        \end{subfigure}
        \begin{subfigure}{0.96\textwidth}
            \includegraphics[width=\textwidth]{fv2-pressure.eps}
            \caption{Formulation variationnelle \eqref{formVar2}}
        \end{subfigure}
        \caption{Résultats du champ de pression pour les deux formulations variationnelles.}
        \label{fig:ques10pressure}
    \end{figure}
    \begin{figure}[h]
        \begin{subfigure}{\textwidth}
            \includegraphics[width=\textwidth]{fv1-stream.eps}
            \caption{Le champ de vitesse pour la formulation variationnelle \eqref{formVar1}.}
        \end{subfigure}
        \begin{subfigure}{\textwidth}
            \includegraphics[width=\textwidth]{fv2-stream.eps}
            \caption{Pour la formulation variationnelle \eqref{formVar2}.}
        \end{subfigure}
        \caption{Champs de vitesse $u:\Omega\longrightarrow\RR^d$ pour les deux formulations variationnelles. Éléments finis $\PP^2$.}
        \label{fig:ques10speed}
    \end{figure}
\end{ques}

\begin{ques}
    On trouvera des animations GIF de la déformation dans les dossiers \verb|deformation-FV*| où \verb|*| désigne 1 ou 2. Les fichiers source \verb|.eps| sont générés par le code \verb|FreeFem++| et sont convertis en GIF par un outil dans la ligne de commande (ImageMagick).

    On remarquera que le comportement des deux formulations variationnelles est différent. En effet dans la deuxième partie, les flux aux bords du domaine dépendent seulement de la partie \textit{symétrique} du gradient du champ de vitesse.

    \textbf{Remarque} Le domaine se déforme en s'étendant sur les côtés, mais le rendu ne reflète pas cela (\verb|FreeFem++| dézoome la fenêtre quand le domaine grandit...).

    Le code est comme suit:
    \lstinputlisting[firstline=76,language=C++]{simulation-projet.edp}
\end{ques}

\begin{ques}
    En essayant d'itérer 100 fois pour un coefficient $\delta = 0.06$, \verb|FreeFem++| refuse d'aller plus loin que l'itération $j=80$ et plante.

    Le compilateur \verb|FreeFem++| renvoie l'erreur suivante 
\begin{verbatim}
u min -0.284417 max 1.28446
v min -0.00724354 max 0.680117
  current line = 81
Exec error : Error move mesh triangles was reverse
  -- number :1
Exec error : Error move mesh triangles was reverse
  -- number :1
err code 7 ,  mpirank 0
    \end{verbatim}
    c'est-à-dire qu'un des triangles du maillage s'est écrasé au cours d'une opération \verb|movemesh|.

    Malheureusement, la documentation de \verb|FreeFem++| (\url{http://www.freefem.org/ff++/ftp/freefem++doc.pdf}) ne donne pas de méthode pour reconsruire le maillage de façon dynamique... Mais qu'une telle méthode existe pour des maillages en dimension 3!
\end{ques}


\end{document}
