\documentclass[11pt,a4paper]{article}

\usepackage{adjustbox}
\usepackage{algorithm}
\usepackage{algorithmic}
\usepackage{amsmath}
\usepackage{amssymb}
\usepackage{amsthm}
\usepackage{amsfonts}
\usepackage{afterpage}
\usepackage{blindtext}
\usepackage[font=footnotesize,labelfont=bf]{caption}
\usepackage{hyperref}
\usepackage[english]{babel}
\usepackage{bbm}
\usepackage{bigints}
\usepackage{bm}
\usepackage{cite}
\usepackage{color}
\usepackage{float}
\usepackage[left=2cm,right=2cm,top=2cm,bottom=2cm]{geometry}
\usepackage{graphicx}
\usepackage[utf8]{inputenc}
\usepackage{mathtools}
\usepackage{mdframed}
\usepackage{pgfplots} 
\usepackage{subfigure}
\usepackage{stmaryrd}
\usepackage{textcomp}
\usepackage{tikz}
\usepackage{url}
\renewcommand{\proofname}{Proof}
\theoremstyle{plain}
\newtheorem{monTheoNumrote}{Théorème}[section] % Environnement numéroté en fonction de la section
\newtheorem*{monTheoNonNumerote}{Théorème}  % Environnement non numéroté
\newtheorem{The}{Theorem}[section]
\newtheorem*{The*}{Theorem}
\newtheorem{Prop}{Proposition}[section]
\newtheorem*{Prop*}{Proposition} 
\newtheorem{Cor}{Corollary}[section]
\newtheorem*{Cor*}{Corollary}
\newtheorem{Conj}{Conjecture}[section]
\newtheorem{Lem}{Lemma}[section]
\renewcommand{\qed}{\unskip\nobreak\quad\qedsymbol}%
\numberwithin{equation}{section} % Numérote les équations section.numéro.
\theoremstyle{definition}
\newtheorem{Def}{Definition}[section]
\newtheorem{Rem}{Remark}[section]
\newtheorem*{Rem*}{Remark}
\newtheorem*{Lem*}{Lemma}
\newtheorem{Que}{Question}
\newcommand{\enstq}[2]{\left\{#1\mathrel{}\middle|\mathrel{}#2\right\}}
\newcommand{\Lp}[2]{L^#1(#2)}
\newcommand{\Sob}[3]{W^{#1,#2}(#3)}
\newcommand{\Rd}[0]{\mathbb{R}^d}
\newcommand{\RN}[0]{\mathbb{R}^N}
\newcommand{\Rn}[0]{\mathbb{R}^n}
\newcommand{\norm}[1]{\left\|#1\right\|}
\newcommand{\sinc}[0]{\textup{sinc}}
\newcommand{\functionDef}[5]{\begin{array}{lllll}
#1 & : & #2 & \longrightarrow & #3 \\
 & & #4 & \longmapsto &\displaystyle #5 \\
\end{array}}
\newcommand{\Theautorefname}{Theorem}
\newcommand{\Propautorefname}{Proposition}
\newcommand{\Corautorefname}{Corollary}
\newcommand{\Lemautorefname}{Lemma}
\newcommand{\Defautorefname}{Definition}
\newcommand{\N}{\mathbb{N}}
\newcommand{\Z}{\mathbb{Z}}
\newcommand{\D}{\mathbb{D}}
\newcommand{\R}{\mathbb{R}}
\newcommand{\A}{\mathcal{A}_{a,b}}
\newcommand{\Crad}{C^\infty_{c,rad}(B)}
\newcommand{\Lrad}{L^2_{rad}(B)}
\newcommand{\Lradab}{L^2_{rad}(\mathcal{A}_{a,b})}
\newcommand{\duality}[2]{\left\langle #1,#2\right\rangle}
\newcommand{\Hrad}{H^1_{rad}(B)}
\newcommand{\Hzrad}{H^1_{0,rad}(B)}
\newcommand{\rmin}{\delta_{\min}}
\newcommand{\rmax}{\delta_{\max}}
\newcommand{\corr}{\gamma}
\newcommand{\question}[1]{\begin{Que} \ 
#1
\end{Que}}
\newcommand{\abs}[1]{\left\lvert #1 \right\rvert}
\newcommand{\CL}[2]{\textup{CL}\left(\enstq{#1}{#2}\right)}
\newcommand{\Script}[1]{`\texttt{#1}`}
\newcommand{\espace}{\text{ }\qquad} 
\newcommand{\loc}{\text{loc}}
\newcommand{\SL}{\textup{SL}\hspace{1.5pt}}
\newcommand{\DL}{\textup{DL}\hspace{1.5pt}}
\newcommand{\fp}{\underset{\varepsilon \to 0}{\textup{f.p.}}}
\newcommand{\scalProd}[2]{\left(#1|#2\right)}
\newcommand{\toDo}[1]{{\color{red}#1}}
\newcommand{\bs}[1]{\boldsymbol{#1}}
\newcommand{\varInRange}[4]{(#1_{#2})_{#3 \leq #2 \leq #4}}
\newcommand{\from}{\colon}
\newcommand{\Cinf}{C^{\infty}}
\newcommand{\isdef}{\mathrel{\mathop:}=}
\newcommand{\defis}{=\mathrel{\mathop:}}

\renewcommand{\algorithmicrequire}{\textbf{Inputs:}}
\renewcommand{\algorithmicensure}{\textbf{Outputs:}}

\pgfplotsset{compat=1.13}
\author{Martin AVERSENG}
\title{Inversion formelle du simple couche}
\begin{document}
	\begin{The*}
		Pour tout $x,y \in [-1,1]$, on a
		\begin{equation}
			\ln |x-y| =  -\ln(2)  - 2\sum_{n=1}^{+\infty} \frac{T_n(x)T_n(y)}{n}
			\label{DeveloppementDuLn}
		\end{equation}
		\begin{proof}
			On sait que, pour tout $x \in (-1,1)$.
			\begin{equation}
				\int_{-1}^{1} -\frac{1}{2\pi}\ln|x-y| \frac{T_n(y)}{\omega(y)}dy = \lambda_n T_n(x),
			\end{equation}
			avec
			\begin{equation}
			\lambda_n = \Bigg|\begin{array}{lr}
			\frac{\ln(2)}{2} & \text{ si } n = 0\\
			\frac{1}{2n} & \text{ si } n \neq 0
			\end{array}
			\end{equation}
			Soit $f \in L^2(-1,1)$, on écrit $f(y) = \frac{\alpha(y)}{\omega(y)}$
			Et, comme $\alpha(y)$ est dans $L^2(-1,1,\omega(x)^2dx) \subset L^2(-1,1,\omega(x)dx)$, on peut écrire 
			\[\alpha(y) = \sum_{n=0}^{+\infty} \alpha_n T_n(y)\]
			avec 
			\[\alpha_n = \mu_n\int_{-1}^{1} \alpha(y) \frac{T_n(y)}{\omega(y)}dy = \mu_n\int_{-1}^1{f(y) T_n(y)dy}\]
			où \begin{equation}
			\mu_n = \Bigg|\begin{array}{lr}
			1/\pi& \text{ si } n = 0\\
			2/\pi & \text{ si } n \neq 0
			\end{array}
			\end{equation}
			Ainsi, on obtient 
			\[ \frac{-1}{2\pi}\int_{-1}^{1} \ln(|x-y|) f(y) = \sum_{n = 0}^{+ \infty} \lambda_n \alpha_n T_n(x)\]
			En échangeant l'ordre de sommation, on obtient
			\[ \frac{-1}{2\pi}\int_{-1}^{1} \ln(|x-y|) f(y) = \int_{-1}^{1} \left[\sum_{n = 0}^{+ \infty} \lambda_n \mu_n T_n(x)T_n(y)\right] f(y) dy.\]
			D'où le résultat en identifiant le noyau des opérateurs de gauche et de droite, puisque leur action coïncide point par point sur les fonctions de $L^2(-1,1)$. 
		\end{proof}
	\end{The*}
	Par la suite, on pose 
	\[\ln |x-y| = \sum_{n=0}^{+\infty}s_n T_n(x)T_n(y)\]
	Où, d'après le théorème précédent, 
	$$s_n = \Bigg|\begin{array}{lr}
	- \ln(2) & \text{ si } n = 0\\
	- \frac{2}{n} & \text{ si } n \neq 0
	\end{array}.$$ 
	D'autre part, remarquons que les polynômes de Tchebitchev vérifient 
	\[\left[-\omega(x)D_x\right]^2 T_n = n^2 T_n.\]
	Ainsi, en appliquant l'opérateur $-\left[\omega(x)D_x\right]^2$ dans la formule (\ref{DeveloppementDuLn}), on obtient	
	\begin{Cor*}
		\[-\frac{1-xy}{2|x-y|^2} = \sum_{n=1}^{+\infty} n T_n(x)T_n(y)\]
	\end{Cor*}
	
	Pour $x,y \in (1,1)$, on pose $L(x,y) = \sum_{n=0}^{+\infty} h_n \frac{Tn(x) T_n(y)}{\omega(x)\omega(y)}$, avec 
	\[h_n = \frac{1}{s_n \mu_n^2} = \Bigg|\begin{array}{lr}
	-\frac{\pi^2}{\ln(2)} & \text{ si } n = 0\\
	-\frac{n\pi^2}{8} & \text{ si } n \neq 0
	\end{array}\]
	Ainsi, 
	\begin{align}
		L(x,y) &= \frac{1}{\sqrt{(1-x^2)(1-y^2)}} \left(-\frac{\pi^2}{\ln(2)} - \frac{\pi^2}{8} \sum_{n=1}^{+\infty} n T_n(x)T_n(y)\right) \\
		&= \frac{\pi^2}{\sqrt{(1-x^2)(1-y^2)}} \left(-\frac{1}{\ln(2)} + \frac{1-xy}{16|x-y|^2}\right)
	\end{align}
	
	\begin{Cor*} 
		Les opérateurs $S$ et $H$ de noyaux respectifs $K(x,y) = \ln(|x-y|)$ et $L(x,y)$ sont inverses l'un de l'autre. 
		\begin{proof}
			Pour que les deux opérateurs soient mutuellement inverses, il suffit de vérifier que pour tout $n$, on a
			\begin{equation}
				HS \frac{T_n}{\omega} = \frac{T_n}{\omega}
			\end{equation}
			Or, 
			\[S\frac{T_n}{\omega} = s_n \mu_n T_n,\]			
			et 
			\[HT_n = \frac{1}{s_n \mu_n^2} \mu_n \frac{T_n}{\omega} \]
			ce qui conclut la démonstration.
		\end{proof}
	\end{Cor*}
\end{document}