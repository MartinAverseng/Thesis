\documentclass[11pt,a4paper]{article}

\usepackage{adjustbox}
\usepackage{algorithm}
\usepackage{algorithmic}
\usepackage{amsmath}
\usepackage{amssymb}
\usepackage{amsthm}
\usepackage{amsfonts}
\usepackage{afterpage}
\usepackage{blindtext}
\usepackage[font=footnotesize,labelfont=bf]{caption}
\usepackage{hyperref}
\usepackage[english]{babel}
\usepackage{bbm}
\usepackage{bigints}
\usepackage{bm}
\usepackage{cite}
\usepackage{color}
\usepackage{float}
\usepackage[left=2cm,right=2cm,top=2cm,bottom=2cm]{geometry}
\usepackage{graphicx}
\usepackage[utf8]{inputenc}
\usepackage{mathtools}
\usepackage{mdframed}
\usepackage{pgfplots} 
\usepackage{subfigure}
\usepackage{stmaryrd}
\usepackage{textcomp}
\usepackage{tikz}
\usepackage{url}
\renewcommand{\proofname}{Proof}
\theoremstyle{plain}
\newtheorem{monTheoNumrote}{Théorème}[section] % Environnement numéroté en fonction de la section
\newtheorem*{monTheoNonNumerote}{Théorème}  % Environnement non numéroté
\newtheorem{The}{Theorem}[section]
\newtheorem*{The*}{Theorem}
\newtheorem{Prop}{Proposition}[section]
\newtheorem*{Prop*}{Proposition} 
\newtheorem{Cor}{Corollary}[section]
\newtheorem*{Cor*}{Corollary}
\newtheorem{Conj}{Conjecture}[section]
\newtheorem{Lem}{Lemma}[section]
\renewcommand{\qed}{\unskip\nobreak\quad\qedsymbol}%
\numberwithin{equation}{section} % Numérote les équations section.numéro.
\theoremstyle{definition}
\newtheorem{Def}{Definition}[section]
\newtheorem{Rem}{Remark}[section]
\newtheorem*{Rem*}{Remark}
\newtheorem*{Lem*}{Lemma}
\newtheorem{Que}{Question}
\newcommand{\enstq}[2]{\left\{#1\mathrel{}\middle|\mathrel{}#2\right\}}
\newcommand{\Lp}[2]{L^#1(#2)}
\newcommand{\Sob}[3]{W^{#1,#2}(#3)}
\newcommand{\Rd}[0]{\mathbb{R}^d}
\newcommand{\RN}[0]{\mathbb{R}^N}
\newcommand{\Rn}[0]{\mathbb{R}^n}
\newcommand{\norm}[1]{\left\|#1\right\|}
\newcommand{\sinc}[0]{\textup{sinc}}
\newcommand{\functionDef}[5]{\begin{array}{lllll}
#1 & : & #2 & \longrightarrow & #3 \\
 & & #4 & \longmapsto &\displaystyle #5 \\
\end{array}}
\newcommand{\Theautorefname}{Theorem}
\newcommand{\Propautorefname}{Proposition}
\newcommand{\Corautorefname}{Corollary}
\newcommand{\Lemautorefname}{Lemma}
\newcommand{\Defautorefname}{Definition}
\newcommand{\N}{\mathbb{N}}
\newcommand{\Z}{\mathbb{Z}}
\newcommand{\D}{\mathbb{D}}
\newcommand{\R}{\mathbb{R}}
\newcommand{\A}{\mathcal{A}_{a,b}}
\newcommand{\Crad}{C^\infty_{c,rad}(B)}
\newcommand{\Lrad}{L^2_{rad}(B)}
\newcommand{\Lradab}{L^2_{rad}(\mathcal{A}_{a,b})}
\newcommand{\duality}[2]{\left\langle #1,#2\right\rangle}
\newcommand{\Hrad}{H^1_{rad}(B)}
\newcommand{\Hzrad}{H^1_{0,rad}(B)}
\newcommand{\rmin}{\delta_{\min}}
\newcommand{\rmax}{\delta_{\max}}
\newcommand{\corr}{\gamma}
\newcommand{\question}[1]{\begin{Que} \ 
#1
\end{Que}}
\newcommand{\abs}[1]{\left\lvert #1 \right\rvert}
\newcommand{\CL}[2]{\textup{CL}\left(\enstq{#1}{#2}\right)}
\newcommand{\Script}[1]{`\texttt{#1}`}
\newcommand{\espace}{\text{ }\qquad} 
\newcommand{\loc}{\text{loc}}
\newcommand{\SL}{\textup{SL}\hspace{1.5pt}}
\newcommand{\DL}{\textup{DL}\hspace{1.5pt}}
\newcommand{\fp}{\underset{\varepsilon \to 0}{\textup{f.p.}}}
\newcommand{\scalProd}[2]{\left(#1|#2\right)}
\newcommand{\toDo}[1]{{\color{red}#1}}
\newcommand{\bs}[1]{\boldsymbol{#1}}
\newcommand{\varInRange}[4]{(#1_{#2})_{#3 \leq #2 \leq #4}}
\newcommand{\from}{\colon}
\newcommand{\Cinf}{C^{\infty}}
\newcommand{\isdef}{\mathrel{\mathop:}=}
\newcommand{\defis}{=\mathrel{\mathop:}}

\renewcommand{\algorithmicrequire}{\textbf{Inputs:}}
\renewcommand{\algorithmicensure}{\textbf{Outputs:}}

\pgfplotsset{compat=1.13}
\author{Martin AVERSENG}
\title{Simple couche et Hyper-singulier sur un segment}
\begin{document}
\maketitle	

\section{Potentiel de simple couche sur un segment}

Commençons par fixer les notations, de la même manière que dans le papier d'Oscar Bruno. 
Soit une fente $\Gamma = (-1,1)\times \{0\}$ dans le plan $\mathbb{R}^2$. On note $\Omega = \mathbb{R}^2 \setminus \Gamma$ le domaine extérieur à la fente. On considère une onde plane $u^{inc}$ de vecteur d'onde $k$ et on cherche le champ $u$ diffracté par  $\Gamma$. Selon la modélisation choisie, le champ $u \in H^1(\Omega)$ est solution de l'une des équations suivantes (condition de Dirichlet ou Neumann) : 
\begin{equation} \left\{\begin{array}{rll}
\Delta u + k^2 u &= 0 & \text{dans $\Omega$}\\
u &= u^{inc} & \text{sur } \Gamma
\end{array}\right.
\label{ProblemeDeDirichlet}
\end{equation}
\begin{equation} \left\{\begin{array}{rll}
\Delta u + k^2 u &= 0 & \text{dans $\Omega$}\\
\frac{\partial u}{\partial n} &= \frac{\partial u^{inc}}{\partial n} & \text{sur } \Gamma
\end{array}\right.
\label{ProblemeDeNeumann}
\end{equation}
Ce problème peut être mis sous forme d'équations intégrales. Soit $G_k$ le noyau de Green définit pour $x \in \mathbb{R}^2$ par
\[G_k(x,x') = \left\{\begin{array}{cl} 
-\frac{1}{2\pi}\ln(|x-x'|), & \text{si $k=0$}\\
 \frac{i}{4}H_0^1(k|x-x'|),& \text{si $k>0$} 
\end{array}\right.,\]
Où $H^1_0$ est la fonction de Hankel de première espèce. 
Soit $S$ l'opérateur de simple couche sur $\Gamma$ définit par
	\[Su(x) = \int_\Gamma G_k(x,x')u(x')d\Gamma(x')\]
et $N$ l'opérateur "hypersingulier" défini par 
\[Nu(x) = \lim_{z \to 0^+} \dfrac{\partial}{\partial z}\int_\Gamma \dfrac{\partial G_k(x + z e_y, x')}{\partial e_y}u(x')d\Gamma(x')\]
avec $\frac{\partial G_k}{\partial e_y} = e_y \cdot \nabla_{x'} G_k(x,x')$ et où $e_y$ est le vecteur $(0,1)$. Soient $\mu$ et $\lambda$ des solutions des équations intégrales suivantes : 
\[ S\lambda = u^{inc}_{|\Gamma},\]
\[ N\mu = \frac{\partial u^{inc}}{\partial n}_{|\Gamma}.\]
Alors, dans le cas du problème de Dirichlet, $\lambda$ est le saut de la dérivée normale de $u$ à travers la fente, où la normale est définie de manière opposée de chaque côté de la fente. Dans le cas du problème de Neumann, l'unique solution (à constante près) $\mu$ est le saut du champ $u$ à travers la fente. On obtient ensuite le champ total dans tout l'espace avec les formules de représentation intégrale. Si l'on résout un problème de Dirichlet, on pose par convention $\mu = 0$ (en effet, on cherche alors une solution continue à travers la fente). De même, si l'on résout un problème de Neumann, on prend par convention $\lambda = 0$. On a alors pour $u$ la formule suivante pour $x \notin \Gamma$
\[u(x) = \mathcal{S}\lambda(x) - \mathcal{D}\mu(x)\]
Où 
\[\mathcal{S}\lambda = \int_{\Gamma}G_k(x,x') \lambda(x')d\Gamma(x'), \text{ et}\]
\[\mathcal{D}\mu = \int_{\Gamma} \frac{\partial{G_k(x,x')}}{\partial e_y} \mu(x')d\Gamma(x').\]

À cause de la présence de "bords" sur la fente $\Gamma$, la solution de l'équation n'est pas infiniment dérivable, même si l'onde incidente $u^{inc}$ l'est, contrairement au cas où le domaine de résolution de l'équation de Helmholtz est plus régulier. Plus précisément, on a le résultat suivant, cf. \cite{costabel2003asymptotics} : 
\begin{The} On suppose que $u_{inc}$ est infiniment dérivable sur $\Gamma$. La solution $u$ du problème \ref{ProblemeDeDirichlet} recherchée vérifie alors le développement suivant : 
\[ u(x) = \]
\end{The}


On pose $\omega(x)= \sqrt{1-x^2}$. 


On s'intéresse aux propriétés de l'opérateur $\alpha \mapsto \dfrac{1}{\omega} S \dfrac{1}{\omega}\alpha$. Selon la remarque du paragraphe 2.3 de \cite{bruno2012second}, on admet la conjecture suivante :
\begin{The} Soit $f$ une fonction dans $H^s(-1,1)$, $s>0$. Alors l'unique solution de l'équation d'inconnue $\alpha \in H^1(-1,1)$ :
\[S\left(\dfrac{\alpha}{\omega}\right) = f\]
est dans $H^{s+1}(-1,1)$. 
\end{The}

Le résultat est probablement un peu faux. En revanche, il est clair que si le second membre est $C^{\infty}$, $\alpha$ est $C^{\infty}$, ce qui est prouvé dans \cite{costabel2003asymptotics} et utilisé dans \cite{bruno2012second}. Nous nous restreignons dans un premier temps à l'analyse de ce cas, qui ne permet malheureusement pas de comprendre l'impact de la régularité du second membre sur la vitesse de convergence. 

\begin{Rem} L'équation 
\end{Rem}

L'intérêt de cette propriété est qu'on obtient une convergence rapide de l'approximation par éléments finis lorsque le pas du maillage $h$ devient petit. Ce fait se base sur une version du lemme de Céa adaptée à notre situation.
Soit $S_{\omega} := \dfrac{1}{\omega}S\dfrac{1}{\omega}$ (ce n'est pas la même notation que celle choisie par Oscar Bruno). De manière immédiate, $S_\omega$ hérite de la propriété de coercivité de $S$. 
\begin{Prop} Pour tout $\alpha$ tel que $\frac{\alpha}{\omega} \in H^{-1/2}(-1,1)$, on a 
\[ (S_\omega \alpha,\alpha) \geq c \norm{\frac{\alpha}{\omega}}_{H^{-1/2}}^2\] 
\begin{proof}
On a
$(S_{\omega}\alpha,\alpha) = \left(\dfrac{1}{\omega}S\dfrac{1}{\omega}\alpha , \alpha\right) = \left(S\dfrac{1}{\omega}\alpha , \dfrac{1}{\omega}\alpha\right) \geq c \norm{\dfrac{\alpha}{\omega}}_{H^{-1/2}}^2$
\end{proof}
\end{Prop}
Soit $V_h$ un sous-espace vectoriel de dimension finie de $\enstq{\alpha}{\alpha/\omega \in H^{-1/2}}$. Soit $\alpha_h$ l'unique solution de la formulation variationnelle : $\forall \beta_h \in V_h$ : 
\[(S_\omega\alpha_h,\alpha_h) = \int_{-1}^{1} f(x)\dfrac{\beta_h(x)}{w(x)}.\]
Le lemme de Céa assure 
\[\norm{(\alpha - \alpha_h)/\omega}_{H^{-1/2}} \leq \inf_{\beta_h \in V_h}C\norm{(\alpha-\beta_h)/\omega}_{H^{-1/2}}\]
\paragraph{Question} Y a-t-il une bonne méthode pour montrer que le terme de droite est d'ordre $O(h)$ pour des éléments finis $\mathbbm{P}_1$ ? ?

Soient $T_n$ les polynômes de Tchebychev de première espèce. 
D'après \cite{bruno2012second}, on a 
\[ S \left(\dfrac{T_n}{\omega}\right) = \lambda_n T_n\]
Avec $\lambda_0 = \frac{\ln(2)}{2}$ et $\lambda_n = \dfrac{1}{2n}$ pour $n\neq 0$. D'autre part, considérons l'opérateur $\Lambda$ qui, à une fonction $g$ définie sur le segment $(-1,1)$ associe la donnée de Neumann de la solution $u$ du problème 
$\left\{\begin{array}{rll}
-\Delta u &= 0 & \text{ dans } \mathbb{R}^2 \setminus \{(-1,1)\times \{0\}\}\\
u &= g & \text{ sur } (-1,1)\times\{0\}
\end{array}\right.$
En prenant la normale du côté des $y$ positifs. Les formules de Calderòn impliquent alors que 
\[S\Lambda g = \frac{1}{2}g\]
Donc $S^{-1} = 2\Lambda$. On a donc 
\[ \omega\Lambda T_n = \mu_n T_n \]
où $\mu_n = \frac{1}{\ln(2)}$ si $n=0$ et $\mu_n = n$ sinon. Or l'équation différentielle vérifiée par les polynômes $T_n$ nous fournit un opérateur différentiel $P$ explicite qui satisfait pour $n \neq 0$ à la relation $PT_n = -\mu_n^2 T_n$. L'opérateur $P$ est donné par  \[ P = (1-x^2) \partial_{xx} - x\partial_x = \left(\omega \partial_x\right)^2\] 
Les polynômes $T_n$ forment une base Hilbertienne de $L^2\left[(-1,1),\omega^{-1}(x)dx\right]$. On a donc pour toute fonction $\varphi = \sum_{n=0}^{+\infty}c_n T_n(x)$ dans cet espace : 
\[\left[P^2 + (\omega\Lambda)^2\right] \varphi = c_0\mu_0^2 T_0 \]
L'intérêt de cette relation est qu'il permet d'exprimer l'opérateur $\omega\Lambda$ en fonction d'un opérateur différentiel donc local, qui permet une discrétisation numérique efficace. Dans l'optique de la résolution d'un problème intégral, on pourrait utiliser $\omega\Lambda$ ou une approximation de celui-ci pour préconditionner l'équation. 
Puisque les deux opérateurs du membre de gauche sont diagonalisés par une même base Hilbertienne, ils commutent sur cet espace de Hilbert. 




































\bibliographystyle{plain}
\bibliography{../Biblio/biblio}  
\end{document}