
\documentclass[]{article}

\usepackage{adjustbox}
\usepackage{algorithm}
\usepackage{algorithmic}
\usepackage{amsmath}
\usepackage{amssymb}
\usepackage{amsthm}
\usepackage{amsfonts}
\usepackage{afterpage}
\usepackage{blindtext}
\usepackage[font=footnotesize,labelfont=bf]{caption}
\usepackage{hyperref}
\usepackage[english]{babel}
\usepackage{bbm}
\usepackage{bigints}
\usepackage{bm}
\usepackage{cite}
\usepackage{color}
\usepackage{float}
\usepackage[left=2cm,right=2cm,top=2cm,bottom=2cm]{geometry}
\usepackage{graphicx}
\usepackage[utf8]{inputenc}
\usepackage{mathtools}
\usepackage{mdframed}
\usepackage{pgfplots} 
\usepackage{subfigure}
\usepackage{stmaryrd}
\usepackage{textcomp}
\usepackage{tikz}
\usepackage{url}
\renewcommand{\proofname}{Proof}
\theoremstyle{plain}
\newtheorem{monTheoNumrote}{Théorème}[section] % Environnement numéroté en fonction de la section
\newtheorem*{monTheoNonNumerote}{Théorème}  % Environnement non numéroté
\newtheorem{The}{Theorem}[section]
\newtheorem*{The*}{Theorem}
\newtheorem{Prop}{Proposition}[section]
\newtheorem*{Prop*}{Proposition} 
\newtheorem{Cor}{Corollary}[section]
\newtheorem*{Cor*}{Corollary}
\newtheorem{Conj}{Conjecture}[section]
\newtheorem{Lem}{Lemma}[section]
\renewcommand{\qed}{\unskip\nobreak\quad\qedsymbol}%
\numberwithin{equation}{section} % Numérote les équations section.numéro.
\theoremstyle{definition}
\newtheorem{Def}{Definition}[section]
\newtheorem{Rem}{Remark}[section]
\newtheorem*{Rem*}{Remark}
\newtheorem*{Lem*}{Lemma}
\newtheorem{Que}{Question}
\newcommand{\enstq}[2]{\left\{#1\mathrel{}\middle|\mathrel{}#2\right\}}
\newcommand{\Lp}[2]{L^#1(#2)}
\newcommand{\Sob}[3]{W^{#1,#2}(#3)}
\newcommand{\Rd}[0]{\mathbb{R}^d}
\newcommand{\RN}[0]{\mathbb{R}^N}
\newcommand{\Rn}[0]{\mathbb{R}^n}
\newcommand{\norm}[1]{\left\|#1\right\|}
\newcommand{\sinc}[0]{\textup{sinc}}
\newcommand{\functionDef}[5]{\begin{array}{lllll}
#1 & : & #2 & \longrightarrow & #3 \\
 & & #4 & \longmapsto &\displaystyle #5 \\
\end{array}}
\newcommand{\Theautorefname}{Theorem}
\newcommand{\Propautorefname}{Proposition}
\newcommand{\Corautorefname}{Corollary}
\newcommand{\Lemautorefname}{Lemma}
\newcommand{\Defautorefname}{Definition}
\newcommand{\N}{\mathbb{N}}
\newcommand{\Z}{\mathbb{Z}}
\newcommand{\D}{\mathbb{D}}
\newcommand{\R}{\mathbb{R}}
\newcommand{\A}{\mathcal{A}_{a,b}}
\newcommand{\Crad}{C^\infty_{c,rad}(B)}
\newcommand{\Lrad}{L^2_{rad}(B)}
\newcommand{\Lradab}{L^2_{rad}(\mathcal{A}_{a,b})}
\newcommand{\duality}[2]{\left\langle #1,#2\right\rangle}
\newcommand{\Hrad}{H^1_{rad}(B)}
\newcommand{\Hzrad}{H^1_{0,rad}(B)}
\newcommand{\rmin}{\delta_{\min}}
\newcommand{\rmax}{\delta_{\max}}
\newcommand{\corr}{\gamma}
\newcommand{\question}[1]{\begin{Que} \ 
#1
\end{Que}}
\newcommand{\abs}[1]{\left\lvert #1 \right\rvert}
\newcommand{\CL}[2]{\textup{CL}\left(\enstq{#1}{#2}\right)}
\newcommand{\Script}[1]{`\texttt{#1}`}
\newcommand{\espace}{\text{ }\qquad} 
\newcommand{\loc}{\text{loc}}
\newcommand{\SL}{\textup{SL}\hspace{1.5pt}}
\newcommand{\DL}{\textup{DL}\hspace{1.5pt}}
\newcommand{\fp}{\underset{\varepsilon \to 0}{\textup{f.p.}}}
\newcommand{\scalProd}[2]{\left(#1|#2\right)}
\newcommand{\toDo}[1]{{\color{red}#1}}
\newcommand{\bs}[1]{\boldsymbol{#1}}
\newcommand{\varInRange}[4]{(#1_{#2})_{#3 \leq #2 \leq #4}}
\newcommand{\from}{\colon}
\newcommand{\Cinf}{C^{\infty}}
\newcommand{\isdef}{\mathrel{\mathop:}=}
\newcommand{\defis}{=\mathrel{\mathop:}}

\renewcommand{\algorithmicrequire}{\textbf{Inputs:}}
\renewcommand{\algorithmicensure}{\textbf{Outputs:}}

\pgfplotsset{compat=1.13}





\begin{document}
	Le théorème de Steinbach et Wendland :
	\begin{The}
		Soient $\mathcal{V}$ et $\mathcal{W}$ deux espaces de Hilbert, et soient $\mathcal{V}_h$, $\mathcal{W}_h$, deux sous-espaces de dimension finie $M$ et $N$ de $\mathcal{V}$ et $\mathcal{W}_h$ respectivement. Soit $\opFromTo{A}{\mathcal{V}}{\mathcal{V^*}}$ et $\opFromTo{B}{\mathcal{W}}{\mathcal{V^*}}$ deux opérateurs satisfaisant les conditions suivantes :
		\begin{itemize}
			\item $A$ est auto-adjoint sur $\mathcal{V}$, tel que 
			\[\duality{Au}{u} \geq c_A\norm{u}_\mathcal{V}^2\]
			\[\duality{Au}{v} \leq C_A \norm{u}_{\mathcal{V}}\norm{v}_{\mathcal{V}}\]
			\item $B$ est continu
			\[\duality{Au}{v} \leq C_B \norm{u}_{\mathcal{V}} \norm{v}_{\mathcal{V}}\]
			\item $B$ satisfait à la condition de stabilité 
			\[c_B\norm{w_h}_{\mathcal{W}} \leq \sup_{v_h \neq 0, v_h \in \mathcal{V}} \dfrac{\abs{\duality{B w_h}{v_h}}}{\norm{v_h}_{V}},\]
			où la constante $c_B$ ne dépend pas du choix de $\mathcal{W}_h$. 
		\end{itemize}
		On pose $\mathcal{T} =\isdef  \opFromTo{\mathcal{B}'\mathcal{A}^{-1}\mathcal{B}}{\mathcal{W}}{\mathcal{W}^*} $. Pour tous ces opérateurs, on note $[\cdot]$ leur matrice de Galerkine. Alors pour tout vecteur $w_h \in \mathbb{R}^N$, 
		\[\gamma_1  w_h^T \left[\mathcal{T} \right]w_h \leq [B]^T[A]^{-1}[B] w_h \leq w_h^T \left[ \mathcal{T}\right] w_h \]
		
	\end{The}
	Soit $\Gamma$ un arc ouvert $C^{\infty}$ ($C^1$ suffirait ?), on note $S_\omega$ le simple couche défini sur $H_\omega^{-1/2}(\Gamma)$ (on définit $H_\omega^{s}([-1,1])$ en transportant l'espace de Sobolev $H^s$ sur le cercle, puis $H_\omega^{s}(\Gamma)$ en utilisant un $C^{\infty}$ difféomorphisme. Il faudrait prouver ce qui suit :
	\begin{Conj}
		$H_{\omega}^s(\Gamma)$ est un espace de Hilbert pour le produit scalaire
		\[(u,v) \mapsto \int_{\Gamma} \frac{1}{\sqrt{d(x,\partial \Gamma)}}u(x)v(x)dx,\]
		et pour tout $s$ et pour $s < t$, l'injection $H^s_{\omega}(\Gamma) \subset H^{t}_{\omega}(\Gamma)$ est compacte. 
		$H^{1/2}_{\omega}(\Gamma)$ est le dual de $H^{-1/2}_\omega(\Gamma)$.
		L'opérateur $A : u \mapsto (\omega(x)\partial_x)^2 u + \scalProd{u}{1}_\omega$ est auto-adjoint continu et elliptique de $H_{\omega}^{1/2} \longrightarrow H_{\omega}^{-3/2}(\Gamma)$. 
	\end{Conj}
	On pose $\Delta_\omega$ l'opérateur défini 
	\begin{The}
		contenu...
	\end{The}
\end{document}
