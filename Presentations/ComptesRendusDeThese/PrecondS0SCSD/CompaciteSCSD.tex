\documentclass[11pt,a4paper]{article}

%\usepackage{adjustbox}
%\usepackage{algorithm}
%\usepackage{algorithmic}
%\usepackage{lmodern}
\usepackage{amsmath}
\usepackage{amssymb}
\usepackage{anyfontsize}
%\usepackage{amsthm}
\usepackage{amsfonts}
%\usepackage{afterpage}
%\usepackage{blindtext}
\usepackage[hidelinks]{hyperref}
\usepackage[english]{babel}
\usepackage{bbm}
\usepackage{bigints}
%\usepackage{bm}
\usepackage{cite}
\usepackage{color}
\usepackage{float}
\usepackage{graphicx}
\usepackage[utf8]{inputenc}
\usepackage{mathptmx}
\usepackage{mathtools}
%\usepackage{mdframed}
\usepackage{pgfplots} 
\usetikzlibrary{external}
\tikzexternalize
\usepackage{subcaption}
\usepackage{stmaryrd}
\SetSymbolFont{stmry}{bold}{U}{stmry}{m}{n}
					%%%
\usepackage{textcomp}
\usepackage{tikz}
\usepackage{url}

\smartqed  % flush right qed marks, e.g. at end of proof

%\renewcommand{\proofname}{Proof}
%\newtheorem{monTheoNumrote}{Théorème} % Environnement numéroté en fonction de la section
%\newtheorem*{monTheoNonNumerote}{Théorème}  % Environnement non numéroté
%\newtheorem{The}{Theorem}
%\newtheorem*{The*}{Theorem}
%\newtheorem{Prop}{Proposition}
%\newtheorem*{Prop*}{Proposition} 
%\newtheorem{Cor}{Corollary}
%\newtheorem*{Cor*}{Corollary}
%\newtheorem{Conj}{Conjecture}
%\newtheorem{Lem}{Lemma}
%\renewcommand{\qed}{\unskip\nobreak\quad\qedsymbol}%
%\theoremstyle{definition}
%\newtheorem{Def}{Definition}
%\newtheorem{Rem}{Remark}
%\newtheorem*{Rem*}{Remark}
%\newtheorem*{Lem*}{Lemma}
%\newtheorem{Que}{Question}
\newcommand{\enstq}[2]{\left\{#1\mathrel{}\middle|\mathrel{}#2\right\}}
\newcommand{\Lp}[2]{L^#1(#2)}
\newcommand{\Sob}[3]{W^{#1,#2}(#3)}
\newcommand{\Rd}[0]{\mathbb{R}^d}
\newcommand{\RN}[0]{\mathbb{R}^N}
\newcommand{\Rn}[0]{\mathbb{R}^n}
\newcommand{\norm}[1]{\left\|#1\right\|}
\newcommand{\sinc}[0]{\textup{sinc}}
\newcommand{\functionDef}[5]{\begin{array}{lllll}
#1 & : & #2 & \longrightarrow & #3 \\
 & & #4 & \longmapsto &\displaystyle #5 \\
\end{array}}
\newcommand{\Theautorefname}{Theorem}
\newcommand{\Propautorefname}{Proposition}
\newcommand{\Corautorefname}{Corollary}
\newcommand{\Lemautorefname}{Lemma}
\newcommand{\Defautorefname}{Definition}
\newcommand{\Conjautorefname}{Conjecture}
\newcommand{\Remautorefname}{Remark}
\renewcommand{\sectionautorefname}{Section}
\renewcommand{\subsectionautorefname}{Section}
%\renewcommand{\algorithmicrequire}{\textbf{Inputs:}}
%\renewcommand{\algorithmicensure}{\textbf{Outputs:}}

\newcommand{\N}{\mathbb{N}}
\newcommand{\Z}{\mathbb{Z}}
\newcommand{\D}{\mathbb{D}}
\newcommand{\R}{\mathbb{R}}
\newcommand{\A}{\mathcal{A}_{a,b}}
\newcommand{\Crad}{C^\infty_{c,\textup{rad}}(B)}
\newcommand{\Lrad}{L^2_{\textup{rad}}(B)}
\newcommand{\Lradab}{L^2_{\textup{rad}}(\mathcal{A}_{a,b})}
\newcommand{\duality}[2]{\left\langle #1,#2\right\rangle}
\newcommand{\Hrad}{H^1_{\textup{rad}}(B)}
\newcommand{\Hzrad}{H^1_{0,\textup{rad}}(B)}
\newcommand{\rmin}{\delta_{\min}}
\newcommand{\rmax}{\delta_{\max}}
\newcommand{\corr}{\gamma}
%\newcommand{\question}[1]{\begin{Que} \ 
%#1
%\end{Que}}
\newcommand{\abs}[1]{\left\lvert #1 \right\rvert}
\newcommand{\CL}[2]{\textup{CL}\left(\enstq{#1}{#2}\right)}
\newcommand{\Script}[1]{`\texttt{#1}`}
\newcommand{\espace}{\text{ }\qquad} 
\newcommand{\loc}{\text{loc}}
\newcommand{\SL}{\textup{SL}\hspace{1.5pt}}
\newcommand{\DL}{\textup{DL}\hspace{1.5pt}}
\newcommand{\fp}{\underset{\varepsilon \to 0}{\textup{f.p.}}}
\newcommand{\scalProd}[2]{\left(#1|#2\right)}
\newcommand{\toDo}[1]{{\color{red}#1}}
\newcommand{\bs}[1]{\boldsymbol{#1}}
\newcommand{\varInRange}[4]{(#1_{#2})_{#3 \leq #2 \leq #4}}
\newcommand{\from}{\colon}
\newcommand{\Cinf}{C^{\infty}}
\newcommand{\isdef}{\mathrel{\mathop:}=}
\newcommand{\defis}{=\mathrel{\mathop:}}


\pgfplotsset{compat=newest}
\usepackage{array}
\usepackage{booktabs}
\setlength{\heavyrulewidth}{1.5pt}
\setlength{\abovetopsep}{4pt}
\author{Martin AVERSENG}
\title{Compacité dans $H^{1/2}(\Gamma)$ pour l'opérateur régulier dans la méthode SCSD.}
\begin{document}
\maketitle
\section*{}
Soit $\Omega$ un domaine (ouvert connexe) borné de $\mathbb{R}^n$ ($n=2$ ou $3$) lipschitzien, c'est à dire qui est localement situé sous le graphe d'une fonction lipschitzienne. On note $\Gamma$ sa frontière. On rappelle que 
\begin{itemize}
\item[-] L'injection canonique $H^2(\Omega) \subset H^1(\Omega)$ est compacte
\item[-] L'espace $H^{1/2}(\Gamma)$ (une définition des espaces de Sobolev au bort d'un domaine borné est donnée dans \cite[Chapitre 1, p.~20]{grisvard2011elliptic}) a pour dual l'espace $H^{-1/2}(\Gamma)$. 
\item[-] L'application trace, notée $\gamma$, est l'unique extension continue à $H^1(\Omega)$ de l'application de restriction des fonctions régulières au bord. C'est une surjection de $H^1(\Omega)$ dans $H^{1/2}(\Gamma)$.  
\end{itemize}

On introduit l'opérateur classique de simple couche $S$ (cf. \cite[Chapitre 3, page 113]{nedelec2001acoustic}) qui associe à toute fonction $q \in C^0(\Gamma)$ la fonction $Sq$ définie pour $x \notin \Gamma$ de la manière suivante 
\[ Sq(x) = \int_{\Gamma} E(x-y)q(y) dy \]
Et définie sur $\Gamma$ par continuité, où $E$ est la solution fondamentale de l'équation de Helmholtz qui satisfait la condition de radiation de Sommerfield choisie avec la convention $+i$, c'est-à-dire 
\[ E(x) = \dfrac{e^{ik|x|}}{4\pi|x|}.\]
D'après le théorème 3.4.1 de la même référence (p. 142), l'opérateur $\gamma \circ S$ s'étend par densité de manière unique en un isomorphisme de $H^{-1/2}(\Gamma)$ sur $H^{1/2}(\Gamma)$, dès lors que $-k^2$ n'est pas une valeur propre du Laplacien sur $\Omega$. 
Dans \cite{alouges2015sparse} est introduite une méthode de décomposition de l'opérateur de simple couche précédent sous la forme 
\[ S = S_0 + R\]
Où $S_0$ sont deux opérateurs de convolution sur $\Gamma$, donc qui associent à une fonction $u$ de $H^{-1/2}(\Gamma)$ une fonction de la forme
\[ u \mapsto \int_{\Gamma} K(x-y)u(y)dy,\]
Où l'intégrale est définie par densité de même que $S$. Les propriétés du noyau $K$ dans chaque cas sont~: 
\begin{itemize}
\item[-] Support compact pour l'opérateur $S_0$. On l'appelle ainsi l'opérateur de champ proche. 
\item[-] Régularité $C^{\infty}$ sur $\mathbb{R}^n$ pour l'opérateur $R$. On l'appelle ainsi l'opérateur régulier.  
\end{itemize}
Nous allons montrer que dans ces conditions, l'opérateur $R$ est une perturbation compacte de l'opérateur $S_0$. L'intérêt de ce résultat tient au fait que dans les méthodes numériques, on s'attend à ce que l'opérateur  $S_h S_{0,h}^{-1} = Id + R_h S_{0,h}^{-1}$ (où l'indice $h$ signifie formellement ici que l'on parle des version discrétisées) ait un conditionnement indépendant de la taille $h$ du maillage, puisqu'il sera une perturbation compacte de l'identité.  

On considère une suite bornée $u_n$ sur $H^{-1/2}(\Gamma)$, dont on cherche à extraire une sous-suite telle que $\gamma \circ R u_n$ converge dans $H^{1/2}$. Le raisonnement est le suivant : 
\begin{itemize}
\item[-] $R$ est continue de $H^{-1/2}(\Gamma)$ dans $H^m(\Omega)$ pour tout $m\in \mathbb{N}$. Donc $Ru_n$ est bornée dans $H^m(\Omega)$.  
\item[-] Les injections compactes de Sobolev permettent d'extraire de $Ru_n$ une sous-suite convergente dans $H^{1}(\Omega)$. 
\item[-] La continuité de l'application trace implique le résultat. 
\end{itemize} 
La seule affirmation méritant démonstration est la première :

\begin{Lem*} Pour tout $m \in \mathbb{N}$, $R$ est continue de $H^{-1/2}(\Gamma)$ dans $H^{m}(\Omega)$.
\begin{proof}
On note $K$ le noyau intégral de l'opérateur $R$. Par hypothèse, $K$ est de classe $C^{\infty}$ sur $\mathbb{R}^n$. 
Soit $u \in H^{-1/2}(\Gamma)$, soit $v= Ru$. Observons que les dérivées de $v$ sont données par 
\[\partial^{\alpha} v = \int_{\Gamma} \partial^\alpha K(x-y)u(y) d\sigma(y).\]
Pour tout $x \in \Omega$, la fonction $G_x : y \in \Omega \mapsto \partial^\alpha K(x-y)$ est dans $H^1(\Omega)$. C'est évident car $\Omega$ est borné et $K$ est $C^{\infty}$. Sa restriction à $\Gamma$ est donc dans $H^{1/2}(\Gamma)$, par le théorème de trace. On en déduit, par définition de $H^{-1/2}(\Gamma)$ : 
\[\abs{\partial^\alpha v} \leq \norm{G_x}_{H^{1/2}(\Gamma)}\norm{u}_{H^{-1/2}(\Gamma)}\] 
Enfin, par application du théorème de trace, il existe une constante $C$ telle que
\[\norm{G_x}_{H^{1/2}(\Gamma)} \leq C \norm{G_x}_{H^{1}(\Omega)} \leq C \sqrt{\abs{\Omega}}\sqrt{\norm{G}_{\infty}^2+\norm{\nabla G}_\infty^2}\]
On en déduit que $\partial^\alpha v \in L^2(\Omega)$ et 
\[ \norm{\partial^\alpha v}_{L^2(\Omega)} \leq C \abs{\Omega}\sqrt{\norm{G}_{\infty}^2+\norm{\nabla G}_\infty^2} \norm{u}_{H^{-1/2}(\Gamma)}\] 
D'où la continuité annoncée. 
\end{proof}
\end{Lem*}




\bibliographystyle{plain}
\bibliography{../Biblio/biblio} 

\end{document}