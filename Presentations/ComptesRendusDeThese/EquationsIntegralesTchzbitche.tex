\documentclass[11pt,a4paper]{article}

%\usepackage{adjustbox}
%\usepackage{algorithm}
%\usepackage{algorithmic}
%\usepackage{lmodern}
\usepackage{amsmath}
\usepackage{amssymb}
\usepackage{anyfontsize}
%\usepackage{amsthm}
\usepackage{amsfonts}
%\usepackage{afterpage}
%\usepackage{blindtext}
\usepackage[hidelinks]{hyperref}
\usepackage[english]{babel}
\usepackage{bbm}
\usepackage{bigints}
%\usepackage{bm}
\usepackage{cite}
\usepackage{color}
\usepackage{float}
\usepackage{graphicx}
\usepackage[utf8]{inputenc}
\usepackage{mathptmx}
\usepackage{mathtools}
%\usepackage{mdframed}
\usepackage{pgfplots} 
\usetikzlibrary{external}
\tikzexternalize
\usepackage{subcaption}
\usepackage{stmaryrd}
\SetSymbolFont{stmry}{bold}{U}{stmry}{m}{n}
					%%%
\usepackage{textcomp}
\usepackage{tikz}
\usepackage{url}

\smartqed  % flush right qed marks, e.g. at end of proof

%\renewcommand{\proofname}{Proof}
%\newtheorem{monTheoNumrote}{Théorème} % Environnement numéroté en fonction de la section
%\newtheorem*{monTheoNonNumerote}{Théorème}  % Environnement non numéroté
%\newtheorem{The}{Theorem}
%\newtheorem*{The*}{Theorem}
%\newtheorem{Prop}{Proposition}
%\newtheorem*{Prop*}{Proposition} 
%\newtheorem{Cor}{Corollary}
%\newtheorem*{Cor*}{Corollary}
%\newtheorem{Conj}{Conjecture}
%\newtheorem{Lem}{Lemma}
%\renewcommand{\qed}{\unskip\nobreak\quad\qedsymbol}%
%\theoremstyle{definition}
%\newtheorem{Def}{Definition}
%\newtheorem{Rem}{Remark}
%\newtheorem*{Rem*}{Remark}
%\newtheorem*{Lem*}{Lemma}
%\newtheorem{Que}{Question}
\newcommand{\enstq}[2]{\left\{#1\mathrel{}\middle|\mathrel{}#2\right\}}
\newcommand{\Lp}[2]{L^#1(#2)}
\newcommand{\Sob}[3]{W^{#1,#2}(#3)}
\newcommand{\Rd}[0]{\mathbb{R}^d}
\newcommand{\RN}[0]{\mathbb{R}^N}
\newcommand{\Rn}[0]{\mathbb{R}^n}
\newcommand{\norm}[1]{\left\|#1\right\|}
\newcommand{\sinc}[0]{\textup{sinc}}
\newcommand{\functionDef}[5]{\begin{array}{lllll}
#1 & : & #2 & \longrightarrow & #3 \\
 & & #4 & \longmapsto &\displaystyle #5 \\
\end{array}}
\newcommand{\Theautorefname}{Theorem}
\newcommand{\Propautorefname}{Proposition}
\newcommand{\Corautorefname}{Corollary}
\newcommand{\Lemautorefname}{Lemma}
\newcommand{\Defautorefname}{Definition}
\newcommand{\Conjautorefname}{Conjecture}
\newcommand{\Remautorefname}{Remark}
\renewcommand{\sectionautorefname}{Section}
\renewcommand{\subsectionautorefname}{Section}
%\renewcommand{\algorithmicrequire}{\textbf{Inputs:}}
%\renewcommand{\algorithmicensure}{\textbf{Outputs:}}

\newcommand{\N}{\mathbb{N}}
\newcommand{\Z}{\mathbb{Z}}
\newcommand{\D}{\mathbb{D}}
\newcommand{\R}{\mathbb{R}}
\newcommand{\A}{\mathcal{A}_{a,b}}
\newcommand{\Crad}{C^\infty_{c,\textup{rad}}(B)}
\newcommand{\Lrad}{L^2_{\textup{rad}}(B)}
\newcommand{\Lradab}{L^2_{\textup{rad}}(\mathcal{A}_{a,b})}
\newcommand{\duality}[2]{\left\langle #1,#2\right\rangle}
\newcommand{\Hrad}{H^1_{\textup{rad}}(B)}
\newcommand{\Hzrad}{H^1_{0,\textup{rad}}(B)}
\newcommand{\rmin}{\delta_{\min}}
\newcommand{\rmax}{\delta_{\max}}
\newcommand{\corr}{\gamma}
%\newcommand{\question}[1]{\begin{Que} \ 
%#1
%\end{Que}}
\newcommand{\abs}[1]{\left\lvert #1 \right\rvert}
\newcommand{\CL}[2]{\textup{CL}\left(\enstq{#1}{#2}\right)}
\newcommand{\Script}[1]{`\texttt{#1}`}
\newcommand{\espace}{\text{ }\qquad} 
\newcommand{\loc}{\text{loc}}
\newcommand{\SL}{\textup{SL}\hspace{1.5pt}}
\newcommand{\DL}{\textup{DL}\hspace{1.5pt}}
\newcommand{\fp}{\underset{\varepsilon \to 0}{\textup{f.p.}}}
\newcommand{\scalProd}[2]{\left(#1|#2\right)}
\newcommand{\toDo}[1]{{\color{red}#1}}
\newcommand{\bs}[1]{\boldsymbol{#1}}
\newcommand{\varInRange}[4]{(#1_{#2})_{#3 \leq #2 \leq #4}}
\newcommand{\from}{\colon}
\newcommand{\Cinf}{C^{\infty}}
\newcommand{\isdef}{\mathrel{\mathop:}=}
\newcommand{\defis}{=\mathrel{\mathop:}}


\pgfplotsset{compat=newest}
\usepackage{array}
\usepackage{booktabs}
\setlength{\heavyrulewidth}{1.5pt}
\setlength{\abovetopsep}{4pt}
\author{Martin AVERSENG}
\title{Simple couche et Hyper-singulier sur un segment}
\begin{document}
\maketitle	

\section{Potentiel de simple couche sur un segment}

Soit $S$ l'opérateur de simple couche sur le segment $(-1,1)$ définit par
\[Su(x) = \dfrac{1}{2\pi}\int_{-1}^1 \ln(|x-y|)u(y)dy\]
On pose $\omega(x)= \sqrt{1-x^2}$. 
On s'intéresse aux propriétés de l'opérateur $\alpha \mapsto \dfrac{1}{\omega} S \dfrac{1}{\omega}\alpha$. Selon la remarque du paragraphe 2.3 de \cite{bruno2012second}, on admet la conjecture suivante :
\begin{The} Soit $f$ une fonction dans $H^s(-1,1)$, $s>0$. Alors l'unique solution de l'équation d'inconnue $\alpha \in H^1(-1,1)$ :
\[S\left(\dfrac{\alpha}{\omega}\right) = f\]
est dans $H^{s+1}(-1,1)$. 
\end{The}
L'intérêt de cette remarque est qu'on obtient une convergence rapide de l'approximation par éléments finis lorsque le pas du maillage $h$ devient petit. Ce fait se base sur une version du lemme de Céa adaptée à notre situation.
Soit $S_{\omega} := \dfrac{1}{\omega}S\dfrac{1}{\omega}$ (ce n'est pas la même notation que celle choisie par Oscar Bruno). De manière immédiate, $S_\omega$ hérite de la propriété de coercivité de $S$. 
\begin{Prop} Pour tout $\alpha$ tel que $\frac{\alpha}{\omega} \in H^{-1/2}(-1,1)$, on a 
\[ (S_\omega \alpha,\alpha) \geq c \norm{\frac{\alpha}{\omega}}_{H^{-1/2}}^2\] 
\begin{proof}
On a
$(S_{\omega}\alpha,\alpha) = \left(\dfrac{1}{\omega}S\dfrac{1}{\omega}\alpha , \alpha\right) = \left(S\dfrac{1}{\omega}\alpha , \dfrac{1}{\omega}\alpha\right) \geq c \norm{\dfrac{\alpha}{\omega}}_{H^{-1/2}}^2$
\end{proof}
\end{Prop}
Soit $V_h$ un sous-espace vectoriel de dimension finie de $\enstq{\alpha}{\alpha/\omega \in H^{-1/2}}$. Soit $\alpha_h$ l'unique solution de la formulation variationnelle : $\forall \beta_h \in V_h$ : 
\[(S_\omega\alpha_h,\alpha_h) = \int_{-1}^{1} f(x)\dfrac{\beta_h(x)}{w(x)}.\]
Le lemme de Céa assure 
\[\norm{(\alpha - \alpha_h)/\omega}_{H^{-1/2}} \leq \inf_{\beta_h \in V_h}C\norm{(\alpha-\beta_h)/\omega}_{H^{-1/2}}\]
\paragraph{Question} Y a-t-il une bonne méthode pour montrer que le terme de droite est d'ordre $O(h)$ pour des éléments finis $\mathbbm{P}_1$ ? ?

Soient $T_n$ les polynômes de Tchebychev de première espèce. 
D'après \cite{bruno2012second}, on a 
\[ S \left(\dfrac{T_n}{\omega}\right) = \lambda_n T_n\]
Avec $\lambda_0 = \frac{\ln(2)}{2}$ et $\lambda_n = \dfrac{1}{2n}$ pour $n\neq 0$. D'autre part, considérons l'opérateur $\Lambda$ qui, à une fonction $g$ définie sur le segment $(-1,1)$ associe la donnée de Neumann de la solution $u$ du problème 
$\left\{\begin{array}{rll}
-\Delta u &= 0 & \text{ dans } \mathbb{R}^2 \setminus \{(-1,1)\times \{0\}\}\\
u &= g & \text{ sur } (-1,1)\times\{0\}
\end{array}\right.$
En prenant la normale du côté des $y$ positifs. Les formules de Calderòn impliquent alors que 
\[S\Lambda g = \frac{1}{2}g\]
Donc $S^{-1} = 2\Lambda$. On a donc 
\[ \omega\Lambda T_n = \mu_n T_n \]
où $\mu_n = \frac{1}{\ln(2)}$ si $n=0$ et $\mu_n = n$ sinon. Or l'équation différentielle vérifiée par les polynômes $T_n$ nous fournit un opérateur différentiel $P$ explicite qui satisfait pour $n \neq 0$ à la relation $PT_n = -\mu_n^2 T_n$. L'opérateur $P$ est donné par  \[ P = (1-x^2) \partial_{xx} - x\partial_x = \left(\omega \partial_x\right)^2\] 
Les polynômes $T_n$ forment une base Hilbertienne de $L^2\left[(-1,1),\omega^{-1}(x)dx\right]$. On a donc pour toute fonction $\varphi = \sum_{n=0}^{+\infty}c_n T_n(x)$ dans cet espace : 
\[\left[P^2 + (\omega\Lambda)^2\right] \varphi = c_0\mu_0^2 T_0 \]
L'intérêt de cette relation est qu'il permet d'exprimer l'opérateur $\omega\Lambda$ en fonction d'un opérateur différentiel donc local, qui permet une discrétisation numérique efficace. Dans l'optique de la résolution d'un problème intégral, on pourrait utiliser $\omega\Lambda$ ou une approximation de celui-ci pour préconditionner l'équation. 
Puisque les deux opérateurs du membre de gauche sont diagonalisée par une même base Hilbertienne, ils commutent sur cet espace de Hilbert. 

\bibliographystyle{plain}
\bibliography{../Biblio/biblio}  
\end{document}