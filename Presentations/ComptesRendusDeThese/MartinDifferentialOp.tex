
\documentclass[]{article}

\usepackage{adjustbox}
\usepackage{algorithm}
\usepackage{algorithmic}
\usepackage{amsmath}
\usepackage{amssymb}
\usepackage{amsthm}
\usepackage{amsfonts}
\usepackage{afterpage}
\usepackage{blindtext}
\usepackage[font=footnotesize,labelfont=bf]{caption}
\usepackage{hyperref}
\usepackage[english]{babel}
\usepackage{bbm}
\usepackage{bigints}
\usepackage{bm}
\usepackage{cite}
\usepackage{color}
\usepackage{float}
\usepackage[left=2cm,right=2cm,top=2cm,bottom=2cm]{geometry}
\usepackage{graphicx}
\usepackage[utf8]{inputenc}
\usepackage{mathtools}
\usepackage{mdframed}
\usepackage{pgfplots} 
\usepackage{subfigure}
\usepackage{stmaryrd}
\usepackage{textcomp}
\usepackage{tikz}
\usepackage{url}
\renewcommand{\proofname}{Proof}
\theoremstyle{plain}
\newtheorem{monTheoNumrote}{Théorème}[section] % Environnement numéroté en fonction de la section
\newtheorem*{monTheoNonNumerote}{Théorème}  % Environnement non numéroté
\newtheorem{The}{Theorem}[section]
\newtheorem*{The*}{Theorem}
\newtheorem{Prop}{Proposition}[section]
\newtheorem*{Prop*}{Proposition} 
\newtheorem{Cor}{Corollary}[section]
\newtheorem*{Cor*}{Corollary}
\newtheorem{Conj}{Conjecture}[section]
\newtheorem{Lem}{Lemma}[section]
\renewcommand{\qed}{\unskip\nobreak\quad\qedsymbol}%
\numberwithin{equation}{section} % Numérote les équations section.numéro.
\theoremstyle{definition}
\newtheorem{Def}{Definition}[section]
\newtheorem{Rem}{Remark}[section]
\newtheorem*{Rem*}{Remark}
\newtheorem*{Lem*}{Lemma}
\newtheorem{Que}{Question}
\newcommand{\enstq}[2]{\left\{#1\mathrel{}\middle|\mathrel{}#2\right\}}
\newcommand{\Lp}[2]{L^#1(#2)}
\newcommand{\Sob}[3]{W^{#1,#2}(#3)}
\newcommand{\Rd}[0]{\mathbb{R}^d}
\newcommand{\RN}[0]{\mathbb{R}^N}
\newcommand{\Rn}[0]{\mathbb{R}^n}
\newcommand{\norm}[1]{\left\|#1\right\|}
\newcommand{\sinc}[0]{\textup{sinc}}
\newcommand{\functionDef}[5]{\begin{array}{lllll}
#1 & : & #2 & \longrightarrow & #3 \\
 & & #4 & \longmapsto &\displaystyle #5 \\
\end{array}}
\newcommand{\Theautorefname}{Theorem}
\newcommand{\Propautorefname}{Proposition}
\newcommand{\Corautorefname}{Corollary}
\newcommand{\Lemautorefname}{Lemma}
\newcommand{\Defautorefname}{Definition}
\newcommand{\N}{\mathbb{N}}
\newcommand{\Z}{\mathbb{Z}}
\newcommand{\D}{\mathbb{D}}
\newcommand{\R}{\mathbb{R}}
\newcommand{\A}{\mathcal{A}_{a,b}}
\newcommand{\Crad}{C^\infty_{c,rad}(B)}
\newcommand{\Lrad}{L^2_{rad}(B)}
\newcommand{\Lradab}{L^2_{rad}(\mathcal{A}_{a,b})}
\newcommand{\duality}[2]{\left\langle #1,#2\right\rangle}
\newcommand{\Hrad}{H^1_{rad}(B)}
\newcommand{\Hzrad}{H^1_{0,rad}(B)}
\newcommand{\rmin}{\delta_{\min}}
\newcommand{\rmax}{\delta_{\max}}
\newcommand{\corr}{\gamma}
\newcommand{\question}[1]{\begin{Que} \ 
#1
\end{Que}}
\newcommand{\abs}[1]{\left\lvert #1 \right\rvert}
\newcommand{\CL}[2]{\textup{CL}\left(\enstq{#1}{#2}\right)}
\newcommand{\Script}[1]{`\texttt{#1}`}
\newcommand{\espace}{\text{ }\qquad} 
\newcommand{\loc}{\text{loc}}
\newcommand{\SL}{\textup{SL}\hspace{1.5pt}}
\newcommand{\DL}{\textup{DL}\hspace{1.5pt}}
\newcommand{\fp}{\underset{\varepsilon \to 0}{\textup{f.p.}}}
\newcommand{\scalProd}[2]{\left(#1|#2\right)}
\newcommand{\toDo}[1]{{\color{red}#1}}
\newcommand{\bs}[1]{\boldsymbol{#1}}
\newcommand{\varInRange}[4]{(#1_{#2})_{#3 \leq #2 \leq #4}}
\newcommand{\from}{\colon}
\newcommand{\Cinf}{C^{\infty}}
\newcommand{\isdef}{\mathrel{\mathop:}=}
\newcommand{\defis}{=\mathrel{\mathop:}}

\renewcommand{\algorithmicrequire}{\textbf{Inputs:}}
\renewcommand{\algorithmicensure}{\textbf{Outputs:}}

\pgfplotsset{compat=1.13}

\begin{document}
	
\begin{Def}
	Let the M-transform be defined by
	\[\mathcal{M}\phi(\xi) \isdef \int_{0}^{+\infty} \cos(2\pi\sqrt{\xi u})\dfrac{\phi(u)}{\sqrt{u}}du\]
\end{Def}
\begin{Def}
	For any function $\phi$ defined on $\R^+$, let $C$ the operator defined by 
	\[C\phi(t) = \phi(t^2), \quad t \in  \R\]
	For any function even function $\phi$ defined on $\mathbb{R}$, $C^{-1}\phi$ is a function defined on $\R^+$ by 
	\[C^{-1}\phi(u) = \phi(\sqrt{u})\] 
\end{Def}
\begin{Def}
	We note $\mathcal{S}\left(\sqrt{\R^+}\right)$ the space of $\phi$ such that $C\phi \in \mathcal{S}(\mathbb{R})$. Let $S_p(\R)$ the subspace of real even functions that belong to the Schwartz class. 
\end{Def}
\begin{Prop}
	$C\mathcal{S}\left(\sqrt{\R^+}\right) = \mathcal{S}_p(\R)$
\end{Prop}
\begin{Prop}
	If $\phi \in \mathcal{S}\left(\sqrt{\R^+}\right)$, then 
	\[\sqrt{x}\phi \in \mathcal{S}\left(\sqrt{\R^+}\right)\]
	For all polynomial $P$, 
	\[C^{-1}P(dx^2)C \phi \in \mathcal{S}\left(\sqrt{\R^+}\right)\]
	Where $dx$ is the differentiation operator. 
\end{Prop}
\begin{Prop}
	The operator $\opFromTo{\mathcal{M}}{\mathcal{S}\left(\sqrt{\R^+}\right)}{\mathcal{S}\left(\sqrt{\R^+}\right)}$ is an involution, and can be rewritten as 
	\[\mathcal{M} = C^{-1}\mathcal{F}C\]
	where $\mathcal{F}$ is the Fourier transform defined on $\mathcal{S}_p(\R)$
	\[\mathcal{F}u(\xi) = \int_{-\infty}^{+\infty}e^{-i2\pi x \xi}u(x)dx\]
	or equivalently
	\[\mathcal{F}u(\xi) = \int_{0}^{+\infty}2\cos(2\pi\xi x)u(x)dx\]
	$\mathcal{F}$ is self-adjoint on $S_p(\R)$. 
\end{Prop}
\begin{Def}For $\psi$ and $\phi$ in $\mathcal{S}\left(\sqrt{\R^+}\right)$, we define the duality product 
	\[\duality{\phi}{\psi}_{\omega}= \int_{0}^{+\infty}\dfrac{\phi(x) \psi(x)}{\sqrt{x}}\]
\end{Def}
\begin{Prop}
	\[\duality{\phi}{\psi}_{\omega} = \duality{C\phi}{C\psi}\]
\end{Prop}
\begin{Prop}
	For any $\phi$, $\psi \in \mathcal{S}\left(\sqrt{\R^+}\right)$, one has
	\[\duality{\mathcal{M}\phi}{\psi}_{\omega} = \duality{\phi}{\mathcal{M}\psi}_{\omega}\]
\end{Prop}

\begin{Def}
	Let $\Delta_\omega$ the operator defined on $\mathcal{S}\left(\sqrt{\R^+}\right)$ by 
	\[\Delta_\omega \phi(x) = 2\sqrt{x}\left(2\sqrt{x} \phi'(x)\right)'\]
\end{Def}
If we call $\Delta$ the usual Laplace operator defined on $S_p(\R)$, we have  
\begin{Prop}
	\[\Delta_{\omega}\phi = C^{-1} \Delta C \phi\]
\end{Prop}
\begin{Cor}
	$\Delta_\omega$ maps $\mathcal{S}\left(\sqrt{\R^+}\right)$ on itself. 
\end{Cor}
\begin{Cor}
	$\duality{\Delta_{\omega}\phi}{\psi}_{\omega} = \duality{\phi}{\Delta_{\omega}\psi}_{\omega}$ 
\end{Cor}
\begin{Prop}
	One has, for all $\xi \in \R^+$
	\[\mathcal{M}(\Delta_\omega \phi) = -\xi\mathcal{M}\phi\]
\end{Prop}
\begin{Def}
	For $s \in \R$, we define $\mathcal{M}^s(\R)$ as 
	\[f\in \mathcal{M}^s(\R) \iff \int_{0}^{+\infty}\frac{\left(1 + \xi\right)^{s}}{\sqrt{\xi}}\abs{\mathcal{M}f}^2(\xi) < + \infty\]
\end{Def}
\begin{Prop}
	\[f\in \mathcal{M}^s(\R^+) \iff Cf \in H^{s}(\R)\]
	and we have
	\[\norm{f}_{\mathcal{M}^s} = \norm{Cf}_{H^s}\]
\end{Prop}
\begin{Def}
	For $s = 0$, we note $L^2_{\omega} = \mathcal{M}^0(\R^+)$. It is a Hilbert space with the scalar product corresponding to $\duality{\cdot}{\cdot}_{\omega}$ defined earlier.
	\begin{proof}
		To prove that $L^2_{\omega}$ is complete, it suffices to show that any Cauchy sequence $f_n$ has a limit in $L^2_{\omega}$. Obviously, $g_n \isdef \dfrac{f_n}{x^{1/4}}$ is a Cauchy sequence in $L^2(\R^+)$, so it admits a limit $g_\infty \in L^2(\R^+)$. Then $f_{\infty} \isdef x^{1/4}g_{\infty}$ belongs to $L^2_\omega$ and we have 
		\[ \norm{f_n - f_\infty}_{L^2_\omega} = \norm{g_n - g_{\infty}}_{L^2}\]
		which ensures $f_n \to f_{\infty}$ in $L^2_{\omega}$.  
	\end{proof} 
\end{Def}
\begin{Prop}
	$\mathcal{M}^s(\R^+)$ is a closed subspace of $L^2_\omega$. It is also a Hilbert space for the scalar product defined by 
	\[\duality{u}{v}_{\omega,s} \isdef \int_{0}^{+\infty} \dfrac{(1+ \xi)^s}{\sqrt{\xi}}\mathcal{M}u(\xi) \mathcal{M}v(\xi)\]
	\begin{proof}
		Let $u_n$ a Cauchy sequence in $\mathcal{M}^s(\R^+)$. Then, by the same arguments as above, 
		\[v_n \isdef \dfrac{(1 + \xi)^{s/2}\mathcal{M}u_n(\xi)}{\xi^{1/4}}\]
		has a limit $v_\infty$ in $L^2$, and
		\[u_{\infty} = \mathcal{M}\left(\dfrac{\xi^{1/4}}{(1+\xi)^{s/2}}v_\infty\right)\]
		is the limit of $u_n$ in $\mathcal{M}^s(\R^+)$. 
	\end{proof}
\end{Prop}
\begin{Prop}
	The injection 
\end{Prop}
\end{document}
