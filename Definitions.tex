
%\usepackage{adjustbox}
%\usepackage{algorithm}
%\usepackage{algorithmic}
\usepackage{lmodern}
\usepackage{amsmath}
\usepackage{amssymb}
\usepackage{amsthm}
\usepackage{amsfonts}

\usepackage{array}
\usepackage{afterpage}

%\usepackage{blindtext}
\usepackage[hidelinks]{hyperref}
\usepackage{cleveref}
\usepackage[english]{babel}
\usepackage{bbm}
\usepackage{bigints}
%\usepackage{bm}
\usepackage{cite}
\usepackage{color}
\usepackage{floatrow}
% Table float box with bottom caption, box width adjusted to content
\newfloatcommand{capbtabbox}{table}[][\FBwidth]

\usepackage{float}
\usepackage[pdftex]{graphicx}
\usepackage[outdir = ./]{epstopdf}
\usepackage[utf8]{inputenc}
\usepackage{mathtools}
%\usepackage{mdframed}
\usepackage{pgfplots} 
\usepackage{subcaption}
%\usepackage{stmaryrd}
%\SetSymbolFont{stmry}{bold}{U}{stmry}{m}{n}
\usepackage{mathtools}				%%%
%\usepackage{textcomp}
\usepackage{tikz}
\usepackage{url}

%\renewcommand{\proofname}{Proof}
\newtheorem{monTheoNumrote}{Théorème} % Environnement numéroté en fonction de la section
%\newtheorem*{monTheoNonNumerote}{Théorème}  % Environnement non numéroté
\newtheorem{The}{Theorem}
\newtheorem{theorem}{Theorem}
%\newtheorem*{The*}{Theorem}
\newtheorem{Prop}{Proposition}
\newtheorem{proposition}{Proposition}
%\newtheorem*{Prop*}{Proposition} 
\newtheorem{Cor}{Corollary}
%\newtheorem*{Cor*}{Corollary}
\newtheorem{Conj}{Conjecture}
\newtheorem{Lem}{Lemma}
%\renewcommand{\qed}{\unskip\nobreak\quad\qedsymbol}%
%\theoremstyle{definition}
\newtheorem{Def}{Definition}
\newtheorem{Rem}{Remark}
%\newtheorem*{Rem*}{Remark}
%\newtheorem*{Lem*}{Lemma}
\newtheorem{Que}{Question}
\newcommand{\enstq}[2]{\left\{#1\mathrel{}\middle|\mathrel{}#2\right\}}
\newcommand{\Lp}[2]{L^#1(#2)}
\newcommand{\Sob}[3]{W^{#1,#2}(#3)}
\newcommand{\Rd}[0]{\mathbb{R}^d}
\newcommand{\RN}[0]{\mathbb{R}^N}
\newcommand{\Rn}[0]{\mathbb{R}^n}
\newcommand{\norm}[1]{\left\|#1\right\|}
\newcommand{\sinc}[0]{\textup{sinc}}
\newcommand{\functionDef}[5]{\begin{array}{lllll}
#1 & : & #2 & \longrightarrow & #3 \\
 & & #4 & \longmapsto &\displaystyle #5 \\
\end{array}}
\newcommand{\Theautorefname}{Theorem}
\newcommand{\Propautorefname}{Proposition}
\newcommand{\Corautorefname}{Corollary}
\newcommand{\Lemautorefname}{Lemma}
\newcommand{\Defautorefname}{Definition}
\newcommand{\Conjautorefname}{Conjecture}
\newcommand{\Remautorefname}{Remark}
%\renewcommand{\sectionautorefname}{Section}
%\renewcommand{\suectionautorefname}{Subsection}
%\renewcommand{\algorithmicrequire}{\textbf{Inputs:}}
%\renewcommand{\algorithmicensure}{\textbf{Outputs:}}

\newcommand{\N}{\mathbb{N}}
\newcommand{\Z}{\mathbb{Z}}
\newcommand{\D}{\mathbb{D}}
\newcommand{\R}{\mathbb{R}}
\newcommand{\A}{\mathcal{A}_{a,b}}
\newcommand{\Crad}{C^\infty_{c,rad}(B)}
\newcommand{\Lrad}{L^2_{rad}(B)}
\newcommand{\Lradab}{L^2_{rad}(\mathcal{A}_{a,b})}
\newcommand{\duality}[2]{\left\langle #1,#2\right\rangle}

\newcommand{\inner}[2]{\left( #1,#2\right)}

\newcommand{\Hrad}{H^1_{rad}(B)}
\newcommand{\Hzrad}{H^1_{0,rad}(B)}
\newcommand{\rmin}{\delta_{\min}}
\newcommand{\rmax}{\delta_{\max}}
\newcommand{\corr}{\gamma}
%\newcommand{\question}[1]{\begin{Que} \ 
%#1
%\end{Que}}
\newcommand{\abs}[1]{\left\lvert #1 \right\rvert}
\newcommand{\CL}[2]{\textup{CL}\left(\enstq{#1}{#2}\right)}
\newcommand{\Script}[1]{`\texttt{#1}`}
\newcommand{\espace}{\text{ }\qquad} 
\newcommand{\loc}{\text{loc}}
\newcommand{\SL}{\textup{SL}\hspace{1.5pt}}
\newcommand{\DL}{\textup{DL}\hspace{1.5pt}}
\newcommand{\fp}{\underset{\varepsilon \to 0}{\textup{f.p.}}}
\newcommand{\scalProd}[2]{\left(#1|#2\right)}
\newcommand{\toDo}[1]{{\color{red}#1}}
\newcommand{\bs}[1]{\boldsymbol{#1}}
\newcommand{\varInRange}[4]{(#1_{#2})_{#3 \leq #2 \leq #4}}
\newcommand{\from}{\colon}
\newcommand{\Cinf}{C^{\infty}}
\newcommand{\isdef}{\mathrel{\mathop:}=}
\newcommand{\defis}{=\mathrel{\mathop:}}
\newcommand{\conj}[1]{\overline{#1}}
\newcommand{\labeleq}[2]{\begin{equation}
	#1
	#2
	\end{equation}}
\newcommand{\opFromTo}[3]{#1 : #2 \longrightarrow #3}
\pgfplotsset{compat=newest}
% \itemequation[label]{text before}{equation}
\makeatletter
\newcommand*{\itemequation}[3][]{%
	\begingroup
	\refstepcounter{equation}%
	\ifx\\#1\\%
	\else
	\label{#1}%
	\fi
	\sbox0{#2}%
	\sbox2{#3}%
	\sbox4{ \@eqnnum}%
	\dimen@=.5\dimexpr\linewidth-\wd2\relax
	% Warning for overlapping
	\let\CenterInSpace=N%
	\ifcase
	\ifdim\wd0>\dimen@
	\z@
	\else
	\ifdim\wd4>\dimen@
	\z@
	\else
	\@ne
	\fi
	\fi
	\let\CenterInSpace=Y%
	\fi
	\ifdim\dimexpr\wd0+\wd2+\wd4\relax>\linewidth
	\@latex@warning{Equation is too large}%
	\fi
	\noindent
	\rlap{\copy0}%
	\ifx\CenterInSpace Y%
	\rlap{\hbox to \linewidth{\kern\wd0\hss\copy2\hss\kern\wd4}}%
	\else
	\rlap{\hbox to \linewidth{\hfill\copy2\hfill}}%
	\fi
	\hbox to \linewidth{\hfill\copy4}%
	\hspace{0pt}% allow linebreak
	\endgroup
	\ignorespaces
}
\makeatother
\usepackage{scalerel,stackengine}
\stackMath
\newcommand\reallywidecheck[1]{%
	\savestack{\tmpbox}{\stretchto{%
			\scaleto{%
				\scalerel*[\widthof{\ensuremath{#1}}]{\kern-.6pt\bigwedge\kern-.6pt}%
				{\rule[-\textheight/2]{1ex}{\textheight}}%WIDTH-LIMITED BIG WEDGE
			}{\textheight}% 
		}{0.5ex}}%
	\stackon[1pt]{#1}{\scalebox{-1}{\tmpbox}}%
}
\parskip 1ex