\documentclass[]{article}

%\usepackage{adjustbox}
%\usepackage{algorithm}
%\usepackage{algorithmic}
%\usepackage{lmodern}
\usepackage{amsmath}
\usepackage{amssymb}
\usepackage{anyfontsize}
%\usepackage{amsthm}
\usepackage{amsfonts}
%\usepackage{afterpage}
%\usepackage{blindtext}
\usepackage[hidelinks]{hyperref}
\usepackage[english]{babel}
\usepackage{bbm}
\usepackage{bigints}
%\usepackage{bm}
\usepackage{cite}
\usepackage{color}
\usepackage{float}
\usepackage{graphicx}
\usepackage[utf8]{inputenc}
\usepackage{mathptmx}
\usepackage{mathtools}
%\usepackage{mdframed}
\usepackage{pgfplots} 
\usetikzlibrary{external}
\tikzexternalize
\usepackage{subcaption}
\usepackage{stmaryrd}
\SetSymbolFont{stmry}{bold}{U}{stmry}{m}{n}
					%%%
\usepackage{textcomp}
\usepackage{tikz}
\usepackage{url}

\smartqed  % flush right qed marks, e.g. at end of proof

%\renewcommand{\proofname}{Proof}
%\newtheorem{monTheoNumrote}{Théorème} % Environnement numéroté en fonction de la section
%\newtheorem*{monTheoNonNumerote}{Théorème}  % Environnement non numéroté
%\newtheorem{The}{Theorem}
%\newtheorem*{The*}{Theorem}
%\newtheorem{Prop}{Proposition}
%\newtheorem*{Prop*}{Proposition} 
%\newtheorem{Cor}{Corollary}
%\newtheorem*{Cor*}{Corollary}
%\newtheorem{Conj}{Conjecture}
%\newtheorem{Lem}{Lemma}
%\renewcommand{\qed}{\unskip\nobreak\quad\qedsymbol}%
%\theoremstyle{definition}
%\newtheorem{Def}{Definition}
%\newtheorem{Rem}{Remark}
%\newtheorem*{Rem*}{Remark}
%\newtheorem*{Lem*}{Lemma}
%\newtheorem{Que}{Question}
\newcommand{\enstq}[2]{\left\{#1\mathrel{}\middle|\mathrel{}#2\right\}}
\newcommand{\Lp}[2]{L^#1(#2)}
\newcommand{\Sob}[3]{W^{#1,#2}(#3)}
\newcommand{\Rd}[0]{\mathbb{R}^d}
\newcommand{\RN}[0]{\mathbb{R}^N}
\newcommand{\Rn}[0]{\mathbb{R}^n}
\newcommand{\norm}[1]{\left\|#1\right\|}
\newcommand{\sinc}[0]{\textup{sinc}}
\newcommand{\functionDef}[5]{\begin{array}{lllll}
#1 & : & #2 & \longrightarrow & #3 \\
 & & #4 & \longmapsto &\displaystyle #5 \\
\end{array}}
\newcommand{\Theautorefname}{Theorem}
\newcommand{\Propautorefname}{Proposition}
\newcommand{\Corautorefname}{Corollary}
\newcommand{\Lemautorefname}{Lemma}
\newcommand{\Defautorefname}{Definition}
\newcommand{\Conjautorefname}{Conjecture}
\newcommand{\Remautorefname}{Remark}
\renewcommand{\sectionautorefname}{Section}
\renewcommand{\subsectionautorefname}{Section}
%\renewcommand{\algorithmicrequire}{\textbf{Inputs:}}
%\renewcommand{\algorithmicensure}{\textbf{Outputs:}}

\newcommand{\N}{\mathbb{N}}
\newcommand{\Z}{\mathbb{Z}}
\newcommand{\D}{\mathbb{D}}
\newcommand{\R}{\mathbb{R}}
\newcommand{\A}{\mathcal{A}_{a,b}}
\newcommand{\Crad}{C^\infty_{c,\textup{rad}}(B)}
\newcommand{\Lrad}{L^2_{\textup{rad}}(B)}
\newcommand{\Lradab}{L^2_{\textup{rad}}(\mathcal{A}_{a,b})}
\newcommand{\duality}[2]{\left\langle #1,#2\right\rangle}
\newcommand{\Hrad}{H^1_{\textup{rad}}(B)}
\newcommand{\Hzrad}{H^1_{0,\textup{rad}}(B)}
\newcommand{\rmin}{\delta_{\min}}
\newcommand{\rmax}{\delta_{\max}}
\newcommand{\corr}{\gamma}
%\newcommand{\question}[1]{\begin{Que} \ 
%#1
%\end{Que}}
\newcommand{\abs}[1]{\left\lvert #1 \right\rvert}
\newcommand{\CL}[2]{\textup{CL}\left(\enstq{#1}{#2}\right)}
\newcommand{\Script}[1]{`\texttt{#1}`}
\newcommand{\espace}{\text{ }\qquad} 
\newcommand{\loc}{\text{loc}}
\newcommand{\SL}{\textup{SL}\hspace{1.5pt}}
\newcommand{\DL}{\textup{DL}\hspace{1.5pt}}
\newcommand{\fp}{\underset{\varepsilon \to 0}{\textup{f.p.}}}
\newcommand{\scalProd}[2]{\left(#1|#2\right)}
\newcommand{\toDo}[1]{{\color{red}#1}}
\newcommand{\bs}[1]{\boldsymbol{#1}}
\newcommand{\varInRange}[4]{(#1_{#2})_{#3 \leq #2 \leq #4}}
\newcommand{\from}{\colon}
\newcommand{\Cinf}{C^{\infty}}
\newcommand{\isdef}{\mathrel{\mathop:}=}
\newcommand{\defis}{=\mathrel{\mathop:}}


\pgfplotsset{compat=newest}
\usepackage{array}
\usepackage{booktabs}
\setlength{\heavyrulewidth}{1.5pt}
\setlength{\abovetopsep}{4pt}
%opening
\title{Démonstration des inégalités de type Poincaré}
\author{Martin Averseng}

\begin{document}

\maketitle



\begin{Def}{Convergence faible}
	Soit $E$ un espace de Hilbert dont le produit scalaire est noté $\inner{\cdot}{\cdot}_E$. On dit que la suite $(u_n)$ converge faiblement vers $u$ si pour tout $v \in E$, 
	\[\lim_{n \to + \infty} \inner{u_n}{v}_E = \inner{u}{v}_E.\]
\end{Def}

Dans ce qui suit, on s'apprête à utiliser le (puissant) résultat d'analyse fonctionnelle suivant: 

\begin{The}
	Soit $E$ un espace de Hilbert et $(u_n)_n$ une suite bornée dans $E$. Alors $(u_n)_n$ admet une sous-suite faiblement convergente. 
	\label{Faible}
\end{The}

On utilise aussi le théorème de Rellich:
\begin{The}
	Soit $\Omega$ un ouvert borné et connexe de frontière régulière. Alors toute suite $(u_n)$ bornée dans $H^1(\Omega)$ admet une sous-suite convergente dans $L^2(\Omega)$. 
\end{The}


Soit $H^1_\#(\Omega) \isdef \enstq{ u \in H^1(\Omega)}{ \int_{\Omega} u(x) dx = 0}$.
\begin{The}
	Il existe une constante $C > 0$ telle que 
	\[\forall u \in H^1_\#(\Omega),\quad \int_{\Omega} u^2(x) dx \leq C \int_{\Omega} \abs{\nabla u(x)}^2dx\,.\]
\end{The}


Avant de commencer la démonstration, voici l'idée principale de la preuve. Si le résultat était faux, cela voudrait dire qu'il y a des fonctions $u \in H^1_\#(\Omega)$ telles que leur norme $L^2$ est égale à $1$ et la norme de leur gradient est aussi petite que l'on veut. (C'est ce qu'on va bientôt formaliser en prenant la négation de la phrase avec quantificateurs $\forall, \exists$ dans notre raisonnement par l'absurde). Intuitivement, si la norme $L^2$ du gradient est presque $0$, c'est que la fonction est presque constante. Et si la fonction est presque constante, la contrainte de moyenne nulle oblige la fonction à être presque nulle. Or nous avons supposé que $u$ a sa norme $L^2$ égale à $1$. C'est cette contradiction que nous allons faire ressortir précisément dans la démonstration. 

\begin{proof}
	Pour commencer, nous allons raisonner par l'absurde et donc supposer que le résultat 
	\[\exists C > 0 : \forall u \in H^1_{\#}(\Omega), \quad  \int_{\Omega} u^2(x) dx \leq C \int_{\Omega} \abs{\nabla u(x)}^2dx\,.\]
	Pour prendre la négation d'une phrase avec quantificateurs, on change les $\exists$ en $\forall$ et inversement, et dans les prédicats sans quantificateurs, on change $<$ en $\geq$, $\leq$ en $>$, $=$ en $\neq$ etc. Ici, cela donne donc:
	\[\forall C > 0 : \exists u \in H^1_{\#}(\Omega), \quad  \int_{\Omega} u^2(x) dx > C \int_{\Omega} \abs{\nabla u(x)}^2dx\,.\]
	Nous supposons par l'absurde que cette phrase est vraie. Cela nous permet d'affirmer que pour tout $n \in \N$, (en prenant $C = n$), il existe une fonction $u_n \in  H^1_{\#}(\Omega)$ telle que 
	\[\int_{\Omega} u_n^2(x) dx > n \int_{\Omega} \abs{\nabla u_n(x)}^2dx\,.\]
	Nous normalisons $u_n$ en posant $v_n = \frac{u}{\norm{u}_{L^2(\Omega}}$. En divisant les deux membres de l'inégalité par $\norm{u}_{L^2(\Omega)}^2$, on obtient 
	\[1 > n \int_{\Omega} \abs{\nabla v_n(x)}^2 dx \]
	Nous en déduisons 
	\[\int_{\Omega} \abs{\nabla v_n(x)}^2 dx < \frac{1}{n}\]
	et donc la suite $(\nabla v_n)_n$ converge vers $0$ dans $L^2(\Omega)$. Par ailleurs, on a 
	\[\norm{v_n}_{H^1_{\#}(\Omega)}^2 = \int_{\Omega} v_n(x)^2 dx +  \int_{\Omega} \abs{\nabla v_n(x)}^2 dx < 1 + \frac{1}{n} \leq 2\]
	donc $(v_n)_n$ est bornée dans $H^1_{\#}(\Omega)$. D'après \autoref{Faible}, puisque $H^1_{\#}(\Omega)$ est un espace de Hilbert, on en déduit que $v_n$ admet une sous-suite $w_n$ faiblement convergente vers une limite $w$ dans $H^1_{\#}(\Omega)$.  Puisque $w_n$ est une sous-suite de $v_n$ et que $\nabla v_n$ converge vers $0$, $\nabla w_n$ converge également vers $0$. Utilisons finalement le théorème de Rellich: puisque  $w_n$ est bornée dans $H^1(\Omega)$, elle admet une sous-suite $z_n$ convergente dans $L^2(\Omega)$, dont la limite est notée $z$. Récapitulons: on a $z_n \to z$ dans $L^2$, $\nabla z_n \to 0$ dans $L^2$ et $z_n$ converge faiblement vers $w$ dans $H^{1}_{\#}(\Omega)$. Il nous reste à jongler avec ces différentes limites pour montrer $z = w = 0$ d'où naîtra une contradiction. Pour ce faire, on se concentre sur la convergence la plus faible, c'est à dire celle de $w$: 
	\[\forall v \in H^{1}_{\#}(\Omega),\quad \inner{z_n}{v}_{H^1} \to \inner{w}{v}_{H^1}\,.\]
	Les convergences plus fortes que nous avons pour $z_n$ nous permettent de calculer autrement le membre de gauche:
	\[\inner{z_n}{v}_{H^1} = \int_{\Omega} z_n(x) v(x)dx + \int_{\Omega} \nabla z_n(x) \cdot \nabla v(x)dx \to \int_{\Omega} z(x)v(x) dx\]
\end{proof}




\end{document}
