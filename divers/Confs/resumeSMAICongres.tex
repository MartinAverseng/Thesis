%%%%%%%%%%%%%%%%%%%%%%%%%%%%%%%%%%%%%%%%%%%%%%%%%%%%%%%%%%%%%%%%%%%%%%%%%%%%
%               PRIERE DE NE RIEN MODIFIER CI-DESSOUS                      %
%       JUSQU'A LA LIGNE "PRIERE DE NE RIEN MODIFIER CI-DESSUS"            %
%   TOUT LE BLOC QUI SUIT SERA SUPPRIME LORS DE L'EDITION FINALE DU POLY   %
%   CONTENANT LES RESUMES                                                  %
%%%%%%%%%%%%%%%%%%%%%%%%%%%%%%%%%%%%%%%%%%%%%%%%%%%%%%%%%%%%%%%%%%%%%%%%%%%%
\documentclass[10pt]{article}
%===  Priere de ne pas utiliser d'autres modules
\usepackage{latexsym}
\usepackage{bbm}              % fontes doubles (pour les ensembles, par ex.)
\usepackage{graphicx}         % pour d'eventuelles figures
\usepackage{epsfig}           % (preferer graphicx, si possible)
\usepackage{amsmath}          % AMSTEX
\usepackage{amsfonts}
%
\setlength{\paperheight}{297mm}\setlength{\paperwidth}{210mm}
\setlength{\oddsidemargin}{10mm}\setlength{\evensidemargin}{10mm}
\setlength{\topmargin}{0mm}\setlength{\headheight}{10mm}\setlength{\headsep}{8mm}
\setlength{\textheight}{240mm}\setlength{\textwidth}{160mm}
\setlength{\marginparsep}{0mm}\setlength{\marginparwidth}{0mm}
\setlength{\footskip}{10mm}
\voffset -13mm\hoffset -10mm\parindent=0cm
\def\titre#1{\begin{center}{\Large{\bf #1}}\end{center}}
\def\orateur#1#2{\begin{center}{\underline{\large{\bf #1}}}, {#2}\end{center}}
\def\auteur#1#2{\begin{center}{\large{\bf #1}}, {#2}\end{center}}
\def\auteurenbasdepage#1#2#3{\small{\bf #1}, \small{#2}\\ \small{\tt #3}\\ }
\def\motscles#1{%
	\ifx#1\IsUndefined\relax\else\noindent{\normalsize{\bf Mots-cl\'es :}} #1\\ \fi}
\renewcommand{\refname}{\normalsize R\'ef\'erences}
%
\begin{document}
	\thispagestyle{empty}
	%%%%%%%%%%%%%%%%%%%%%%%%%%%%%%%%%%%%%%%%%%%%%%%%%%%%%%%%%%%%%%%%%%%%%%%%%%%%
	%               PRIERE DE NE RIEN MODIFIER CI-DESSUS                       %
	%%%%%%%%%%%%%%%%%%%%%%%%%%%%%%%%%%%%%%%%%%%%%%%%%%%%%%%%%%%%%%%%%%%%%%%%%%%%
	%
	% DANS TOUTE LA SUITE NOUS PRIONS LES AUTEURS DE BIEN VOULOIR UTILISER
	% LA SYNTAXE TeX STRICTE POUR LES LETTRES ACCENTUEES.
	% ON PEUT AU BESOIN LES REMPLACER APR\'ES LA FRAPPE DU DOCUMENT
	% PAR LEUR \'EQUIVALENT TeX.
	% DANS LE CAS CONTRAIRE LES LETTRES ACCENTU\'ES N'APPARA\^ITRONT PAS
	% DANS LE DOCUMENT FINAL.
	%
	%                   ORATEUR ET CO-AUTEURS
	%---------------------------------------------------------------
	%
	% LES AUTEURS SONT PRIES DE FOURNIR LES BONS ARGUMENTS AUX MACROS CI-DESSOUS :
	%   \Titre         : Titre de la communication
	%   \NomOrateur    : Pr\'enom(s) NOM de l'Orateur
	%   \AdresseCourteOrateur : Exemple : Universit\'e de Rennes 1
	%   \AdresseLongueOrateur : Exemple : IRMAR, Universit\'e de Rennes 1, 263 avenue du G\'en\'eral Leclerc, 35000 Rennes
	%   \EmailOrateur : Adresse electronique
	%
	% ET DE MEME POUR LES EVENTUELS CO-AUTEURS :
	%   \NomAuteurI ...
	%   \AdresseCourteAuteurI ...
	%---------------------------------------------------------------
	% DEFINIR ICI LE TITRE DE VOTRE COMMUNICATION
	\def\Titre{\'El\'ements finis de fronti\`ere sur des domaines singuliers}
	%
	% DEFINIR ICI LES NOMS, ADRESSES, ... DE l'ORATEUR OU UNIQUE AUTEUR
	\def\NomOrateur{Martin AVERSENG}
	\def\AdresseCourteOrateur{CMAP, \'Ecole Polytechnique}
	\def\AdresseLongueOrateur{CMAP, \'Ecole Polytechnique, Route de Saclay, 91128 Palaiseau }
	\def\EmailOrateur{martin.averseng@polytechnique.edu}
	%
	% DEFINIR ICI LES NOMS, ADRESSES, ... DES EVENTUELS CO-AUTEURS
	%\def\NomAuteurI{Pr\'enom NOM}
	%\def\AdresseCourteAuteurI{Adresse Courte}
	%\def\AdresseLongueAuteurI{Adresse Longue}
	%\def\EmailAuteurI{email}
	%=== et ainsi de suite II, III, IV, V ... pour les suivants
	%
	%=== Liste des mots-cles separes par des virgules si besoin
	% N'enlever le signe % que si necessaire
	\def\listmotcles{\'Eléments finis de fronti\`ere, Pr\'econditionnement, Singularit\'es.}
	%
	%
	%                   DEBUT DE LA COMMUNICATION
	%---------------------------------------------------------------
	% NE PAS MODIFIER LA LIGNE SUIVANTE
	% Le titre est a definir dans la macro \Titre (23 lignes plus haut)
	\titre{\Titre}%
	%---------------------------------------------------------------
	% TITRE & AUTEUR(S)
	% RETIRER LES SIGNES % SI NECESSAIRE ET PLACER DANS L'ORDRE SOUHAITE
	% DANS LES LIGNES SUIVANTES NE MODIFIER QUE LES SIGNES COMMENTAIRES '%'
	% Les noms, adresses, email de l'orateur et des co-auteurs sont a definir
	% dans les macros \NomOrateur, \AdresseCourteOrateur etc. plus haut
	%---------------------------------------------------------------
	\orateur{\NomOrateur}{\AdresseCourteOrateur}
	% NE PAS MODIFIER LES 4 LIGNES SUIVANTES sauf a retirer le signe commentaire '%'
	%\auteur{\NomAuteurI}{\AdresseCourteAuteurI}
	%\auteur{\NomAuteurII}{\AdresseCourteAuteurII}
	%\auteur{\NomAuteurIII}{\AdresseCourteAuteurIII}
	%\auteur{\NomAuteurIV}{\AdresseCourteAuteurIV}
	%
	\motscles{\listmotcles}
	%---------------------------------------------------------------
	% TEXTE DE LA COMMUNICATION
	%---------------------------------------------------------------
	
	Les \'el\'ements finis de fronti\`ere sont une m\'ethode num\'erique largement utilis\'ee pour r\'esoudre les porbl\`emes de diffraction d'une onde (acoustique, \'elastique, \'el\'ectromag\'tique...) par un objet $\mathcal{O}$. Ces probl\`emes trouvent de nombreuses applications industrielles. Pour calculer le champ diffract\'e $u_d$ d\'efinit sur le {\bf domaine volumique non born\'e} $\mathbb{R}^3\setminus \mathcal{O}$, on r\'esout une \'equation dite int\'egrale de la forme
	\[\forall x \in \partial \mathcal{O}, \quad \int_{\mathcal{O}} G(x,y) \lambda(y) d\sigma(y) = u_{\text{inc}}(x)\]
	dont l'inconnue $\lambda$ est une fonction d\'efinie sur un {\bf domaine surfacique born\'e}: la fronti\`ere de l'objet diffractant. La solution $u_d$ est ensuite donn\'ee par $\forall x \in \mathbb{R}^3\setminus \mathcal{O}, \quad u_d(x) = \mathcal{F} \lambda(x)$
	o\`u $\mathcal{F}\lambda(x)$  est une fonction que l'on peut \'evaluer rapidement une fois que $\lambda$ a \'et\'e d\'etermin\'e.
	Cette approche permet ainsi de r\'eduire consid\'erablement la taille du maillage utilis\'e pour discr\'etiser le probl\`eme, et donc le nombre d'inconnues dans le syst\`eme lin\'eaire qui en r\'esulte. En contrepartie, la matrice du syst\`eme est {\bf dense}, et il est donc difficile de  l'inverser directement dans les applications int\'eressantes. L'approche la plus r\'epandue consiste \`a r\'esoudre le syst\`eme par une m\'ethode it\'erative de type Krylov, dont la complexit\'e d\'epend de mani\`ere cruciale du conditionnement du syst\`eme. Lorsque l'objet diffractant a une g\'eom\'etrie r\'eguli\`ere, de bonnes strat\'egies de pr\'econditionnement sont connues pour r\'eduire le nombre d'it\'erations, ce qui fait que cette approche se r\'ev\`ele particuli\`erement comp\'etitive. 
	
	Malheureusement, dans beaucoup d'applications r\'eelles, l'hypoth\`ese d'une g\'eom\'etrie r\'eguli\`ere n'est pas satisfaite. Par exemple, pour le probl\`eme du radar, l'objet diffractant est un avion, dont les ailes pr\'esentent des bords en lesquels la courbure est tr\`es grande. En pr\'esence de telles singularit\'es g\'eom\'etriques, deux difficult\'es apparaissent dans la m\'ethode d\'ecrite pr\'ec\'edemment.
	\begin{itemize}
		\item[(i)] Au voisinage des singularit\'es g\'eom\'etriques, l'inconnue $\lambda$ est singuli\`ere. Ceci se traduit num\'eriquement par des taux de convergence en maillage s\'ev\`erement d\'egrad\'es. 
		\item[(ii)] Les approches connues pour construire de bons pr\'econditionneurs s'effondrent car elles utilisent de mani\`ere essentielle la r\'egularit\'e de la fronti\`ere de l'objet. 
	\end{itemize} 
	\`A ce jour, il n'existe pas de m\'ethode qui rem\'edie simultan\'ement aux deux points ci-dessus. En particulier, un raffinement de maillage pr\`es de la singularit\'e, s'il permet de retrouver un bon taux de convergence en maillage, m\`ene \`a des syst\`emes lin\'eaires catastrophiquement mal conditionn\'es. 

	Dans notre expos\'e, nous d\'ecrivons un cadre th\'orique et num\'erique qui permet de r\'esoudre ces deux difficult\'es, en combinant plusieurs id\'ees apparues r\'ecemment dans la litt\'erature \cite{bruno2012second,hiptmair2017closed,ramaciotti2017some}. L'approche passe par l'introduction d'op\'erateurs int\'egraux "\`a poids", d'un maillage raffin\'e (d'une mani\`ere explicite) et par des pr\'econditionneurs construits \`a l'aide d'un nouveau calcul pseudo-diff\'rentiel sur ces g\'eom\'etries singuli\`eres, g\'en\'eralisant l'approche de \cite{antoine2007generalized}. 
	
	
	%---------------------------------------------------------------
	% REFERENCES BIBLIOGRAPHIQUES
	%---------------------------------------------------------------
	% NE PAS MODIFIER LES 2 LIGNES SUIVANTES
	\bibliographystyle{plain}
	\begin{thebibliography}{99}
		
		\bibitem{bruno2012second}
		O.~P. Bruno and S.~K. Lintner.
		\newblock Second-kind integral solvers for TE and TM problems of diffraction by
		open arcs.
		\newblock {\em Radio Science}, 47(6), 2012.
		
		\bibitem{hiptmair2017closed}
		R. Hiptmair, C. Jerez-Hanckes, and C.~A. Urz{\'u}a~Torres.
		\newblock Closed-form exact inverses of the weakly singular and hypersingular
		operators on disks.
		\newblock {\em arXiv preprint arXiv:1703.08556}, 2017.
		
		\bibitem{ramaciotti2017some}
		P. Ramaciotti and J.-C. N{\'e}d{\'e}lec.
		\newblock About some boundary integral operators on the unit disk related to
		the laplace equation.
		\newblock {\em SIAM Journal on Numerical Analysis}, 55(4):1892--1914, 2017.
		
		\bibitem{antoine2007generalized}
		X. Antoine and M. Darbas.
		\newblock Generalized combined field integral equations for the iterative
		solution of the three-dimensional helmholtz equation.
		\newblock {\em ESAIM: Mathematical Modelling and Numerical Analysis},
		41(1):147--167, 2007.
		
		% NE PAS MODIFIER LA LIGNE SUIVANTE
	\end{thebibliography}
	%
	%---------------------------------------------------------------
	% NOM & ADRESSE COMPLETE & EMAIL DU OU DES AUTEURS
	% RETIRER LES SIGNES % SI NECESSAIRE ET PLACER DANS L'ORDRE SOUHAITE
	% DANS LES LIGNES SUIVANTES NE MODIFIER QUE LES SIGNES COMMENTAIRES '%'
	%---------------------------------------------------------------
	\vfill
	\auteurenbasdepage{\NomOrateur}{\AdresseLongueOrateur}{\EmailOrateur}
	% Les noms, adresses, email de l'orateur et des co-auteurs sont a definir
	% dans les macros \NomOrateur, \AdresseCourteOrateur etc. plus haut
	%
	%\auteurenbasdepage{\NomAuteurI}{\AdresseLongueAuteurI}{\EmailAuteurI}
	%\auteurenbasdepage{\NomAuteurII}{\AdresseLongueAuteurII}{\EmailAuteurII}
	%\auteurenbasdepage{\NomAuteurIII}{\AdresseLongueAuteurIII}{\EmailAuteurIII}
	%\auteurenbasdepage{\NomAuteurIV}{\AdresseLongueAuteurIV}{\EmailAuteurIV}
	%%%%%%%%%%%%%%%%%%%%%%%%%%%%%%%%%%%%%%%%%%%%%%%%%%%%%%%%%%%%%%%%%%%%%%%%%%%
\end{document}