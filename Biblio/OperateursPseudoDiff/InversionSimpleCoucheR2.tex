\documentclass[11pt,a4paper]{article}

\usepackage{adjustbox}
\usepackage{algorithm}
\usepackage{algorithmic}
\usepackage{amsmath}
\usepackage{amssymb}
\usepackage{amsthm}
\usepackage{amsfonts}
\usepackage{afterpage}
\usepackage{blindtext}
\usepackage[font=footnotesize,labelfont=bf]{caption}
\usepackage{hyperref}
\usepackage[english]{babel}
\usepackage{bbm}
\usepackage{bigints}
\usepackage{bm}
\usepackage{cite}
\usepackage{color}
\usepackage{float}
\usepackage[left=2cm,right=2cm,top=2cm,bottom=2cm]{geometry}
\usepackage{graphicx}
\usepackage[utf8]{inputenc}
\usepackage{mathtools}
\usepackage{mdframed}
\usepackage{pgfplots} 
\usepackage{subfigure}
\usepackage{stmaryrd}
\usepackage{textcomp}
\usepackage{tikz}
\usepackage{url}
\renewcommand{\proofname}{Proof}
\theoremstyle{plain}
\newtheorem{monTheoNumrote}{Théorème}[section] % Environnement numéroté en fonction de la section
\newtheorem*{monTheoNonNumerote}{Théorème}  % Environnement non numéroté
\newtheorem{The}{Theorem}[section]
\newtheorem*{The*}{Theorem}
\newtheorem{Prop}{Proposition}[section]
\newtheorem*{Prop*}{Proposition} 
\newtheorem{Cor}{Corollary}[section]
\newtheorem*{Cor*}{Corollary}
\newtheorem{Conj}{Conjecture}[section]
\newtheorem{Lem}{Lemma}[section]
\renewcommand{\qed}{\unskip\nobreak\quad\qedsymbol}%
\numberwithin{equation}{section} % Numérote les équations section.numéro.
\theoremstyle{definition}
\newtheorem{Def}{Definition}[section]
\newtheorem{Rem}{Remark}[section]
\newtheorem*{Rem*}{Remark}
\newtheorem*{Lem*}{Lemma}
\newtheorem{Que}{Question}
\newcommand{\enstq}[2]{\left\{#1\mathrel{}\middle|\mathrel{}#2\right\}}
\newcommand{\Lp}[2]{L^#1(#2)}
\newcommand{\Sob}[3]{W^{#1,#2}(#3)}
\newcommand{\Rd}[0]{\mathbb{R}^d}
\newcommand{\RN}[0]{\mathbb{R}^N}
\newcommand{\Rn}[0]{\mathbb{R}^n}
\newcommand{\norm}[1]{\left\|#1\right\|}
\newcommand{\sinc}[0]{\textup{sinc}}
\newcommand{\functionDef}[5]{\begin{array}{lllll}
#1 & : & #2 & \longrightarrow & #3 \\
 & & #4 & \longmapsto &\displaystyle #5 \\
\end{array}}
\newcommand{\Theautorefname}{Theorem}
\newcommand{\Propautorefname}{Proposition}
\newcommand{\Corautorefname}{Corollary}
\newcommand{\Lemautorefname}{Lemma}
\newcommand{\Defautorefname}{Definition}
\newcommand{\N}{\mathbb{N}}
\newcommand{\Z}{\mathbb{Z}}
\newcommand{\D}{\mathbb{D}}
\newcommand{\R}{\mathbb{R}}
\newcommand{\A}{\mathcal{A}_{a,b}}
\newcommand{\Crad}{C^\infty_{c,rad}(B)}
\newcommand{\Lrad}{L^2_{rad}(B)}
\newcommand{\Lradab}{L^2_{rad}(\mathcal{A}_{a,b})}
\newcommand{\duality}[2]{\left\langle #1,#2\right\rangle}
\newcommand{\Hrad}{H^1_{rad}(B)}
\newcommand{\Hzrad}{H^1_{0,rad}(B)}
\newcommand{\rmin}{\delta_{\min}}
\newcommand{\rmax}{\delta_{\max}}
\newcommand{\corr}{\gamma}
\newcommand{\question}[1]{\begin{Que} \ 
#1
\end{Que}}
\newcommand{\abs}[1]{\left\lvert #1 \right\rvert}
\newcommand{\CL}[2]{\textup{CL}\left(\enstq{#1}{#2}\right)}
\newcommand{\Script}[1]{`\texttt{#1}`}
\newcommand{\espace}{\text{ }\qquad} 
\newcommand{\loc}{\text{loc}}
\newcommand{\SL}{\textup{SL}\hspace{1.5pt}}
\newcommand{\DL}{\textup{DL}\hspace{1.5pt}}
\newcommand{\fp}{\underset{\varepsilon \to 0}{\textup{f.p.}}}
\newcommand{\scalProd}[2]{\left(#1|#2\right)}
\newcommand{\toDo}[1]{{\color{red}#1}}
\newcommand{\bs}[1]{\boldsymbol{#1}}
\newcommand{\varInRange}[4]{(#1_{#2})_{#3 \leq #2 \leq #4}}
\newcommand{\from}{\colon}
\newcommand{\Cinf}{C^{\infty}}
\newcommand{\isdef}{\mathrel{\mathop:}=}
\newcommand{\defis}{=\mathrel{\mathop:}}

\renewcommand{\algorithmicrequire}{\textbf{Inputs:}}
\renewcommand{\algorithmicensure}{\textbf{Outputs:}}

\pgfplotsset{compat=1.13}
\title{Inversion de l'opérateur de simple couche sur un hyperplan}
\author{Martin Averseng}
\begin{document}
\maketitle

\section{Notations}

On considère le noyau de Green $G$ de l'équation de Helmholtz dans $\mathbb{R}^3$,
\[ -(\Delta+k^2)G = \delta_0 \text{ dans }\mathbb{R}^3,\]
associée à la fréquence $k$, $k > 0$ et vérifiant la condition de radiation à l'infini.
\[\lim_{r \to \infty} r\dfrac{\partial G}{\partial r} - ik G = 0,\]
où $r$ désigne $|x|$, $\dfrac{\partial}{\partial r}$ désigne l'opérateur $x\cdot \nabla$, et $\delta_0$ représente la fonction dirac centrée à l'origine. On rappelle qu'alors l'expression de $G$ est donnée par 
\[ G(x) = \dfrac{e^{ik|x|}}{4\pi|x|}\]
On définit un opérateur \[\functionDef{K}{\mathcal{S}(\mathbb{R}^{2})}{\mathcal{S}(\mathbb{R}^{2})}{u}{ x \in \mathbb{R}^2 \mapsto\int_{\mathbb{R}^{2}}\tilde{G}(x-y)u(y)dy}\]
Où $\tilde{G}$ est défini sur $\mathbb{R}^{2}$ par :
\[ \tilde{G}(x_1,x_2) = G(x_1,x_2,0)\]
On souhaite montrer le théorème suivant 
\begin{The} L'opérateur $K$ est inversible, et son inverse est donné par l'opérateur $L$ défini par 
\[ \functionDef{L}{\mathcal{S}(\mathbb{R}^{2})}{\mathcal{S}(\mathbb{R}^{2})}{u}{x \in \mathbb{R}^d \mapsto\int_{\mathbb{R}^{2}}(\Delta\tilde{G}+k^2\tilde{G})(x-y)u(y)dy}\]
\end{The}

\section{Démonstration}

L'opérateur précédent étant exprimé comme une convolution par le noyau $\tilde{G}$, nous allons étudier la transformée de Fourier de ce dernier : 
\begin{Lem} 
\[\mathcal{F}(\tilde{G})(\xi) = \dfrac{1}{4\pi|\xi|} \mathcal{F}(HJ_0)\left(-\dfrac{k}{|\xi|}\right),\]
Où $H$ est la fonction d'Heaviside. 
\end{Lem}
\begin{proof}
Soit $\varphi$ une fonction test. Par définition :
\begin{align*}
\duality{\mathcal{F}(\tilde{G})}{\varphi} &= \duality{\tilde{G}}{\mathcal{F}(\varphi)}\\
 &= \int_{\mathbb{R}^2} G(x) \int_{\mathbb{R}^2} e^{-i x \cdot \xi} \varphi(\xi) d\xi
\end{align*}
On effectue un changement de variable polaire dans la première intégrale : 
\begin{align*}
\duality{\mathcal{F}(\tilde{G})}{\varphi} &= \int_{0}^{+\infty} rG(r)dr \int_{\mathbbm{S}^2} d\sigma(u) \int_{\mathbb{R}^2} e^{-ir \xi \cdot u} \varphi(\xi)d \xi\\
\end{align*}
En utilisant le théorème de Fubini, on permute les deux dernières intégrales, puisque l'intégrande est majorée sur un domaine compact. 
\begin{align*}
\duality{\mathcal{F}(\tilde{G})}{\varphi} &= \int_{0}^{+\infty} e^{ikr}dr \int_{\mathbb{R}^2}\varphi(\xi) \left( \int_{\mathbbm{S}^2}e^{-ir \xi \cdot u} d\sigma(u) \right)d \xi\\
&= \int_{0}^{+\infty}  e^{ikr}dr \int_{\mathbb{R}^2} J_0(r|\xi|) \varphi(\xi) d\xi
\end{align*}
On effectue enfin le changement de variable $r' = r|\xi|$ : 
\begin{align*}
\duality{\mathcal{F}(\tilde{G})}{\varphi} &= \int_{0}^{+\infty}\int_{\mathbb{R}^2} \dfrac{e^{i k \frac{r'}{|\xi|}}}{|\xi|} J_0(r')\varphi(\xi) d\xi dr'
\end{align*}
Ok je vois que le calcul n'aboutit pas comme ça... Il faut que j'approche $G$ par $G\phi_n$ pour tronquer $G$ et rendre tout intégrable. Ensuite j'utilise que si $G_n \to G$ alors $\mathcal{F}(G_n) \to \mathcal{F}(G)$

\end{proof}
\begin{Def} Soit $E$ un sous-espace vectoriel de dimension $p$ de $\mathbb{R}^{d-1}$. On définit la distribution de mesure uniforme sur $E$ $\mu_{E}$ par 
\[\duality{\mu_{E}}{f} = \int_{\mathbb{R}^{p}} f\left(\sum_{i=1}^p \lambda_i e_i\right)d\lambda_1 ...d\lambda_p\]
Où $(e_1,...,e_p)$ est une base orthonormée de $E$.
\end{Def}
Notons que cette définition est correcte, et notamment qu'elle ne dépend pas du choix de la base en vertu du théorème de changement de variables. 

\begin{Lem} La distribution $\mu_E$ a pour transformée de Fourier la distribution $\mu_{E^{\perp}}$ 
\begin{proof}
Nous notons $(e_1,e_2,...,e_d)$ une base orthonormée adaptée à la décomposition $\mathbb{R}^d = E\bigoplus E^{\perp}$ et $f$ une fonction test. Par définition de la transformée de Fourier, on a 
\[\duality{\hat{\mu}_E}{f} = \int_{\mathbb{R}^p} \hat{f}\left(\sum_{i=1}^p \lambda_i e_i\right)d\lambda_1...d\lambda_n\]
Et l'on peut expliciter $\hat{f}$ dans l'égalité précédente, soit
\[ \duality{\hat{\mu}_E}{f} = \int_{\mathbb{R}^p} \int_{\mathbb{R}^d}\exp\left(i\left(\sum_{i=1}^p\lambda_i e_i\right)\cdot \left(\sum_{i=1}^d\nu_i e_i\right)\right)f\left(\sum_{i=1}^d\nu_i e_i\right)d\nu_1...d\nu_d d\lambda_1 ...d\lambda_p\]
\[\duality{\hat{\mu}_E}{f} = \int_{\mathbb{R}^p} \int_{\mathbb{R}^d}\exp\left(i\sum_{i=1}^p\lambda_i \nu_i\right)f\left(\sum_{i=1}^d\nu_i e_i\right)d\nu_1...d\nu_d d\lambda_1 ...d\lambda_p\]
Décomposons la seconde intégrale en intégrant d'abord sur $E^{\perp}$ :

\[\duality{\hat{\mu}_E}{f} = \int_{\mathbb{R}^p} \int_{\mathbb{R}^p}\exp\left(i\sum_{i=1}^p\lambda_i \nu_i \right)\left(\int_{\mathbb{R}^{d-p}}f\left(\sum_{i=1}^d\nu_i e_i\right)d\nu_{p+1}...d\nu_d \right)d\nu_1... d\nu_p d\lambda_1 ...d\lambda_p\]

Pour éclaircir les notations et voir le résultat, il reste à poser 
\[F(\nu_1,\nu_2,...,\nu_p) = \int_{\mathbb{R}^{d-p}}f\left(\sum_{i=1}^d\nu_i e_i\right)d\nu_{p+1}...d\nu_d \]
Avec cette définition, on peut réécrire 
\[\duality{\hat{\mu}_E}{f} = \int_{\mathbb{R}^p} \int_{\mathbb{R}^p}\exp\left(i\sum_{i=1}^p\lambda_i \nu_i \right)F(\nu_1,...,\nu_p)d\nu_1... d\nu_p d\lambda_1 ...d\lambda_p\]
Le théorème d'inversion de la transformée de Fourier assure alors 
\[\duality{\hat{\mu}_E}{f} = F(0,0,...,0)\]
C'est-à-dire  
\[\duality{\hat{\mu}_E}{f} = \duality{\mu_{E^\perp}}{f}\]
\end{proof}
\end{Lem}
Nous n'aurons pas besoin de ce lemme. 
À la place, nous sommes obligés de passer par un calcul assez technique. D'abord, il est classique que
\begin{Lem} La transformée de Fourier de la fonction de Bessel $J_0$ est donnée par : 
\[\mathcal{F}(J_0)(\omega) = \int_{-\infty}^{+\infty} e^{-i\omega r} J_0(r)dr = 2\dfrac{\mathbbm{1}_{]-1,1[}(\omega)}{\sqrt{1-w^2}}\]
\end{Lem}
Cela permet d'obtenir, si $H$ est la fonction d'Heaviside :  
\begin{Lem} On a la formule suivante : 
\[ \mathcal{F}(H  J_0) = \int_{0}^{+\infty} e^{i\omega r} J_0(r) = \mathbbm{1}_{-1<\omega<1}\dfrac{1}{\sqrt{1-\omega^2}} + i \mathbbm{1}_{\omega \notin ]-1,1[}\dfrac{1}{\sqrt{\omega^2-1}} \]
\begin{proof}
Notez que l'intégrale écrite ci-dessus est impropre puisque $J_0$ n'est pas intégrable en l'infini. On doit donc travailler avec les distributions. Étant donné que la transformée de Fourier de $J_0$ est à support compact, les transformes de Fourier des distributions $H$ et $J_0$ sont convolables et avec les conventions choisies pour la transformée de Fourier, on a :
\begin{align*}
\mathcal{F}(HJ_0) &= \dfrac{1}{2\pi}\mathcal{F}(H)*\mathcal{F}(J_0)\\
& = \dfrac{1}{2\pi}\left(\pi \delta - i \text{v.p.}\left(\dfrac{1}{\omega}\right)\right)*\mathcal{F}(J_0)\\
& = \dfrac{1}{2} \mathcal{F}(J_0) - \dfrac{i}{2\pi} \text{v.p.}\left(\dfrac{1}{\omega}\right)*\mathcal{F}(J_0)
\end{align*}
Seul le dernier terme reste à calculer, étant donné le résultat du lemme précédent. Soit $\varphi$ une fonction test, par définition du produit de convolution dans $\mathcal{D}(\mathbb{R})\times \mathcal{E}(\mathbb{R})$, on a 
\begin{align*}
\duality{\text{v.p.}\left(\dfrac{1}{\omega}\right)*\mathcal{F}(J_0)}{\varphi} &= \duality{\text{v.p.}\left(\dfrac{1}{\omega}\right)}{\mathcal{F}(J_0)*\varphi} \\
& = \lim_{\varepsilon\to0} \int_{\abs{\omega}> \varepsilon} \dfrac{\varphi * \mathcal{F}(J_0)(\omega)}{\omega}
\end{align*} 
En tenant compte de la parité de la transformée de Fourier de $J_0$. Pour un $\varepsilon$ fixé, on peut réécrire l'intégrale précédente en permutant l'ordre d'intégration : 
\begin{align*}
\int_{\abs{\omega}> \varepsilon} \dfrac{\varphi * \mathcal{F}(J_0)(\omega)}{\omega} &= 2\int_{\abs{\omega}> \varepsilon} \dfrac{1}{\omega} \int_{-\infty}^{+\infty} \varphi(u)\dfrac{\mathbbm{1}_{]-1,1[}(\omega-u)}{\sqrt{1-(\omega-u)^2}}\\
& = 2 \int_{-\infty}^{+\infty} \varphi(u) \int_{\abs{\omega}> \varepsilon} \dfrac{\mathbbm{1}_{]-1,1[}(\omega-u)}{\omega\sqrt{1-(\omega-u)^2}}
\end{align*}
Nous allons calculer la deuxième intégrale : 
\begin{align*}
I_\varepsilon(u) = \int_{u-1}^{u+1} \dfrac{\mathbbm{1}_{\abs{\omega}> \varepsilon}}{\omega\sqrt{1-(\omega-u)^2}}d\omega
\end{align*}
Il faut montrer qu'elle converge dans $L^1(\mathbb{R})$ vers la fonction 
\[I(u) = \mathbbm{1}_{u \notin [-1,1]} \dfrac{\pi}{\sqrt{u^2-1}}\]
Il est aisé de voir que $I_\varepsilon(u) = I_{\varepsilon}(-u)$. Nous ne le calculerons donc que pour $u$ positif. Soit $\varepsilon$ un réel positif suffisamment petit pour rendre possibles tous les prochains calculs, Lorsque $u>1+\varepsilon$, un calcul élémentaire permet de montrer que 
\[I_\varepsilon(u) = \dfrac{\pi}{\sqrt{u^2-1}}.\]
Lorsque $u \in ]1-\varepsilon,1+\varepsilon[$, on a alors l'expression
\[I_\varepsilon(u) = \int_{\varepsilon}^{u+1} \dfrac{1}{\omega\sqrt{1-(\omega-u)^2}}d\omega,\]
dont il est facile de montrer qu'elle vérifie 
\[|I_\varepsilon(u)| \leq C + D\ln\left(\dfrac{1}{\varepsilon}\right),\]
ce qui assure que l'intégrale de $I_\varepsilon$ sur $]1-\varepsilon,1+\varepsilon[$ tend vers 0 comme $\varepsilon\ln(\varepsilon)$. 

Lorsque $\varepsilon < u < 1-\varepsilon$, le calcul est ramené à 
\[I_\varepsilon(u) = \int_{]u-1,-\varepsilon[\cup]\varepsilon,u+1[} \dfrac{1}{\omega\sqrt{1-(\omega-u)^2}} = \int_{]-1,-u-\varepsilon[\cup]-u+\varepsilon,1[} \dfrac{1}{	(\omega+u)\sqrt{1-\omega^2}}.\]
Les changements de variables $\omega = \sin(\theta)$ et $t = \tan(\theta/2)$ mènent à l'expression suivante : 
\[ I_{\varepsilon}(u) = \int_{]-1,h(u-\varepsilon)[\cup]h(u+\varepsilon),1[} \dfrac{1}{u(t-u{-})(t-u^{+})}.\]
Où $u^{-}$ et $u^{+}$ sont les deux racines distinctes du polynôme 
\[P(u) = t^2 - \dfrac{2}{u}t + 1.\]
de discriminant $\Delta = \dfrac{4}{u^2}(1-u^2) >0$, et où $h(x)$ est la fonction définie pour $x \in ]0,1[$ par 
\[h(x) = \dfrac{1-\sqrt{1-x^2}}{x}\]  
On note que $h(u) = u^{-}$, $h$ tend vers $0$ en $0$, vaut $1$ en $x=1$ et est strictement croissante. Ainsi, comme on pouvait s'y attendre, l'intégration "évite" le pôle non intégrable $u^{-}$. On peut noter que $u^+$ est en dehors du domaine d'intégration (en effet, $u^{+}>1$ puisque $u^+ u^- = 1$). En décomposant en éléments simples 
\[ \dfrac{1}{u(t-u^-)(t-u^+)} = \dfrac{1}{2\sqrt{1-u^2}}\left(\dfrac{1}{t-u^+} - \dfrac{1}{t-u^-}\right)\] 
et en intégrant cette expression, on trouve
\[ I_\varepsilon(u) = \dfrac{1}{2\sqrt{1-u^2}} \left(\left[\ln\left(\dfrac{u^+-t}{u^--t}\right)\right]_{-1}^{h(u-\varepsilon)} + \left[\ln\left(\dfrac{u^+-t}{t-u^-}\right)\right]_{h(u+\varepsilon)}^{1}\right)\]
Ce qui conduit à 
\[I_{\varepsilon}(u) = \dfrac{1}{2\sqrt{1-u^2}}\ln\left(\dfrac{(u^+ - h(u-\varepsilon))(h(u+\varepsilon)-u^-)}{(u^+-h(u+\varepsilon))(u^--h(u-\varepsilon))}\right)\]
Un développement limité en $\varepsilon$ permet de constater que la limite ponctuelle de $I_\varepsilon$ est bien nulle sur l'intervalle $]\varepsilon,1-\varepsilon[$. D'autre part des majorations grossières de $I_\varepsilon(u)$ montrent qu'elle est dominée indépendamment de $\varepsilon$ par une fonction intégrable sur $]-1,1[$. Le théorème de convergence dominée assure donc que 
\[ \lim_{\varepsilon\to 0} \int_{\varepsilon}^{1-\varepsilon} |I_{\varepsilon}(u)|du = 0\]
De même que précédemment, il est évident que 
\[ \lim_{\varepsilon \to 0} \int_{0}^{\varepsilon}|I_{\varepsilon}(u)| = 0 \]
en majorant encore grossièrement $I_\varepsilon$ sur l'intervalle d'intégration. 
\end{proof}
\end{Lem}

On déduit des résultats précédents la formule suivante :
\begin{Lem} Pour tout $\xi \in \mathbb{R}^{2}$
\[ \mathcal{F}(\tilde{G})(\xi) = \dfrac{1}{4\pi}\left(\mathbbm{1}_{k < |\xi|}\dfrac{1}{\sqrt{|\xi|^2-k^2}} + i \mathbbm{1}_{k > |\xi|}\dfrac{1}{\sqrt{k^2-|\xi|^2}}\right) \]
\end{Lem}
En particulier, on peut constater que 
\[(|\xi|^2-k^2) (\mathcal{F}(\tilde{G}))^2 = \dfrac{1}{16\pi^2}\]

Prouvons maintenant le théorème. Soit $u\in \mathcal{S}(\mathbb{R}^2)$ et soit $v = Ku$. En changeant la convolution en produit dans le domaine de Fourier, cela s'écrit :
\[v(x) = \dfrac{1}{4\pi^2}\int_{\mathbb{R}^2} e^{i x\cdot \xi}\mathcal{F}(\tilde{G})(\xi)\mathcal{F}(u)(\xi)\] 
Appliquons l'opérateur $-\Delta - k^2Id$ à cette expression et nommons $w$ le résultat :
\begin{align*}
w(x) &= -\Delta(v)(x) - k^2 v(x) \\
&= \dfrac{1}{4\pi^2}\int_{\mathbb{R}^2} (|\xi|^2-k^2)e^{i x\cdot \xi}\mathcal{F}(\tilde{G})(\xi)\mathcal{F}(u)(\xi)\\
\end{align*}
Appliquons l'opérateur $K$ à $w$ : 
\[Kw(x) = \dfrac{1}{2\pi} \int_{\mathbb{R}^2}e^{i x\cdot \xi}\mathcal{F}(\tilde{G})(\xi)\dfrac{\mathcal{F}(u)(\xi)}{\mathcal{F}(\tilde{G})(\xi)} = u(x)\]
Par transformée de Fourier inverse. En permuttant la convolution et les dérivations, on en déduit que 
\[ u(x) = (-\Delta \tilde{G} - k^2 \tilde{G}) * Ku (x)\]
Ce qui conclut la démonstration. 
\end{document}

% Christiansen en thèse avec Nédelec 
% demander les références sur l'inversion exacte
% Ramaciotti 
% Oscar Bruno
% Regarder l'hypersingulier comment le faire
% Bendali (papier récent) 
% Sophie Borel
% Invesion analytique en Fourier pour vérifier
% Bouquin de Nédélec
% Relire aboud terasse sur page 189 (se focaliser sur l'hypersingulier). 
% Bouquin de Dautray-JL-Lions : chapitre éq intégrales. (le truc en 9 vol.)

%%Se concnentrer sur : 
1°) Grisvard traces sur 1 coin (exprimer les conditions de raccord). 
2°) Faire un point sur l'hypersingulier, comprendre bien Calderon. 



