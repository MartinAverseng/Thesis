\documentclass[11pt,a4paper]{article}
\usepackage[utf8]{inputenc}
\usepackage[english]{babel}
\usepackage{amsmath}
\usepackage{bbm}
\usepackage{amsthm}
\usepackage{amsfonts}
\usepackage{amssymb}
\usepackage{graphicx}
\usepackage{lmodern}
\usepackage[left=2cm,right=2cm,top=2cm,bottom=2cm]{geometry}
\author{Martin}
\title{Quadrature pour les fonctions de Bessel}
\begin{document}
\renewcommand{\proofname}{Preuve}
\maketitle
\theoremstyle{plain}
\newtheorem{The}{Théorème}[section]
\newtheorem{Prop}{Proposition}[section]
\theoremstyle{definition}
\newtheorem{Def}{Définition}[section]
\newcommand{\enstq}[2]{\left\{#1\mathrel{}\middle|\mathrel{}#2\right\}}
\newcommand{\Lp}[2]{L^#1(#2)}
\newcommand{\Sob}[3]{W^{#1,#2}(#3)}
\newcommand{\RN}[0]{\mathbb{R}^N}
\newcommand{\norm}[1]{\left\|#1\right\|}
\newcommand{\sinc}[0]{\textup{sinc}}

\section{Introduction}

Soit $x \in \mathbb{R}^3$. On souhaite donner une approximation de $J_0(|x|)$ sous la forme 
\[J_0(|x|) \approx \sum_{k=0}^{N_\xi} w_k \exp(i\xi_k\cdot x)\]Une telle expression se prête au calcul rapide d'une somme de la forme \[\sum_{l=0}^N J_0(|x-y|)f(y)\] à l'aide de transformées de Fourier rapides non-uniformes (les variables x et y sont séparées dans la forme approchée). Or une telle expression existe sous une forme continue car on a pour tout $x$ \[J_0(|x|) = \frac{1}{2\pi}\int_{\mathcal{C}} e^{i\xi \cdot x} d\xi\] où $\mathcal{C}$ est le cercle unité. Le but de ce petit document est de majorer la différence entre cette intégrale et son approximation par une somme de Riemann. 

\begin{Prop} Soit $N\in \mathbb{N}$ et $X\in \mathbb{R}^3$. On suppose que $\frac{|X|}{N}  < 1$. On a alors la majoration suivante : 

\begin{equation}
\left| J_0(|x|) - \frac{1}{N}\sum_{j} e^{ix\sin\left(\frac{2j\pi}{N}\right)}\right|\leq C_N \left(e\frac{|X|}{N} \right)^N
\label{Erreur}
\end{equation}
Où $C_N \leq 3$ et $C_N \underset{N\to+\infty} {\to}2$
\end{Prop}

\section{Quadrature pour les fonctions périodiques}

\begin{Prop} \label{erreurRect}
Soit f une fonction de classe $C^2$ $2\pi$ périodique. On a alors \[\frac{1}{2\pi}\int_0^{2\pi}f - \frac{1}{N}\sum_{j = 0}^{N-1}f \left(2\frac{j\pi}{N}\right) = - \sum_{k\in \mathbb{Z}^*} c_{kN}(f)\] où $c_k(f)$ représentent les coefficients de Fourier de f

\begin{proof}
Comme $f$ est $C^2$, elle est égale à sa série de Fourier (qui converge normalement).En injectant l'expression \[f(x) = \sum_{k\in \mathbb{Z}} c_{k}(f)e^{ikx}\] dans le membre de gauche, et en remarquant que \[\frac{1}{N}\sum_{j=0}^{N-1}e^{ik\frac{2j \pi }{N}} = \int_0^{2\pi} e^{ikx} + \delta(k\in N\mathbb{Z}^*)\]
\end{proof}
\end{Prop}

Le résultat suivant montre que la méthode des rectangles pour une fonction périodique est précise à n'importe quel ordre. 

\begin{Prop} 
Dans le cas où f est de classe $C^m$, on a la majoration \[\left| \int_0^{2\pi}f - \frac{1}{N}\sum_{j = 0}^{N-1}f \left(2\frac{j\pi}{N}\right)\right| \leq \frac{1}{N^m} \zeta(m) \norm{f^{(p)}}_{\infty}\]
Où $\zeta$ est la fonction zêta de Riemann. 
 

\begin{proof}
On effectue une majoration par inégalité triangulaire dans la proposition précédente et on applique l'identité \[c_k\left(f^{(p)}\right) = (ik)^p c_k(f)\]
\end{proof}
\end{Prop}

Sur des cas particuliers, il peut être difficile d'estimer le terme $\norm{f^{(p)}}_{\infty}$. Et même si celle-ci est estimée précisément, il se peut que l'inégalité donnée par la proposition précédente soit sous-optimale. Nous utilisons uniquement la forme exacte donnée par la proposition \ref{erreurRect} dans la suite. 

\section{Résultats sur les fonctions de Bessel}

On note $J_n$ la fonction de Bessel d'ordre $n$. 
On a la définition suivante, qui permet d'identifier les coefficients de Fourier de $u \mapsto e^{iz\sin u}$

\begin{Def}
Par définition, la fonction de Bessel d'ordre $n$ est donnée par 
\[J_n(z) = \frac{1}{2\pi}\int_{0}^{2\pi} e^{iz\sin(u)}e^{-inu}du\]
\end{Def}

J'ai ensuite trouvé ce résultat, dont la preuve que je n'ai pas encore bien décortiquée, utilise la théorie des séries de Laurent.  

\begin{Prop}
Pour tout $R>1$, pour tout $t\in\mathbb{C}$, pour tout $n\in \mathbb{Z}$ on a la majoration suivante : \[|J_n(t)| \leq R^{-|n|}e^{R|t|} \]

\end{Prop}

\section{Majoration de l'erreur}

On peut maintenant prouver la majoration annoncée, que je rappelle ici :

\begin{Prop} Soit $N\in \mathbb{N}$ et $X\in \mathbb{R}^3$. On suppose que $\frac{|X|}{N} \leq  < 1$. On a alors la majoration suivante : \[\left| J_0(|x|) - \frac{1}{N}\sum_{j} e^{ix\sin\left(\frac{2j\pi}{N}\right)}\right|\leq C_N \left(e\frac{|X|}{N} \right)^N\] 
Où $C_N \leq 3$ et $C_N \underset{N\to+\infty} {\to}2$

 
\begin{proof}
D'après le résultat sur les quadratures périodiques, et l'identification des coefficients de Fourier de la fonction dont on approche l'intégrale, en notant $\varepsilon(X,N)$ le terme d'erreur à l'intérieur de la valeur absolue, on a \[\varepsilon(X,N) = -\sum_{k\in N\mathbb{Z}^*} J_k(X)\]
On utilise alors la majoration des fonctions $J_k$ : pour toute famille $R_k>1$ on a 

\[|\varepsilon(X,N)| \leq 2\sum_{k\in \mathbb{N}^*} R_k^{-Nk}e^{R_k|X|}\]

On peut alors optimiser chaque terme en $R_k$, soit $R_k = \frac{Nk}{|X|}> 1$ et on trouve 

\[|\varepsilon(X,N)| \leq 2\sum_{k\in \mathbb{N}^*} \left(\frac{e|X|}{N}\right)^{kN} \times \frac{1}{k^{kN}}\]

On applique l'inégalité de Hölder pour trouver : 

\[|\varepsilon(X,N)| \leq 2\left(\frac{e|X|}{N}\right)^{N} \sum_{k\in \mathbb{N}^*}  \frac{1}{k^{kN}} \]
On pose $\gamma_N = \frac{1}{k^{kN}}$. On a alors \[0 \leq \gamma_N - 1 = \sum_{k\geq 2} \frac{1}{k^{kN}}\leq \sum_{k\geq 2} \frac{1}{2^{kN}}\] Donc \[0 \leq \gamma_N - 1 \leq \frac{1}{2^{2N} - 2^N}\]. On en déduit que $C_N = 2\gamma_N \leq 3$ et $C_N$ tend vers $2$ quand $N$ tend vers l'infini. 

\end{proof}
\end{Prop}

\section{Développement asymptotique en fonction de $\varepsilon$ et de $|X|$ }

\subsection{En fonction de $\varepsilon$}

Dans cette section, on suppose que $\varepsilon$ tend vers 0 et $|X|$ est fixé. On cherche à donner un équivalent de $N$ tel que 
\begin{equation}
C_N \left(e\frac{|X|}{N} \right)^N = \varepsilon
\label{eqN}
\end{equation}
Pour commencer, il est clair que $\underset{\varepsilon\to 0}{\lim} N(\varepsilon) = +\infty$. En prenant le logarithme, on a \[N\log N - N\log(e|X|) = \log\left(\frac{1}{\varepsilon}\right) + \log(C_N)\] 
On a donc l'équivalent suivant : 
$N \log(N) \underset{\varepsilon\to 0}{\sim} \log\left(\frac{1}{\varepsilon}\right) $. On en déduit également en passant au logarithme que $\log(N)  \underset{\varepsilon\to 0}{\sim} \log\left(\log\left(\frac{1}{\varepsilon}\right)\right)$. Il s'ensuit que \[ N(\varepsilon)\underset{\varepsilon\to 0}{\sim}\frac{\log\left(\frac{1}{\varepsilon}\right) }{\log\left(\log\left(\frac{1}{\varepsilon}\right)\right)}\]. 

\subsection{En fonction de $|X|$}

On suppose à présent que $\varepsilon$ est fixé et $|X|$ tend vers l'infini. On cherche un équivalent des $N$ positifs vérifiant (\ref{eqN}). On suppose que $\varepsilon < 1$. En passant (\ref{eqN}) au logarithme, on peut écrire \[\log(N) - \log(e|X|) = \frac{1}{N}\log\left(\frac{C_N}{\varepsilon}\right)\]. On en déduit d'abord que $N>e|X|$, donc $N\underset{|X| \to +\infty}{\longrightarrow} +\infty$. Ceci entraîne, en le réinjectant dans l'équation précédente, que $N \underset{|X| \to +\infty}{\sim} e|X|$. En écrivant ensuite \[\log(N) - \log(e|X|) = \log\left(1 + \frac{N-e|X|}{e|X|}\right)\underset{|X| \to +\infty}{\sim} \frac{N-e|X|}{e|X|}\], on a finalement le développement asymptotique suivant pour $N$ \[N = e|X| + \log\left(\frac{C_N}{\varepsilon}\right) + o(1)\]


\end{document}

