\documentclass[11pt,a4paper]{article}
\IfFileExists{Definitions.tex}{
\usepackage{adjustbox}
\usepackage{algorithm}
\usepackage{algorithmic}
\usepackage{amsmath}
\usepackage{amssymb}
\usepackage{amsthm}
\usepackage{amsfonts}
\usepackage{afterpage}
\usepackage{blindtext}
\usepackage[font=footnotesize,labelfont=bf]{caption}
\usepackage{hyperref}
\usepackage[english]{babel}
\usepackage{bbm}
\usepackage{bigints}
\usepackage{bm}
\usepackage{cite}
\usepackage{color}
\usepackage{float}
\usepackage[left=2cm,right=2cm,top=2cm,bottom=2cm]{geometry}
\usepackage{graphicx}
\usepackage[utf8]{inputenc}
\usepackage{mathtools}
\usepackage{mdframed}
\usepackage{pgfplots} 
\usepackage{subfigure}
\usepackage{stmaryrd}
\usepackage{textcomp}
\usepackage{tikz}
\usepackage{url}
\renewcommand{\proofname}{Proof}
\theoremstyle{plain}
\newtheorem{monTheoNumrote}{Théorème}[section] % Environnement numéroté en fonction de la section
\newtheorem*{monTheoNonNumerote}{Théorème}  % Environnement non numéroté
\newtheorem{The}{Theorem}[section]
\newtheorem*{The*}{Theorem}
\newtheorem{Prop}{Proposition}[section]
\newtheorem*{Prop*}{Proposition} 
\newtheorem{Cor}{Corollary}[section]
\newtheorem*{Cor*}{Corollary}
\newtheorem{Conj}{Conjecture}[section]
\newtheorem{Lem}{Lemma}[section]
\renewcommand{\qed}{\unskip\nobreak\quad\qedsymbol}%
\numberwithin{equation}{section} % Numérote les équations section.numéro.
\theoremstyle{definition}
\newtheorem{Def}{Definition}[section]
\newtheorem{Rem}{Remark}[section]
\newtheorem*{Rem*}{Remark}
\newtheorem*{Lem*}{Lemma}
\newtheorem{Que}{Question}
\newcommand{\enstq}[2]{\left\{#1\mathrel{}\middle|\mathrel{}#2\right\}}
\newcommand{\Lp}[2]{L^#1(#2)}
\newcommand{\Sob}[3]{W^{#1,#2}(#3)}
\newcommand{\Rd}[0]{\mathbb{R}^d}
\newcommand{\RN}[0]{\mathbb{R}^N}
\newcommand{\Rn}[0]{\mathbb{R}^n}
\newcommand{\norm}[1]{\left\|#1\right\|}
\newcommand{\sinc}[0]{\textup{sinc}}
\newcommand{\functionDef}[5]{\begin{array}{lllll}
#1 & : & #2 & \longrightarrow & #3 \\
 & & #4 & \longmapsto &\displaystyle #5 \\
\end{array}}
\newcommand{\Theautorefname}{Theorem}
\newcommand{\Propautorefname}{Proposition}
\newcommand{\Corautorefname}{Corollary}
\newcommand{\Lemautorefname}{Lemma}
\newcommand{\Defautorefname}{Definition}
\newcommand{\N}{\mathbb{N}}
\newcommand{\Z}{\mathbb{Z}}
\newcommand{\D}{\mathbb{D}}
\newcommand{\R}{\mathbb{R}}
\newcommand{\A}{\mathcal{A}_{a,b}}
\newcommand{\Crad}{C^\infty_{c,rad}(B)}
\newcommand{\Lrad}{L^2_{rad}(B)}
\newcommand{\Lradab}{L^2_{rad}(\mathcal{A}_{a,b})}
\newcommand{\duality}[2]{\left\langle #1,#2\right\rangle}
\newcommand{\Hrad}{H^1_{rad}(B)}
\newcommand{\Hzrad}{H^1_{0,rad}(B)}
\newcommand{\rmin}{\delta_{\min}}
\newcommand{\rmax}{\delta_{\max}}
\newcommand{\corr}{\gamma}
\newcommand{\question}[1]{\begin{Que} \ 
#1
\end{Que}}
\newcommand{\abs}[1]{\left\lvert #1 \right\rvert}
\newcommand{\CL}[2]{\textup{CL}\left(\enstq{#1}{#2}\right)}
\newcommand{\Script}[1]{`\texttt{#1}`}
\newcommand{\espace}{\text{ }\qquad} 
\newcommand{\loc}{\text{loc}}
\newcommand{\SL}{\textup{SL}\hspace{1.5pt}}
\newcommand{\DL}{\textup{DL}\hspace{1.5pt}}
\newcommand{\fp}{\underset{\varepsilon \to 0}{\textup{f.p.}}}
\newcommand{\scalProd}[2]{\left(#1|#2\right)}
\newcommand{\toDo}[1]{{\color{red}#1}}
\newcommand{\bs}[1]{\boldsymbol{#1}}
\newcommand{\varInRange}[4]{(#1_{#2})_{#3 \leq #2 \leq #4}}
\newcommand{\from}{\colon}
\newcommand{\Cinf}{C^{\infty}}
\newcommand{\isdef}{\mathrel{\mathop:}=}
\newcommand{\defis}{=\mathrel{\mathop:}}

\renewcommand{\algorithmicrequire}{\textbf{Inputs:}}
\renewcommand{\algorithmicensure}{\textbf{Outputs:}}

\pgfplotsset{compat=1.13}}{
\usepackage{adjustbox}
\usepackage{algorithm}
\usepackage{algorithmic}
\usepackage{amsmath}
\usepackage{amssymb}
\usepackage{amsthm}
\usepackage{amsfonts}
\usepackage{afterpage}
\usepackage{blindtext}
\usepackage[font=footnotesize,labelfont=bf]{caption}
\usepackage{hyperref}
\usepackage[english]{babel}
\usepackage{bbm}
\usepackage{bigints}
\usepackage{bm}
\usepackage{cite}
\usepackage{color}
\usepackage{float}
\usepackage[left=2cm,right=2cm,top=2cm,bottom=2cm]{geometry}
\usepackage{graphicx}
\usepackage[utf8]{inputenc}
\usepackage{mathtools}
\usepackage{mdframed}
\usepackage{pgfplots} 
\usepackage{subfigure}
\usepackage{stmaryrd}
\usepackage{textcomp}
\usepackage{tikz}
\usepackage{url}
\renewcommand{\proofname}{Proof}
\theoremstyle{plain}
\newtheorem{monTheoNumrote}{Théorème}[section] % Environnement numéroté en fonction de la section
\newtheorem*{monTheoNonNumerote}{Théorème}  % Environnement non numéroté
\newtheorem{The}{Theorem}[section]
\newtheorem*{The*}{Theorem}
\newtheorem{Prop}{Proposition}[section]
\newtheorem*{Prop*}{Proposition} 
\newtheorem{Cor}{Corollary}[section]
\newtheorem*{Cor*}{Corollary}
\newtheorem{Conj}{Conjecture}[section]
\newtheorem{Lem}{Lemma}[section]
\renewcommand{\qed}{\unskip\nobreak\quad\qedsymbol}%
\numberwithin{equation}{section} % Numérote les équations section.numéro.
\theoremstyle{definition}
\newtheorem{Def}{Definition}[section]
\newtheorem{Rem}{Remark}[section]
\newtheorem*{Rem*}{Remark}
\newtheorem*{Lem*}{Lemma}
\newtheorem{Que}{Question}
\newcommand{\enstq}[2]{\left\{#1\mathrel{}\middle|\mathrel{}#2\right\}}
\newcommand{\Lp}[2]{L^#1(#2)}
\newcommand{\Sob}[3]{W^{#1,#2}(#3)}
\newcommand{\Rd}[0]{\mathbb{R}^d}
\newcommand{\RN}[0]{\mathbb{R}^N}
\newcommand{\Rn}[0]{\mathbb{R}^n}
\newcommand{\norm}[1]{\left\|#1\right\|}
\newcommand{\sinc}[0]{\textup{sinc}}
\newcommand{\functionDef}[5]{\begin{array}{lllll}
#1 & : & #2 & \longrightarrow & #3 \\
 & & #4 & \longmapsto &\displaystyle #5 \\
\end{array}}
\newcommand{\Theautorefname}{Theorem}
\newcommand{\Propautorefname}{Proposition}
\newcommand{\Corautorefname}{Corollary}
\newcommand{\Lemautorefname}{Lemma}
\newcommand{\Defautorefname}{Definition}
\newcommand{\N}{\mathbb{N}}
\newcommand{\Z}{\mathbb{Z}}
\newcommand{\D}{\mathbb{D}}
\newcommand{\R}{\mathbb{R}}
\newcommand{\A}{\mathcal{A}_{a,b}}
\newcommand{\Crad}{C^\infty_{c,rad}(B)}
\newcommand{\Lrad}{L^2_{rad}(B)}
\newcommand{\Lradab}{L^2_{rad}(\mathcal{A}_{a,b})}
\newcommand{\duality}[2]{\left\langle #1,#2\right\rangle}
\newcommand{\Hrad}{H^1_{rad}(B)}
\newcommand{\Hzrad}{H^1_{0,rad}(B)}
\newcommand{\rmin}{\delta_{\min}}
\newcommand{\rmax}{\delta_{\max}}
\newcommand{\corr}{\gamma}
\newcommand{\question}[1]{\begin{Que} \ 
#1
\end{Que}}
\newcommand{\abs}[1]{\left\lvert #1 \right\rvert}
\newcommand{\CL}[2]{\textup{CL}\left(\enstq{#1}{#2}\right)}
\newcommand{\Script}[1]{`\texttt{#1}`}
\newcommand{\espace}{\text{ }\qquad} 
\newcommand{\loc}{\text{loc}}
\newcommand{\SL}{\textup{SL}\hspace{1.5pt}}
\newcommand{\DL}{\textup{DL}\hspace{1.5pt}}
\newcommand{\fp}{\underset{\varepsilon \to 0}{\textup{f.p.}}}
\newcommand{\scalProd}[2]{\left(#1|#2\right)}
\newcommand{\toDo}[1]{{\color{red}#1}}
\newcommand{\bs}[1]{\boldsymbol{#1}}
\newcommand{\varInRange}[4]{(#1_{#2})_{#3 \leq #2 \leq #4}}
\newcommand{\from}{\colon}
\newcommand{\Cinf}{C^{\infty}}
\newcommand{\isdef}{\mathrel{\mathop:}=}
\newcommand{\defis}{=\mathrel{\mathop:}}

\renewcommand{\algorithmicrequire}{\textbf{Inputs:}}
\renewcommand{\algorithmicensure}{\textbf{Outputs:}}

\pgfplotsset{compat=1.13}}

\author{François Alouges, Martin Averseng}
\title{Sparse Bessel Decomposition for fast 2D non-uniform convolution}

\begin{document}
\maketitle
	
	
\abstract{We describe an efficient algorithm for computing the matrix vector products that appear in the numerical resolution of boundary integral equations in 2 space dimension. This work is an extension of the so-called Sparse Cardinal Sine Decomposition algorithm proposed in \cite{Alouges2015}, which is restricted to three-dimensional setups. Although the approach is analog, significant differences appear throughout the analysis of the method. This leads, in particular, to longer series for the same accuracy, with a slight increase in the global complexity, which remains nevertheless sub-quadratic.}

\section{Introduction}
The Boundary Element Method requires the resolution of fully populated linear systems $Au = b$. As the number of unknowns gets large, the storage of the matrix $A$ and the computational cost for solving the system through direct methods (e.g. LU factorization) become prohibitive. Instead, iterative methods can be used, which require very fast evaluations of matrix vector products. In the context of boundary integral formulations, this takes the form of discrete convolutions:
\begin{equation}
	q_k = \sum_{l=1}^{N_z} G(\boldsymbol{z}_k - \boldsymbol{z}_l) f_l, \quad k \in \left\{1, \cdots, N_z\right\}.
	\label{discreteConv}					
\end{equation}
Here, $G$ is the Green's kernel of the partial differential equation being solved, $\bs{z} = \varInRange{\bs{z}}{k}{1}{N_z}$  is a set of points in $\mathbb{R}^2$, with diameter $\rmax$, $\abs{\cdot}$ is the Euclidian norm and $f = \varInRange{f}{k}{1}{N_z}$ is a vector (typically the values of a function at the points $\bs{z}_k$). For example, the resolution of the Laplace equation with Dirichlet boundary conditions leads to \eqref{discreteConv} with $G(\bs{x}) = \log \abs{\bs{x}}$ (the kernel of the single layer potential). 

In principle, the effective computation of the $(q_k)_{1 \leq k\leq N_z}$ using \eqref{discreteConv} requires $O(N_z^2)$ operations. However, several more efficient algorithms have emerged to compute an approximation of \eqref{discreteConv} with only quasilinear complexity in $N_z$. Among those are the celebrated Fast Multipole Method (see for example \cite{greengard1988rapid,rokhlin1990rapid, rokhlin1993diagonal, coifman1993fast, cheng1999fast} and references therein), the Hierarchical Matrix \cite{borm2003introduction}, and more recently, the Sparse Cardinal Sine Decomposition (SCSD) \cite{Alouges2015}.

One of the key ingredients in all those methods consists in writing the following local variable separation:
\[G(\bs{z}_k- \bs{z}_l) \approx \sum_j \lambda_j G^j_1(\bs{z}_k)G^j_2(\bs{z}_l),\] 
which needs to be valid for $\bs{z}_k$ and $\bs{z}_l$ distant from each other, and up to a controlled accuracy. This eventually results in compressed matrix representations and accelerated matrix-vector product. Notice that, to be fully effective, the former separation is made locally with the help of a geometrical splitting of the cloud of points $\bs{z}$ (octree).

Here we present an alternative compression and acceleration technique, which we call the Sparse Bessel Decomposition (SBD). This is an extension of the SCSD adapted to 2-dimensional problems. The SBD and SCSD methods achieve performances almost comparable to the aforementioned algorithms, they are flexible with respect to the kernel $G$, and do not rely on the construction of octrees, which makes them easier to implement. In addition, they express in an elegant way the intuition according to which a discrete convolution is nothing but a product of Fourier spectra. 

The method heavily relies on the Non Uniform Fast Fourier Transform (NUFFT) of type III (see the seminal paper \cite{NuFFT} and also refer to \cite{greengard2004accelerating,poplau2006calculation} and references therein for numerical aspects and open source codes).  The NUFFT is a fast algorithm, which we denote by $\operatorname{NUFFT}_\pm[\bs{z},\boldsymbol{\xi}](\alpha)$ for later use, that returns, for arbitrary sets of points $\bs{z},\bs{\xi}$ in $\mathbb{R}^2$ and a complex vector $\alpha$, the vector $q$ defined by:
\[ q_\nu = \sum_{k = 1}^{N_z} e^{\pm i \bs{z}_k \cdot \boldsymbol{\xi}_\nu} \alpha_k, \quad \nu \in \left\{1,\cdots,N_\xi \right\}.\]
This algorithm generalizes the classical Fast Fourier Transform (FFT) \cite{cooley1965algorithm}, (in fact, Gauss himself had found an equally powerful algorithm for computing Fourier transforms, see the interesting paper \cite{gaussFFT}), to nonequispaced data, preserving the quasi-linear complexity in $N_{z,\xi} \isdef \max(N_z,N_\xi)$.
The SBD method first produces a discrete and sparse approximation of the spectrum of $G$,
\begin{equation}
	\label{Gapprox}
	G(\bs{x}) \approx G_{\text{approx}}(\boldsymbol{x}) \isdef \sum_{\nu=1}^{N_\xi} e^{i  \boldsymbol{x}\cdot \boldsymbol{\xi}_\nu} \hat{\omega}_\nu, \quad \abs{\bs{x}} \leq \rmax.
\end{equation}
This approximation is replaced in \eqref{discreteConv} to yield 
\begin{equation}
	\begin{split}	q_{k} &\approx \left(\sum_{\nu = 1}^{N_\xi} e^{+i  \bs{z}_k  \cdot\boldsymbol{\xi}_\nu } \left[\hat{\omega}_\nu \sum_{l=1}^{N_z} e^{- i \bs{z}_l \cdot \boldsymbol{\xi}_\nu} f_l) \right]\right)_{1 \leq k \leq N_z}\\
		&= \operatorname{NuFFT}_+[\bs{z},\boldsymbol{\xi}]\left(\hat{\omega} \odot \operatorname{NuFFT}_-[\bs{z},\boldsymbol{\xi}]\big(f\big)\right).
	\end{split}
	\label{far convolution}					
\end{equation}
(Here, $\odot$ denotes the elementwise product between vectors.) The decomposition \eqref{Gapprox} is obtained "off-line" and depends on the value of $\rmax$, but is independent of the choice of the vector $f$ and can thus be used for any evaluation of the matrix vector product \eqref{discreteConv}. 
The approximation \eqref{far convolution} reduces the complexity from $O(N_z^2)$ to $O(N_{z,\xi}\log (N_{z,\xi}))$, stemming from the NUFFT complexity.

The NUFFT has already been used in the literature for the fast evaluation of quantities of the form \eqref{discreteConv}. In particular, our algorithm shares many ideas with the approach in \cite{potts2004fast}. The method presented therein also relies on an approximation of the form \eqref{Gapprox}. However, we choose the set of frequencies $\bs{\xi}$ in a different way, leading to a sparser representation of $G$ (see Remark \autoref{RemarqueQuiTuePotts}). 
 
The kernel $G$ is usually singular near the origin, and consequently the approximation can only be accurate for $|\bs{z}|$ above some threshold $\rmin$. Part of the SBD algorithm is thus dedicated to computing a local correction using a sparse matrix product to account for the closer interactions. This threshold must be chosen so as to balance the time spent for computing the far \eqref{far convolution} and those close contributions. As a matter of fact, we shall prove the following:

\begin{The} Assume the points $\bs{z}$ are uniformly distributed on a curve, and $G(\bs{x}) = \log \abs{\bs{x}}$. Let $\varepsilon > 0$ the desired accuracy of the method, and assume $N_z > 2\abs{\log{\varepsilon}}$. Fix 
	\[a =\dfrac{\abs{\log\varepsilon}^{2/3}}{N_z^{1/2+ \alpha}}\]
	for some $\alpha \in \left[0,\frac{1}{6}\right]$, and let 
	\[\rmin = a \rmax.\] 
	Then:
	\label{The:GlobalComplexity}
	\begin{itemize}
		\item[(i)]  The number of operations needed for the computation of representation \eqref{Gapprox} is bounded by
		      \[ C_{\textup{off}}(N_z,\varepsilon,\alpha) \leq C \abs{\log \varepsilon} N_z^{2 - 3\alpha},\]
		      for some constant $C$ independent of $\varepsilon$, $N_z$ and $\alpha$. 
		\item[(ii)] Once the off-line computations have been done, \eqref{discreteConv} is evaluated for any choice of vector $f$ at a precision at least $\varepsilon \displaystyle\sum_l \abs{f_l}$ with a number of operations bounded by
		      \[C_{\textup{on}}(N_z,\varepsilon,\alpha) \leq C \abs{\log\varepsilon}^{2/3} C_{\textup{NUFFT}}(\varepsilon) N_z ^{4/3 + \alpha} \log(N_z),\] 
		      where $C_{\textup{NUFFT}}(\varepsilon)$ represents the complexity factor of the Type III \textup{NUFFT} related to the target tolerance. 
	\end{itemize} 
\end{The}

The next section gives a brief overview of our algorithm before treating each steps in full details. We will then prove \autoref{The:GlobalComplexity} using the following steps. We will first introduce the Fourier-Bessel series (\autoref{sec:FourierBesselSeries}). In \autoref{sec:SBD} we present the Sparse Bessel Decomposition, and provide some numerical analysis. In section \autoref{sec:ApplicationLaplace}, we apply this method to the Laplace kernel, and give estimates for the number of terms to reach a fixed accuracy. We will then show how such decompositions lead to an approximation of the form \eqref{Gapprox} in \autoref{sec:circQuad}. In \autoref{sec:choiceOfa}, we detail the choice of $\rmin$ leading to the estimation of complexity stated in the previous theorem. We conclude with some numerical examples of accelerated matrix-vector products \eqref{discreteConv}. 



%\toDo{En parler plus tard}
%The choice $\lambda = 1$ in the previous theorem yields the minimal complexity for on-line computations. This is the appropriate choice when \eqref{discreteConv} needs to be evaluated a lot of times for different $\phi$. In the case where \eqref{discreteConv} only has to be computed a few times, $\lambda$ can be chosen so as to minimize the total computation time (off-line $+$ on-line), that is $\lambda = \frac{N_z^{1/6}}{C_{\textup{NuFFT}(\varepsilon)}^{1/4}}$, yielding a complexity of $O\left( \abs{\log(\varepsilon)}^{3/4}C_{\textup{NuFFT}}(\varepsilon)^{3/4}N_z^{3/2}\log(N_z)\right)$. 

\section{Summary of the algorithm}
\label{sec:overview}

The SBD algorithm can be summarized as follows:

\subsection*{Off-line part}
\begin{itemize}
	\item[]\textbf{Inputs:} A radially symmetric kernel $G$, a set of $N_z$ points $\bs{z}$ in $\mathbb{R}^2$ of diameter $\rmax$, a value for the parameter $\rmin$, a tolerance $\varepsilon > 0$.
	\item[]\textbf{Sparse spectrum sampling:} Compute a set of $N_\xi$ complex weights $\hat{\omega}$ and $N_\xi$ frequencies $\boldsymbol{\xi}$ so that \eqref{Gapprox} is valid for $\rmin \leq |\boldsymbol{x}| \leq \rmax$ up to tolerance $\varepsilon$. 
	\item[]\textbf{Correction Matrix:} Determine the set $\mathcal{P}$ of all pairs $(k,l)$ such that $\abs{\bs{z}_k - \bs{z}_l} \leq \delta_{\min}$. Assemble the close correction sparse matrix:
	      \[B_{kl} = \delta_{(k,l) \in \mathcal{P}} \left( G({\bs{z}_k - \bs{z}_l}) - \sum_{\nu = 1}^{N_{\xi}}e^{i (\bs{z}_k - \bs{z}_l)\cdot \boldsymbol{\xi}_\nu} \hat{\omega}_\nu\right).\]
	\item[] \textbf{Outputs}: The set of weights $\hat{\omega}$, the frequencies $\boldsymbol{\xi}$ and the sparse matrix $B$. 
\end{itemize}
	
\subsection*{On-line part}
	
\begin{itemize}
	\item[] \textbf{Input:} All outputs of the off-line part, and a complex vector $f$ of size $N_z$. 
	\item[] \textbf{Far approximation:} Compute, for all $k$, 
	      \[ q^{\text{far}} = \sum_{l=1}^{N_z} G_{\textup{approx}}(\bs{z}_k - \bs{z}_l) f_l.\]
	      For this, follow three steps
	      \begin{itemize}
	      	\item[(i)] \textbf{Space $\rightarrow$ Fourier: } Compute $\hat{f} = \textup{NUFFT}_-[\bs{z},\boldsymbol{\xi}](f).$
	      	\item[(ii)] \textbf{Fourier multiply} Perform elementwise multiplication by $\hat{\omega}$: $\hat{g}_{\nu} = \hat{\omega}_\nu \hat{f_\nu}.$
	      	\item[(iii)] \textbf{Fourier $\rightarrow$ Space: } Compute $q^{\text{far}} =  \textup{NUFFT}_+[\bs{z},\boldsymbol{\xi}](\hat{g}).$
	      \end{itemize}
	\item[] \textbf{Close correction:} Compute the sparse matrix product:
	      \[q^{\textup{close}} = Bf.\]
	\item[] \textbf{Output:} The vector $q = q^{\textup{far}} + q^{\textup{close}}$, with, for any $k \in \left\{1,\cdots,N_z\right\},$	
	      \[ \abs{q_k - \sum_{l = 1}^{N_z} G({\bs{z}_k - \bs{z}_l}) f_l} \leq \varepsilon \sum_{l=1}^{N_z} \abs{f_l}.\]
\end{itemize}

\subsection*{More on sparse spectrum sampling}
The main novelty in our algorithm is the method for producing an approximation of the form \eqref{Gapprox}. We proceed in two steps, uncoupling the radial and circular coordinates. 

\paragraph{-Sparse Bessel Decomposition:} In a first part, we compute a Bessel series approximating $G(r)$ on $[\rmin,\rmax]$, which we call a Sparse Bessel Decomposition. It takes the form
\[G(r) \approx G(\rmax) + \sum_{p=1}^P \alpha_p J_0(\rmax \rho_p r), \quad r \in [\rmin,\rmax],\]
where $J_0$ is the Bessel function of first kind and order zero and $(\rho_p)_{p \in \N*}$ is the sequence of its roots (see \autoref{defJ0} for more details). Here is how we compute the coefficients $\varInRange{\alpha}{p}{1}{P}$. We choose a starting value for $P$ and compute the weights $\alpha_1,\cdots, \alpha_{P}$ that minimize
\[\bigintsss_{\rmin \leq |\bs{x}|\leq \rmax} \abs{\nabla \left(G(\bs{x}) - \sum_{p=1}^P \alpha_p J_0(\rho_p \rmax \abs{\bs{x}})\right)}^2 d\bs{x}.\]
This amounts to solving a linear system. We keep increasing $P$ until the residual error goes below the required tolerance. To choose the stopping criterion, we suggest monitoring the $L^{\infty}$ error near $\rmin$ where it is usually the highest. 

To keep the number of coefficients $\alpha_p$ as low as possible, we have to compute them for increasing values of $P$ until the accuracy is reached. A priori, all coefficients must be recomputed for each new value of $P$. When the initial guess for $P$ is far from the target, this can severely increase the computation time. To avoid this issue, we use a Gram-Schmidt process detailed in \autoref{Cholesky}.

\paragraph{-Circular quadrature:} In a second step, we use approximations of the form:
\[J_0(\rho_p \abs{\bs{x}}) \approx \dfrac{1}{M_p}\sum_{m=0}^{M_p-1}e^{i \rho_p \bs{\xi}_p^m \cdot \bs{x}}, \quad p \in \{1,\cdots,P\}\]
which are discrete versions of the identity:
\[ J_0(\rho_p\abs{\bs{x}}) = \frac{1}{2\pi}\int_{\abs{\bs{\xi}}=1}{e^{i \rho_p\bs{x} \cdot \bs{\xi}}} d\sigma(\bs{\xi}).\]
We sum them to eventually obtain the formula \eqref{Gapprox}. 
\begin{Rem}
	In the SCSD method \cite{Alouges2015}, the Bessel functions are replaced by cardinal sine functions, since, for $\bs{x} \in \R^3$
	\[ \frac{1}{4\pi}\int_{\abs{\bs{\xi}}=1} e^{i \bs{x}\cdot \bs{\xi}} d\sigma(\bs{\xi}) = \sinc(\abs{\bs{x}}),\]
	where the integral is now taken over $\mathcal{S}^2 \subset \R^3$.
\end{Rem}


\section{Series of Bessel functions and error estimates}
\label{sec:FourierBesselSeries}
In this section, we give a short introduction to Fourier-Bessel series. A reference on this topic is \cite{watson1995treatise}, chapter XVIII. 
The main result needed for our purpose is \autoref{DecroissanceFourierBessel},  an equivalent statement of which can be found in to Theorem $1$ in \cite{tolstov2012fourier} chapter 8, section 20. 

In \autoref{FunctionalFramework}, we provide a self-contained proof for the characterization of the Laplacian eigenfunctions with Dirichlet conditions on the unit ball. The results we need for the subsequent analysis are mainly estimations \eqref{EncadrementCp}, \autoref{EncadrementRhop}, and \autoref{epEstUneBaseDeHilbert}). Those are classical results, and the reader already familiar with them may skip this first subsection. 
The way we define the Fourier-Bessel series is closer to \cite{wolf2013integral} (chapter 6). That is, we keep the analysis in two variables but with radial symmetry, rather than going to one variable. We prefer this approach because it does not depend on the space dimension. In $\R^3$, for instance, the radial eigenvalues of the Laplacian are proportional to
\[  \bs{x} \mapsto \dfrac{J_{1/2}(2\pi p\abs{\bs{x}})}{\bs{|x|}^{1/2}}, \quad  p\in\N^*.\] 
Therefore, our approach generalizes \cite{Alouges2015} to any dimension.  
\subsection{Radial eigenvalues of the Laplace operator with Dirichlet conditions}
\label{FunctionalFramework}
Let $J_0$ be the Bessel function of first kind (see \autoref{defJ0}), and $(\rho_p)_{p \in \N*}$ the sequence of its roots. 
The aim of this paragraph is to show that the functions 
\[ \bs{x} \mapsto J_0(\rho_p \abs{\bs{x}})\]
can be used to construct a Hilbert basis of a space of radial functions (see \autoref{epEstUneBaseDeHilbert}).

In the following, $B$ denotes the unit ball in $\mathbb{R}^2$, and $\mathcal{C}$ its boundary. We say that a function $u\from\mathbb{R}^2\rightarrow \mathbb{R}$ is radial if there exists $\tilde{u}\from\mathbb{R}^+\to \R$ such that for any $\bs{z} \in \R^2$, 
\[ u(\bs{x}) = \tilde{u}(\abs{\bs{x}}).\] 
We will drop the $\texttildelow$ superscript and use the abusive notation $u(r)$ for $u(\bs{x})$ when $\abs{\bs{x}} = r$. With this notation, in particular, integrals of radial functions on $B$ reduce to 1-D integrals:
\begin{equation}
	\int_{B} u(\bs{x})d\bs{x} = 2\pi \int_0^1 r u(r) dr.
\end{equation}
We let $\Lrad$ the subspace of radial functions that are square integrable on $B$. $\Lrad$ is a closed subspace of $L^2(B)$ thus it is a Hilbert space associated with the usual $L^2$ scalar product. 
Let 
\[\Crad = \enstq{\varphi \in \Cinf_c(B)}{\varphi \text{ is radial }}\]
the set of smooth compactly supported radial functions. Moreover, let
\[\Hrad = \enstq{ u \in \Lrad}{ \forall 1 \leq i \leq 2, \dfrac{\partial u}{\partial x_i} \in L^2(B)}.\]
This is a Hilbert space with the scalar product
\[\duality{u}{v}_{\Hrad} = \int_{B} u(\bs{x})v(\bs{x}) + \nabla u(\bs{x}) \cdot \nabla v(\bs{x}) dx.\]
Finally, let $\Hzrad$ the closure of $\Crad$ in $\Hrad$. 

Let us introduce the radial averaging operator, also studied in \cite{SphericalAverage}:  
\begin{Lem} \label{RadialAveraging}
	Let $\textup{Rad}\from L^2(B) \to \Lrad$ be defined by:
	\[{\mathop\textup{Rad }} v(r) = \frac{1}{2\pi} \int_{\mathcal{C}} v(r\bs{u})d\sigma(\bs{u}),\]
	where $\sigma$ is the usual uniform measure on $\mathcal{C}$. Then, 
	\item[(i)] For any $u \in \Lrad$ and $v \in L^2(B)$,
	\[ \int_{B} u(\bs{x})v(\bs{x})d\bs{x} = \int_B u(\bs{x}){\mathop\textup{Rad }} v(\bs{x})d\bs{x}.\]
	\item[(ii)] If $v \in C^{2}(B),$
	\[ \Delta\textup{Rad }v   = \textup{Rad }\Delta v.\]
	\begin{proof}
		To prove (i), write
		\begin{equation}
			\begin{split}
				\int_{B}u(\bs{x}) v(\bs{x}) d\bs{x} &= \int_{0}^{1} \int_{\mathcal{C}}r u(r \bs{u}) v(r\bs{u})d\sigma(\bs{u})dr\\
				&=\int_{0}^1 r u(r)\left( \int_{\mathcal{C}} v(r\bs{u})d\sigma(\bs{u})\right) dr\\
				&=2\pi \int_{0}^{1} r u(r) \textup{Rad }v(r) dr\\
			\end{split}
		\end{equation}	
		For (ii), writing $v(\bs{x}) = v(r,\theta)$ with $(r,\theta)$ the standard polar coordinates, we have
		\[ \Delta v = \dfrac{\partial^2v}{\partial r^2} + \dfrac{1}{r} \dfrac{\partial v}{\partial r} + \dfrac{1}{r^2}\dfrac{\partial^2 v}{\partial \theta^2},\]
		therefore, 
		\begin{equation}
			\begin{split}
				\textup{Rad } \Delta v &= \int_{0}^{2\pi} \dfrac{\partial^2v}{\partial r^2} + \dfrac{1}{r} \dfrac{\partial v}{\partial r} + \dfrac{1}{r^2}\dfrac{\partial^2 v}{\partial \theta^2} d\theta\\
				&= \dfrac{\partial^2}{\partial r^2} \textup{Rad } v + \dfrac{1}{r}\dfrac{\partial }{\partial r} \textup{Rad }v \\
				&= \Delta \textup{Rad } v.
			\end{split}
		\end{equation}
		Indeed, the term $\dfrac{1}{r^2}\displaystyle\int_{0}^{2\pi} \dfrac{\partial^2 v}{\partial \theta^2} d\theta$ vanishes since $v$ is periodic in $\theta$. \qedhere
	\end{proof}
\end{Lem}

The standard Sobolev results apply to radial functions:

\begin{Prop} \text{ }\label{SobolevRadialResults}
	\begin{itemize}
		\item[(i)] $\Crad$ is a dense subspace of $\Lrad$. 
		\item[(ii)] The canonical injection $\Hzrad \xhookrightarrow{} \Lrad$ is compact.
		\item[(iii)] There exists a constant $C > 0$ such that for any $u \in \Hzrad$
		      \begin{equation}
		      	\label{inegaliteDePoincare}
		      	\int_{B}\abs{u(\bs{x})}^2dx \leq C \int_B\abs{\nabla u (\bs{x})}^2d\bs{x}.
		      \end{equation}
	\end{itemize}
\end{Prop}
\noindent Therefore, 
\[ \norm{u}_{\Hzrad} \isdef \sqrt{\int_B \abs{\nabla u}^2}\]
defines a norm on $\Hzrad$ equivalent to $\norm{\cdot}_{H^1(B)}$.
\begin{proof}
	Only (i) needs a proof, the other two results are just straightforward consequences of the analogous results of classical Sobolev spaces theory.
	Let $f \in \Lrad$ and $\varepsilon >0$. We know that $\mathcal{C}_c^\infty(B)$ is dense in $L^2(B)$ thus there exists a function $\chi \in \mathcal{C}_c^\infty(B)$ such that $\norm{f - \chi}_{L^2(B)} \leq \varepsilon$. Let $\chi_r = \textup{Rad }\chi$. It can be shown (see \cite{SphericalAverage}) that $\chi_r \in \Crad$.
	For any $r \in (0,1)$: 
	\[\abs{f(r) - \chi_r(r)}^2 = \abs{\frac{1}{2\pi} \int_{\mathcal{C}} \left[f(r\bs{u}) - \chi(r \bs{u})\right]d\sigma(\bs{u})}^2 \]
	by Jensen inequality, 
	\[\left|f(r) - \chi_r(r)\right|^2 \leq \frac{1}{2\pi} \int_{\mathcal{C}}\left|f(r\bs{u}) - \chi(r\bs{u}) \right|^2 d\sigma({\bs{u}}).\]
	Thus, \[ 2\pi \int_{0}^1 r \abs{ f(r) - \chi_r(r)}^2 dr \leq \int_{0}^1 \int_{\mathcal{C}}\left| f(r\bs{u}) - \varphi(r\bs{u})\right|^2 r dr d\sigma(\bs{u}),\]
	which proves that $\norm{f - \chi_r}_{\Lrad} \leq \varepsilon$.\qedhere
\end{proof}
We now briefly recall some facts on Bessel functions. All the results that we will be using on this topic can be found in the comprehensive book \cite{abramowitz1964handbook} of Abramowitz and Stegun. 
\begin{Def}
	\label{defJ0}
	The Bessel function of the first kind and order $\alpha$, $J_\alpha$ is defined by the following series: 
	\[J_\alpha(r) \isdef \sum_{m=0}^\infty \frac{(-1)^m}{m! \, \Gamma(m+1+\alpha)} {\bigg(\frac{r}{2}\bigg)}^{2m+\alpha}.\]
	When $\alpha=0$, we get a $\Cinf$ solution of Bessel's differential equation 
	\begin{equation}
		\label{BesselDifferentialEquation}
		r^2f''(r) + r f'(r) + r^2 f(r) = 0.
	\end{equation}
	The roots $(\rho_p)_{p \in \N^*}$ of $J_0$, behave, for large $p$, as 
	\[ \rho_p \underset{p \to \infty}{\sim} \pi p.\]
	More precisely, $\rho_p = \pi p - \frac{\pi}{4} + O\left(\frac{1}{p}\right)$. It is well-known that, for any $p$, 
	\begin{equation}
		\pi(p - 1/4)\leq \rho_p \leq \pi(p - 1/8)
		\label{EncadrementRhop}
	\end{equation}.
\end{Def}

For any $p\in \N*$, we introduce:
\[e_p(\bs{x}) = C_p J_0(\rho_p \abs{\bs{x}}),\]
where the constant $C_p$ is chosen such that $\norm{e_p}_{\Hrad} = 1$, that is 
\[C_p = \frac{1}{\left(\int_{B} \abs{\nabla e_p}^2\right)^{1/2}}.\]
Explicitly, 
\[C_p = \dfrac{1}{\sqrt{\pi}\rho_p\abs{J_0'(\rho_p)}},\]
One can check, using asymptotic expansions of Bessel functions, that
\begin{equation}
	\label{equivalentCp}
	C_p = \dfrac{1}{\sqrt{2 \pi p}} + O\left(\frac{1}{p^{3/2}}\right), 
\end{equation}
Moreover, we observed numerically that:
\begin{equation}
	\label{EncadrementCp}
	\frac{1}{\sqrt{2\pi p}} \leq C_p \leq \frac{1}{\sqrt{2\pi (p-1/4)}}.
\end{equation}
The first part can be established with asymptotic expansions of $J_1$. The second inequality however, seems difficult to prove. The first $1000$ terms are shown in \autoref{figure:encadrementCp}\\
\begin{figure}[H]
	\centering
	% This file was created by matlab2tikz.
%
%The latest updates can be retrieved from
%  http://www.mathworks.com/matlabcentral/fileexchange/22022-matlab2tikz-matlab2tikz
%where you can also make suggestions and rate matlab2tikz.
%
\definecolor{mycolor1}{rgb}{0.00000,0.44700,0.74100}%
\definecolor{mycolor2}{rgb}{0.85000,0.32500,0.09800}%
%
\begin{tikzpicture}[%
trim axis left, trim axis right
]

\begin{axis}[%
width=0.497\textwidth,
height=5cm,
at={(0\textwidth,0cm)},
scale only axis,
xmode=log,
xmin=1,
xmax=1000,
xtick={  10,  100, 1000},
xminorticks=true,
xlabel style={font=\color{white!15!black}},
xlabel={$p$},
ymin=-0.0189948529191399,
ymax=0.132767171487086,
axis background/.style={fill=white}
]
\addplot [color=mycolor1, draw=none, mark size=1.2pt, mark=*, mark options={solid, mycolor1}, forget plot]
  table[row sep=crcr]{%
1	0.132767171487086\\
2	0.0647992584224122\\
3	0.0427467943219646\\
4	0.0318741938710847\\
5	0.0254058687817951\\
6	0.0211181335940633\\
7	0.0180679247448419\\
8	0.0157872615934254\\
9	0.0140176357160373\\
10	0.0126046339596677\\
11	0.0114503461375337\\
12	0.010489691458639\\
13	0.00967772529463251\\
14	0.00898241169652203\\
15	0.00838030002370815\\
16	0.00785382987845828\\
17	0.00738959119599603\\
18	0.00697716661640846\\
19	0.00660834082139972\\
20	0.00627654794136312\\
21	0.00597647740387686\\
22	0.00570378764364676\\
23	0.00545489474163086\\
24	0.0052268140715106\\
25	0.00501704006712922\\
26	0.00482345381787241\\
27	0.00464425125712475\\
28	0.00447788678154271\\
29	0.00432302856663958\\
30	0.00417852284254994\\
31	0.00404336510167203\\
32	0.00391667671816132\\
33	0.00379768582860285\\
34	0.003685711594553\\
35	0.00358015116904253\\
36	0.00348046884014952\\
37	0.00338618693886539\\
38	0.0032968781855891\\
39	0.00321215921649975\\
40	0.00313168508293216\\
41	0.00305514455729416\\
42	0.00298225611086544\\
43	0.00291276445387578\\
44	0.00284643754826952\\
45	0.00278306401948192\\
46	0.00272245090640411\\
47	0.00266442169908276\\
48	0.00260881462212814\\
49	0.00255548112866411\\
50	0.00250428457529295\\
51	0.00245509905321017\\
52	0.00240780835438992\\
53	0.0023623050549928\\
54	0.00231848970077042\\
55	0.00227627008146847\\
56	0.00223556058308616\\
57	0.00219628160841534\\
58	0.00215835905761108\\
59	0.00212172386165932\\
60	0.00208631156255934\\
61	0.00205206193485963\\
62	0.00201891864386239\\
63	0.00198682893642976\\
64	0.00195574336080684\\
65	0.0019256155123486\\
66	0.00189640180238215\\
67	0.00186806124780037\\
68	0.00184055527924198\\
69	0.00181384756596947\\
70	0.00178790385577821\\
71	0.00176269182844679\\
72	0.00173818096141631\\
73	0.0017143424065249\\
74	0.0016911488767406\\
75	0.001668574541972\\
76	0.0016465949331117\\
77	0.00162518685356372\\
78	0.00160432829758506\\
79	0.00158399837483536\\
80	0.00156417724059166\\
81	0.00154484603113647\\
82	0.00152598680387817\\
83	0.00150758248180893\\
84	0.00148961680192716\\
85	0.00147207426730778\\
86	0.00145494010252212\\
87	0.00143820021212782\\
88	0.00142184114200128\\
89	0.00140585004327143\\
90	0.00139021463866507\\
91	0.00137492319106869\\
92	0.00135996447414732\\
93	0.00134532774485896\\
94	0.00133100271772224\\
95	0.00131697954071464\\
96	0.00130324877267562\\
97	0.00128980136211188\\
98	0.00127662862729605\\
99	0.00126372223757842\\
100	0.00125107419581649\\
101	0.00123867682185308\\
102	0.00122652273696189\\
103	0.0012146048492061\\
104	0.00120291633963765\\
105	0.00119145064928761\\
106	0.00118020146689179\\
107	0.00116916271730827\\
108	0.00115832855057918\\
109	0.00114769333158815\\
110	0.0011372516302921\\
111	0.00112699821247353\\
112	0.00111692803099261\\
113	0.00110703621750519\\
114	0.001097318074611\\
115	0.00108776906841968\\
116	0.00107838482149525\\
117	0.00106916110616773\\
118	0.00106009383817884\\
119	0.00105117907065666\\
120	0.00104241298838281\\
121	0.00103379190235175\\
122	0.00102531224459179\\
123	0.00101697056324412\\
124	0.00100876351787593\\
125	0.00100068787502083\\
126	0.000992740503931389\\
127	0.000984918372533228\\
128	0.000977218543565472\\
129	0.00096963817090745\\
130	0.000962174496065948\\
131	0.000954824844834334\\
132	0.00094758662409089\\
133	0.000940457318755694\\
134	0.000933434488872509\\
135	0.000926515766824565\\
136	0.000919698854676243\\
137	0.000912981521625333\\
138	0.000906361601570094\\
139	0.000899836990782443\\
140	0.000893405645675305\\
141	0.000887065580672086\\
142	0.000880814866161206\\
143	0.000874651626538103\\
144	0.00086857403832985\\
145	0.000862580328394591\\
146	0.000856668772198477\\
147	0.000850837692160988\\
148	0.000845085456067984\\
149	0.000839410475548474\\
150	0.000833811204610457\\
151	0.000828286138240486\\
152	0.000822833811051416\\
153	0.000817452795989659\\
154	0.000812141703086189\\
155	0.000806899178264153\\
156	0.000801723902185358\\
157	0.000796614589141598\\
158	0.000791569985993723\\
159	0.000786588871144245\\
160	0.000781670053552119\\
161	0.000776812371782842\\
162	0.000772014693094514\\
163	0.000767275912556986\\
164	0.000762594952204543\\
165	0.000757970760215443\\
166	0.000753402310127438\\
167	0.000748888600077269\\
168	0.000744428652066809\\
169	0.000740021511258071\\
170	0.000735666245289535\\
171	0.000731361943622222\\
172	0.000727107716902209\\
173	0.000722902696348671\\
174	0.00071874603316191\\
175	0.000714636897954701\\
176	0.000710574480196735\\
177	0.000706557987684597\\
178	0.000702586646025072\\
179	0.000698659698139537\\
180	0.00069477640377924\\
181	0.000690936039063228\\
182	0.000687137896028256\\
183	0.000683381282191364\\
184	0.000679665520132655\\
185	0.000675989947082511\\
186	0.000672353914534574\\
187	0.000668756787859159\\
188	0.000665197945935114\\
189	0.000661676780791653\\
190	0.000658192697266413\\
191	0.00065474511266328\\
192	0.00065133345643309\\
193	0.000647957169856772\\
194	0.000644615705738039\\
195	0.000641308528113615\\
196	0.000638035111959701\\
197	0.000634794942919514\\
198	0.000631587517031518\\
199	0.000628412340467399\\
200	0.000625268929280054\\
201	0.000622156809158891\\
202	0.000619075515186474\\
203	0.00061602459161314\\
204	0.000613003591627859\\
205	0.000610012077141509\\
206	0.000607049618574385\\
207	0.000604115794652804\\
208	0.000601210192207935\\
209	0.000598332405979285\\
210	0.000595482038431516\\
211	0.000592658699566817\\
212	0.000589862006749931\\
213	0.00058709158453274\\
214	0.000584347064489066\\
215	0.000581628085049024\\
216	0.000578934291341371\\
217	0.000576265335039183\\
218	0.000573620874208647\\
219	0.000571000573164726\\
220	0.000568404102327058\\
221	0.000565831138083173\\
222	0.000563281362654156\\
223	0.00056075446396342\\
224	0.000558250135508143\\
225	0.000555768076238472\\
226	0.000553307990432295\\
227	0.000550869587582437\\
228	0.000548452582280312\\
229	0.000546056694103125\\
230	0.00054368164750751\\
231	0.000541327171723616\\
232	0.000538993000651189\\
233	0.000536678872759655\\
234	0.000534384530991305\\
235	0.000532109722665819\\
236	0.000529854199386115\\
237	0.000527617716949313\\
238	0.000525400035259027\\
239	0.000523200918237432\\
240	0.000521020133743111\\
241	0.000518857453489119\\
242	0.000516712652964602\\
243	0.000514585511353083\\
244	0.000512475811460744\\
245	0.00051038333964093\\
246	0.000508307885722203\\
247	0.000506249242937296\\
248	0.000504207207855822\\
249	0.000502181580315675\\
250	0.000500172163359514\\
251	0.0004981787631686\\
252	0.000496201189002843\\
253	0.000494239253137296\\
254	0.000492292770806424\\
255	0.00049036156014215\\
256	0.000488445442120344\\
257	0.000486544240502651\\
258	0.000484657781785192\\
259	0.000482785895145499\\
260	0.000480928412387227\\
261	0.000479085167894411\\
262	0.000477255998580173\\
263	0.00047544074383965\\
264	0.000473639245500479\\
265	0.000471851347779273\\
266	0.000470076897237215\\
267	0.000468315742732761\\
268	0.000466567735382339\\
269	0.000464832728517939\\
270	0.000463110577642034\\
271	0.000461401140393392\\
272	0.000459704276503325\\
273	0.000458019847760394\\
274	0.000456347717970429\\
275	0.000454687752921235\\
276	0.000453039820345946\\
277	0.00045140378988906\\
278	0.000449779533070682\\
279	0.000448166923254112\\
280	0.000446565835611201\\
281	0.000444976147091714\\
282	0.000443397736392237\\
283	0.0004418304839231\\
284	0.000440274271780172\\
285	0.000438728983714665\\
286	0.000437194505104044\\
287	0.000435670722923609\\
288	0.000434157525719181\\
289	0.000432654803578902\\
290	0.000431162448109257\\
291	0.000429680352404649\\
292	0.000428208411025865\\
293	0.000426746519973653\\
294	0.000425294576663848\\
295	0.000423852479902287\\
296	0.000422420129865264\\
297	0.000420997428072001\\
298	0.000419584277365326\\
299	0.000418180581886807\\
300	0.000416786247056544\\
301	0.000415401179553632\\
302	0.000414025287291286\\
303	0.000412658479400863\\
304	0.000411300666208314\\
305	0.000409951759215987\\
306	0.000408611671083969\\
307	0.000407280315609881\\
308	0.000405957607712448\\
309	0.00040464346341107\\
310	0.000403337799809167\\
311	0.000402040535076642\\
312	0.000400751588433224\\
313	0.00039947088013137\\
314	0.000398198331438282\\
315	0.000396933864622362\\
316	0.000395677402937\\
317	0.000394428870602814\\
318	0.000393188192794325\\
319	0.000391955295624857\\
320	0.000390730106131221\\
321	0.000389512552259497\\
322	0.000388302562852383\\
323	0.000387100067631874\\
324	0.000385904997189712\\
325	0.000384717282970959\\
326	0.000383536857262889\\
327	0.000382363653181006\\
328	0.000381197604655936\\
329	0.00038003864642322\\
330	0.000378886714007542\\
331	0.000377741743715188\\
332	0.000376603672620046\\
333	0.000375472438549185\\
334	0.000374347980078404\\
335	0.000373230236513811\\
336	0.000372119147886041\\
337	0.000371014654937163\\
338	0.000369916699111128\\
339	0.00036882522254178\\
340	0.000367740168045305\\
341	0.000366661479109354\\
342	0.000365589099879715\\
343	0.000364522975156767\\
344	0.000363463050381485\\
345	0.000362409271627895\\
346	0.000361361585593967\\
347	0.000360319939591403\\
348	0.000359284281537864\\
349	0.000358254559949422\\
350	0.00035723072392857\\
351	0.000356212723159555\\
352	0.000355200507897502\\
353	0.00035419402896264\\
354	0.000353193237729643\\
355	0.000352198086121636\\
356	0.00035120852660242\\
357	0.000350224512168484\\
358	0.000349245996341674\\
359	0.00034827293316142\\
360	0.000347305277177856\\
361	0.000346342983445602\\
362	0.000345386007515769\\
363	0.000344434305428187\\
364	0.000343487833707634\\
365	0.000342546549355172\\
366	0.000341610409839932\\
367	0.000340679373097341\\
368	0.000339753397519127\\
369	0.000338832441947545\\
370	0.000337916465670052\\
371	0.000337005428414194\\
372	0.000336099290339398\\
373	0.000335198012033633\\
374	0.000334301554505645\\
375	0.000333409879179625\\
376	0.000332522947891878\\
377	0.000331640722882609\\
378	0.000330763166792147\\
379	0.000329890242654285\\
380	0.000329021913893168\\
381	0.000328158144316415\\
382	0.000327298898111783\\
383	0.000326444139838067\\
384	0.000325593834427984\\
385	0.000324747947174409\\
386	0.000323906443732369\\
387	0.0003230692901115\\
388	0.000322236452671154\\
389	0.000321407898117743\\
390	0.000320583593498069\\
391	0.000319763506197779\\
392	0.00031894760393314\\
393	0.000318135854751045\\
394	0.000317328227022573\\
395	0.000316524689438769\\
396	0.000315725211005313\\
397	0.000314929761043858\\
398	0.000314138309181144\\
399	0.000313350825350778\\
400	0.000312567279785458\\
401	0.000311787643016093\\
402	0.000311011885865575\\
403	0.00031023997944768\\
404	0.000309471895162838\\
405	0.000308707604690595\\
406	0.000307947079994042\\
407	0.000307190293308501\\
408	0.000306437217142852\\
409	0.000305687824276424\\
410	0.000304942087750337\\
411	0.000304199980871722\\
412	0.000303461477205724\\
413	0.000302726550572396\\
414	0.000301995175046921\\
415	0.000301267324953169\\
416	0.000300542974861928\\
417	0.000299822099587344\\
418	0.000299104674186479\\
419	0.000298390673953097\\
420	0.000297680074416995\\
421	0.000296972851339339\\
422	0.000296268980713554\\
423	0.000295568438757332\\
424	0.000294871201914404\\
425	0.000294177246849658\\
426	0.000293486550447142\\
427	0.000292799089808282\\
428	0.000292114842247004\\
429	0.00029143378529084\\
430	0.000290755896675154\\
431	0.000290081154341593\\
432	0.000289409536436303\\
433	0.000288741021308825\\
434	0.000288075587506542\\
435	0.000287413213774235\\
436	0.000286753879052526\\
437	0.000286097562474774\\
438	0.000285444243365962\\
439	0.00028479390123537\\
440	0.000284146515783235\\
441	0.000283502066892538\\
442	0.000282860534627671\\
443	0.000282221899231994\\
444	0.000281586141130052\\
445	0.000280953240920034\\
446	0.000280323179374653\\
447	0.000279695937439151\\
448	0.000279071496227967\\
449	0.000278449837024741\\
450	0.0002778309412792\\
451	0.000277214790605607\\
452	0.000276601366781204\\
453	0.000275990651743996\\
454	0.000275382627590304\\
455	0.000274777276576099\\
456	0.000274174581110564\\
457	0.000273574523758091\\
458	0.000272977087235615\\
459	0.000272382254409953\\
460	0.000271790008297135\\
461	0.000271200332061294\\
462	0.000270613209011561\\
463	0.000270028622601171\\
464	0.000269446556427466\\
465	0.000268866994226569\\
466	0.000268289919876485\\
467	0.000267715317391337\\
468	0.000267143170924022\\
469	0.000266573464760889\\
470	0.000266006183323286\\
471	0.000265441311163572\\
472	0.000264878832965998\\
473	0.0002643187335436\\
474	0.000263760997839313\\
475	0.000263205610920858\\
476	0.000262652557982745\\
477	0.000262101824342942\\
478	0.000261553395442204\\
479	0.000261007256843415\\
480	0.000260463394229804\\
481	0.000259921793404283\\
482	0.000259382440285894\\
483	0.000258845320910916\\
484	0.000258310421431984\\
485	0.000257777728115194\\
486	0.000257247227339885\\
487	0.000256718905597086\\
488	0.000256192749489292\\
489	0.000255668745727133\\
490	0.000255146881131818\\
491	0.000254627142629804\\
492	0.000254109517255463\\
493	0.000253593992147305\\
494	0.000253080554548646\\
495	0.000252569191806051\\
496	0.00025205989136734\\
497	0.000251552640782027\\
498	0.000251047427699769\\
499	0.000250544239868589\\
500	0.000250043065134431\\
501	0.000249543891441606\\
502	0.000249046706829015\\
503	0.000248551499432592\\
504	0.000248058257478867\\
505	0.000247566969292068\\
506	0.000247077623285685\\
507	0.000246590207965802\\
508	0.000246104711929096\\
509	0.000245621123860618\\
510	0.000245139432536012\\
511	0.000244659626817967\\
512	0.000244181695655543\\
513	0.000243705628084623\\
514	0.000243231413227463\\
515	0.000242759040288476\\
516	0.000242288498556897\\
517	0.00024181977740656\\
518	0.000241352866290789\\
519	0.000240887754744845\\
520	0.000240424432385922\\
521	0.000239962888908929\\
522	0.000239503114089157\\
523	0.000239045097779833\\
524	0.000238588829911235\\
525	0.000238134300490689\\
526	0.000237681499601461\\
527	0.000237230417401868\\
528	0.000236781044125722\\
529	0.000236333370079667\\
530	0.000235887385643618\\
531	0.000235443081269882\\
532	0.000235000447484257\\
533	0.000234559474881157\\
534	0.000234120154126716\\
535	0.000233682475957231\\
536	0.000233246431176504\\
537	0.000232812010659833\\
538	0.000232379205346911\\
539	0.000231948006246929\\
540	0.000231518404435249\\
541	0.000231090391052957\\
542	0.000230663957305977\\
543	0.000230239094466622\\
544	0.000229815793869603\\
545	0.000229394046915132\\
546	0.00022897384506404\\
547	0.00022855517984155\\
548	0.000228138042834614\\
549	0.000227722425690802\\
550	0.000227308320117414\\
551	0.000226895717884812\\
552	0.000226484610820865\\
553	0.000226074990812508\\
554	0.000225666849807071\\
555	0.000225260179807396\\
556	0.000224854972876498\\
557	0.000224451221131572\\
558	0.000224048916748876\\
559	0.000223648051959291\\
560	0.000223248619050098\\
561	0.000222850610361869\\
562	0.000222454018292018\\
563	0.000222058835289696\\
564	0.000221665053859565\\
565	0.00022127266655847\\
566	0.00022088166599521\\
567	0.000220492044832099\\
568	0.000220103795781856\\
569	0.000219716911610046\\
570	0.000219331385131083\\
571	0.000218947209211562\\
572	0.000218564376766039\\
573	0.000218182880759699\\
574	0.000217802714207682\\
575	0.000217423870171318\\
576	0.000217046341762783\\
577	0.000216670122139329\\
578	0.000216295204508166\\
579	0.000215921582121137\\
580	0.00021554924827849\\
581	0.000215178196325772\\
582	0.000214808419654267\\
583	0.000214439911700337\\
584	0.000214072665945642\\
585	0.000213706675916914\\
586	0.000213341935184408\\
587	0.000212978437361899\\
588	0.000212616176107572\\
589	0.00021225514512202\\
590	0.000211895338149137\\
591	0.000211536748973895\\
592	0.000211179371425008\\
593	0.000210823199371379\\
594	0.000210468226723881\\
595	0.000210114447434018\\
596	0.000209761855494817\\
597	0.000209410444938385\\
598	0.000209060209836798\\
599	0.000208711144302764\\
600	0.000208363242487408\\
601	0.000208016498580488\\
602	0.00020767090681173\\
603	0.000207326461447943\\
604	0.000206983156793905\\
605	0.000206640987192808\\
606	0.000206299947024702\\
607	0.000205960030706498\\
608	0.000205621232692188\\
609	0.000205283547472623\\
610	0.000204946969573516\\
611	0.000204611493557882\\
612	0.000204277114022711\\
613	0.000203943825602293\\
614	0.000203611622963118\\
615	0.000203280500809866\\
616	0.000202950453877859\\
617	0.000202621476940168\\
618	0.000202293564800504\\
619	0.000201966712297663\\
620	0.000201640914304191\\
621	0.000201316165725496\\
622	0.000200992461498295\\
623	0.000200669796593722\\
624	0.000200348166014219\\
625	0.000200027564793981\\
626	0.000199707988000064\\
627	0.00019938943072928\\
628	0.00019907188811108\\
629	0.00019875535530578\\
630	0.000198439827504115\\
631	0.000198125299926355\\
632	0.00019781176782474\\
633	0.000197499226480158\\
634	0.000197187671204579\\
635	0.000196877097338177\\
636	0.000196567500251987\\
637	0.000196258875343469\\
638	0.000195951218041612\\
639	0.000195644523803828\\
640	0.000195338788114396\\
641	0.000195034006486239\\
642	0.000194730174460478\\
643	0.000194427287606658\\
644	0.000194125341520746\\
645	0.000193824331827352\\
646	0.000193524254175292\\
647	0.00019322510424491\\
648	0.000192926877739197\\
649	0.000192629570388902\\
650	0.000192333177951198\\
651	0.000192037696210345\\
652	0.000191743120974364\\
653	0.000191449448079251\\
654	0.000191156673383652\\
655	0.000190864792774637\\
656	0.00019057380216192\\
657	0.000190283697481641\\
658	0.0001899944746937\\
659	0.000189706129782863\\
660	0.000189418658758322\\
661	0.000189132057653696\\
662	0.000188846322526359\\
663	0.000188561449456337\\
664	0.000188277434548745\\
665	0.000187994273932235\\
666	0.000187711963758108\\
667	0.00018743050019987\\
668	0.000187149879456117\\
669	0.000186870097745429\\
670	0.000186591151311255\\
671	0.000186313036419694\\
672	0.000186035749356162\\
673	0.000185759286431164\\
674	0.000185483643975193\\
675	0.000185208818342497\\
676	0.000184934805907089\\
677	0.000184661603064074\\
678	0.000184389206231428\\
679	0.000184117611847778\\
680	0.00018384681637218\\
681	0.000183576816285003\\
682	0.000183307608086158\\
683	0.000183039188298428\\
684	0.000182771553462135\\
685	0.000182504700140029\\
686	0.000182238624912845\\
687	0.000181973324383744\\
688	0.000181708795173208\\
689	0.000181445033923033\\
690	0.000181182037292782\\
691	0.000180919801963775\\
692	0.000180658324634653\\
693	0.000180397602023374\\
694	0.000180137630867216\\
695	0.000179878407922551\\
696	0.000179619929963293\\
697	0.000179362193782229\\
698	0.000179105196192131\\
699	0.000178848934021536\\
700	0.000178593404118743\\
701	0.000178338603348704\\
702	0.000178084528596134\\
703	0.000177831176761956\\
704	0.000177578544765966\\
705	0.000177326629543284\\
706	0.00017707542804879\\
707	0.000176824937252906\\
708	0.000176575154144487\\
709	0.000176326075727706\\
710	0.000176077699026278\\
711	0.000175830021077461\\
712	0.000175583038937166\\
713	0.000175336749677957\\
714	0.000175091150387496\\
715	0.000174846238170989\\
716	0.000174602010148739\\
717	0.000174358463457924\\
718	0.000174115595251934\\
719	0.00017387340269881\\
720	0.000173631882983249\\
721	0.000173391033305048\\
722	0.000173150850879988\\
723	0.00017291133293873\\
724	0.000172672476728586\\
725	0.000172434279510414\\
726	0.000172196738560171\\
727	0.000171959851169801\\
728	0.000171723614646124\\
729	0.000171488026310618\\
730	0.000171253083498302\\
731	0.00017101878355974\\
732	0.000170785123860373\\
733	0.000170552101779853\\
734	0.000170319714710931\\
735	0.000170087960061682\\
736	0.000169856835255056\\
737	0.00016962633772466\\
738	0.000169396464922533\\
739	0.000169167214311372\\
740	0.000168938583367861\\
741	0.000168710569584007\\
742	0.000168483170464251\\
743	0.000168256383525689\\
744	0.000168030206299852\\
745	0.000167804636331592\\
746	0.000167579671178197\\
747	0.000167355308410055\\
748	0.000167131545612209\\
749	0.000166908380380582\\
750	0.000166685810323752\\
751	0.000166463833064956\\
752	0.000166242446238751\\
753	0.000166021647492576\\
754	0.000165801434486301\\
755	0.000165581804892456\\
756	0.00016536275639556\\
757	0.000165144286691676\\
758	0.000164926393490861\\
759	0.000164709074513603\\
760	0.000164492327493937\\
761	0.000164276150175002\\
762	0.000164060540315258\\
763	0.000163845495682491\\
764	0.000163631014057142\\
765	0.0001634170932312\\
766	0.000163203731008643\\
767	0.000162990925202777\\
768	0.000162778673641117\\
769	0.000162566974161171\\
770	0.000162355824611549\\
771	0.000162145222851962\\
772	0.000161935166753446\\
773	0.000161725654198364\\
774	0.00016151668307951\\
775	0.000161308251300563\\
776	0.000161100356776966\\
777	0.000160892997433493\\
778	0.000160686171206681\\
779	0.000160479876042396\\
780	0.000160274109899161\\
781	0.000160068870743713\\
782	0.000159864156554335\\
783	0.000159659965319081\\
784	0.000159456295037774\\
785	0.000159253143718008\\
786	0.000159050509378922\\
787	0.000158848390050537\\
788	0.00015864678377131\\
789	0.000158445688590358\\
790	0.00015824510256679\\
791	0.000158045023768594\\
792	0.000157845450275751\\
793	0.000157646380174681\\
794	0.000157447811564015\\
795	0.000157249742551269\\
796	0.000157052171253724\\
797	0.000156855095796438\\
798	0.000156658514316232\\
799	0.000156462424957038\\
800	0.000156266825874996\\
801	0.000156071715231798\\
802	0.000155877091201573\\
803	0.00015568295196533\\
804	0.000155489295714517\\
805	0.000155296120648574\\
806	0.00015510342497671\\
807	0.000154911206915687\\
808	0.000154719464693143\\
809	0.000154528196543602\\
810	0.000154337400711579\\
811	0.000154147075449362\\
812	0.000153957219017675\\
813	0.000153767829687013\\
814	0.000153578905736085\\
815	0.000153390445451151\\
816	0.000153202447127132\\
817	0.000153014909067606\\
818	0.000152827829585256\\
819	0.000152641206999204\\
820	0.000152455039637678\\
821	0.000152269325837784\\
822	0.00015208406394307\\
823	0.000151899252307297\\
824	0.000151714889289334\\
825	0.000151530973259595\\
826	0.000151347502593158\\
827	0.000151164475675092\\
828	0.00015098189089624\\
829	0.000150799746657437\\
830	0.000150618041365957\\
831	0.000150436773436624\\
832	0.000150255941292921\\
833	0.000150075543365213\\
834	0.000149895578090753\\
835	0.000149716043915671\\
836	0.000149536939292094\\
837	0.000149358262680588\\
838	0.000149180012548156\\
839	0.000149002187370906\\
840	0.000148824785629831\\
841	0.000148647805815028\\
842	0.00014847124642281\\
843	0.000148295105957263\\
844	0.000148119382928247\\
845	0.0001479440758545\\
846	0.000147769183260094\\
847	0.000147594703677312\\
848	0.000147420635644657\\
849	0.000147246977707738\\
850	0.000147073728419489\\
851	0.000146900886338397\\
852	0.000146728450031386\\
853	0.000146556418070265\\
854	0.000146384789035281\\
855	0.000146213561511122\\
856	0.000146042734092244\\
857	0.000145872305377326\\
858	0.000145702273972148\\
859	0.000145532638488266\\
860	0.000145363397545228\\
861	0.000145194549768135\\
862	0.000145026093787859\\
863	0.000144858028243489\\
864	0.00014469035177811\\
865	0.000144523063042135\\
866	0.000144356160693526\\
867	0.000144189643393577\\
868	0.000144023509812685\\
869	0.000143857758625021\\
870	0.000143692388512529\\
871	0.000143527398162258\\
872	0.000143362786267698\\
873	0.000143198551527224\\
874	0.00014303469264787\\
875	0.000142871208339335\\
876	0.000142708097320199\\
877	0.00014254535831193\\
878	0.000142382990043544\\
879	0.000142220991250497\\
880	0.000142059360672242\\
881	0.000141898097055337\\
882	0.000141737199151226\\
883	0.000141576665717347\\
884	0.000141416495516467\\
885	0.00014125668731757\\
886	0.000141097239894528\\
887	0.000140938152026981\\
888	0.000140779422499904\\
889	0.000140621050104706\\
890	0.000140463033636573\\
891	0.000140305371897353\\
892	0.000140148063694223\\
893	0.000139991107838577\\
894	0.000139834503149139\\
895	0.000139678248447517\\
896	0.00013952234256176\\
897	0.000139366784326356\\
898	0.000139211572578457\\
899	0.000139056706162988\\
900	0.000138902183927536\\
901	0.000138748004726796\\
902	0.000138594167419015\\
903	0.000138440670869322\\
904	0.00013828751394529\\
905	0.000138134695522041\\
906	0.000137982214478694\\
907	0.000137830069698586\\
908	0.00013767826007105\\
909	0.000137526784489861\\
910	0.000137375641853232\\
911	0.00013722483106493\\
912	0.000137074351033162\\
913	0.000136924200671462\\
914	0.000136774378897142\\
915	0.000136624884632619\\
916	0.000136475716805862\\
917	0.000136326874348169\\
918	0.000136178356196837\\
919	0.000136030161292044\\
920	0.000135882288580635\\
921	0.000135734737012783\\
922	0.000135587505542656\\
923	0.000135440593130864\\
924	0.000135293998740904\\
925	0.000135147721341156\\
926	0.000135001759904663\\
927	0.000134856113408688\\
928	0.000134710780835601\\
929	0.000134565761170435\\
930	0.000134421053405109\\
931	0.000134276656533538\\
932	0.000134132569554746\\
933	0.000133988791472861\\
934	0.000133845321295789\\
935	0.000133702158034543\\
936	0.000133559300705466\\
937	0.000133416748330006\\
938	0.000133274499931835\\
939	0.00013313255453995\\
940	0.000132990911187347\\
941	0.000132849568910576\\
942	0.000132708526750847\\
943	0.00013256778375359\\
944	0.000132427338968011\\
945	0.000132287191446867\\
946	0.000132147340247801\\
947	0.000132007784432231\\
948	0.00013186852306446\\
949	0.0001317295552139\\
950	0.000131590879954402\\
951	0.000131452496361595\\
952	0.000131314403517324\\
953	0.000131176600505878\\
954	0.000131039086415319\\
955	0.000130901860338817\\
956	0.000130764921371762\\
957	0.000130628268614652\\
958	0.000130491901169982\\
959	0.000130355818146022\\
960	0.00013022001865437\\
961	0.000130084501808847\\
962	0.000129949266728824\\
963	0.00012981431253567\\
964	0.000129679638355418\\
965	0.00012954524331743\\
966	0.000129411126555512\\
967	0.000129277287204799\\
968	0.000129143724406644\\
969	0.000129010437303956\\
970	0.000128877425044749\\
971	0.000128744686779481\\
972	0.000128612221662383\\
973	0.000128480028851241\\
974	0.000128348107506948\\
975	0.000128216456794839\\
976	0.000128085075882467\\
977	0.000127953963941385\\
978	0.000127823120146697\\
979	0.000127692543676394\\
980	0.000127562233711576\\
981	0.000127432189438226\\
982	0.000127302410043661\\
983	0.00012717289471964\\
984	0.000127043642661473\\
985	0.000126914653067578\\
986	0.000126785925137707\\
987	0.000126657458078494\\
988	0.000126529251095686\\
989	0.000126401303402135\\
990	0.000126273614210914\\
991	0.000126146182739539\\
992	0.000126019008209077\\
993	0.000125892089842372\\
994	0.000125765426867153\\
995	0.000125639018513146\\
996	0.000125512864012522\\
997	0.000125386962602336\\
998	0.000125261313520975\\
999	0.000125135916011487\\
1000	0.000125010769318701\\
};
\label{addplot:EncadremeentCp0}
\addplot [color=mycolor2, draw=none, mark size=1.2pt, mark=*, mark options={solid, mycolor2}, forget plot]
  table[row sep=crcr]{%
1	-0.0189948529191399\\
2	-0.00397149732309154\\
3	-0.00164595259012812\\
4	-0.00089210793902128\\
5	-0.000557987702367813\\
6	-0.000381498117149603\\
7	-0.000277144053604905\\
8	-0.000210387986559035\\
9	-0.000165127605646687\\
10	-0.000133038300255794\\
11	-0.000109466034875827\\
12	-9.1644016226855e-05\\
13	-7.78438301836104e-05\\
14	-6.69407118586429e-05\\
15	-5.81771999738079e-05\\
16	-5.10280890999582e-05\\
17	-4.51199432545124e-05\\
18	-4.01813616910385e-05\\
19	-3.60112678738922e-05\\
20	-3.24581267268087e-05\\
21	-2.94059891254861e-05\\
22	-2.67649159780836e-05\\
23	-2.44642788632676e-05\\
24	-2.2447990528307e-05\\
25	-2.06710551259315e-05\\
26	-1.90970367944798e-05\\
27	-1.76961774718265e-05\\
28	-1.64439804002381e-05\\
29	-1.53201321559937e-05\\
30	-1.4307673807834e-05\\
31	-1.33923575130757e-05\\
32	-1.25621426132483e-05\\
33	-1.1806797706182e-05\\
34	-1.11175839561106e-05\\
35	-1.04870012512803e-05\\
36	-9.90858332128308e-06\\
37	-9.37673133238892e-06\\
38	-8.88657790898417e-06\\
39	-8.43387539239959e-06\\
40	-8.01490351198808e-06\\
41	-7.62639272355869e-06\\
42	-7.26546022877805e-06\\
43	-6.92955635783754e-06\\
44	-6.61641941790414e-06\\
45	-6.32403752487409e-06\\
46	-6.05061621417047e-06\\
47	-5.79455084115388e-06\\
48	-5.55440298843912e-06\\
49	-5.32888021953504e-06\\
50	-5.11681866044444e-06\\
51	-4.91716794803754e-06\\
52	-4.72897820258389e-06\\
53	-4.55138871036098e-06\\
54	-4.38361806365251e-06\\
55	-4.22495555219005e-06\\
56	-4.07475362262932e-06\\
57	-3.93242126350835e-06\\
58	-3.79741818201662e-06\\
59	-3.66924966821447e-06\\
60	-3.54746205877277e-06\\
61	-3.43163871463492e-06\\
62	-3.32139645708995e-06\\
63	-3.21638238975908e-06\\
64	-3.11627107552059e-06\\
65	-3.02076200886514e-06\\
66	-2.92957736058863e-06\\
67	-2.84245995407773e-06\\
68	-2.7591714506503e-06\\
69	-2.67949071619533e-06\\
70	-2.60321235390215e-06\\
71	-2.53014537821006e-06\\
72	-2.46011201976426e-06\\
73	-2.39294663950673e-06\\
74	-2.32849475267916e-06\\
75	-2.26661213720281e-06\\
76	-2.20716402721255e-06\\
77	-2.15002437797818e-06\\
78	-2.09507519799423e-06\\
79	-2.04220593891247e-06\\
80	-1.99131293909804e-06\\
81	-1.94229891548048e-06\\
82	-1.89507249892529e-06\\
83	-1.849547806132e-06\\
84	-1.80564405027894e-06\\
85	-1.76328518353142e-06\\
86	-1.72239956341969e-06\\
87	-1.68291965640943e-06\\
88	-1.64478175213034e-06\\
89	-1.60792571002322e-06\\
90	-1.57229471975384e-06\\
91	-1.53783508549665e-06\\
92	-1.50449602132063e-06\\
93	-1.47222946234038e-06\\
94	-1.44098989518504e-06\\
95	-1.41073419424043e-06\\
96	-1.38142147532161e-06\\
97	-1.35301295411949e-06\\
98	-1.32547182174481e-06\\
99	-1.29876312260357e-06\\
100	-1.27285364470708e-06\\
101	-1.24771181553296e-06\\
102	-1.22330760732314e-06\\
103	-1.19961244560152e-06\\
104	-1.17659912579615e-06\\
105	-1.15424173607881e-06\\
106	-1.13251558486738e-06\\
107	-1.11139713110386e-06\\
108	-1.09086391941737e-06\\
109	-1.07089452472398e-06\\
110	-1.05146849105342e-06\\
111	-1.03256628347648e-06\\
112	-1.01416923670161e-06\\
113	-9.96259508445618e-07\\
114	-9.78820038799455e-07\\
115	-9.61834507151593e-07\\
116	-9.45287294995545e-07\\
117	-9.29163449292503e-07\\
118	-9.13448650274873e-07\\
119	-8.98129177362428e-07\\
120	-8.83191881295708e-07\\
121	-8.68624154715114e-07\\
122	-8.54413905737594e-07\\
123	-8.40549533309698e-07\\
124	-8.27019902782666e-07\\
125	-8.13814324929218e-07\\
126	-8.00922532961934e-07\\
127	-7.88334662882306e-07\\
128	-7.76041236827396e-07\\
129	-7.64033141864573e-07\\
130	-7.52301617112927e-07\\
131	-7.4083823253801e-07\\
132	-7.29634879959029e-07\\
133	-7.18683752731764e-07\\
134	-7.07977336200649e-07\\
135	-6.97508393820989e-07\\
136	-6.87269953725256e-07\\
137	-6.77255298175972e-07\\
138	-6.67457953129613e-07\\
139	-6.57871675691091e-07\\
140	-6.48490447785477e-07\\
141	-6.39308461725108e-07\\
142	-6.30320115435623e-07\\
143	-6.21520002019871e-07\\
144	-6.12902901542256e-07\\
145	-6.04463774589448e-07\\
146	-5.96197753943706e-07\\
147	-5.8810013725541e-07\\
148	-5.80166380714786e-07\\
149	-5.72392092723639e-07\\
150	-5.64773028566279e-07\\
151	-5.57305082526938e-07\\
152	-5.49984285114213e-07\\
153	-5.42806794956441e-07\\
154	-5.35768897025335e-07\\
155	-5.28866993976251e-07\\
156	-5.22097604149785e-07\\
157	-5.15457357908033e-07\\
158	-5.08942990418149e-07\\
159	-5.02551339875978e-07\\
160	-4.96279343509265e-07\\
161	-4.90124033025729e-07\\
162	-4.84082531948538e-07\\
163	-4.78152051952563e-07\\
164	-4.72329888312473e-07\\
165	-4.66613420568862e-07\\
166	-4.61000105533849e-07\\
167	-4.55487475403693e-07\\
168	-4.50073137203688e-07\\
169	-4.44754767237043e-07\\
170	-4.39530111973063e-07\\
171	-4.34396981940921e-07\\
172	-4.29353251618636e-07\\
173	-4.24396856768539e-07\\
174	-4.19525793549091e-07\\
175	-4.1473811318582e-07\\
176	-4.10031924191756e-07\\
177	-4.05405387260416e-07\\
178	-4.00856715043751e-07\\
179	-3.96384169709663e-07\\
180	-3.9198606216484e-07\\
181	-3.87660750167385e-07\\
182	-3.83406635440231e-07\\
183	-3.79222164115234e-07\\
184	-3.75105823846589e-07\\
185	-3.71056144476967e-07\\
186	-3.67071693818666e-07\\
187	-3.63151078208723e-07\\
188	-3.59292941398692e-07\\
189	-3.55495964110553e-07\\
190	-3.51758858596618e-07\\
191	-3.4808037363554e-07\\
192	-3.4445928986937e-07\\
193	-3.40894418138227e-07\\
194	-3.37384602477897e-07\\
195	-3.33928713125431e-07\\
196	-3.30525652514346e-07\\
197	-3.27174348613291e-07\\
198	-3.23873757146487e-07\\
199	-3.20622860372488e-07\\
200	-3.17420665862933e-07\\
201	-3.14266204504143e-07\\
202	-3.11158534049838e-07\\
203	-3.08096732348773e-07\\
204	-3.05079901341543e-07\\
205	-3.02107164507071e-07\\
206	-2.99177667417716e-07\\
207	-2.96290575518832e-07\\
208	-2.93445073018539e-07\\
209	-2.90640366440442e-07\\
210	-2.87875678295357e-07\\
211	-2.85150251855271e-07\\
212	-2.82463346601425e-07\\
213	-2.79814240777831e-07\\
214	-2.77202227616513e-07\\
215	-2.74626619001239e-07\\
216	-2.72086740915611e-07\\
217	-2.69581935441465e-07\\
218	-2.6711155975967e-07\\
219	-2.64674985928082e-07\\
220	-2.62271600215414e-07\\
221	-2.59900802435098e-07\\
222	-2.57562006167333e-07\\
223	-2.55254637426816e-07\\
224	-2.52978136883186e-07\\
225	-2.50731954976047e-07\\
226	-2.48515556910966e-07\\
227	-2.46328417885522e-07\\
228	-2.44170024976675e-07\\
229	-2.42039876918732e-07\\
230	-2.39937482993113e-07\\
231	-2.37862363250407e-07\\
232	-2.35814047511163e-07\\
233	-2.33792076476114e-07\\
234	-2.31796000615958e-07\\
235	-2.29825378839088e-07\\
236	-2.27879780823059e-07\\
237	-2.25958784461078e-07\\
238	-2.24061976417111e-07\\
239	-2.22188952458957e-07\\
240	-2.20339317014151e-07\\
241	-2.1851268272588e-07\\
242	-2.16708667899468e-07\\
243	-2.14926902386559e-07\\
244	-2.13167021034799e-07\\
245	-2.11428667129532e-07\\
246	-2.09711490950504e-07\\
247	-2.0801514988289e-07\\
248	-2.06339308306269e-07\\
249	-2.0468363726156e-07\\
250	-2.03047813895907e-07\\
251	-2.01431523016993e-07\\
252	-1.99834454095438e-07\\
253	-1.98256304484445e-07\\
254	-1.96696775423e-07\\
255	-1.95155575921646e-07\\
256	-1.93632419209777e-07\\
257	-1.92127025622213e-07\\
258	-1.90639119490577e-07\\
259	-1.89168430475561e-07\\
260	-1.87714694788177e-07\\
261	-1.86277652747258e-07\\
262	-1.84857049556619e-07\\
263	-1.83452635416081e-07\\
264	-1.820641651884e-07\\
265	-1.80691398732336e-07\\
266	-1.79334099792428e-07\\
267	-1.77992037442287e-07\\
268	-1.76664983753128e-07\\
269	-1.75352715570121e-07\\
270	-1.74055015289554e-07\\
271	-1.72771666862026e-07\\
272	-1.71502460455386e-07\\
273	-1.70247188124861e-07\\
274	-1.6900564725475e-07\\
275	-1.67777637782862e-07\\
276	-1.66562964309946e-07\\
277	-1.6536143432333e-07\\
278	-1.64172858752032e-07\\
279	-1.62997051522673e-07\\
280	-1.61833831002767e-07\\
281	-1.60683018557428e-07\\
282	-1.59544437439152e-07\\
283	-1.58417915563369e-07\\
284	-1.57303283176979e-07\\
285	-1.5620037330244e-07\\
286	-1.55109021848787e-07\\
287	-1.54029068832884e-07\\
288	-1.5296035515977e-07\\
289	-1.51902725953335e-07\\
290	-1.50856027669732e-07\\
291	-1.49820111428056e-07\\
292	-1.48794828791488e-07\\
293	-1.47780034875922e-07\\
294	-1.46775586906678e-07\\
295	-1.45781345439744e-07\\
296	-1.44797171919286e-07\\
297	-1.43822931342186e-07\\
298	-1.42858489926567e-07\\
299	-1.41903716888159e-07\\
300	-1.40958484218245e-07\\
301	-1.40022664130157e-07\\
302	-1.39096132945049e-07\\
303	-1.38178767428165e-07\\
304	-1.37270447342352e-07\\
305	-1.36371053893747e-07\\
306	-1.35480471175065e-07\\
307	-1.34598584056178e-07\\
308	-1.33725279960473e-07\\
309	-1.32860447310534e-07\\
310	-1.32003977193484e-07\\
311	-1.31155762583823e-07\\
312	-1.30315696900141e-07\\
313	-1.29483676225561e-07\\
314	-1.28659598863656e-07\\
315	-1.2784336378413e-07\\
316	-1.27034871177933e-07\\
317	-1.26234024011573e-07\\
318	-1.25440726139736e-07\\
319	-1.24654882638353e-07\\
320	-1.23876400803802e-07\\
321	-1.23105189042683e-07\\
322	-1.22341156094663e-07\\
323	-1.21584214585191e-07\\
324	-1.20834276251536e-07\\
325	-1.20091255051413e-07\\
326	-1.19355066274807e-07\\
327	-1.18625625766811e-07\\
328	-1.17902852148077e-07\\
329	-1.17186664150282e-07\\
330	-1.16476982392477e-07\\
331	-1.15773727937807e-07\\
332	-1.15076822848614e-07\\
333	-1.14386192073823e-07\\
334	-1.13701759230089e-07\\
335	-1.13023451042693e-07\\
336	-1.12351195014071e-07\\
337	-1.11684918979726e-07\\
338	-1.11024551885386e-07\\
339	-1.10370024897222e-07\\
340	-1.09721269181406e-07\\
341	-1.09078215237979e-07\\
342	-1.08440799229115e-07\\
343	-1.07808954097344e-07\\
344	-1.07182615227686e-07\\
345	-1.06561718782316e-07\\
346	-1.05946201478524e-07\\
347	-1.05336002365064e-07\\
348	-1.04731059713536e-07\\
349	-1.04131313238831e-07\\
350	-1.03536703655038e-07\\
351	-1.02947172897494e-07\\
352	-1.02362662790512e-07\\
353	-1.0178311637965e-07\\
354	-1.01208478042736e-07\\
355	-1.00638692157595e-07\\
356	-1.00073705100456e-07\\
357	-9.95134622483462e-08\\
358	-9.8957910754649e-08\\
359	-9.84069982168378e-08\\
360	-9.78606740087429e-08\\
361	-9.73188861719265e-08\\
362	-9.67815853014642e-08\\
363	-9.62487221034536e-08\\
364	-9.57202478391039e-08\\
365	-9.51961133255352e-08\\
366	-9.46762728215589e-08\\
367	-9.41606789206517e-08\\
368	-9.36492846603798e-08\\
369	-9.31420451877329e-08\\
370	-9.26389162048125e-08\\
371	-9.21398529696305e-08\\
372	-9.16448115173552e-08\\
373	-9.11537489933778e-08\\
374	-9.06666223210451e-08\\
375	-9.01833907551719e-08\\
376	-8.97040121072834e-08\\
377	-8.92284451881054e-08\\
378	-8.87566503626758e-08\\
379	-8.82885875519435e-08\\
380	-8.7824217676058e-08\\
381	-8.73635016551688e-08\\
382	-8.6906401297604e-08\\
383	-8.64528795219144e-08\\
384	-8.60028980254057e-08\\
385	-8.5556420836852e-08\\
386	-8.51134117629826e-08\\
387	-8.46738337223485e-08\\
388	-8.4237653075192e-08\\
389	-8.3804833739265e-08\\
390	-8.33753417417427e-08\\
391	-8.2949142887756e-08\\
392	-8.2526203759592e-08\\
393	-8.21064916056713e-08\\
394	-8.16899725641917e-08\\
395	-8.12766146607302e-08\\
396	-8.08663870310866e-08\\
397	-8.04592569236817e-08\\
398	-8.0055194029427e-08\\
399	-7.96541669290107e-08\\
400	-7.92561462015229e-08\\
401	-7.88611005386741e-08\\
402	-7.84690018518219e-08\\
403	-7.80798201649446e-08\\
404	-7.76935257240652e-08\\
405	-7.73100917728087e-08\\
406	-7.69294891123096e-08\\
407	-7.65516900980145e-08\\
408	-7.61766677515041e-08\\
409	-7.58043935400465e-08\\
410	-7.54348424836238e-08\\
411	-7.50679870487048e-08\\
412	-7.47038009230039e-08\\
413	-7.43422591265031e-08\\
414	-7.39833354579389e-08\\
415	-7.36270046042264e-08\\
416	-7.32732415853476e-08\\
417	-7.29220226425298e-08\\
418	-7.25733229067771e-08\\
419	-7.22271182862499e-08\\
420	-7.1883384800131e-08\\
421	-7.15420997998706e-08\\
422	-7.12032389715844e-08\\
423	-7.08667799997897e-08\\
424	-7.05327004579814e-08\\
425	-7.02009776976098e-08\\
426	-6.9871589736259e-08\\
427	-6.95445144804907e-08\\
428	-6.92197310581122e-08\\
429	-6.8897216931596e-08\\
430	-6.85769513397716e-08\\
431	-6.82589138545353e-08\\
432	-6.79430839367612e-08\\
433	-6.76294408252787e-08\\
434	-6.73179642030064e-08\\
435	-6.70086346410415e-08\\
436	-6.67014327104809e-08\\
437	-6.63963384273103e-08\\
438	-6.60933314744483e-08\\
439	-6.57923954205941e-08\\
440	-6.54935099486664e-08\\
441	-6.51966556297623e-08\\
442	-6.49018150333802e-08\\
443	-6.46089711731079e-08\\
444	-6.43181039539087e-08\\
445	-6.40291967224371e-08\\
446	-6.37422318261471e-08\\
447	-6.3457191390448e-08\\
448	-6.31740587619944e-08\\
449	-6.28928170653964e-08\\
450	-6.26134493142416e-08\\
451	-6.23359387441624e-08\\
452	-6.20602688128358e-08\\
453	-6.17864236440724e-08\\
454	-6.1514387361683e-08\\
455	-6.12441432013e-08\\
456	-6.09756762859348e-08\\
457	-6.07089706283759e-08\\
458	-6.04440110185678e-08\\
459	-6.01807821354328e-08\\
460	-5.99192695460715e-08\\
461	-5.96594573742948e-08\\
462	-5.94013314092479e-08\\
463	-5.91448773290537e-08\\
464	-5.88900800346792e-08\\
465	-5.86369259814035e-08\\
466	-5.83854005142825e-08\\
467	-5.8135490643707e-08\\
468	-5.78871813816662e-08\\
469	-5.76404596275282e-08\\
470	-5.73953117255499e-08\\
471	-5.7151724686122e-08\\
472	-5.69096847424788e-08\\
473	-5.66691800152341e-08\\
474	-5.64301957384217e-08\\
475	-5.6192720365722e-08\\
476	-5.59567406854811e-08\\
477	-5.57222444852457e-08\\
478	-5.54892192194956e-08\\
479	-5.52576530088444e-08\\
480	-5.50275330857275e-08\\
481	-5.47988472376915e-08\\
482	-5.45715841404615e-08\\
483	-5.43457320256735e-08\\
484	-5.41212794580304e-08\\
485	-5.38982143361011e-08\\
486	-5.36765253356108e-08\\
487	-5.3456201576374e-08\\
488	-5.32372312900264e-08\\
489	-5.30196037074049e-08\\
490	-5.28033078373014e-08\\
491	-5.25883327995302e-08\\
492	-5.23746680469728e-08\\
493	-5.21623026994433e-08\\
494	-5.19512267649347e-08\\
495	-5.1741428919172e-08\\
496	-5.15328997252595e-08\\
497	-5.13256285250563e-08\\
498	-5.11196048824658e-08\\
499	-5.09148194716147e-08\\
500	-5.07112626335626e-08\\
501	-5.05089237101686e-08\\
502	-5.03077935976037e-08\\
503	-5.01078617487494e-08\\
504	-4.99091205030666e-08\\
505	-4.97115584252583e-08\\
506	-4.95151675217187e-08\\
507	-4.93199378004405e-08\\
508	-4.91258607127065e-08\\
509	-4.89329269326433e-08\\
510	-4.8741127356422e-08\\
511	-4.85504531022585e-08\\
512	-4.83608957324577e-08\\
513	-4.81724466983025e-08\\
514	-4.798509622983e-08\\
515	-4.77988369995686e-08\\
516	-4.76136607918676e-08\\
517	-4.74295577257422e-08\\
518	-4.72465202516759e-08\\
519	-4.70645409311743e-08\\
520	-4.68836105493864e-08\\
521	-4.67037218898625e-08\\
522	-4.65248664038853e-08\\
523	-4.63470359868268e-08\\
524	-4.61702234222372e-08\\
525	-4.59944208275331e-08\\
526	-4.5819620653198e-08\\
527	-4.56458150166483e-08\\
528	-4.54729961463229e-08\\
529	-4.53011569367945e-08\\
530	-4.51302898385464e-08\\
531	-4.49603881902405e-08\\
532	-4.47914436652042e-08\\
533	-4.46234494910769e-08\\
534	-4.44563991175428e-08\\
535	-4.4290284439974e-08\\
536	-4.41251001292997e-08\\
537	-4.39608368596467e-08\\
538	-4.37974900791005e-08\\
539	-4.36350519050777e-08\\
540	-4.34735154541954e-08\\
541	-4.33128743981825e-08\\
542	-4.31531222977455e-08\\
543	-4.29942523805238e-08\\
544	-4.28362582072239e-08\\
545	-4.26791326724185e-08\\
546	-4.25228710021486e-08\\
547	-4.23674655358752e-08\\
548	-4.22129102783941e-08\\
549	-4.20591990124564e-08\\
550	-4.19063268530806e-08\\
551	-4.17542855846165e-08\\
552	-4.16030707661719e-08\\
553	-4.1452675847431e-08\\
554	-4.13030948331894e-08\\
555	-4.11543223943767e-08\\
556	-4.10063515365877e-08\\
557	-4.08591778189304e-08\\
558	-4.07127950241559e-08\\
559	-4.05671974901267e-08\\
560	-4.04223792216385e-08\\
561	-4.02783353337099e-08\\
562	-4.01350597201144e-08\\
563	-3.99925477179153e-08\\
564	-3.98507933319081e-08\\
565	-3.97097906779109e-08\\
566	-3.95695357591208e-08\\
567	-3.94300223582889e-08\\
568	-3.92912458124783e-08\\
569	-3.91532001264849e-08\\
570	-3.90158809704388e-08\\
571	-3.88792825711803e-08\\
572	-3.87434007098619e-08\\
573	-3.86082300574131e-08\\
574	-3.84737649516964e-08\\
575	-3.83400016179536e-08\\
576	-3.82069342830249e-08\\
577	-3.80745591721521e-08\\
578	-3.79428699570639e-08\\
579	-3.78118635291358e-08\\
580	-3.76815342262304e-08\\
581	-3.75518776074557e-08\\
582	-3.74228885657857e-08\\
583	-3.72945633264621e-08\\
584	-3.71668973375705e-08\\
585	-3.70398852700404e-08\\
586	-3.6913523127069e-08\\
587	-3.67878068008309e-08\\
588	-3.66627316283896e-08\\
589	-3.65382933908975e-08\\
590	-3.64144868703065e-08\\
591	-3.62913092910588e-08\\
592	-3.61687552130618e-08\\
593	-3.60468208615572e-08\\
594	-3.59255025728089e-08\\
595	-3.58047956838803e-08\\
596	-3.56846956428569e-08\\
597	-3.55651988970251e-08\\
598	-3.54463018936713e-08\\
599	-3.53279996367917e-08\\
600	-3.5210288684695e-08\\
601	-3.50931653736453e-08\\
602	-3.49766252627504e-08\\
603	-3.48606646882743e-08\\
604	-3.47452799864811e-08\\
605	-3.46304670495456e-08\\
606	-3.45162222137319e-08\\
607	-3.44025421483707e-08\\
608	-3.42894225235923e-08\\
609	-3.41768598977055e-08\\
610	-3.40648506069741e-08\\
611	-3.39533908766398e-08\\
612	-3.38424779311453e-08\\
613	-3.37321067744867e-08\\
614	-3.36222756303073e-08\\
615	-3.35129787254473e-08\\
616	-3.34042152827507e-08\\
617	-3.32959791959908e-08\\
618	-3.31882687998331e-08\\
619	-3.30810806525861e-08\\
620	-3.29744106464247e-08\\
621	-3.28682553396575e-08\\
622	-3.27626125118385e-08\\
623	-3.26574779441202e-08\\
624	-3.25528484168558e-08\\
625	-3.244872126551e-08\\
626	-3.23450926043023e-08\\
627	-3.22419597686974e-08\\
628	-3.21393193170039e-08\\
629	-3.20371683626419e-08\\
630	-3.19355030198309e-08\\
631	-3.18343215122141e-08\\
632	-3.17336198429885e-08\\
633	-3.16333952365966e-08\\
634	-3.15336449174808e-08\\
635	-3.1434365554972e-08\\
636	-3.13355539294236e-08\\
637	-3.12372080424339e-08\\
638	-3.11393247853786e-08\\
639	-3.10418999394102e-08\\
640	-3.09449318391941e-08\\
641	-3.08484174871282e-08\\
642	-3.07523542186772e-08\\
643	-3.06567388141943e-08\\
644	-3.0561568498122e-08\\
645	-3.04668403838804e-08\\
646	-3.03725531392018e-08\\
647	-3.02787017680828e-08\\
648	-3.0185284716211e-08\\
649	-3.00922999851849e-08\\
650	-2.99997443553579e-08\\
651	-2.99076143850385e-08\\
652	-2.98159088529815e-08\\
653	-2.97246237623838e-08\\
654	-2.96337578920003e-08\\
655	-2.95433074670726e-08\\
656	-2.94532707112438e-08\\
657	-2.93636452930457e-08\\
658	-2.92744279928314e-08\\
659	-2.91856168121996e-08\\
660	-2.90972089755925e-08\\
661	-2.90092024846089e-08\\
662	-2.89215942306242e-08\\
663	-2.88343825483039e-08\\
664	-2.87475651061797e-08\\
665	-2.86611389066493e-08\\
666	-2.85751013961999e-08\\
667	-2.8489451353586e-08\\
668	-2.84041851150718e-08\\
669	-2.83193020145234e-08\\
670	-2.82347989433163e-08\\
671	-2.81506722377145e-08\\
672	-2.80669220087404e-08\\
673	-2.79835445926579e-08\\
674	-2.79005386571995e-08\\
675	-2.78179008716961e-08\\
676	-2.77356294597908e-08\\
677	-2.76537227561491e-08\\
678	-2.75721787623695e-08\\
679	-2.74909947028945e-08\\
680	-2.74101684683004e-08\\
681	-2.73296981712079e-08\\
682	-2.72495820352603e-08\\
683	-2.71698175069446e-08\\
684	-2.70904025878593e-08\\
685	-2.70113353906254e-08\\
686	-2.69326141388859e-08\\
687	-2.68542359460611e-08\\
688	-2.67761998129501e-08\\
689	-2.66985028529731e-08\\
690	-2.66211445110187e-08\\
691	-2.65441214564177e-08\\
692	-2.64674321348579e-08\\
693	-2.63910747699825e-08\\
694	-2.63150474744123e-08\\
695	-2.62393474725897e-08\\
696	-2.61639744314479e-08\\
697	-2.60889256864516e-08\\
698	-2.60141987951101e-08\\
699	-2.59397927582228e-08\\
700	-2.58657053553435e-08\\
701	-2.57919353652269e-08\\
702	-2.57184807894717e-08\\
703	-2.56453388525202e-08\\
704	-2.55725083331271e-08\\
705	-2.54999882320917e-08\\
706	-2.54277756628341e-08\\
707	-2.53558696261535e-08\\
708	-2.52842683456933e-08\\
709	-2.52129701561188e-08\\
710	-2.51419722818724e-08\\
711	-2.50712743898873e-08\\
712	-2.50008743707397e-08\\
713	-2.49307701150059e-08\\
714	-2.48609607345074e-08\\
715	-2.47914438977759e-08\\
716	-2.47222184945883e-08\\
717	-2.46532827485879e-08\\
718	-2.45846339952394e-08\\
719	-2.45162721235204e-08\\
720	-2.44481950240072e-08\\
721	-2.43804012534099e-08\\
722	-2.4312888813327e-08\\
723	-2.42456572596694e-08\\
724	-2.41787030397234e-08\\
725	-2.41120259314442e-08\\
726	-2.40456244915421e-08\\
727	-2.39794972767271e-08\\
728	-2.391364239962e-08\\
729	-2.38480577507971e-08\\
730	-2.37827431082138e-08\\
731	-2.37176966955133e-08\\
732	-2.3652916514294e-08\\
733	-2.35884008992215e-08\\
734	-2.35241495172289e-08\\
735	-2.34601600368478e-08\\
736	-2.3396430903766e-08\\
737	-2.33329620069611e-08\\
738	-2.32697502378088e-08\\
739	-2.32067951522197e-08\\
740	-2.31440955289486e-08\\
741	-2.3081649369594e-08\\
742	-2.30194554529106e-08\\
743	-2.29575127796977e-08\\
744	-2.2895819795643e-08\\
745	-2.28343752795013e-08\\
746	-2.27731775659379e-08\\
747	-2.27122259888191e-08\\
748	-2.26515183276987e-08\\
749	-2.25910534723539e-08\\
750	-2.25308310897177e-08\\
751	-2.24708487372993e-08\\
752	-2.24111059710097e-08\\
753	-2.23516007924474e-08\\
754	-2.22923325354785e-08\\
755	-2.22332995347685e-08\\
756	-2.21745005690721e-08\\
757	-2.21159351943001e-08\\
758	-2.20576009679618e-08\\
759	-2.19994974459681e-08\\
760	-2.19416225188951e-08\\
761	-2.18839766308321e-08\\
762	-2.18265568951992e-08\\
763	-2.17693634230187e-08\\
764	-2.17123944379338e-08\\
765	-2.16556487186992e-08\\
766	-2.15991245999803e-08\\
767	-2.15428226368886e-08\\
768	-2.1486739498755e-08\\
769	-2.14308755186465e-08\\
770	-2.13752286981617e-08\\
771	-2.13197985932112e-08\\
772	-2.12645839825498e-08\\
773	-2.1209583200843e-08\\
774	-2.11547958040015e-08\\
775	-2.11002209038469e-08\\
776	-2.10458562799332e-08\\
777	-2.09917015991934e-08\\
778	-2.09377557514046e-08\\
779	-2.08840179594105e-08\\
780	-2.08304860027653e-08\\
781	-2.07771597704465e-08\\
782	-2.07240382632534e-08\\
783	-2.06711205930077e-08\\
784	-2.06184042061963e-08\\
785	-2.05658903240646e-08\\
786	-2.05135768371889e-08\\
787	-2.04614624133015e-08\\
788	-2.04095459421794e-08\\
789	-2.03578275348448e-08\\
790	-2.03063050818741e-08\\
791	-2.02549779171335e-08\\
792	-2.02038448193775e-08\\
793	-2.01529055665617e-08\\
794	-2.01021588264183e-08\\
795	-2.00516031556575e-08\\
796	-2.0001237777123e-08\\
797	-1.99510622467258e-08\\
798	-1.99010752321982e-08\\
799	-1.98512761784286e-08\\
800	-1.98016634200826e-08\\
801	-1.97522368461378e-08\\
802	-1.9702994902282e-08\\
803	-1.96539367003368e-08\\
804	-1.9605061463146e-08\\
805	-1.95563683025313e-08\\
806	-1.95078562192919e-08\\
807	-1.94595247693385e-08\\
808	-1.94113725093814e-08\\
809	-1.93633986622643e-08\\
810	-1.93156024508312e-08\\
811	-1.92679830979259e-08\\
812	-1.92205399374146e-08\\
813	-1.91732716370296e-08\\
814	-1.91261773085927e-08\\
815	-1.90792560639252e-08\\
816	-1.9032507792005e-08\\
817	-1.89859310495422e-08\\
818	-1.89395245042689e-08\\
819	-1.88932881561854e-08\\
820	-1.88472213391577e-08\\
821	-1.88013220547845e-08\\
822	-1.87555910802217e-08\\
823	-1.87100261950235e-08\\
824	-1.86646276212343e-08\\
825	-1.86193934714751e-08\\
826	-1.85743238567682e-08\\
827	-1.85294174448458e-08\\
828	-1.84846741246858e-08\\
829	-1.84400924529982e-08\\
830	-1.83956718746714e-08\\
831	-1.83514121676609e-08\\
832	-1.83073113335652e-08\\
833	-1.82633692613621e-08\\
834	-1.82195852849176e-08\\
835	-1.81759581829866e-08\\
836	-1.81324882886358e-08\\
837	-1.80891738255085e-08\\
838	-1.80460147936046e-08\\
839	-1.80030095275896e-08\\
840	-1.79601582495081e-08\\
841	-1.79174599601595e-08\\
842	-1.78749129942091e-08\\
843	-1.78325175737015e-08\\
844	-1.77902731435253e-08\\
845	-1.77481781493682e-08\\
846	-1.77062328132749e-08\\
847	-1.76644359139999e-08\\
848	-1.76227867854095e-08\\
849	-1.75812849834145e-08\\
850	-1.75399291757472e-08\\
851	-1.749871947343e-08\\
852	-1.74576546552174e-08\\
853	-1.74167341659981e-08\\
854	-1.73759574506605e-08\\
855	-1.73353245092045e-08\\
856	-1.72948331211842e-08\\
857	-1.72544835086441e-08\\
858	-1.72142745613613e-08\\
859	-1.71742067234248e-08\\
860	-1.71342784405226e-08\\
861	-1.70944891575431e-08\\
862	-1.7054838430397e-08\\
863	-1.70153250378391e-08\\
864	-1.69759490908916e-08\\
865	-1.69367100344431e-08\\
866	-1.68976063141812e-08\\
867	-1.68586381521507e-08\\
868	-1.68198043271062e-08\\
869	-1.67811048390476e-08\\
870	-1.67425386887743e-08\\
871	-1.67041052101524e-08\\
872	-1.66658038480705e-08\\
873	-1.66276349355954e-08\\
874	-1.65895961412588e-08\\
875	-1.65516882422168e-08\\
876	-1.65139090180233e-08\\
877	-1.64762600229906e-08\\
878	-1.64387395917842e-08\\
879	-1.64013467252033e-08\\
880	-1.63640818673372e-08\\
881	-1.63269433528512e-08\\
882	-1.62899315148124e-08\\
883	-1.62530449099307e-08\\
884	-1.62162837602509e-08\\
885	-1.61796471775943e-08\\
886	-1.6143134717872e-08\\
887	-1.61067452708608e-08\\
888	-1.60704793916722e-08\\
889	-1.60343350819048e-08\\
890	-1.5998313229737e-08\\
891	-1.59624125029012e-08\\
892	-1.59266323462859e-08\\
893	-1.58909725378464e-08\\
894	-1.58554318563375e-08\\
895	-1.58200108568707e-08\\
896	-1.57847084292229e-08\\
897	-1.57495235741933e-08\\
898	-1.5714456957916e-08\\
899	-1.56795063599446e-08\\
900	-1.56446727794801e-08\\
901	-1.56099549952771e-08\\
902	-1.55753526742686e-08\\
903	-1.55408651503208e-08\\
904	-1.55064926454784e-08\\
905	-1.54722333833845e-08\\
906	-1.54380875860838e-08\\
907	-1.54040551425538e-08\\
908	-1.53701346095048e-08\\
909	-1.53363262089812e-08\\
910	-1.53026294968939e-08\\
911	-1.52690435850644e-08\\
912	-1.52355680294036e-08\\
913	-1.52022023858223e-08\\
914	-1.51689463212534e-08\\
915	-1.51357992805856e-08\\
916	-1.5102760597685e-08\\
917	-1.50698306056185e-08\\
918	-1.50370071949624e-08\\
919	-1.50042915869619e-08\\
920	-1.49716824493495e-08\\
921	-1.49391792270137e-08\\
922	-1.49067824750659e-08\\
923	-1.48744899730602e-08\\
924	-1.48423029422418e-08\\
925	-1.48102202723877e-08\\
926	-1.4778241297364e-08\\
927	-1.47463662392155e-08\\
928	-1.47145936546522e-08\\
929	-1.46829242098079e-08\\
930	-1.46513561283257e-08\\
931	-1.46198902983841e-08\\
932	-1.45885254987377e-08\\
933	-1.45572617293865e-08\\
934	-1.45260977690853e-08\\
935	-1.44950341729455e-08\\
936	-1.44640704968779e-08\\
937	-1.44332054086149e-08\\
938	-1.44024392412234e-08\\
939	-1.4371771106525e-08\\
940	-1.43412010045196e-08\\
941	-1.43107283800958e-08\\
942	-1.42803527891644e-08\\
943	-1.4250074120703e-08\\
944	-1.4219891042444e-08\\
945	-1.41898039984767e-08\\
946	-1.41598125447118e-08\\
947	-1.41299155709262e-08\\
948	-1.41001137432539e-08\\
949	-1.40704059514718e-08\\
950	-1.40407917514906e-08\\
951	-1.40112711433105e-08\\
952	-1.39818434607974e-08\\
953	-1.39525080378178e-08\\
954	-1.39232656515276e-08\\
955	-1.38941143035254e-08\\
956	-1.38650547709673e-08\\
957	-1.38360859436304e-08\\
958	-1.38072084876484e-08\\
959	-1.37784214038206e-08\\
960	-1.37497233598793e-08\\
961	-1.372111557707e-08\\
962	-1.3692596723125e-08\\
963	-1.36641664649773e-08\\
964	-1.36358251356938e-08\\
965	-1.3607571736074e-08\\
966	-1.35794054889615e-08\\
967	-1.35513272825349e-08\\
968	-1.35233358955489e-08\\
969	-1.34954314390257e-08\\
970	-1.34676128027422e-08\\
971	-1.34398803197655e-08\\
972	-1.3412233434984e-08\\
973	-1.33846718153308e-08\\
974	-1.33571950167166e-08\\
975	-1.33298027060746e-08\\
976	-1.33024943282933e-08\\
977	-1.32752702164396e-08\\
978	-1.32481290382458e-08\\
979	-1.32210713488234e-08\\
980	-1.3194096593061e-08\\
981	-1.31672039938024e-08\\
982	-1.31403938841146e-08\\
983	-1.31136655978636e-08\\
984	-1.30870188019827e-08\\
985	-1.30604521642042e-08\\
986	-1.30339675719071e-08\\
987	-1.30075626936232e-08\\
988	-1.2981238417531e-08\\
989	-1.29549936334072e-08\\
990	-1.2928828674319e-08\\
991	-1.29027427631101e-08\\
992	-1.28767355667136e-08\\
993	-1.28508075292189e-08\\
994	-1.28249570963135e-08\\
995	-1.27991847120867e-08\\
996	-1.27734900434717e-08\\
997	-1.27478727574015e-08\\
998	-1.27223326318315e-08\\
999	-1.26968685565387e-08\\
1000	-1.267148086459e-08\\
};
\label{addplot:EncadremeentCp1}
\addplot [color=black, dashed, line width=2.0pt, forget plot]
  table[row sep=crcr]{%
1	0\\
2	0\\
3	0\\
4	0\\
5	0\\
6	0\\
7	0\\
8	0\\
9	0\\
10	0\\
11	0\\
12	0\\
13	0\\
14	0\\
15	0\\
16	0\\
17	0\\
18	0\\
19	0\\
20	0\\
21	0\\
22	0\\
23	0\\
24	0\\
25	0\\
26	0\\
27	0\\
28	0\\
29	0\\
30	0\\
31	0\\
32	0\\
33	0\\
34	0\\
35	0\\
36	0\\
37	0\\
38	0\\
39	0\\
40	0\\
41	0\\
42	0\\
43	0\\
44	0\\
45	0\\
46	0\\
47	0\\
48	0\\
49	0\\
50	0\\
51	0\\
52	0\\
53	0\\
54	0\\
55	0\\
56	0\\
57	0\\
58	0\\
59	0\\
60	0\\
61	0\\
62	0\\
63	0\\
64	0\\
65	0\\
66	0\\
67	0\\
68	0\\
69	0\\
70	0\\
71	0\\
72	0\\
73	0\\
74	0\\
75	0\\
76	0\\
77	0\\
78	0\\
79	0\\
80	0\\
81	0\\
82	0\\
83	0\\
84	0\\
85	0\\
86	0\\
87	0\\
88	0\\
89	0\\
90	0\\
91	0\\
92	0\\
93	0\\
94	0\\
95	0\\
96	0\\
97	0\\
98	0\\
99	0\\
100	0\\
101	0\\
102	0\\
103	0\\
104	0\\
105	0\\
106	0\\
107	0\\
108	0\\
109	0\\
110	0\\
111	0\\
112	0\\
113	0\\
114	0\\
115	0\\
116	0\\
117	0\\
118	0\\
119	0\\
120	0\\
121	0\\
122	0\\
123	0\\
124	0\\
125	0\\
126	0\\
127	0\\
128	0\\
129	0\\
130	0\\
131	0\\
132	0\\
133	0\\
134	0\\
135	0\\
136	0\\
137	0\\
138	0\\
139	0\\
140	0\\
141	0\\
142	0\\
143	0\\
144	0\\
145	0\\
146	0\\
147	0\\
148	0\\
149	0\\
150	0\\
151	0\\
152	0\\
153	0\\
154	0\\
155	0\\
156	0\\
157	0\\
158	0\\
159	0\\
160	0\\
161	0\\
162	0\\
163	0\\
164	0\\
165	0\\
166	0\\
167	0\\
168	0\\
169	0\\
170	0\\
171	0\\
172	0\\
173	0\\
174	0\\
175	0\\
176	0\\
177	0\\
178	0\\
179	0\\
180	0\\
181	0\\
182	0\\
183	0\\
184	0\\
185	0\\
186	0\\
187	0\\
188	0\\
189	0\\
190	0\\
191	0\\
192	0\\
193	0\\
194	0\\
195	0\\
196	0\\
197	0\\
198	0\\
199	0\\
200	0\\
201	0\\
202	0\\
203	0\\
204	0\\
205	0\\
206	0\\
207	0\\
208	0\\
209	0\\
210	0\\
211	0\\
212	0\\
213	0\\
214	0\\
215	0\\
216	0\\
217	0\\
218	0\\
219	0\\
220	0\\
221	0\\
222	0\\
223	0\\
224	0\\
225	0\\
226	0\\
227	0\\
228	0\\
229	0\\
230	0\\
231	0\\
232	0\\
233	0\\
234	0\\
235	0\\
236	0\\
237	0\\
238	0\\
239	0\\
240	0\\
241	0\\
242	0\\
243	0\\
244	0\\
245	0\\
246	0\\
247	0\\
248	0\\
249	0\\
250	0\\
251	0\\
252	0\\
253	0\\
254	0\\
255	0\\
256	0\\
257	0\\
258	0\\
259	0\\
260	0\\
261	0\\
262	0\\
263	0\\
264	0\\
265	0\\
266	0\\
267	0\\
268	0\\
269	0\\
270	0\\
271	0\\
272	0\\
273	0\\
274	0\\
275	0\\
276	0\\
277	0\\
278	0\\
279	0\\
280	0\\
281	0\\
282	0\\
283	0\\
284	0\\
285	0\\
286	0\\
287	0\\
288	0\\
289	0\\
290	0\\
291	0\\
292	0\\
293	0\\
294	0\\
295	0\\
296	0\\
297	0\\
298	0\\
299	0\\
300	0\\
301	0\\
302	0\\
303	0\\
304	0\\
305	0\\
306	0\\
307	0\\
308	0\\
309	0\\
310	0\\
311	0\\
312	0\\
313	0\\
314	0\\
315	0\\
316	0\\
317	0\\
318	0\\
319	0\\
320	0\\
321	0\\
322	0\\
323	0\\
324	0\\
325	0\\
326	0\\
327	0\\
328	0\\
329	0\\
330	0\\
331	0\\
332	0\\
333	0\\
334	0\\
335	0\\
336	0\\
337	0\\
338	0\\
339	0\\
340	0\\
341	0\\
342	0\\
343	0\\
344	0\\
345	0\\
346	0\\
347	0\\
348	0\\
349	0\\
350	0\\
351	0\\
352	0\\
353	0\\
354	0\\
355	0\\
356	0\\
357	0\\
358	0\\
359	0\\
360	0\\
361	0\\
362	0\\
363	0\\
364	0\\
365	0\\
366	0\\
367	0\\
368	0\\
369	0\\
370	0\\
371	0\\
372	0\\
373	0\\
374	0\\
375	0\\
376	0\\
377	0\\
378	0\\
379	0\\
380	0\\
381	0\\
382	0\\
383	0\\
384	0\\
385	0\\
386	0\\
387	0\\
388	0\\
389	0\\
390	0\\
391	0\\
392	0\\
393	0\\
394	0\\
395	0\\
396	0\\
397	0\\
398	0\\
399	0\\
400	0\\
401	0\\
402	0\\
403	0\\
404	0\\
405	0\\
406	0\\
407	0\\
408	0\\
409	0\\
410	0\\
411	0\\
412	0\\
413	0\\
414	0\\
415	0\\
416	0\\
417	0\\
418	0\\
419	0\\
420	0\\
421	0\\
422	0\\
423	0\\
424	0\\
425	0\\
426	0\\
427	0\\
428	0\\
429	0\\
430	0\\
431	0\\
432	0\\
433	0\\
434	0\\
435	0\\
436	0\\
437	0\\
438	0\\
439	0\\
440	0\\
441	0\\
442	0\\
443	0\\
444	0\\
445	0\\
446	0\\
447	0\\
448	0\\
449	0\\
450	0\\
451	0\\
452	0\\
453	0\\
454	0\\
455	0\\
456	0\\
457	0\\
458	0\\
459	0\\
460	0\\
461	0\\
462	0\\
463	0\\
464	0\\
465	0\\
466	0\\
467	0\\
468	0\\
469	0\\
470	0\\
471	0\\
472	0\\
473	0\\
474	0\\
475	0\\
476	0\\
477	0\\
478	0\\
479	0\\
480	0\\
481	0\\
482	0\\
483	0\\
484	0\\
485	0\\
486	0\\
487	0\\
488	0\\
489	0\\
490	0\\
491	0\\
492	0\\
493	0\\
494	0\\
495	0\\
496	0\\
497	0\\
498	0\\
499	0\\
500	0\\
501	0\\
502	0\\
503	0\\
504	0\\
505	0\\
506	0\\
507	0\\
508	0\\
509	0\\
510	0\\
511	0\\
512	0\\
513	0\\
514	0\\
515	0\\
516	0\\
517	0\\
518	0\\
519	0\\
520	0\\
521	0\\
522	0\\
523	0\\
524	0\\
525	0\\
526	0\\
527	0\\
528	0\\
529	0\\
530	0\\
531	0\\
532	0\\
533	0\\
534	0\\
535	0\\
536	0\\
537	0\\
538	0\\
539	0\\
540	0\\
541	0\\
542	0\\
543	0\\
544	0\\
545	0\\
546	0\\
547	0\\
548	0\\
549	0\\
550	0\\
551	0\\
552	0\\
553	0\\
554	0\\
555	0\\
556	0\\
557	0\\
558	0\\
559	0\\
560	0\\
561	0\\
562	0\\
563	0\\
564	0\\
565	0\\
566	0\\
567	0\\
568	0\\
569	0\\
570	0\\
571	0\\
572	0\\
573	0\\
574	0\\
575	0\\
576	0\\
577	0\\
578	0\\
579	0\\
580	0\\
581	0\\
582	0\\
583	0\\
584	0\\
585	0\\
586	0\\
587	0\\
588	0\\
589	0\\
590	0\\
591	0\\
592	0\\
593	0\\
594	0\\
595	0\\
596	0\\
597	0\\
598	0\\
599	0\\
600	0\\
601	0\\
602	0\\
603	0\\
604	0\\
605	0\\
606	0\\
607	0\\
608	0\\
609	0\\
610	0\\
611	0\\
612	0\\
613	0\\
614	0\\
615	0\\
616	0\\
617	0\\
618	0\\
619	0\\
620	0\\
621	0\\
622	0\\
623	0\\
624	0\\
625	0\\
626	0\\
627	0\\
628	0\\
629	0\\
630	0\\
631	0\\
632	0\\
633	0\\
634	0\\
635	0\\
636	0\\
637	0\\
638	0\\
639	0\\
640	0\\
641	0\\
642	0\\
643	0\\
644	0\\
645	0\\
646	0\\
647	0\\
648	0\\
649	0\\
650	0\\
651	0\\
652	0\\
653	0\\
654	0\\
655	0\\
656	0\\
657	0\\
658	0\\
659	0\\
660	0\\
661	0\\
662	0\\
663	0\\
664	0\\
665	0\\
666	0\\
667	0\\
668	0\\
669	0\\
670	0\\
671	0\\
672	0\\
673	0\\
674	0\\
675	0\\
676	0\\
677	0\\
678	0\\
679	0\\
680	0\\
681	0\\
682	0\\
683	0\\
684	0\\
685	0\\
686	0\\
687	0\\
688	0\\
689	0\\
690	0\\
691	0\\
692	0\\
693	0\\
694	0\\
695	0\\
696	0\\
697	0\\
698	0\\
699	0\\
700	0\\
701	0\\
702	0\\
703	0\\
704	0\\
705	0\\
706	0\\
707	0\\
708	0\\
709	0\\
710	0\\
711	0\\
712	0\\
713	0\\
714	0\\
715	0\\
716	0\\
717	0\\
718	0\\
719	0\\
720	0\\
721	0\\
722	0\\
723	0\\
724	0\\
725	0\\
726	0\\
727	0\\
728	0\\
729	0\\
730	0\\
731	0\\
732	0\\
733	0\\
734	0\\
735	0\\
736	0\\
737	0\\
738	0\\
739	0\\
740	0\\
741	0\\
742	0\\
743	0\\
744	0\\
745	0\\
746	0\\
747	0\\
748	0\\
749	0\\
750	0\\
751	0\\
752	0\\
753	0\\
754	0\\
755	0\\
756	0\\
757	0\\
758	0\\
759	0\\
760	0\\
761	0\\
762	0\\
763	0\\
764	0\\
765	0\\
766	0\\
767	0\\
768	0\\
769	0\\
770	0\\
771	0\\
772	0\\
773	0\\
774	0\\
775	0\\
776	0\\
777	0\\
778	0\\
779	0\\
780	0\\
781	0\\
782	0\\
783	0\\
784	0\\
785	0\\
786	0\\
787	0\\
788	0\\
789	0\\
790	0\\
791	0\\
792	0\\
793	0\\
794	0\\
795	0\\
796	0\\
797	0\\
798	0\\
799	0\\
800	0\\
801	0\\
802	0\\
803	0\\
804	0\\
805	0\\
806	0\\
807	0\\
808	0\\
809	0\\
810	0\\
811	0\\
812	0\\
813	0\\
814	0\\
815	0\\
816	0\\
817	0\\
818	0\\
819	0\\
820	0\\
821	0\\
822	0\\
823	0\\
824	0\\
825	0\\
826	0\\
827	0\\
828	0\\
829	0\\
830	0\\
831	0\\
832	0\\
833	0\\
834	0\\
835	0\\
836	0\\
837	0\\
838	0\\
839	0\\
840	0\\
841	0\\
842	0\\
843	0\\
844	0\\
845	0\\
846	0\\
847	0\\
848	0\\
849	0\\
850	0\\
851	0\\
852	0\\
853	0\\
854	0\\
855	0\\
856	0\\
857	0\\
858	0\\
859	0\\
860	0\\
861	0\\
862	0\\
863	0\\
864	0\\
865	0\\
866	0\\
867	0\\
868	0\\
869	0\\
870	0\\
871	0\\
872	0\\
873	0\\
874	0\\
875	0\\
876	0\\
877	0\\
878	0\\
879	0\\
880	0\\
881	0\\
882	0\\
883	0\\
884	0\\
885	0\\
886	0\\
887	0\\
888	0\\
889	0\\
890	0\\
891	0\\
892	0\\
893	0\\
894	0\\
895	0\\
896	0\\
897	0\\
898	0\\
899	0\\
900	0\\
901	0\\
902	0\\
903	0\\
904	0\\
905	0\\
906	0\\
907	0\\
908	0\\
909	0\\
910	0\\
911	0\\
912	0\\
913	0\\
914	0\\
915	0\\
916	0\\
917	0\\
918	0\\
919	0\\
920	0\\
921	0\\
922	0\\
923	0\\
924	0\\
925	0\\
926	0\\
927	0\\
928	0\\
929	0\\
930	0\\
931	0\\
932	0\\
933	0\\
934	0\\
935	0\\
936	0\\
937	0\\
938	0\\
939	0\\
940	0\\
941	0\\
942	0\\
943	0\\
944	0\\
945	0\\
946	0\\
947	0\\
948	0\\
949	0\\
950	0\\
951	0\\
952	0\\
953	0\\
954	0\\
955	0\\
956	0\\
957	0\\
958	0\\
959	0\\
960	0\\
961	0\\
962	0\\
963	0\\
964	0\\
965	0\\
966	0\\
967	0\\
968	0\\
969	0\\
970	0\\
971	0\\
972	0\\
973	0\\
974	0\\
975	0\\
976	0\\
977	0\\
978	0\\
979	0\\
980	0\\
981	0\\
982	0\\
983	0\\
984	0\\
985	0\\
986	0\\
987	0\\
988	0\\
989	0\\
990	0\\
991	0\\
992	0\\
993	0\\
994	0\\
995	0\\
996	0\\
997	0\\
998	0\\
999	0\\
1000	0\\
};
\label{addplot:EncadremeentCp2}
\end{axis}
\end{tikzpicture}%
	\captionsetup{width=0.5\textwidth}
	\caption{Numerical values of the first $1000$ terms of the sequences ${v_p = \sqrt{2\pi p}C_p - 1}$~\ref{addplot:EncadremeentCp0} and ${w_p = \sqrt{2\pi (p-1/4)}C_p - 1}$~\ref{addplot:EncadremeentCp1}.}
	\label{figure:encadrementCp}
\end{figure}
\begin{equation}
	\label{estimationNormeInfinieEp}
	\norm{e_p}_{\infty} = O\left(\frac{1}{\sqrt{p}}\right).
\end{equation}
For any $p$, 
\begin{equation}
	\label{epEstUnVP}
	-\Delta e_p = \rho_p^2 e_p.
\end{equation}


\begin{The} 
	\label{epEstUneBaseDeHilbert}
	The family $\left\{e_p, p\in \N^*\right\}$ is a Hilbert basis of $\Hzrad$.
\end{The}
\begin{proof}
	First, notice that $(e_p)_{p \in \N^*}$ is a free family in $\Hzrad$, as it is composed of eigenvectors of the Laplace operator with distinct eigenvalues. On the other hand, in \autoref{SobolevRadialResults}, we have shown all the hypothesis required to apply Theorem 7.3.2 of \cite{allaire2005analyse} (p. 219). Accordingly, the eigenvalues of the bilinear form  
	\[(u,v)\mapsto \int_{B} \nabla u \cdot \nabla v\]
	form an increasing sequence of positive numbers $(\lambda_n)_{n \in \N^*}$ going to infinity, and there exists a Hilbert basis of associated eigenvectors $\left\{u_n, n\in \N^*\right\}$ in $\Hzrad$, that is 							
	\begin{equation}
		\label{formulationVaritionnellePropHilbert}
		\forall n \in \N, \forall v \in \Hzrad, \quad \int_{B}\nabla u_n \cdot \nabla v = \lambda_n \int_B u_n v.
	\end{equation}
	Without loss of generality, we can assume that for all $n$, $\norm{u_n}_{\Hrad} = 1$. In view of \eqref{epEstUnVP}, it is sufficient to show that for any $n \in \N^*$, $\sqrt{\lambda_n}$ is a root of $J_0$. We first obtain that $u_n \in \Cinf_0(B)$ and 
	\[ - \Delta u_n = \lambda_n u_n, \quad n \in \N^*\]
	Indeed, restricting assertion \eqref{formulationVaritionnellePropHilbert} to $v \in \Crad$, and by integration by part, we get, 
	\begin{equation}
		\label{aux1}
		{\mathop-}\int_{B} u_n \Delta v = \lambda_n \int_{B} u_n v.
	\end{equation}
	More generally, for $v \in C^{\infty}_c(B)$, remark that, using \autoref{RadialAveraging},
	\begin{align*}
		-\duality{u_n}{\Delta v} & = -\duality{u_n}{\textup{Rad }\Delta v},  & \text{from (i)}          \\
		                         & =-\duality{u_n}{\Delta \textup{Rad }v},   & \text{from (ii)}         \\
		                         & =\lambda_n \duality{u_n}{\textup{Rad }v}, & \text{from \eqref{aux1}} \\
		                         & =\lambda_n \duality{u_n}{v},              & \text{again from (i)},   
	\end{align*}	
	so $-\Delta u_n = \lambda_n u_n$ in the sense of distributions. By elliptic regularity, this implies that $u_n \in \Crad$ for all $n \in \N$ since $B$ is a $\Cinf$ domain. We now fix $n \in \N$, and let 
	\[f(r) = u_n\left( \frac{r}{\sqrt{\lambda_n}}\right).\]
	Using the classical expression of the Laplace operator in polar coordinates, it is easy to check that $f$ is a solution of Bessel's equation:	
	\[ f''(r) + r f'(r) + r^2 f(r) = 0.\]
	All smooth solutions of this equation are proportional to $J_0$, so there exists a constant $A_n$ such that
	\[u_n(\bs{x}) = A_n J_0\big(\sqrt{\lambda_n} \abs{\bs{x}}\big).\]
	Since $u_n$ is in $\Crad$, it must vanish on $\mathcal{C}$ so $\sqrt{\lambda_n}$ must be a root of $J_0$. \qedhere
\end{proof}

\subsection{Truncature error for Fourier-Bessel series of smooth functions}
\label{FourierBesselTruncError}
We now introduce the Fourier-Bessel series and prove a bound for the norm of the remainder. 
In \autoref{epEstUneBaseDeHilbert}, we have shown that any function $f \in \Hzrad$ can be expanded through its so-called Fourier-Bessel series as
\[f = \sum_{p\in \mathbb{N}^*}c_p(f)e_{p}.\]
The generalized Fourier coefficients are obtained by the orthonormal projection: 
\[c_p(f) = \displaystyle \int_B \nabla f(\bs{x}) \cdot \nabla e_{p}(\bs{x}) dx = \rho_k^2 \int_{B}{f(\bs{x}) e_p(\bs{x})d\bs{x}}.\]
Most references on this topic focus on proving pointwise convergence of the series even for not very regular functions $f$ (e.g. piecewise continuous, square integrable, etc.) \cite{stempak2002convergence,guadalupe1993mean,Balodis1999,colzani1993equiconvergence}.  In such cases, the Fourier-Bessel series may exhibit a Gibb's phenomenon \cite{wilton1928gibbs}. On the contrary here, we need to establish that the Fourier-Bessel series of very smooth functions converges exponentially fast. To this aim, we first introduce the following terminology: 
\begin{Def}
	We say that a radial function $f$ satisfies the multi-Dirichlet condition of order $n \in \N^*$ if 
	\begin{itemize}
		\item[(i)] $f$ is in $H^{2n}(B) ;$
		\item[(ii)] for all $s \leq n-1$, the $s-th$ iterate of $-\Delta$ on $f$, denoted by $(-\Delta)^s f$, vanishes on $\mathcal{C}$ (with the convention $(-\Delta)^0 f = f$). 
	\end{itemize} 
\end{Def}
\begin{Prop} 
	\label{DecroissanceFourierBessel}
	If $f$ satisfies the multi-Dirichlet condition of order $n$, then for any $p \in \mathbb{N}^*$:
	\[ c_p(f) = \dfrac{1}{\rho_{p}^{2n-2}} \int_{B}\left(-\Delta\right)^{n} f(\bs{x})e_p(\bs{x}) d\bs{x}.\] 
	\begin{proof}
		Let $f \in \Hzrad$, we have:
		\[c_p(f) = \int_{B} \nabla  f(\bs{x})\cdot \nabla e_p(\bs{x})d\bs{x}. \] 
		If $f$ satisfies the multi-Dirichlet condition of order $n=1$, then by integration by parts:
		\[c_p(f) = \int_{B}(-\Delta)f(\bs{x}) e_p(\bs{x})d\bs{x},\]
		since $e_p$ vanishes on $\mathcal{C}$.
		Assume the result is true for some $n \geq 1$ and let $f$ satisfy the multi-Dirichlet condition of order $n+1$. Then, using the fact that $e_p$ is an eigenvector of $-\Delta$ associated to the eigenvalue $\rho_p^2$, 
		\[c_p(f) = \frac{1}{\rho_p^{2n}}\int_{B}(-\Delta)^{n}f(\bs{x})~ (-\Delta) e_p(\bs{x}) d\bs{x}.\]
		The result follows from integration by parts where we successively use that $(-\Delta)^{n}f$ and $e_p$ vanish on $\mathcal{C}$.
	\end{proof}
	\label{PropDecrCond}
\end{Prop}

\begin{Cor} If $f$ satisfies the multi-Dirichlet condition of order $n$, there exists a constant $C$ independent of the function $f$ such that for all $p \in \mathbb{N^*}$, 
	\[ |c_p(f)| \leq  C \dfrac{\norm{(-\Delta)^n f}_{\Lrad}}{(\pi p)^{2n-1}}.\] 
\end{Cor}
\noindent Notice that this is similar to the fact that the Fourier coefficients of smooth functions decay fast. 
\begin{proof}
	We apply the result of the previous proposition and remark that since $e_p$ is an eigenfunction of the Laplace operator with unit norm in $\Hzrad$, $\norm{e_p}_{\Lrad} = \dfrac{1}{\rho_p}\norm{e_p}_{\Hzrad} = \dfrac{1}{\rho_p}$. To conclude, recall $\rho_{p} \sim p\pi$ for large $p$.\qedhere
\end{proof}

\begin{Cor} Let the remainder be defined as 
	\[R_P(f) = \displaystyle\sum_{k = P+1}^{+\infty} c_{k}(f) e_{k}.\]
	If $f$ satisfies the multi-Dirichlet condition of order $n$, there exists a constant $C$ independent of $n$ and $P$ such that: 
	\[\norm{R_P(f)}_{\Hzrad} \leq C\dfrac{\norm{(-\Delta)^n f}_{\Lrad}}{(\pi P)^{2n}}\sqrt{\dfrac{P^3}{n}}.\]
	\label{EstimationRest}
																																																								
	\begin{proof}
		One has 
		\[R_P(f) = \sum_{p=P+1}^{+\infty}c_p(f) e_p,\]
		and Parseval's identity implies
		\[\norm{R_P(f)}_{\Lrad} = \sum_{p=P+1}^{+\infty}|c_p(f)|^2.\]
																																				
		According to the previous results, we find that:
		\[\norm{R_P}_{\Lrad} \leq C \norm{(-\Delta)^n f}_{\Lrad}\sqrt{\sum_{p= P+1}^{+\infty} \dfrac{1}{(\pi p)^{4n-2}}}.\]
		The announced result follows from $\displaystyle\sum_{p > P} \frac{1}{p^{\alpha}} \propto \frac{1}{P^{\alpha-1}}$ for $\alpha > 1$. \qedhere
	\end{proof}
\end{Cor}

\subsection{Other boundary conditions}

When we replace the Dirichlet boundary condition by the following Robin boundary conditions
\begin{equation}
	\label{robinCondition}
	\dfrac{\partial u}{\partial n} + H u = 0
\end{equation}
for some constant $H \geq 0$, the same analysis can be conducted, leading to Dini series (also covered in \cite{watson1995treatise}). This time, we construct a Hilbert basis of $H^1(B)$ with respect to the bilinear form
\[a_H(u,v) \isdef \int_{B} \nabla u(\bs{x} ) \cdot \nabla v(\bs{x} ) d\bs{x} + H \int_{\mathcal{C}}{u(\bs{x} )v(\bs{x} )}d\sigma(\bs{x} ).\]
\begin{The}
	Let $(\rho_p^H)_{p \in \N*}$ the sequence of positive roots of the function
	\[r J_0'(r) + H J_0(r).\]
	\item[-] If $H>0$, the functions 
	\[e_p^H(r) = C_p J_0(\rho_p^H r),\]
	with $C_p$ such that $a_H(e_p^H,e_p^H) = 1$, form a Hilbert basis of $H^1(B)$. 
	\item[-]If $H = 0$, a constant function must be added to form a complete family. 
\end{The}
It can be easily checked that the truncature error estimates \autoref{EstimationRest} extend to this case, for functions satisfying multi-Robin conditions of order $n \geq 1$, that is 
\item[-] $u \in H^{2n}(B)$
\item[-] For all $s\leq n-1$, $(-\Delta)^s u$ satisfies \eqref{robinCondition}.

%The case $H < 0$ can be treated, with a slight change. In this instance, the function $r J_0'(r) + H J_0(r)$ also has two purely imaginary roots $\pm i \lambda_0$, and an eigenfunction proportional to $I_0(\lambda_0 r)$ must be added, where $I_0$ is the hyperbolic Bessel function of first kind. 

\section{Sparse Bessel Decomposition}
\label{sec:SBD}
\subsection{Choice of the coefficients}

Consider the kernel $G$ involved in \eqref{discreteConv}. Up to a rescaling, we can assume without loss of generality that the diameter $\rmax$ of $\bs{z}$ is bounded by some $b \leq 1$. 
If we wish to approximate $G$ in series of Bessel functions, the multi-Dirichlet condition of order $n$ is too restrictive in applications of interest. Two kinds of complications are encountered:
\begin{itemize}
	\item[(i)] $G$ is usually singular near the origin, therefore not in $H^{2n}(B)$ (even for $n=1$). 
	\item[(ii)] Conditions of the type 
	      \[(-\Delta)^s G \text{ vanishes on }\mathcal{C}\]   
	      may not be fulfilled (though for a special category of kernels, among which the Laplace and Helmholtz equation, they are indeed up to appropriate rescaling). 
\end{itemize}
Here is a strategy to overcome this problem. For the approximation 
\[G \approx \sum_{p = 1}^P \alpha_p e_p,\]
we know by \autoref{epEstUneBaseDeHilbert} that the minimal $H^1_0$ error on $B$ is reached for $\alpha_p = c_p(G)$. However, if the closest interaction in \eqref{discreteConv} are computed explicitly, it can be sufficient to approximate $G$ in a domain of the form $a \leq r \leq b$ for some $a$. For this reason, we propose to define the coefficients $(\alpha_1,\cdots,\alpha_P)$ as the minimizers of the quadratic form
\[ Q^P(t_1,t_2,...,t_P) = \bigintsss_{\mathcal{A}(a,b)} \left|\nabla \!\left( G(\bs{x}) - \sum_{p=1}^P t_p J_0(\rho_p |\bs{x}|)\right)\right|^2d\bs{x},\]
where $\mathcal{A}(a,b)$ is the annulus $\enstq{\bs{x} \in \R^2}{a < \abs{\bs{x}} < b}$. In the sequel, those coefficients will be called the SBD coefficients of $G$ of order $P$. Obviously, for any radial function $\tilde{G}$ defined on $B$ that coincides with $G$ on $\mathcal{A}(a,b)$, one has 
\[ Q^P(\alpha_1,\cdots,\alpha_P) \leq \bigintsss_{B} \left|\nabla \!\left( \tilde{G}(\bs{x}) - \sum_{p=1}^P c_p(\tilde{G}) J_0(\rho_p |\bs{x}|)\right)\right|^2d\bs{x} \]
In particular, when $\tilde{G}$ satisfies the multi-Dirichlet condition of order $n$ for some $n \geq 1$, this gives an error estimation via \autoref{EstimationRest}. By setting the parameters $a>0$ and $b<1$, we allow such extensions to exist and therefore ensure the fast decay of the coefficients. 
\begin{Rem}
	\label{RemarqueQuiTuePotts}
	The SBD weights do not depend on any specific extension of $G$ outside the annulus. Therefore, they provide a sparser approximation of the kernel than the usual approach where an explicit regularization $\tilde{G}$ of the kernel is constructed and the coefficients $c_p(\tilde{G})$ are used (see e.g. \cite{potts2004fast}). 
\end{Rem}

For $b = 1$, the next result shows that the $H_0^1$ norm on $\mathcal{A}(a,1)$ controls the $L^{\infty}$ norm on the same domain, thus ruling out any risk of Gibb's phenomenon in this case.
\begin{Lem}
	Let $e \in \Lrad$ and $a\in (0,1)$ such that:
	\begin{itemize}
		\item[(i)]$\displaystyle \int_{\mathcal{A}(a,1)} \abs{\nabla e(\bs{x})}^2 d\bs{x} < +\infty$,
		\item[(ii)] The function $e$ vanishes on $\mathcal{C}$.
	\end{itemize}
	Then the restriction of $e$ to $\mathcal{A}(a,1)$ is equal almost everywhere to a continuous function with
	\[\abs{e(\bs{x}_0)} \leq \sqrt{\dfrac{-\log \abs{\bs{x}_0}}{2\pi}}\sqrt{\int_{\mathcal{A}(a,1)} \abs{\nabla e(\bs{x})}^2} d\bs{x},\quad  \forall \bs{x}_0 \in \mathcal{A}(a,1).\]
	\begin{proof}
		It is sufficient to show the inequality for smooth $e$, the general result following by density. Let $\bs{x}_0 \in \mathcal{A}(a,1)$, we have, since $e(1)=0$:
		\begin{align*}
			\abs{e(\bs{x}_0)} & \leq \int_{\abs{\bs{x}_0}}^1 \abs{e'(r)}dr,                                                                             \\
			                  & \leq \sqrt{2\pi \int_{\abs{\bs{x}_0}}^1 r \abs{e'(r)}^2 dr} \sqrt{\int_{\abs{\bs{x}_0}}^1 \frac{1}{2\pi r} dr}.\qedhere 
		\end{align*}					
	\end{proof}
\end{Lem}

\subsection{Numerical computation}
\label{sub:Chol}

For a given kernel $G$, the SBD coefficients $\alpha_p$ are obtained numerically by solving the following linear system: 
\begin{equation}
	\sum_{q = 1}^P \left(\int_{\mathcal{A}(a,b)} \rho_ q J_1(\rho_p|\bs{x}|) J_1(\rho_q|\bs{x}|) d\bs{x}\right) \alpha_q = -\int_{\mathcal{A}(a,b)} G'(\bs{x}) J_1(\rho_p|\bs{x}|)d\bs{x} \quad 1\leq p \leq P,
	\label{LinearSystem}
\end{equation}
Where $J_1$ is the Bessel function of first kind and order $1$ (in fact, $J_0' = - J_1$). We solve this system for increasing values of $P$ until a required tolerance is reached. It turns out that the matrix $A^P$ whose entries are given by
\[ A^P_{k,l} = \int_{\mathcal{A}(a,b)} \!\!\!\!\!\!\!\! \nabla e_k \cdot \nabla e_l,\quad k,l \in \{1,\cdots,P\},\]
is explicit:
\begin{Prop}
	For $(i,j) \in \{1,\cdots,P\}^2$, the non-diagonal entries of $A^P$ are
	\begin{equation*}
		A_{i,j} = \frac{2\pi C_i C_j \rho_i \rho_j}{\rho_j^2 - \rho_i^2}\bigg[F_{i,j}(b) - F_{j,i}(b) - F_{i,j}(a) + F_{j,i}(a)\bigg]
	\end{equation*}
	where 
	\[	 F_{i,j}(r) =  \rho_i r J_0(\rho_i r)J_0'(\rho_j r)\]
	while the diagonal entries are
	\begin{equation*}
		A_{i,i} = 2\pi C_i^2 \big(F_i(b) - F_i(a)\big)
	\end{equation*}
	where 
	\[F_i(r) = \rho_i^2r^2\left[\dfrac{1}{2}J_0(\rho_ir)^2 + \frac{1}{2}J_0'(\rho_ir)^2\right] + \rho_irJ_0(\rho_i r)J_0'(\rho_ir)\]
\end{Prop}

\subsection{Stability}
\newcommand{\Pa}{\gamma}
\newcommand{\Pastar}{\Pa^*}
Here we discuss the numerical stability of inverting the matrix $A$, in the case $b=1$. It will appear in the sequel that the conditioning of $A$ depends mainly on the parameter $\Pa \isdef Pa$. We were able to compute an accurate estimate of the conditioning of $A$ when $\Pa$ is small enough. For large $\Pa$, we will show some numerical evidence for a conjectured bound of the conditioning of $A$.   
\vspace{0.3cm}
\noindent \begin{minipage}{0.5\textwidth}
\subsubsection*{Conditioning of $A$ for small $\Pa$:}
\begin{The} Assume $b=1$. The eigenvalues of $A$ lie in the interval~ ${[F(\Pa) - \frac{\pi^4}{144}\frac{\gamma^4}{P},1]}$ where
	\[F(\Pa) = 1 - \int_{0}^{\pi \gamma} \frac{t}{2}(J_1(t)^2 - J_0(t)J_2(t) )dt.\]
	\label{The:lowBoundCon}
\end{The}
A plot of $F$ is provided at right, \autoref{figure:Fconditionnement}, and some numerical approximations of $\lambda_{\min}$ are shown in function of $\Pa$ for several values of $P$. 			
This estimate is only useful when ${F(\Pa) > 0}$, that is ${\Pa < \Pastar}$ where $\Pastar$ is the first positive root of $F(z)$. One has 
\[\Pastar \approx 1.471.\]
In particular, for $\gamma = 1$, $A$ is well conditioned, the ratio of its largest to its smallest eigenvalues being of the order $F(1)^{-1} < 2$. 
\end{minipage}%
%
\begin{minipage}{0.5\textwidth}
	\begin{figure}[H]
		\centering			
		% This file was created by matlab2tikz.
%
%The latest updates can be retrieved from
%  http://www.mathworks.com/matlabcentral/fileexchange/22022-matlab2tikz-matlab2tikz
%where you can also make suggestions and rate matlab2tikz.
%
\definecolor{mycolor1}{rgb}{0.00000,0.44700,0.74100}%
\definecolor{mycolor2}{rgb}{0.85000,0.32500,0.09800}%
%
\begin{tikzpicture}[%
trim axis left, trim axis right
]

\begin{axis}[%
width=0.666\textwidth,
height=5cm,
at={(0\textwidth,0cm)},
scale only axis,
xmin=0,
xmax=1.471,
xtick={0,0.5,1,1.471},
xticklabels={{0},{0.5},{1},{$\Pastar$}},
xlabel style={font=\color{white!15!black}},
xlabel={$\Pa$},
ymin=0.000784968713614553,
ymax=1,
ytick={  0, 0.5,   1},
ylabel style={font=\color{white!15!black}},
ylabel={$\lambda_{\min}(\gamma)$},
axis background/.style={fill=white},
axis x line*=bottom,
axis y line*=left,
legend style={at={(0.03,0.03)}, anchor=south west, legend cell align=left, align=left, fill=none, draw=none}
]
\addplot [color=black, dashed]
  table[row sep=crcr]{%
0	1\\
0.0377179487179487	0.999996924368755\\
0.0754358974358974	0.999951019631364\\
0.113153846153846	0.999753962524939\\
0.150871794871795	0.999230836707368\\
0.188589743589744	0.998148295462124\\
0.226307692307692	0.996225460697404\\
0.264025641025641	0.993147059356719\\
0.30174358974359	0.988578196925494\\
0.339461538461538	0.982180094373035\\
0.377179487179487	0.973626072973842\\
0.414897435897436	0.962617062865565\\
0.452615384615385	0.948895936193492\\
0.490333333333333	0.932260022937621\\
0.528051282051282	0.912571254166255\\
0.565769230769231	0.889763489274682\\
0.60348717948718	0.863846715341323\\
0.641205128205128	0.834907951752229\\
0.678923076923077	0.803108844795402\\
0.716641025641026	0.768680087834499\\
0.754358974358974	0.731912945847479\\
0.792076923076923	0.693148291895203\\
0.829794871794872	0.65276367154735\\
0.867512820512821	0.611158994539515\\
0.905230769230769	0.568741507318095\\
0.942948717948718	0.52591072341511\\
0.980666666666667	0.483043980060138\\
1.01838461538462	0.440483249887302\\
1.05610256410256	0.398523768313533\\
1.09382051282051	0.357404943781229\\
1.13153846153846	0.317303904363371\\
1.16925641025641	0.278331905924849\\
1.20697435897436	0.240533690448257\\
1.24469230769231	0.203889744904201\\
1.28241025641026	0.168321277797193\\
1.32012820512821	0.133697608532016\\
1.35784615384615	0.0998455596647925\\
1.3955641025641	0.0665603586637413\\
1.43328205128205	0.0336174976170942\\
1.471	0.000784968713614553\\
};
\label{addplot:Fconditionnement0}
\addlegendentry{$F(\Pa)$}

\addplot [color=mycolor1, draw=none, mark size=3.2pt, mark=o, mark options={solid, mycolor1}]
  table[row sep=crcr]{%
0	1\\
0.0377179487179487	0.999997101718933\\
0.0754358974358974	0.999953837628687\\
0.113153846153846	0.999768064727491\\
0.150871794871795	0.999274691332944\\
0.188589743589744	0.998253158228471\\
0.226307692307692	0.996437445862295\\
0.264025641025641	0.993528166936056\\
0.30174358974359	0.989206213238702\\
0.339461538461538	0.983147359366953\\
0.377179487179487	0.975037186998085\\
0.414897435897436	0.964585683345223\\
0.452615384615385	0.951540886680145\\
0.490333333333333	0.935700999339013\\
0.528051282051282	0.916924462099192\\
0.565769230769231	0.895137579688189\\
0.60348717948718	0.870339400871073\\
0.641205128205128	0.842603682640683\\
0.678923076923077	0.812077900502996\\
0.716641025641026	0.778979399401877\\
0.754358974358974	0.743588906144569\\
0.792076923076923	0.706241738238206\\
0.829794871794872	0.667317140366257\\
0.867512820512821	0.627226253700506\\
0.905230769230769	0.586399271316457\\
0.942948717948718	0.545272352895573\\
0.980666666666667	0.504274862829322\\
1.01838461538462	0.463817458492996\\
1.05610256410256	0.424281492103155\\
1.09382051282051	0.386010104001158\\
1.13153846153846	0.349301282587764\\
1.16925641025641	0.314403052778701\\
1.20697435897436	0.28151083785058\\
1.24469230769231	0.250766926311195\\
1.28241025641026	0.222261873135081\\
1.32012820512821	0.196037579696504\\
1.35784615384615	0.172091733932315\\
1.3955641025641	0.150383254696418\\
1.43328205128205	0.130838372736364\\
1.471	0.113356993771856\\
};
\label{addplot:Fconditionnement1}
\addlegendentry{$P=$50}

\addplot [color=mycolor2, draw=none, mark size=3.2pt, mark=x, mark options={solid, mycolor2}]
  table[row sep=crcr]{%
0	1\\
0.0377179487179487	0.999996942735631\\
0.0754358974358974	0.999951311454305\\
0.113153846153846	0.999755422851358\\
0.150871794871795	0.999235378334443\\
0.188589743589744	0.998159158691741\\
0.226307692307692	0.996247438091064\\
0.264025641025641	0.993186628281385\\
0.30174358974359	0.988643564539687\\
0.339461538461538	0.98228117319645\\
0.377179487179487	0.973774419785799\\
0.414897435897436	0.962825828820624\\
0.452615384615385	0.949179891140021\\
0.490333333333333	0.932635731385603\\
0.528051282051282	0.913057493588433\\
0.565769230769231	0.890382012862488\\
0.60348717948718	0.864623470398719\\
0.641205128205128	0.835874871008106\\
0.678923076923077	0.804306330425303\\
0.716641025641026	0.770160306217196\\
0.754358974358974	0.733744044260151\\
0.792076923076923	0.695419635574523\\
0.829794871794872	0.655592179790133\\
0.867512820512821	0.614696626660516\\
0.905230769230769	0.573183912160477\\
0.942948717948718	0.531507018641233\\
0.980666666666667	0.49010756883268\\
1.01838461538462	0.449403512520266\\
1.05610256410256	0.409778385638738\\
1.09382051282051	0.37157251918959\\
1.13153846153846	0.335076456193314\\
1.16925641025641	0.300526706408263\\
1.20697435897436	0.268103839085765\\
1.24469230769231	0.237932792022353\\
1.28241025641026	0.210085168511418\\
1.32012820512821	0.184583209139121\\
1.35784615384615	0.161405067469997\\
1.3955641025641	0.140490989899645\\
1.43328205128205	0.121750000065475\\
1.471	0.105066714406351\\
};
\label{addplot:Fconditionnement2}
\addlegendentry{$P=$500}

\end{axis}
\end{tikzpicture}%
		\captionsetup{width=0.7\textwidth}
		\caption{Graph of $F$\ref{addplot:Fconditionnement0}, and numerical estimation of the curves $\lambda_{\min}(\Pa)$, where \ref{addplot:Fconditionnement2} corresponds to ${P=50}$~and~\ref{addplot:Fconditionnement3} to ${P=500}$. }
		\label{figure:Fconditionnement}
	\end{figure}
\end{minipage}
\begin{proof}
	Let $u \in \text{span}\{e_1,e_2,\cdots,e_P\}$ and let $\alpha$ its coordinates on this basis. Then 
	\[\alpha^T A \alpha = \int_{\mathcal{A}(a,1)} \!\!\!\!\! \nabla{u}^2 < \int_{B} \nabla{u}^2 = \norm{\alpha}_2^2,\]
	showing that the eigenvalues of $A$ are bounded by $1$. For the lowest eigenvalue, fix $P>0$, and let us note $A = A(a)$ to highlight the dependence in $a$. We have
	\[\lambda_{\min}(A) \geq 1 - \text{tr}(I - A),\]
	which yields:
	\[\lambda_{\min} \geq 1 - 2\pi \sum_p^P \int_{0}^a u \rho_p^2 C_p^2 J_1(\rho_p u)^2 du.\]	
	\begin{eqnarray*}
		\lambda_{\min} &\geq& 1 - 2\pi \sum_{p=1}^{P}C_p^2\int_{0}^{\rho_p a} u J_1(u)^2 du
	\end{eqnarray*}
	We now use \ref{EncadrementCp}, which implies
	\[2\pi C_p^2 \leq \dfrac{1}{p} + \dfrac{1}{3p^2},\]
	combined with \ref{EncadrementRhop}, to get
	\[\lambda_{\min} \geq 1 - \int_{0}^{P} \frac{dt}{t} \int_{0}^{\pi t a} u J_1(u)^2du \hspace{1pt} - \frac{1}{3}\sum_{p=1}^P \frac{1}{p^2} \int_{0}^{\rho_p a} u J_1(u)^2du\]
	Note that $\int u J_1(u)^2 = \frac{u^2}{2} \left(J_1^2(u) - J_0(u)J_2(u)\right)$, and $J_1(u) \leq \dfrac{u}{2}$ for all $u \geq 0$. Therefore
	\[\lambda_{\min} \geq 1 - \int_{0}^{\pi\gamma} \dfrac{t}{2} \left(J_1(t)^2 - J_0(t)J_2(t)\right)dt - \frac{1}{48} \sum_{p=1}^P \frac{(\rho_p a)^4}{p^2}\]
	Using again \ref{EncadrementRhop},
	\[\lambda_{\min} \geq 1 - F(\gamma) - \gamma^4\frac{\pi^4}{48} \frac{1}{P^4} \sum_{p=1}^P p^2,\]
	which obviously implies the announced result.
\end{proof}
\subsubsection*{Conditioning of $A$ for large $\Pa$:}

The behavior of $\lambda_{\min}$ is more difficult to study for large $\Pa$. Based on numerical observations, we make the following conjecture:

	\begin{Conj}
		For any $P \geq 10$, and $\gamma \geq 1.4$, the minimal eigenvalue of $A$ is bounded below as
		\[\lambda_{\min}(\gamma) \geq 180 \exp(-5.8\gamma)\]
	\end{Conj}
	\begin{figure}[H]
		\centering
		% This file was created by matlab2tikz.
%
%The latest updates can be retrieved from
%  http://www.mathworks.com/matlabcentral/fileexchange/22022-matlab2tikz-matlab2tikz
%where you can also make suggestions and rate matlab2tikz.
%
\definecolor{mycolor1}{rgb}{0.00000,0.44700,0.74100}%
\definecolor{mycolor2}{rgb}{0.85000,0.32500,0.09800}%
\definecolor{mycolor3}{rgb}{0.92900,0.69400,0.12500}%
\definecolor{mycolor4}{rgb}{0.49400,0.18400,0.55600}%
%
\begin{tikzpicture}[%
trim axis left, trim axis right
]

\begin{axis}[%
width=0.761\linewidth,
height=5cm,
at={(0\linewidth,0cm)},
scale only axis,
xmin=0,
xmax=10,
xtick={0,1.471,2,4,6,8,10},
xticklabels={{0},{$\Pastar$},{2},{4},{6},{8},{10}},
xlabel style={font=\color{white!15!black}},
xlabel={$\Pa$},
ymode=log,
ymin=1.03349875794293e-18,
ymax=180,
ytick={ 1e-20,  1e-16,  1e-12,  1e-08, 0.0001,      1},
yminorticks=true,
ylabel style={font=\color{white!15!black}},
ylabel={$\lambda_{\min}(\gamma)$},
axis background/.style={fill=white},
axis x line*=bottom,
axis y line*=left,
legend style={
	% the /tikz/ prefix is necessary here...
	% otherwise, it might end-up with `/pgfplots/column 2`
	% which is not what we want. compare pgfmanual.pdf
	/tikz/column 5/.style={
		column sep=5pt,
	},
	draw=none,
},
]
\addplot [color=red]
  table[row sep=crcr]{%
0	180\\
0.163333333333333	69.6632048372352\\
0.326666666666667	26.9609006010811\\
0.49	10.4343485620525\\
0.653333333333333	4.03827867345208\\
0.816666666666667	1.56288574676962\\
0.98	0.604864610635595\\
1.14333333333333	0.234093373719456\\
1.30666666666667	0.0905983035803216\\
1.47	0.0350631565567887\\
1.6	0.0164709442695767\\
1.95	0.00215419398357044\\
2.3	0.000281741692698372\\
2.65	3.68482977902385e-05\\
3	4.8192975524277e-06\\
3.35	6.30303983946532e-07\\
3.7	8.24358960734311e-08\\
4.05	1.07815865590435e-08\\
4.4	1.41009698768366e-09\\
4.75	1.84423090589269e-10\\
5.1	2.41202389903467e-11\\
5.45	3.15462628401092e-12\\
5.8	4.12585754053077e-13\\
6.15	5.39610683237929e-14\\
6.5	7.05743440252295e-15\\
6.85	9.23024356134787e-16\\
7.2	1.20720068147351e-16\\
7.55	1.57886785507241e-17\\
7.9	2.06496214095756e-18\\
8.25	2.70071281132799e-19\\
8.6	3.53219535825914e-20\\
8.95	4.61967077601725e-21\\
9.3	6.0419529256463e-22\\
9.65	7.90212050288092e-23\\
10	1.03349875794293e-23\\
};
\label{addplot:minEigA10}
\addlegendentry{$\gamma \mapsto 180\exp(-5.8 \gamma)$}

\addplot [color=black,line width = 2pt]
  table[row sep=crcr]{%
0	1\\
0.163333333333333	0.998947980761158\\
0.326666666666667	0.984576199165217\\
0.49	0.932420261543713\\
0.653333333333333	0.824985189209352\\
0.816666666666667	0.666979162874063\\
0.98	0.483799940289479\\
1.14333333333333	0.304992132857605\\
1.30666666666667	0.145956468633108\\
1.47	0.00165597587137634\\
1.6	-0.113625277851508\\
1.95	-0.465781102510399\\
2.3	-0.826987216510686\\
2.65	-1.15635561677895\\
3	-1.51285907234606\\
3.35	-1.86738664882572\\
3.7	-2.20367369473937\\
4.05	-2.56142889568751\\
4.4	-2.91194539677722\\
4.75	-3.25264587087012\\
5.1	-3.61043555667851\\
5.45	-3.95834574863172\\
5.8	-4.30240467404448\\
6.15	-4.6595456328905\\
6.5	-5.00579703589411\\
6.85	-5.35257422107669\\
7.2	-5.70863848956782\\
7.55	-6.05396027668521\\
7.9	-6.40294775689022\\
8.25	-6.75767839040843\\
8.6	-7.10266265207968\\
8.95	-7.45339310678558\\
9.3	-7.80667043429857\\
9.65	-8.15179903500974\\
10	-8.50381916638207\\
};
\label{addplot:minEigA11}
\addlegendentry{Theorem 4}

\addplot [color=mycolor1, draw=none, mark size=2.5pt, mark=o, mark options={solid, mycolor1}]
  table[row sep=crcr]{%
0	1\\
0.163333333333333	0.99892728189841\\
0.326666666666667	0.984286954771341\\
0.49	0.931268489105689\\
0.653333333333333	0.822510408414748\\
0.816666666666667	0.66394975423407\\
0.98	0.483874225525722\\
1.14333333333333	0.317158559751126\\
1.30666666666667	0.18773983217823\\
1.47	0.101483921190678\\
1.6	0.0588966988176822\\
1.95	0.0114569496014205\\
2.3	0.00190964070579921\\
2.65	0.000292270567114186\\
3	4.23713821428007e-05\\
3.35	5.90799513867324e-06\\
3.7	7.99254091343372e-07\\
4.05	1.05499154147602e-07\\
4.4	1.36399978356425e-08\\
4.75	1.73215798033711e-09\\
5.1	2.16499687336789e-10\\
5.45	2.66737455553842e-11\\
5.8	3.24317523043376e-12\\
6.15	3.89530586146216e-13\\
6.5	4.61323971601161e-14\\
6.85	5.27158843930575e-15\\
7.2	4.22996783184077e-16\\
7.55	-1.0918533325837e-16\\
7.9	-1.28286676674982e-16\\
8.25	-4.88514411450117e-17\\
8.6	-2.5550023789678e-16\\
8.95	-8.44734935398252e-17\\
9.3	-5.60913619543321e-16\\
9.65	-2.85115003655551e-16\\
10	-4.93834544031521e-16\\
};
\label{addplot:minEigA12}
\addlegendentry{$P=50$}

\addplot [color=mycolor2, draw=none, mark size=2.5pt, mark=x, mark options={solid, mycolor2}]
  table[row sep=crcr]{%
0	1\\
0.163333333333333	0.998941076683106\\
0.326666666666667	0.984480316721095\\
0.49	0.932051622472123\\
0.653333333333333	0.824311984063844\\
0.816666666666667	0.666849404139121\\
0.98	0.487449130772276\\
1.14333333333333	0.320702945664034\\
1.30666666666667	0.190661511720668\\
1.47	0.103546565642497\\
1.6	0.0603245136609374\\
1.95	0.011857030145304\\
2.3	0.00199733343781949\\
2.65	0.000309184688567637\\
3	4.53932569624144e-05\\
3.35	6.42014873425208e-06\\
3.7	8.82670910690578e-07\\
4.05	1.1866099419991e-07\\
4.4	1.56626384228797e-08\\
4.75	2.03603885159933e-09\\
5.1	2.61258363672367e-10\\
5.45	3.31508663269631e-11\\
5.8	4.16581021021911e-12\\
6.15	5.19089361549225e-13\\
6.5	6.43884613536526e-14\\
6.85	7.99990416342404e-15\\
7.2	1.14763137965058e-15\\
7.55	3.3301771079726e-16\\
7.9	1.27907579588265e-16\\
8.25	1.95138298142733e-16\\
8.6	1.70829548357877e-16\\
8.95	1.3811461816499e-16\\
9.3	4.03254001966438e-17\\
9.65	3.34978458784725e-18\\
10	1.04409290769306e-17\\
};
\label{addplot:minEigA13}
\addlegendentry{$P=150$}

\addplot [color=mycolor3, draw=none, mark size=1.8pt, mark=square, mark options={solid, mycolor3}]
  table[row sep=crcr]{%
0	1\\
0.163333333333333	0.998945911371084\\
0.326666666666667	0.984548069087991\\
0.49	0.9323259167487\\
0.653333333333333	0.824942618874636\\
0.816666666666667	0.667863575835183\\
0.98	0.488698032445799\\
1.14333333333333	0.321939155663546\\
1.30666666666667	0.191678130291283\\
1.47	0.104261858505617\\
1.6	0.0608178833274197\\
1.95	0.011993336026893\\
2.3	0.0020266159852749\\
2.65	0.000314695780498572\\
3	4.63511250902993e-05\\
3.35	6.57780256473213e-06\\
3.7	9.0758995078207e-07\\
4.05	1.22477541558302e-07\\
4.4	1.62325394784829e-08\\
4.75	2.11937880894763e-09\\
5.1	2.73233367078898e-10\\
5.45	3.48458512582576e-11\\
5.8	4.40238302109596e-12\\
6.15	5.51686060797427e-13\\
6.5	6.86733455063838e-14\\
6.85	8.5018442931035e-15\\
7.2	1.11328997411796e-15\\
7.55	2.14404343289962e-16\\
7.9	2.84867961191653e-17\\
8.25	1.22328780875022e-16\\
8.6	9.06412319296986e-17\\
8.95	6.08896760756933e-17\\
9.3	7.64509594275996e-18\\
9.65	9.37327075265564e-17\\
10	8.251272980649e-17\\
};
\label{addplot:minEigA14}
\addlegendentry{$P=500$}

\addplot [color=mycolor4, draw=none, mark size=4.3pt, mark=diamond, mark options={solid, mycolor4}]
  table[row sep=crcr]{%
0	1\\
0.163333333333333	0.998947293307687\\
0.326666666666667	0.984567433907324\\
0.49	0.932404305241269\\
0.653333333333333	0.825122809886625\\
0.816666666666667	0.668153277663492\\
0.98	0.48905464998925\\
1.14333333333333	0.322291953636624\\
1.30666666666667	0.191968022953862\\
1.47	0.104465584429471\\
1.6	0.060958221880423\\
1.95	0.0120319076067111\\
2.3	0.00203483885434308\\
2.65	0.000316228108759031\\
3	4.66143207611073e-05\\
3.35	6.62053867198976e-06\\
3.7	9.14244039670307e-07\\
4.05	1.23480153987469e-07\\
4.4	1.63796669758557e-08\\
4.75	2.14050350193756e-09\\
5.1	2.76211581929756e-10\\
5.45	3.52592464051164e-11\\
5.8	4.45899479893919e-12\\
6.15	5.59357740724311e-13\\
6.5	6.9663664069227e-14\\
6.85	8.68920646305474e-15\\
7.2	1.09990106622466e-15\\
7.55	1.7311745862756e-16\\
7.9	2.10372346828528e-16\\
8.25	2.05419021925366e-17\\
8.6	8.91051726498131e-17\\
8.95	2.31549918952292e-17\\
9.3	5.55111512312578e-17\\
9.65	6.93889390390723e-17\\
10	3.06730476622694e-17\\
};
\label{addplot:minEigA15}
\addlegendentry{$P=1500$}

\addplot [color=black, dashed, forget plot]
  table[row sep=crcr]{%
0	2.22044604925031e-16\\
0.163333333333333	2.22044604925031e-16\\
0.326666666666667	2.22044604925031e-16\\
0.49	2.22044604925031e-16\\
0.653333333333333	2.22044604925031e-16\\
0.816666666666667	2.22044604925031e-16\\
0.98	2.22044604925031e-16\\
1.14333333333333	2.22044604925031e-16\\
1.30666666666667	2.22044604925031e-16\\
1.47	2.22044604925031e-16\\
1.6	2.22044604925031e-16\\
1.95	2.22044604925031e-16\\
2.3	2.22044604925031e-16\\
2.65	2.22044604925031e-16\\
3	2.22044604925031e-16\\
3.35	2.22044604925031e-16\\
3.7	2.22044604925031e-16\\
4.05	2.22044604925031e-16\\
4.4	2.22044604925031e-16\\
4.75	2.22044604925031e-16\\
5.1	2.22044604925031e-16\\
5.45	2.22044604925031e-16\\
5.8	2.22044604925031e-16\\
6.15	2.22044604925031e-16\\
6.5	2.22044604925031e-16\\
6.85	2.22044604925031e-16\\
7.2	2.22044604925031e-16\\
7.55	2.22044604925031e-16\\
7.9	2.22044604925031e-16\\
8.25	2.22044604925031e-16\\
8.6	2.22044604925031e-16\\
8.95	2.22044604925031e-16\\
9.3	2.22044604925031e-16\\
9.65	2.22044604925031e-16\\
10	2.22044604925031e-16\\
};
\label{addplot:minEigA16}
\end{axis}
\end{tikzpicture}%
		\captionsetup{width=0.888\textwidth}
		\caption{Estimated minimal eigenvalue of $A$ in function of $\gamma$ for $P=50$\ref{addplot:minEigA12}, $P=150$\ref{addplot:minEigA13}, $P=500$\ref{addplot:minEigA14} and $P=1500$\ref{addplot:minEigA15}. Lower bound\ref{addplot:minEigA10}, lower bound established in \autoref{The:lowBoundCon}\ref{addplot:minEigA11}.}
	\end{figure}


\begin{Rem} The conditioning of the system matrix can be an issue. For $b \neq 1$, we sometimes observed numerical instabilities. For $b=1$, the error in the matrix inversion $\alpha = A^{-1}b$ is of the order $\kappa(a,P) \varepsilon_{\textup{mach}}$ where $\varepsilon_{\textup{mach}}$ is the machine precision. For any $a$, the function $P \mapsto \kappa(a,P)$ is increasing as shown in Figure \ref{ConditionNumIncrease}. This means that the error $e_{\textup{cond}}(a,P)$ related to the numerical error in the inversion of matrix $A$ increases with $P$. Since for any $a \in (0,1)$, the theoretical error of quadrature $e_{H_0^1}(P)$ decreases with $P$, there exists some value $P = P_{\times}(a)$ at which 
	\[e_{\textup{cond}}(a,P_{\times}(a)) \approx  e_{H_0^1}(P_{\times}(a)) \defis e_{\min}(a).\]
	Therefore, the error in the SBD is bounded below by $e_{\min(a)}$. 
	In our tests, we found that the function $e_{\min}(a)$ did not seem to depend much on $a$ and was located at about $e_{\min} \approx 10^{-10}$, as shown in Figure \ref{emina}. If one needs to compute a more precise approximation, it would be necessary to increase the machine precision. 
\end{Rem}

\toDo{(En cours) étude numérique de l'influence de b sur le conditionnement}
%
%\begin{figure}[H]
%	\centering 
%	% This file was created by matlab2tikz.
%
%The latest updates can be retrieved from
%  http://www.mathworks.com/matlabcentral/fileexchange/22022-matlab2tikz-matlab2tikz
%where you can also make suggestions and rate matlab2tikz.
%
\definecolor{mycolor1}{rgb}{0.91318,0.60037,0.71279}%
\definecolor{mycolor2}{rgb}{0.83389,0.36044,0.50807}%
\definecolor{mycolor3}{rgb}{0.76149,0.21640,0.36215}%
\definecolor{mycolor4}{rgb}{0.69537,0.12992,0.25814}%
\definecolor{mycolor5}{rgb}{0.63500,0.07800,0.18400}%
\definecolor{mycolor6}{rgb}{0.57987,0.04683,0.13115}%
%
\begin{tikzpicture}

\begin{axis}[%
width=4.038in,
height=2.996in,
at={(1.241in,1.05in)},
scale only axis,
xmode=log,
xmin=10,
xmax=8000,
xminorticks=true,
xlabel={Number of components $P$},
xmajorgrids,
xminorgrids,
ymode=log,
ymin=1,
ymax=1e+20,
yminorticks=true,
ylabel={Condition number $\kappa(a,P)$},
ymajorgrids,
yminorgrids,
axis background/.style={fill=white},
legend style={legend cell align=left,align=left,fill=none,draw=none}
]
\addplot [color=mycolor1,solid,line width=2.0pt]
  table[row sep=crcr]{%
10	147.798299754529\\
13	3622.03814452639\\
18	1039985.78822602\\
25	3931398057.45341\\
34	213178666225072\\
46	1.60869915492932e+16\\
63	1.31606803419788e+16\\
85	1.54499974752314e+16\\
116	5.61909338321288e+16\\
158	5.03590043795304e+16\\
215	3.15883369595641e+16\\
292	3.18715460201047e+16\\
398	7.145627737116e+16\\
541	2.10355165726059e+17\\
735	2.01238279086446e+17\\
999	2.25325778060716e+17\\
};
\addlegendentry{a = 0.200};

\addplot [color=mycolor2,solid,line width=2.0pt]
  table[row sep=crcr]{%
10	1.29005620540376\\
13	1.79538301401442\\
18	4.528696194657\\
25	32.232651935565\\
34	732.503698438176\\
46	68663.4990134139\\
63	57521691.6511078\\
85	449391751064.261\\
116	5.68378920784668e+15\\
158	1.99322360045982e+16\\
215	2.80073923332499e+17\\
292	3.42975837856168e+16\\
398	4.96920552017871e+16\\
541	9.01638555339965e+16\\
735	6.34252229355046e+17\\
999	1.17508911049251e+17\\
};
\addlegendentry{a = 0.069};

\addplot [color=mycolor3,solid,line width=2.0pt]
  table[row sep=crcr]{%
10	1.00524281269282\\
13	1.01415209509248\\
18	1.04772662016641\\
25	1.16041246150576\\
34	1.51104098042757\\
46	2.83266080660356\\
63	11.6814720895289\\
85	134.998092493818\\
116	6860.08658805013\\
158	1951112.7165527\\
215	5489031351.3837\\
292	321765067476248\\
398	1.11176919996093e+17\\
541	1.92260401469188e+17\\
735	9.19594027577622e+17\\
999	3.6997056034168e+17\\
};
\addlegendentry{a = 0.024};

\addplot [color=mycolor4,solid,line width=2.0pt]
  table[row sep=crcr]{%
10	1.00007974313807\\
13	1.00022189361027\\
18	1.00079051736883\\
25	1.00283817381305\\
34	1.00928843434519\\
46	1.02928617170875\\
63	1.09476439659201\\
85	1.28933439849195\\
116	1.98991953838061\\
158	5.40572703933634\\
215	38.9079771830317\\
292	988.160767295899\\
398	124194.699378968\\
541	112596675.964176\\
735	1499902696195.28\\
999	3.33626843306488e+16\\
};
\addlegendentry{a = 0.008};

\addplot [color=mycolor5,solid,line width=2.0pt]
  table[row sep=crcr]{%
10	1.0000011584179\\
13	1.00000323819172\\
18	1.00001165188446\\
25	1.00004260657273\\
34	1.00014359341668\\
46	1.00047351093979\\
63	1.00163041608515\\
85	1.00524200692462\\
116	1.01733196098756\\
158	1.05564620848043\\
215	1.17527035359592\\
292	1.56316816837835\\
398	3.15311942928251\\
541	13.9215882400081\\
735	195.480100302912\\
999	11243.4949456575\\
};
\addlegendentry{a = 0.003};

\addplot [color=mycolor6,solid,line width=2.0pt]
  table[row sep=crcr]{%
10	1.00000001672649\\
13	1.0000000467829\\
18	1.00000016854915\\
25	1.00000061781131\\
34	1.00000209092295\\
46	1.00000694640183\\
63	1.0000242503846\\
85	1.00007975255676\\
116	1.00027394432461\\
158	1.00092954690405\\
215	1.00311530059942\\
292	1.01021566498329\\
398	1.03327043433277\\
541	1.10497570113474\\
735	1.33086565541114\\
999	2.13527133554061\\
};
\addlegendentry{a = 0.001};

\end{axis}
\end{tikzpicture}%
%	\caption{Evolution of the condition number $\kappa(a,P)$ of the matrix $A$ as a function of the number of components $P$ in the radial approximation of $G$ on $\mathcal{A}(a)$ for different values of the parameter $a$.}
%	\label{ConditionNumIncrease}
%\end{figure}
%
%\begin{figure}[H]
%	\centering 
%	% This file was created by matlab2tikz.
%
%The latest updates can be retrieved from
%  http://www.mathworks.com/matlabcentral/fileexchange/22022-matlab2tikz-matlab2tikz
%where you can also make suggestions and rate matlab2tikz.
%
\definecolor{mycolor1}{rgb}{0.91318,0.60037,0.71279}%
\definecolor{mycolor2}{rgb}{0.83389,0.36044,0.50807}%
\definecolor{mycolor3}{rgb}{0.76149,0.21640,0.36215}%
\definecolor{mycolor4}{rgb}{0.69537,0.12992,0.25814}%
\definecolor{mycolor5}{rgb}{0.63500,0.07800,0.18400}%
\definecolor{mycolor6}{rgb}{0.57987,0.04683,0.13115}%
\definecolor{mycolor7}{rgb}{0.52952,0.02811,0.09349}%
\definecolor{mycolor8}{rgb}{0.48355,0.01688,0.06664}%
%
\begin{tikzpicture}

\begin{axis}[%
width=4.77in,
height=3.865in,
at={(1.339in,1.049in)},
scale only axis,
xmode=log,
xmin=1,
xmax=4900,
xminorticks=true,
xlabel={Number of components $P$},
xmajorgrids,
xminorgrids,
ymode=log,
ymin=1e-25,
ymax=50,
yminorticks=true,
ylabel={Condition number $\kappa(a,P)$},
ymajorgrids,
yminorgrids,
axis background/.style={fill=white},
legend style={legend cell align=left,align=left,fill=none,draw=none}
]
\addplot [color=mycolor1,solid,line width=2.0pt]
  table[row sep=crcr]{%
1	1.05579604011155\\
2	0.56752407610207\\
3	0.314117393718553\\
4	0.17410210250843\\
5	0.0959321640664235\\
6	0.0524934151248995\\
7	0.0372105746576767\\
8	0.0265415082956011\\
9	0.0186816510732841\\
10	0.0130203423400554\\
11	0.00900968944744696\\
12	0.00620283415199197\\
13	0.00425560784827672\\
14	0.00291329047919353\\
15	0.0019918587205523\\
16	0.00136127233548589\\
17	0.000930393242544625\\
18	0.000636242798880193\\
19	0.000435496666598922\\
20	0.000298417028048892\\
21	0.000204742183157247\\
22	0.000140682560648031\\
23	9.83818725250885e-05\\
24	6.95526575231753e-05\\
25	4.91475212962555e-05\\
26	3.47185001894346e-05\\
27	2.45205453732744e-05\\
28	1.73154151803523e-05\\
29	1.22279172427575e-05\\
30	8.63554777952302e-06\\
31	6.09944916218907e-06\\
32	4.30867713019367e-06\\
33	3.04468815492953e-06\\
34	2.1519010671156e-06\\
35	1.52111196527471e-06\\
36	1.07583426922631e-06\\
37	7.61036431651263e-07\\
38	5.38604233213391e-07\\
39	3.81252702741364e-07\\
40	2.72171666981791e-07\\
41	1.94372094419748e-07\\
42	1.38813758265854e-07\\
43	9.91486865942193e-08\\
44	7.08046061781431e-08\\
45	5.05704966791143e-08\\
46	3.61258503112083e-08\\
47	2.58023340471425e-08\\
48	1.8435021065244e-08\\
49	1.31708426387434e-08\\
50	9.40917344038894e-09\\
51	6.72570976689713e-09\\
52	4.80585837792091e-09\\
53	3.43491945997698e-09\\
54	2.45539011345386e-09\\
55	1.75500991517197e-09\\
56	1.25494592495556e-09\\
57	9.00978402995634e-10\\
58	6.46367404044668e-10\\
59	4.63340033007853e-10\\
60	3.31692895372271e-10\\
61	2.36791919405732e-10\\
62	1.68165925629182e-10\\
63	1.18316467734303e-10\\
64	8.40447711425441e-11\\
65	6.13060713305913e-11\\
66	4.92390572759405e-11\\
67	4.67055283337459e-11\\
68	5.29940535898277e-11\\
69	6.92810253610787e-11\\
70	9.59761159435857e-11\\
};
\addlegendentry{a = 0.100};

\addplot [color=mycolor1,dashed,line width=2.0pt,forget plot]
  table[row sep=crcr]{%
1	0\\
2	2.22477863102719e-23\\
3	2.77555756156289e-24\\
4	4.96506830649455e-23\\
5	2.22180088862786e-23\\
6	3.88578058618805e-23\\
7	5.82205865834546e-23\\
8	4.07921986653155e-23\\
9	4.04127281044026e-23\\
10	1.11610550040293e-22\\
11	6.75322301446426e-23\\
12	1.15943913215929e-22\\
13	5.88784672006416e-23\\
14	8.55399867968964e-23\\
15	9.38778750585887e-23\\
16	2.83229295878123e-22\\
17	1.859105666227e-22\\
18	1.77272104129687e-22\\
19	8.65772941473847e-22\\
20	5.21387876217762e-21\\
21	3.45889676039865e-21\\
22	2.35104184462165e-21\\
23	9.65723534931088e-22\\
24	1.5368512415031e-20\\
25	1.46384876671513e-21\\
26	4.61319772392902e-20\\
27	1.30237217834415e-20\\
28	2.16929197294826e-19\\
29	2.52831947623171e-19\\
30	1.77587817998022e-19\\
31	1.36373296452971e-18\\
32	9.67924171504593e-19\\
33	1.53322045831105e-18\\
34	2.12633048895956e-18\\
35	4.1703176321583e-18\\
36	1.01107868317464e-17\\
37	1.56105776723988e-17\\
38	3.27535096964755e-17\\
39	2.73736823670769e-17\\
40	1.20575158359541e-16\\
41	4.38935022145886e-16\\
42	4.00849058594131e-17\\
43	1.19020023524606e-15\\
44	8.16055560263524e-17\\
45	1.76132889200764e-15\\
46	5.63377799570921e-15\\
47	4.05421067197434e-16\\
48	6.32380925365716e-16\\
49	1.443131244558e-14\\
50	1.20419446164009e-14\\
51	1.09317753495868e-13\\
52	5.85615545391057e-14\\
53	1.46502649048766e-14\\
54	3.94786466400594e-13\\
55	7.13018420361705e-13\\
56	7.45067308630207e-13\\
57	7.09835125150986e-13\\
58	1.28798402804104e-12\\
59	7.28889769238423e-12\\
60	1.14953276539183e-11\\
61	2.94345574151171e-12\\
62	9.43359784162189e-12\\
63	7.70566993246674e-11\\
64	7.72053234959431e-11\\
65	7.95557902959822e-11\\
66	5.83491623603286e-10\\
67	9.92594649985646e-10\\
68	3.31397521712588e-10\\
69	2.59124270055854e-09\\
70	1.40274484767565e-09\\
};
\addplot [color=mycolor2,solid,line width=2.0pt]
  table[row sep=crcr]{%
1	1.36738403388351\\
2	0.843061636481511\\
3	0.543123179000804\\
4	0.35457576455956\\
5	0.232048139504867\\
6	0.151592224565225\\
7	0.098710620709471\\
8	0.0640501106240792\\
9	0.0451184135984222\\
10	0.035738341136589\\
11	0.0280562810320619\\
12	0.021865445881522\\
13	0.0169406045224267\\
14	0.0130631387200082\\
15	0.0100354566622967\\
16	0.00768690325594479\\
17	0.00587464227627343\\
18	0.00448199403711325\\
19	0.00341537738741327\\
20	0.00260037417137404\\
21	0.00197884025984996\\
22	0.00150549725950055\\
23	0.00114535935323756\\
24	0.000871484772840159\\
25	0.000663344321543136\\
26	0.000505109134021797\\
27	0.000384836985596326\\
28	0.000293396294157589\\
29	0.000223824038407372\\
30	0.000170870142090163\\
31	0.000130531944734447\\
32	0.000101094399848733\\
33	7.88201830976831e-05\\
34	6.14344212386264e-05\\
35	4.78733529720721e-05\\
36	3.72979509397631e-05\\
37	2.90562554976503e-05\\
38	2.26345028875308e-05\\
39	1.76305412216138e-05\\
40	1.3733010995054e-05\\
41	1.06969573390181e-05\\
42	8.33261988653078e-06\\
43	6.49130196217484e-06\\
44	5.05722274990461e-06\\
45	3.94030769612641e-06\\
46	3.07060963855577e-06\\
47	2.39309751304262e-06\\
48	1.86533001356182e-06\\
49	1.45404393814275e-06\\
50	1.13371180576749e-06\\
51	8.84161238001724e-07\\
52	6.89702266853942e-07\\
53	5.38025861640534e-07\\
54	4.19820298525053e-07\\
55	3.28468119814573e-07\\
56	2.57938076675401e-07\\
57	2.02527000947583e-07\\
58	1.59055686843601e-07\\
59	1.24895272790582e-07\\
60	9.80727796751069e-08\\
61	7.7015967825389e-08\\
62	6.0479965480198e-08\\
63	4.74883616874422e-08\\
64	3.7298795163565e-08\\
65	2.92930644185674e-08\\
66	2.30054841843241e-08\\
67	1.80690102880021e-08\\
68	1.41931320030153e-08\\
69	1.11488192011677e-08\\
70	8.75609185158055e-09\\
71	6.88028434225885e-09\\
72	5.40428368722701e-09\\
73	4.2471657302201e-09\\
74	3.33711813738091e-09\\
75	2.62196753197941e-09\\
76	2.06027594984448e-09\\
77	1.61936952736141e-09\\
78	1.27281563067072e-09\\
79	1.0027649821609e-09\\
80	7.9072037806327e-10\\
81	6.23503026986327e-10\\
82	4.91668927793398e-10\\
83	3.8773206867404e-10\\
84	3.05966363356447e-10\\
85	2.4150725863592e-10\\
86	1.90706561653542e-10\\
87	1.50798484810366e-10\\
88	1.19428467115767e-10\\
89	9.48197076411361e-11\\
90	7.56088525122323e-11\\
91	6.06985572915164e-11\\
92	4.94755347801856e-11\\
93	4.14290823869123e-11\\
94	3.5695890687748e-11\\
95	3.20090620675728e-11\\
96	3.00865998781319e-11\\
97	2.96900282137358e-11\\
98	3.05671044031897e-11\\
};
\addlegendentry{a = 0.072};

\addplot [color=mycolor2,dashed,line width=2.0pt,forget plot]
  table[row sep=crcr]{%
1	0\\
2	1.11022302462516e-23\\
3	2.22044604925031e-23\\
4	0\\
5	5.55111512312578e-24\\
6	3.84858480204777e-23\\
7	2.28878339926112e-23\\
8	3.89073378206267e-23\\
9	6.47365704913894e-23\\
10	4.7873339488914e-23\\
11	1.41308747195736e-22\\
12	7.60276084431977e-23\\
13	7.64160951068693e-23\\
14	5.94605414376984e-23\\
15	7.53533208195874e-23\\
16	1.09152572660774e-22\\
17	8.19612325799379e-23\\
18	1.56450131553851e-22\\
19	1.3287936532864e-22\\
20	1.27283739212406e-22\\
21	2.20394825545861e-22\\
22	1.69542043222262e-22\\
23	1.95040939962708e-22\\
24	1.83065854168062e-22\\
25	1.45893144444947e-21\\
26	8.14278830061547e-22\\
27	9.40184596683517e-22\\
28	2.15604076549439e-21\\
29	9.0432184681972e-22\\
30	4.42038411598277e-21\\
31	8.24448351866376e-22\\
32	2.25225648573866e-20\\
33	3.49051083162865e-21\\
34	1.50091067354678e-20\\
35	3.58843952300961e-21\\
36	4.12254235417844e-20\\
37	5.20988280653321e-20\\
38	1.69416698638455e-19\\
39	1.00637135350824e-19\\
40	1.32742966962613e-19\\
41	1.54432570359504e-20\\
42	2.6742013312111e-19\\
43	1.27366155163431e-19\\
44	2.72741495665273e-19\\
45	7.52823532544522e-19\\
46	4.3304371549036e-18\\
47	4.10171191422793e-18\\
48	1.13056229355537e-17\\
49	4.35895009599993e-18\\
50	4.09634916684467e-18\\
51	6.87890703803702e-18\\
52	1.75087021000909e-17\\
53	1.46721756869399e-17\\
54	4.2283094445913e-18\\
55	3.10676318751359e-17\\
56	1.39841474217533e-16\\
57	6.20988733703952e-17\\
58	1.75872875108898e-16\\
59	1.00239426429888e-15\\
60	4.15532849380704e-16\\
61	1.09469906116515e-15\\
62	1.97650655893348e-15\\
63	4.97826707409313e-15\\
64	5.82678426030198e-15\\
65	1.25683108559622e-14\\
66	7.29404270268071e-16\\
67	3.68452125599008e-15\\
68	1.93696145766401e-15\\
69	1.85252610219029e-14\\
70	1.36322748941072e-15\\
71	8.05271196511116e-14\\
72	1.71284813588888e-13\\
73	1.43054626997324e-14\\
74	6.58638310061249e-14\\
75	3.18435203329461e-14\\
76	3.8065864885628e-14\\
77	4.33311319656003e-13\\
78	2.17150276189351e-12\\
79	6.40760576350305e-14\\
80	5.14409805631515e-12\\
81	8.38203677130283e-12\\
82	2.44734264603997e-13\\
83	6.458453014519e-12\\
84	1.15532382065754e-12\\
85	2.23715818149666e-12\\
86	9.78781544521109e-12\\
87	1.55257101809898e-11\\
88	1.74708894411194e-10\\
89	6.76421111725687e-12\\
90	2.56769274459671e-10\\
91	4.61901800938979e-10\\
92	6.32580160971888e-10\\
93	5.47183243008705e-10\\
94	3.26745578050539e-09\\
95	1.16935480200441e-09\\
96	1.07334172115444e-09\\
97	2.03065722905992e-09\\
98	8.78219043145147e-10\\
};
\addplot [color=mycolor3,solid,line width=2.0pt]
  table[row sep=crcr]{%
1	1.68720547709227\\
2	1.14288612081717\\
3	0.814663950445935\\
4	0.592822492469299\\
5	0.435224577633747\\
6	0.320590733963456\\
7	0.236283787952479\\
8	0.174001482547981\\
9	0.127943439302936\\
10	0.0939109794245128\\
11	0.068807130473084\\
12	0.0503277066754078\\
13	0.0422685830505807\\
14	0.0357200351782523\\
15	0.0300455744894097\\
16	0.0251710054859369\\
17	0.0210143410760839\\
18	0.0174922680617988\\
19	0.0145238413108673\\
20	0.0120332910490584\\
21	0.00995182251491622\\
22	0.00821795399893954\\
23	0.00677765779364936\\
24	0.00558402545304171\\
25	0.00459672542544887\\
26	0.00378157294504478\\
27	0.00310936928693017\\
28	0.00255570591938348\\
29	0.00210007005867929\\
30	0.00172542282833366\\
31	0.00141750261558293\\
32	0.00116457107508472\\
33	0.000956832433628207\\
34	0.000786247377321025\\
35	0.000646217763816459\\
36	0.000531245217977094\\
37	0.000436835535015945\\
38	0.00035930751316382\\
39	0.000295652807022151\\
40	0.000243350634572259\\
41	0.00020036967723458\\
42	0.000165046098231869\\
43	0.000136005799674432\\
44	0.000112534501730099\\
45	9.41074468521208e-05\\
46	7.86881631786862e-05\\
47	6.57837413142737e-05\\
48	5.49894610553103e-05\\
49	4.5960126966893e-05\\
50	3.84127783892296e-05\\
51	3.21014882249671e-05\\
52	2.68256621924756e-05\\
53	2.24149594543199e-05\\
54	1.87310080197456e-05\\
55	1.56514423226284e-05\\
56	1.30782651952543e-05\\
57	1.09282721658666e-05\\
58	9.13122619872198e-06\\
59	7.63102598666521e-06\\
60	6.37684372239988e-06\\
61	5.32895056748472e-06\\
62	4.45365671897235e-06\\
63	3.72252316171995e-06\\
64	3.11163646271595e-06\\
65	2.6009550539996e-06\\
66	2.17433100146991e-06\\
67	1.81776672381062e-06\\
68	1.51992811359136e-06\\
69	1.27087118695357e-06\\
70	1.06268777955165e-06\\
71	8.88704986934385e-07\\
72	7.43307805528559e-07\\
73	6.21782733745135e-07\\
74	5.20181217211757e-07\\
75	4.35201309212374e-07\\
76	3.64085872561759e-07\\
77	3.0589627053601e-07\\
78	2.57122180435942e-07\\
79	2.16082244541838e-07\\
80	1.816254027176e-07\\
81	1.5265605890491e-07\\
82	1.28303557112019e-07\\
83	1.07834813078966e-07\\
84	9.06291113267343e-08\\
85	7.61624736611566e-08\\
86	6.40233475124319e-08\\
87	5.38108042569263e-08\\
88	4.5227398715042e-08\\
89	3.80150257939249e-08\\
90	3.19549000593611e-08\\
91	2.68566715533325e-08\\
92	2.25750467208741e-08\\
93	1.89782483062118e-08\\
94	1.59540558541948e-08\\
95	1.34118312011822e-08\\
96	1.12749249936428e-08\\
97	9.47860945288426e-09\\
98	7.9683704079514e-09\\
99	6.69841782041658e-09\\
100	5.633233435276e-09\\
101	4.73710404236272e-09\\
102	3.98279942359636e-09\\
103	3.34960548187269e-09\\
104	2.81733925078242e-09\\
105	2.3693069728381e-09\\
106	1.99355687513503e-09\\
107	1.67711311505059e-09\\
108	1.41153488897316e-09\\
109	1.18840270957321e-09\\
110	1.00077590658998e-09\\
111	8.44553760259714e-10\\
112	7.12988335038744e-10\\
113	6.02303540375715e-10\\
114	5.0928594674815e-10\\
115	4.3121284321046e-10\\
116	3.65780294941942e-10\\
117	3.11280334841513e-10\\
118	2.66789257352684e-10\\
119	2.30257590771998e-10\\
120	2.00540029027252e-10\\
121	1.76965997411571e-10\\
122	1.58675295125477e-10\\
123	1.45389034145182e-10\\
124	1.36579192400177e-10\\
125	1.32021504839486e-10\\
126	1.31530342173392e-10\\
127	1.34954714070545e-10\\
128	1.42169831462979e-10\\
129	1.53064227959021e-10\\
130	1.67683200658075e-10\\
131	1.86019644132784e-10\\
132	2.0809709511127e-10\\
133	2.33861374709932e-10\\
134	2.63651767085094e-10\\
135	2.98033153711685e-10\\
136	3.36098704423193e-10\\
};
\addlegendentry{a = 0.052};

\addplot [color=mycolor3,dashed,line width=2.0pt,forget plot]
  table[row sep=crcr]{%
1	0\\
2	2.22044604925031e-23\\
3	5.55111512312578e-24\\
4	2.28878339926112e-23\\
5	5.55111512312578e-24\\
6	5.87474804595221e-23\\
7	2.55893763326045e-23\\
8	6.01086065024839e-23\\
9	2.60376157333902e-23\\
10	9.22220506951241e-23\\
11	7.85659307318451e-23\\
12	6.2336220642932e-23\\
13	7.07937228699105e-23\\
14	1.49802989663091e-22\\
15	3.84592537276713e-23\\
16	1.48977906681221e-22\\
17	7.09839011114711e-23\\
18	7.64160951068693e-23\\
19	9.86397388431474e-23\\
20	1.25269617972065e-22\\
21	7.92007186133385e-23\\
22	1.27025356028142e-22\\
23	1.02601818093154e-22\\
24	1.91900251345923e-22\\
25	1.53675059173487e-22\\
26	1.37789032189683e-22\\
27	1.43586594594763e-22\\
28	2.5998360402772e-22\\
29	1.42610729654992e-22\\
30	1.65194160787524e-22\\
31	2.80125132135825e-22\\
32	3.0383482358728e-22\\
33	9.98034162451427e-22\\
34	1.97128101720433e-22\\
35	5.76077507190005e-22\\
36	1.33426107017847e-21\\
37	2.58090477428097e-22\\
38	6.57234885034038e-22\\
39	1.72779809753628e-21\\
40	1.43865348373599e-21\\
41	9.27596112482726e-21\\
42	4.8544792979095e-21\\
43	4.14540114705901e-22\\
44	1.12212824094189e-20\\
45	5.36918791381209e-21\\
46	3.38651413957945e-21\\
47	2.38868322039879e-20\\
48	2.88509067097716e-20\\
49	1.69564363368784e-21\\
50	5.32657634101586e-20\\
51	1.52547799928499e-21\\
52	1.92953255395561e-20\\
53	1.27953394624447e-19\\
54	6.83370043529149e-21\\
55	2.07425473907872e-19\\
56	2.74976238400869e-19\\
57	3.23548703932203e-19\\
58	5.39860142870983e-19\\
59	6.53081676744134e-19\\
60	4.37084701298004e-19\\
61	5.56363681019415e-19\\
62	2.74137295284829e-18\\
63	4.91736763444852e-19\\
64	1.47343274344608e-18\\
65	2.76721286022318e-18\\
66	2.71182738278353e-18\\
67	6.2382721007492e-18\\
68	2.11629504077234e-18\\
69	4.13163241813551e-18\\
70	4.73510975998532e-18\\
71	2.82043780505722e-19\\
72	1.06691274251813e-17\\
73	1.71766374988434e-17\\
74	1.77786125828707e-17\\
75	7.01125492615319e-17\\
76	7.66800477189075e-17\\
77	1.25307964438798e-16\\
78	5.00952182446821e-17\\
79	1.46986197313126e-16\\
80	1.40709974202214e-16\\
81	3.82031763275854e-16\\
82	4.43310097495555e-16\\
83	6.25301773024052e-16\\
84	1.14350322416412e-15\\
85	7.79956819067754e-16\\
86	8.28425147234484e-17\\
87	3.85299153180895e-15\\
88	1.31224752174738e-16\\
89	3.10768965068337e-16\\
90	5.59490519015385e-15\\
91	5.14625000350327e-15\\
92	2.91822854091476e-14\\
93	1.14654996607389e-14\\
94	9.59291965104338e-15\\
95	1.778145045005e-15\\
96	3.1056043949371e-14\\
97	1.12429941623363e-14\\
98	4.55206032735048e-14\\
99	3.24330949952212e-14\\
100	2.41330879916505e-13\\
101	3.57605422946382e-13\\
102	3.18765363272001e-13\\
103	5.80745037183236e-13\\
104	2.84403143503518e-13\\
105	3.9687531139835e-13\\
106	4.82677239512501e-13\\
107	1.90412211079007e-12\\
108	1.54470161296098e-12\\
109	1.22759110423744e-12\\
110	4.17635383369268e-12\\
111	2.16276729764958e-12\\
112	8.84114908785974e-13\\
113	1.58030627396241e-12\\
114	6.00394252889787e-12\\
115	1.07000806247984e-11\\
116	2.09328624747762e-11\\
117	3.71464556890223e-12\\
118	2.46779335430567e-11\\
119	7.36187854593714e-12\\
120	8.84903432669901e-11\\
121	4.56142787432926e-11\\
122	1.75067695045342e-11\\
123	1.59601652521656e-10\\
124	1.40268835688374e-11\\
125	2.91503749923709e-11\\
126	3.44561895749978e-10\\
127	2.20956899197124e-12\\
128	3.24343338225524e-10\\
129	3.53390215321753e-10\\
130	6.7828474949545e-10\\
131	1.54716659985265e-09\\
132	1.22608158543271e-09\\
133	5.55290909578692e-09\\
134	3.52823175082004e-09\\
135	1.12237058546679e-09\\
136	8.01123998125197e-09\\
};
\addplot [color=mycolor4,solid,line width=2.0pt]
  table[row sep=crcr]{%
1	2.01138699661251\\
2	1.4563290934228\\
3	1.11214375402881\\
4	0.87018146919847\\
5	0.689534312537481\\
6	0.550171041054667\\
7	0.440602910033227\\
8	0.353495592715883\\
9	0.28379609815564\\
10	0.227827387600173\\
11	0.182807972545654\\
12	0.14657675488104\\
13	0.117424407297604\\
14	0.0939834250774796\\
15	0.0751522552182666\\
16	0.0600402841541454\\
17	0.0484496703187993\\
18	0.0430919550353099\\
19	0.0382175898150878\\
20	0.0338074350392068\\
21	0.0298371559607316\\
22	0.0262783680982337\\
23	0.0231009449290194\\
24	0.0202739106806957\\
25	0.0177663059284559\\
26	0.0155481017569823\\
27	0.0135909037379669\\
28	0.0118676246463045\\
29	0.0103532618665403\\
30	0.0090247722824337\\
31	0.00786106392043884\\
32	0.0068432432160952\\
33	0.00595393334761241\\
34	0.00517785812493399\\
35	0.00450112501487121\\
36	0.00391149234845889\\
37	0.00339822913725563\\
38	0.00295160161652896\\
39	0.00256327979360638\\
40	0.002225705766143\\
41	0.0019324249626691\\
42	0.00167767619364279\\
43	0.00145647577520158\\
44	0.00126442008839511\\
45	0.00109773553518133\\
46	0.000953088284071146\\
47	0.000827546512000676\\
48	0.000718597758700756\\
49	0.000624063181421608\\
50	0.000542047448011118\\
51	0.000470866185668495\\
52	0.000409091470042089\\
53	0.000355474743686468\\
54	0.000308926828957556\\
55	0.00026854163068446\\
56	0.000233475534230809\\
57	0.000203013934394924\\
58	0.000176576936368189\\
59	0.000153606378534121\\
60	0.000133638523673429\\
61	0.000116393237276746\\
62	0.000102351715989268\\
63	8.99956928410717e-05\\
64	7.91250925007425e-05\\
65	6.95602456586286e-05\\
66	6.11504781051764e-05\\
67	5.37518548742177e-05\\
68	4.72460923131024e-05\\
69	4.15268670166391e-05\\
70	3.64991238579471e-05\\
71	3.20782813552078e-05\\
72	2.8191580695669e-05\\
73	2.47760113047946e-05\\
74	2.17741258774495e-05\\
75	1.91352752776375e-05\\
76	1.68161519358279e-05\\
77	1.47781965331717e-05\\
78	1.29871927221359e-05\\
79	1.1412868006655e-05\\
80	1.00299332350673e-05\\
81	8.81480056813544e-06\\
82	7.74662564140272e-06\\
83	6.80819234633034e-06\\
84	5.98380504213125e-06\\
85	5.25905294557205e-06\\
86	4.62210011908937e-06\\
87	4.0624296513414e-06\\
88	3.57073028789401e-06\\
89	3.1387191548049e-06\\
90	2.75914934411148e-06\\
91	2.42558968377438e-06\\
92	2.13238990331988e-06\\
93	1.87458324818479e-06\\
94	1.64796152235525e-06\\
95	1.44905639976045e-06\\
96	1.27397427451825e-06\\
97	1.12024850196235e-06\\
98	9.85017394317111e-07\\
99	8.66234617813433e-07\\
100	7.61669881388372e-07\\
101	6.69945040332465e-07\\
102	5.89174224963074e-07\\
103	5.18150452322175e-07\\
104	4.55812052546634e-07\\
105	4.00947282486186e-07\\
106	3.52719898621956e-07\\
107	3.11317211565409e-07\\
108	2.74719317516769e-07\\
109	2.4246002006123e-07\\
110	2.13946874882964e-07\\
111	1.88823984448305e-07\\
112	1.66639199772334e-07\\
113	1.47049948662215e-07\\
114	1.29758964906301e-07\\
115	1.1451613568525e-07\\
116	1.01062928248297e-07\\
117	8.91883926712467e-08\\
118	7.87078731079305e-08\\
119	6.94575983395396e-08\\
120	6.12925306064938e-08\\
121	5.40951261562839e-08\\
122	4.77437502865996e-08\\
123	4.21360777380642e-08\\
124	3.71834318890762e-08\\
125	3.28150311368347e-08\\
126	2.89642945183743e-08\\
127	2.55599119647343e-08\\
128	2.25599126046916e-08\\
129	1.99110705700889e-08\\
130	1.75726393436548e-08\\
131	1.55103858645589e-08\\
132	1.36895232927259e-08\\
133	1.20821668225801e-08\\
134	1.06654165499265e-08\\
135	9.41197519921388e-09\\
136	8.30897262105168e-09\\
137	7.33456273493971e-09\\
138	6.47318199042957e-09\\
139	5.71405678329029e-09\\
140	5.04469488404879e-09\\
141	4.45324177533735e-09\\
142	3.93080856753159e-09\\
143	3.4694824790904e-09\\
144	3.06295344643104e-09\\
145	2.70435673854763e-09\\
146	2.3876789434496e-09\\
147	2.10806572198408e-09\\
148	1.86120852063709e-09\\
149	1.64328550766868e-09\\
150	1.45091583192425e-09\\
151	1.28110810848625e-09\\
152	1.13140918855947e-09\\
153	1.00035624228667e-09\\
154	8.84838424752843e-10\\
155	7.82527820319956e-10\\
156	6.92017998460415e-10\\
157	6.12137451838635e-10\\
158	5.4133320048777e-10\\
159	4.78938222414627e-10\\
160	4.23619805900444e-10\\
161	3.74831277127896e-10\\
162	3.31680460874395e-10\\
163	2.93490121094919e-10\\
164	2.59873900176899e-10\\
165	2.30114594046427e-10\\
166	2.03793870667823e-10\\
167	1.80680359562757e-10\\
168	1.60267354942789e-10\\
169	1.42263534286258e-10\\
170	1.26420207635647e-10\\
171	1.12501119531316e-10\\
172	1.00312647077772e-10\\
173	8.96673846284557e-11\\
174	8.03992428188849e-11\\
175	7.23847648487208e-11\\
176	6.54947207578971e-11\\
177	5.99822413960283e-11\\
178	5.54121193374613e-11\\
179	5.17270670741254e-11\\
180	4.88924456476525e-11\\
181	4.68896033112287e-11\\
182	4.56608084675736e-11\\
183	4.51865211914537e-11\\
184	4.54498660928948e-11\\
185	4.64339677819225e-11\\
186	4.81219508685626e-11\\
187	5.0493387249162e-11\\
188	5.35211874819197e-11\\
};
\addlegendentry{a = 0.037};

\addplot [color=mycolor4,dashed,line width=2.0pt,forget plot]
  table[row sep=crcr]{%
1	0\\
2	5.55111512312578e-24\\
3	1.11022302462516e-23\\
4	1.11022302462516e-23\\
5	4.44089209850063e-23\\
6	3.33066907387547e-23\\
7	4.50974724488293e-23\\
8	2.27189177800304e-23\\
9	1.59443642914704e-23\\
10	6.40185826276718e-23\\
11	7.62268367585609e-23\\
12	7.22178529824907e-23\\
13	7.71216775670361e-23\\
14	1.11782974907561e-22\\
15	1.32429294620149e-22\\
16	1.51420990482005e-22\\
17	1.21905524622053e-22\\
18	9.03362975600329e-23\\
19	1.04037138981502e-22\\
20	9.10875199633605e-23\\
21	1.16076724156625e-22\\
22	7.76910196764399e-23\\
23	1.77389195744815e-22\\
24	1.31977689088111e-22\\
25	1.97397811835209e-22\\
26	1.3178053842863e-22\\
27	2.02520626346233e-22\\
28	1.45568935721255e-22\\
29	1.2240806741115e-22\\
30	1.6529429861931e-22\\
31	2.53831719534846e-22\\
32	1.483366091517e-22\\
33	1.79919511185587e-22\\
34	2.03224317466995e-22\\
35	1.53121752677306e-22\\
36	1.90566144427564e-22\\
37	2.1149971718852e-22\\
38	4.08974074279458e-22\\
39	6.41038889154974e-22\\
40	3.10835340658942e-22\\
41	2.3307500989403e-22\\
42	3.03089950447922e-22\\
43	2.34644419851946e-22\\
44	4.12221009515456e-22\\
45	4.73165408895375e-22\\
46	9.89908791241165e-22\\
47	4.30525795757699e-22\\
48	4.45817318378107e-22\\
49	5.64241017518359e-22\\
50	4.4368140309594e-22\\
51	6.83826145968148e-22\\
52	1.90838505675251e-22\\
53	5.80622494556764e-22\\
54	1.96154434014933e-21\\
55	2.51924190714724e-21\\
56	2.18116230760762e-21\\
57	4.14908660303587e-21\\
58	4.61048642690759e-22\\
59	6.86878615652588e-21\\
60	6.72429600228737e-21\\
61	8.92227762354652e-21\\
62	4.55463378580251e-21\\
63	1.458823904207e-21\\
64	1.9532384877268e-21\\
65	5.41692253517054e-21\\
66	8.55264563426541e-21\\
67	5.43345886840975e-21\\
68	7.18545270268766e-21\\
69	3.22219362498624e-20\\
70	5.51441900173073e-20\\
71	4.88770679029155e-20\\
72	5.03671963978749e-20\\
73	5.57998115256481e-20\\
74	1.80925365194792e-19\\
75	4.34462165623173e-20\\
76	1.77256551346087e-19\\
77	8.68946613003081e-20\\
78	1.39678831271206e-20\\
79	3.68745222681549e-19\\
80	1.34540691851866e-19\\
81	2.08872427085442e-19\\
82	1.01575573794147e-19\\
83	2.24290984154409e-20\\
84	1.05616018110497e-18\\
85	6.53761203331483e-19\\
86	9.2711900024116e-19\\
87	4.02593922306649e-19\\
88	2.21007238276233e-18\\
89	2.60300064757395e-18\\
90	5.65003431878154e-18\\
91	1.09699632989021e-19\\
92	2.20591088272283e-18\\
93	1.52574812726872e-19\\
94	5.21282378092759e-18\\
95	3.50797556173291e-18\\
96	8.54097138278184e-18\\
97	8.4577481024413e-18\\
98	1.42454198820019e-17\\
99	8.13145340899704e-18\\
100	2.71561038030848e-18\\
101	9.20188783146517e-17\\
102	7.96858449396165e-17\\
103	3.19051405324187e-17\\
104	3.58099904978582e-17\\
105	2.86016221444358e-17\\
106	8.32059127233288e-17\\
107	7.95029873716436e-17\\
108	2.56470559449498e-16\\
109	9.8746857272025e-17\\
110	1.82283278813061e-16\\
111	5.74732851432056e-16\\
112	1.12922850875534e-16\\
113	3.91876216682841e-16\\
114	9.34284278315045e-16\\
115	2.24261299910606e-16\\
116	5.27769448114018e-16\\
117	6.45245780761892e-16\\
118	7.88849587092355e-16\\
119	9.0478642967336e-16\\
120	3.333691695417e-16\\
121	1.85417088699333e-16\\
122	3.62127589981276e-15\\
123	2.39664420027785e-15\\
124	4.26451965125597e-15\\
125	3.65245996547963e-15\\
126	3.09551285990687e-15\\
127	6.31997609862978e-15\\
128	6.78388252651858e-15\\
129	4.68519858084083e-16\\
130	1.37008990644469e-14\\
131	1.24940671417102e-14\\
132	1.53177332000467e-14\\
133	2.75150934500586e-15\\
134	2.41410249588354e-14\\
135	2.95292378593856e-14\\
136	8.72392364449189e-14\\
137	4.45422730411308e-14\\
138	7.41652444987469e-15\\
139	1.27475404156757e-13\\
140	1.65278443188682e-13\\
141	8.47973517493691e-14\\
142	2.82256858843409e-14\\
143	9.98853771986671e-14\\
144	2.50354567094614e-13\\
145	9.65505380586981e-14\\
146	9.20242741711721e-13\\
147	2.7566723754955e-13\\
148	4.90902123940174e-13\\
149	3.53170613136609e-13\\
150	9.02514347944925e-13\\
151	1.32646711128721e-12\\
152	1.30486158888918e-12\\
153	2.69314313376202e-12\\
154	4.95608146208262e-12\\
155	9.10428772369842e-13\\
156	2.56931965890125e-12\\
157	7.07663999237448e-12\\
158	7.88708724288802e-12\\
159	5.56472837984218e-12\\
160	9.49168487632707e-12\\
161	1.57605253750744e-11\\
162	1.19391647165794e-11\\
163	5.22718100973987e-11\\
164	2.19973030301292e-11\\
165	6.39181946836155e-11\\
166	1.0504336127764e-11\\
167	2.30084791859482e-11\\
168	9.65831717800894e-11\\
169	3.58919113039356e-11\\
170	6.37575268581377e-11\\
171	1.86916249727129e-11\\
172	1.24603436329073e-10\\
173	1.2478683349267e-10\\
174	2.0266563919298e-10\\
175	6.5562428808472e-11\\
176	3.63107419849381e-10\\
177	1.83723889419769e-10\\
178	8.32551575274553e-10\\
179	2.74880247365181e-10\\
180	3.10647297014567e-10\\
181	6.54209456581327e-10\\
182	5.84504483054287e-10\\
183	1.01982902976698e-09\\
184	3.11426458092063e-09\\
185	1.90431183832519e-09\\
186	4.47792592637043e-09\\
187	2.52627618956532e-09\\
188	1.65049029049885e-09\\
};
\addplot [color=mycolor5,solid,line width=2.0pt]
  table[row sep=crcr]{%
1	2.33785298552374\\
2	1.77712696941612\\
3	1.42428995284118\\
4	1.17100932993662\\
5	0.976773215204521\\
6	0.821997454422956\\
7	0.695662773853549\\
8	0.590900095395101\\
9	0.503095024328748\\
10	0.428965469648694\\
11	0.366070082653099\\
12	0.312527468054759\\
13	0.266846671936286\\
14	0.22782002656246\\
15	0.194452646411172\\
16	0.165914326543339\\
17	0.141505595604512\\
18	0.120632978261443\\
19	0.102790415131509\\
20	0.0875449107491941\\
21	0.0745251640817308\\
22	0.0634123621457143\\
23	0.0539325873742587\\
24	0.0473254406088617\\
25	0.0434948297355553\\
26	0.0399145586424927\\
27	0.0365782710724165\\
28	0.0334778199114343\\
29	0.0306036732837933\\
30	0.0279454779972199\\
31	0.0254921262693029\\
32	0.023232201340309\\
33	0.0211543434663346\\
34	0.0192468656355307\\
35	0.0174983348213997\\
36	0.0158979822758485\\
37	0.0144349089537688\\
38	0.0130990122115673\\
39	0.0118804618190071\\
40	0.0107701554379003\\
41	0.00975930114223988\\
42	0.00883973707233565\\
43	0.00800394132038607\\
44	0.00724486912282529\\
45	0.00655571749566208\\
46	0.00593060496618802\\
47	0.00536382609002839\\
48	0.00485018185699637\\
49	0.00438488241564849\\
50	0.00396355330095899\\
51	0.00358218510867792\\
52	0.00323704946473891\\
53	0.00292495626165357\\
54	0.00264266593165674\\
55	0.00238743752190329\\
56	0.00215674261641263\\
57	0.00194824919448111\\
58	0.00175981470481812\\
59	0.001589640026598\\
60	0.00143583432085492\\
61	0.00129698829003644\\
62	0.00117155623421361\\
63	0.00105826915963503\\
64	0.00095596578664825\\
65	0.000863585911717202\\
66	0.000780163663819611\\
67	0.000704820795577543\\
68	0.000636809214501266\\
69	0.000575405296594944\\
70	0.000519917095729294\\
71	0.000469867562101545\\
72	0.000424624892221281\\
73	0.000383800405705514\\
74	0.000346921024826585\\
75	0.000313597102825192\\
76	0.000283512499140048\\
77	0.00025634643540462\\
78	0.000231800850577724\\
79	0.000209623858624219\\
80	0.000189586949798315\\
81	0.00017148299505898\\
82	0.00015512439031129\\
83	0.000140341336055005\\
84	0.000126980247276531\\
85	0.000115064064384995\\
86	0.0001049008096774\\
87	9.56287941105138e-05\\
88	8.71711865197966e-05\\
89	7.94607056842089e-05\\
90	7.24282061068671e-05\\
91	6.60138848740388e-05\\
92	6.016744995474e-05\\
93	5.48374599689261e-05\\
94	4.99776621962766e-05\\
95	4.55467742819238e-05\\
96	4.15088657339524e-05\\
97	3.78288168856322e-05\\
98	3.44732743950971e-05\\
99	3.14142771942194e-05\\
100	2.86280440091602e-05\\
101	2.60868602155462e-05\\
102	2.37723724203498e-05\\
103	2.16627910605283e-05\\
104	1.97393823810899e-05\\
105	1.79872147980653e-05\\
106	1.63911214809964e-05\\
107	1.49363910377787e-05\\
108	1.36106688954918e-05\\
109	1.24025960226426e-05\\
110	1.13017457228359e-05\\
111	1.02985613801287e-05\\
112	9.38429583818845e-06\\
113	8.55143949163306e-06\\
114	7.79305935738606e-06\\
115	7.10171347950705e-06\\
116	6.47129920183787e-06\\
117	5.8975121657312e-06\\
118	5.37453621252126e-06\\
119	4.89759518984556e-06\\
120	4.46368626993987e-06\\
121	4.06758824667008e-06\\
122	3.70738646404689e-06\\
123	3.37863736987742e-06\\
124	3.07935261911751e-06\\
125	2.80662163865131e-06\\
126	2.55776300761568e-06\\
127	2.33144686445641e-06\\
128	2.12508765384811e-06\\
129	1.93687711513846e-06\\
130	1.76531719908724e-06\\
131	1.60926337811063e-06\\
132	1.46696487046682e-06\\
133	1.33723093309612e-06\\
134	1.21896858917836e-06\\
135	1.11117551959694e-06\\
136	1.01293329901608e-06\\
137	9.23401052510542e-07\\
138	8.41809453255848e-07\\
139	7.67455129224714e-07\\
140	6.99695389183574e-07\\
141	6.37943367465255e-07\\
142	5.81663424537027e-07\\
143	5.30366952133932e-07\\
144	4.83608430190685e-07\\
145	4.4098179508012e-07\\
146	4.02117104947308e-07\\
147	3.66677427088291e-07\\
148	3.34820261294055e-07\\
149	3.06026107566026e-07\\
150	2.7972537175458e-07\\
151	2.55675567384372e-07\\
152	2.33679174233004e-07\\
153	2.13560178430328e-07\\
154	1.95213516018811e-07\\
155	1.78422792540545e-07\\
156	1.63058953450701e-07\\
157	1.49051647913012e-07\\
158	1.36220421609323e-07\\
159	1.24511109866887e-07\\
160	1.13797628209511e-07\\
161	1.04012609636328e-07\\
162	9.50621634743243e-08\\
163	8.6891186246163e-08\\
164	7.9407217690175e-08\\
165	7.25880444640836e-08\\
166	6.6340458015901e-08\\
167	6.06349672693796e-08\\
168	5.54248145157032e-08\\
169	5.06546480316672e-08\\
170	4.62969604875241e-08\\
171	4.23206105715224e-08\\
172	3.86817062825173e-08\\
173	3.53526239393887e-08\\
174	3.23116577938265e-08\\
175	2.95370741199008e-08\\
176	2.69992090906612e-08\\
177	2.46783344870494e-08\\
178	2.25562786226874e-08\\
179	2.06163104365942e-08\\
180	1.88430280267937e-08\\
181	1.7223499071406e-08\\
182	1.57444079995628e-08\\
183	1.43921741191377e-08\\
184	1.31559985128149e-08\\
185	1.20259588953786e-08\\
186	1.09929745306658e-08\\
187	1.0048712528743e-08\\
188	9.18554876605526e-09\\
189	8.39649860751024e-09\\
190	7.67516894484288e-09\\
191	7.01572444583576e-09\\
192	6.41331121542521e-09\\
193	5.86331738716694e-09\\
194	5.36050270838473e-09\\
195	4.90077001202849e-09\\
196	4.48037340561314e-09\\
197	4.09589917538256e-09\\
198	3.74423070326202e-09\\
199	3.4225218215056e-09\\
200	3.12916847988731e-09\\
201	2.86129697713022e-09\\
202	2.61614374608143e-09\\
203	2.39174946514709e-09\\
204	2.18631690529492e-09\\
205	1.99924832244847e-09\\
206	1.82829262840301e-09\\
207	1.67169167397674e-09\\
208	1.52821133525549e-09\\
209	1.39768108198268e-09\\
210	1.27832189278365e-09\\
211	1.1688805479082e-09\\
212	1.06934772148293e-09\\
213	9.79108349952185e-10\\
214	8.96459795285409e-10\\
215	8.20779888499601e-10\\
216	7.51502415852201e-10\\
217	6.88098911183488e-10\\
218	6.30086205433145e-10\\
219	5.77087710951218e-10\\
220	5.28634025442898e-10\\
221	4.84308593229343e-10\\
222	4.43764136548452e-10\\
223	4.06687128418071e-10\\
224	3.72795128100734e-10\\
225	3.41818129356852e-10\\
226	3.13514991745478e-10\\
227	2.87666335196946e-10\\
228	2.64119837112275e-10\\
229	2.42657449689432e-10\\
230	2.23096652263166e-10\\
231	2.05270467290575e-10\\
232	1.89044779830283e-10\\
233	1.74296133081953e-10\\
234	1.61020530242695e-10\\
235	1.4927659108821e-10\\
236	1.38757005885282e-10\\
237	1.29477317756255e-10\\
238	1.21210153025686e-10\\
239	1.13891562847357e-10\\
240	1.07487796441319e-10\\
241	1.02037489568829e-10\\
242	9.74309521950545e-11\\
243	9.36304367371577e-11\\
244	9.06150710022757e-11\\
245	8.83670914220147e-11\\
246	8.68700666956101e-11\\
247	8.61097859683468e-11\\
248	8.60755910991884e-11\\
249	8.67568239470984e-11\\
250	8.81454909062995e-11\\
251	9.02349306386441e-11\\
252	9.30202581628237e-11\\
253	9.64970325867398e-11\\
254	1.00661257107504e-10\\
255	1.05508934922227e-10\\
256	1.11132436586558e-10\\
257	1.17477139127686e-10\\
258	1.24525723066427e-10\\
259	1.32271971153841e-10\\
260	1.40880640486785e-10\\
261	1.50660817155313e-10\\
};
\addlegendentry{a = 0.027};

\addplot [color=mycolor5,dashed,line width=2.0pt,forget plot]
  table[row sep=crcr]{%
1	0\\
2	1.11022302462516e-23\\
3	0\\
4	3.14325000830454e-23\\
5	6.20633538311818e-24\\
6	4.57966997657877e-23\\
7	2.48253415324727e-23\\
8	4.23670474327502e-23\\
9	5.0876810486276e-23\\
10	2.35922392732846e-23\\
11	3.64010961612205e-23\\
12	7.65671649645901e-23\\
13	1.13254992956686e-22\\
14	5.67124590114613e-23\\
15	5.59262042547163e-23\\
16	8.2198806119046e-23\\
17	9.44505558607342e-23\\
18	8.62211666897098e-23\\
19	1.24034062927838e-22\\
20	1.32908349812642e-22\\
21	1.22504398325739e-22\\
22	1.31246018692376e-22\\
23	1.40029904828549e-22\\
24	1.28225051901857e-22\\
25	1.39885416385914e-22\\
26	1.91493952167789e-22\\
27	1.57402522245094e-22\\
28	1.82997458651767e-22\\
29	1.11282206764648e-22\\
30	1.4026133054323e-22\\
31	1.31029854841794e-22\\
32	2.24751887538258e-22\\
33	2.05742592206104e-22\\
34	2.6632706211441e-22\\
35	1.62526093413884e-22\\
36	1.88503070197661e-22\\
37	2.40996486214869e-22\\
38	3.01643223515183e-22\\
39	1.85677332850501e-22\\
40	2.70082573750449e-22\\
41	2.27752478719339e-22\\
42	2.83983081346884e-22\\
43	2.33952832032117e-22\\
44	2.11617896542959e-22\\
45	3.32977990940715e-22\\
46	3.48789698575139e-22\\
47	3.76293471805956e-22\\
48	2.32927826726076e-22\\
49	3.2326012269156e-22\\
50	3.29853734673738e-22\\
51	3.77225741262437e-22\\
52	3.06212059319088e-22\\
53	3.90964657596312e-22\\
54	5.19584352647444e-22\\
55	4.46268664701931e-22\\
56	2.83296875069997e-22\\
57	3.56348516137261e-22\\
58	3.76233968499088e-22\\
59	3.22192652748256e-22\\
60	3.76877767871856e-22\\
61	5.26340173742779e-22\\
62	7.03439918732384e-22\\
63	1.07792986541252e-21\\
64	5.60924867889454e-22\\
65	1.64074309788134e-21\\
66	3.69606560510412e-22\\
67	4.44401350377387e-22\\
68	1.07247783017264e-21\\
69	7.46364730180184e-22\\
70	1.84794749886408e-21\\
71	6.13597981037541e-22\\
72	2.05312068227667e-21\\
73	8.2334546779834e-22\\
74	2.8311147378652e-21\\
75	9.14363315220606e-22\\
76	1.32558638246001e-21\\
77	1.3474075970204e-21\\
78	4.61073692971295e-21\\
79	4.35513054015908e-21\\
80	2.85697066227967e-21\\
81	8.32929945530809e-21\\
82	1.3417823447873e-20\\
83	1.22686124733488e-21\\
84	1.12725313469374e-20\\
85	1.36371903018802e-21\\
86	1.30171762568585e-20\\
87	1.43186090149524e-20\\
88	2.41243375512367e-21\\
89	1.85109533907843e-20\\
90	8.29857843991997e-21\\
91	5.13101939430609e-22\\
92	3.24962609957792e-20\\
93	5.11044594073065e-20\\
94	1.95830195742915e-20\\
95	1.79651242709319e-20\\
96	5.80631096154693e-20\\
97	2.11184686963339e-20\\
98	1.00609591459175e-20\\
99	6.68488019833624e-20\\
100	1.10654587862086e-20\\
101	4.40768151017626e-21\\
102	6.68059173863345e-20\\
103	7.68810406723207e-20\\
104	8.27517857019637e-20\\
105	8.16545826306154e-20\\
106	1.17839262714022e-19\\
107	2.62032341660691e-21\\
108	2.60917035572281e-20\\
109	3.88885220544424e-19\\
110	2.87381623164573e-19\\
111	5.92189771120036e-19\\
112	2.67009390417575e-19\\
113	8.53285551655148e-19\\
114	1.10367655137818e-18\\
115	1.66508172286364e-19\\
116	3.95639574660362e-19\\
117	7.81634129366339e-19\\
118	1.22267434968669e-19\\
119	2.96970591324204e-19\\
120	1.15927753122291e-18\\
121	2.58202345042005e-18\\
122	1.34589756114861e-18\\
123	1.01068958030541e-18\\
124	1.73176299835036e-18\\
125	2.93445212555351e-18\\
126	5.10228052278747e-18\\
127	2.45167706765983e-18\\
128	3.82317521763762e-18\\
129	1.73529573621065e-18\\
130	1.23766540311221e-17\\
131	1.10542411523772e-17\\
132	1.55957921073202e-19\\
133	2.16697582944481e-17\\
134	2.36982656623318e-18\\
135	5.70600406685274e-19\\
136	4.01252883206475e-18\\
137	4.39399801541535e-18\\
138	1.66449801992369e-17\\
139	3.15682841398418e-17\\
140	2.58085060265177e-17\\
141	5.06609564103968e-17\\
142	1.74522861126649e-17\\
143	1.34920887667814e-17\\
144	7.26691988666934e-17\\
145	1.69762627602149e-18\\
146	8.77260845107884e-17\\
147	2.69147524804127e-17\\
148	8.17499750365101e-17\\
149	9.17502972882877e-17\\
150	1.18040710737227e-16\\
151	1.51241065003683e-17\\
152	2.85018068968258e-16\\
153	1.92156146455303e-16\\
154	8.84643790586077e-18\\
155	2.03123251198692e-16\\
156	2.08753964265845e-16\\
157	4.75808584533758e-16\\
158	1.54486368249226e-17\\
159	3.94299033646902e-16\\
160	2.52428852581562e-16\\
161	1.28540027937319e-16\\
162	1.43224782457984e-16\\
163	1.04624663324714e-17\\
164	2.48087641337699e-16\\
165	2.75331085256823e-15\\
166	1.32344983938574e-15\\
167	9.16974371414343e-16\\
168	2.24872714733276e-15\\
169	5.5050289820925e-15\\
170	4.53364924665242e-15\\
171	4.65489845139694e-15\\
172	5.75388593814883e-15\\
173	1.26502513076309e-15\\
174	8.74275566326222e-15\\
175	1.67033326863112e-15\\
176	4.50573252914034e-15\\
177	5.64706380219456e-15\\
178	1.24789472950622e-14\\
179	5.45158382088432e-15\\
180	2.0939430040982e-14\\
181	1.40012086385767e-14\\
182	9.86362819854466e-15\\
183	5.94666730100198e-16\\
184	2.45621617369036e-14\\
185	7.73762510234665e-15\\
186	2.51990278440824e-14\\
187	1.72156923503168e-14\\
188	6.96366013824185e-14\\
189	3.81415428948472e-14\\
190	6.27563020617642e-15\\
191	7.75228071771984e-14\\
192	3.44066969520647e-14\\
193	8.12453207068781e-15\\
194	5.2977066096069e-14\\
195	1.3676259973808e-13\\
196	2.81382486186857e-14\\
197	5.86840071592622e-14\\
198	2.61694562327141e-13\\
199	2.92337606067141e-13\\
200	2.29347968318927e-13\\
201	4.33913299425741e-13\\
202	6.54713250507515e-13\\
203	4.40255537105706e-13\\
204	8.80494729584915e-13\\
205	3.35985077803169e-13\\
206	4.20644568954742e-13\\
207	1.37687078743154e-12\\
208	4.06353606606185e-13\\
209	1.78359564676873e-12\\
210	4.84160886498095e-13\\
211	8.7182982994859e-13\\
212	2.06512348638291e-12\\
213	2.51943635247405e-12\\
214	3.92128351624237e-12\\
215	2.64553160516575e-12\\
216	2.26550773541899e-12\\
217	7.9368401935835e-12\\
218	1.71872021367998e-12\\
219	7.8428865137387e-12\\
220	1.60395173685007e-11\\
221	2.00836308237305e-12\\
222	1.21702069996355e-11\\
223	1.38009838725294e-11\\
224	3.86945856565267e-12\\
225	4.46802427178225e-12\\
226	1.88463179916468e-12\\
227	2.22685594249754e-11\\
228	1.82319080395526e-11\\
229	2.22109734543531e-11\\
230	1.30857259809185e-11\\
231	7.08631477564699e-11\\
232	4.66640840970114e-11\\
233	2.18229198137683e-12\\
234	6.45915112811687e-11\\
235	9.63154678454492e-11\\
236	4.3247700696413e-11\\
237	7.43950868049021e-11\\
238	7.33324329256252e-11\\
239	3.16491932887389e-10\\
240	1.19989951141115e-10\\
241	3.80294679186584e-11\\
242	1.01028574512405e-10\\
243	6.07168023572346e-11\\
244	2.51693376530078e-10\\
245	1.12851316753675e-10\\
246	2.44705837149153e-10\\
247	8.19597659449227e-10\\
248	4.29892452226874e-10\\
249	7.90658257275546e-10\\
250	2.17757291594036e-10\\
251	2.61319669150895e-09\\
252	4.01192187032582e-10\\
253	7.33524407111908e-10\\
254	4.37593369425178e-10\\
255	1.73021165503617e-09\\
256	1.21020860085715e-09\\
257	4.02749703484928e-11\\
258	5.58553298263585e-09\\
259	1.55918721755978e-09\\
260	5.82623966537068e-10\\
261	2.38173801466841e-09\\
};
\addplot [color=mycolor6,solid,line width=2.0pt]
  table[row sep=crcr]{%
1	2.66550930875225\\
2	2.1018184856806\\
3	1.74439273351831\\
4	1.48498904070358\\
5	1.28321244564065\\
6	1.11961830409684\\
7	0.983344175920656\\
8	0.867689578158914\\
9	0.768214913658022\\
10	0.68181381502587\\
11	0.606216819469391\\
12	0.53970649623647\\
13	0.480944632629771\\
14	0.42886259783066\\
15	0.382589164084015\\
16	0.341401478889668\\
17	0.304690852871296\\
18	0.271938312508576\\
19	0.242696752896447\\
20	0.216577649573502\\
21	0.193240980215372\\
22	0.172387445008651\\
23	0.153752358863051\\
24	0.137100777272639\\
25	0.12222354521192\\
26	0.108934046164473\\
27	0.0970654895549092\\
28	0.0864686180448477\\
29	0.0770097469674105\\
30	0.0685690703555246\\
31	0.0610391841081803\\
32	0.0543237885869594\\
33	0.0486059808363284\\
34	0.0457662823701472\\
35	0.0430576039963513\\
36	0.0404782035503968\\
37	0.038025628624125\\
38	0.0356970422669449\\
39	0.0334891758557219\\
40	0.0313983763019317\\
41	0.0294210790575695\\
42	0.0275530054683277\\
43	0.0257899901444225\\
44	0.0241280235512154\\
45	0.0225625396584297\\
46	0.0210894574885083\\
47	0.0197043001317589\\
48	0.0184029853046996\\
49	0.0171811890949156\\
50	0.0160350172514674\\
51	0.0149604899970122\\
52	0.0139535865200338\\
53	0.0130107457641202\\
54	0.0121283730013371\\
55	0.0113028567335389\\
56	0.0105311293460248\\
57	0.00980986476691781\\
58	0.00913615349284935\\
59	0.00850703666552421\\
60	0.00791981298007682\\
61	0.00737185872327606\\
62	0.00686075719634616\\
63	0.00638415709770435\\
64	0.00593979435825442\\
65	0.005525811640398\\
66	0.00514000873414089\\
67	0.00478061831661991\\
68	0.00444594700252798\\
69	0.00413437166961428\\
70	0.00384434287397717\\
71	0.00357438682228661\\
72	0.00332310615492126\\
73	0.00308940683486103\\
74	0.00287189237483476\\
75	0.00266967199788892\\
76	0.00248155806348516\\
77	0.00230660456336507\\
78	0.00214392416645826\\
79	0.00199267861852181\\
80	0.0018520776792883\\
81	0.0017213777892886\\
82	0.00159988052923943\\
83	0.00148693092550989\\
84	0.0013819602413081\\
85	0.00128443971688919\\
86	0.00119375620608642\\
87	0.00110953492538535\\
88	0.0010312265073833\\
89	0.000958502063906685\\
90	0.000890871294136897\\
91	0.00082808309448712\\
92	0.000769721140844837\\
93	0.000715469289674697\\
94	0.000665074414146538\\
95	0.000618266189971628\\
96	0.000574762626259773\\
97	0.00053433385820334\\
98	0.000496764926924964\\
99	0.000461854970876363\\
100	0.000429416439459818\\
101	0.000399274331551513\\
102	0.000371265461176051\\
103	0.000345237751788741\\
104	0.000321049560203068\\
105	0.000298569030699358\\
106	0.000277673479593066\\
107	0.000258248810113937\\
108	0.000240188957311815\\
109	0.000223421007844227\\
110	0.000207835292481029\\
111	0.000193340038608714\\
112	0.000179856054814742\\
113	0.000167330231723106\\
114	0.000155692611284231\\
115	0.000144861085147774\\
116	0.000134786983288837\\
117	0.000125436787712818\\
118	0.000116749483785927\\
119	0.000109230990911868\\
120	0.000102199518946744\\
121	9.56153101396851e-05\\
122	8.9451810143526e-05\\
123	8.36867858207135e-05\\
124	7.82885316126247e-05\\
125	7.32393159599276e-05\\
126	6.85140166551967e-05\\
127	6.40900092272467e-05\\
128	5.99537357084401e-05\\
129	5.60832521050436e-05\\
130	5.24610597669728e-05\\
131	4.90717715906364e-05\\
132	4.59008061342026e-05\\
133	4.29343671708438e-05\\
134	4.01594204273437e-05\\
135	3.7563668164875e-05\\
136	3.51355222014682e-05\\
137	3.28640757762955e-05\\
138	3.07390747229519e-05\\
139	2.87508882799159e-05\\
140	2.68904798605973e-05\\
141	2.51516266502883e-05\\
142	2.35253036948713e-05\\
143	2.20030218396339e-05\\
144	2.05793125793718e-05\\
145	1.92486468439945e-05\\
146	1.80024010081326e-05\\
147	1.68385048171515e-05\\
148	1.5748628692247e-05\\
149	1.47301243536013e-05\\
150	1.3776669896437e-05\\
151	1.28858742631799e-05\\
152	1.20512990777755e-05\\
153	1.12725257679358e-05\\
154	1.05430579884214e-05\\
155	9.86082559339962e-06\\
156	9.22342823539424e-06\\
157	8.62665926870676e-06\\
158	8.06823480026253e-06\\
159	7.54693496496373e-06\\
160	7.05901404085552e-06\\
161	6.60241559335972e-06\\
162	6.17519701506808e-06\\
163	5.77602087181717e-06\\
164	5.40284883498998e-06\\
165	5.05371794901066e-06\\
166	4.72711155286021e-06\\
167	4.42160190550212e-06\\
168	4.1358458999774e-06\\
169	3.8685807912664e-06\\
170	3.61862014530701e-06\\
171	3.38484980177967e-06\\
172	3.16622406826284e-06\\
173	2.96176199077181e-06\\
174	2.77054376729424e-06\\
175	2.59170734029368e-06\\
176	2.42444510867301e-06\\
177	2.2680007787379e-06\\
178	2.12166636925915e-06\\
179	1.98477936930175e-06\\
180	1.8567200097408e-06\\
181	1.73690869553766e-06\\
182	1.6248035445976e-06\\
183	1.51999310560313e-06\\
184	1.42210673637067e-06\\
185	1.33050027040227e-06\\
186	1.244759152641e-06\\
187	1.16449640374938e-06\\
188	1.08935082021588e-06\\
189	1.01928986584454e-06\\
190	9.5369982711091e-07\\
191	8.92268478658309e-07\\
192	8.34721766374003e-07\\
193	7.81063997479237e-07\\
194	7.30842885321437e-07\\
195	6.83774120435743e-07\\
196	6.39738652719046e-07\\
197	5.98668796314428e-07\\
198	5.60154238993249e-07\\
199	5.24065784990313e-07\\
200	4.90459016511124e-07\\
201	4.58922057244848e-07\\
202	4.29385900613255e-07\\
203	4.01861528498415e-07\\
204	3.7601308688906e-07\\
205	3.5189773983646e-07\\
206	3.29881838290191e-07\\
207	3.0920817106761e-07\\
208	2.89806763031919e-07\\
209	2.71674501561847e-07\\
210	2.54654590747805e-07\\
211	2.38682743791685e-07\\
212	2.23720722836163e-07\\
213	2.09716300769713e-07\\
214	1.9657743210999e-07\\
215	1.84252929358308e-07\\
216	1.72694293087972e-07\\
217	1.61871907877753e-07\\
218	1.51736771147881e-07\\
219	1.42232089217487e-07\\
220	1.33319729478387e-07\\
221	1.24963672298861e-07\\
222	1.17129928423054e-07\\
223	1.09786451485405e-07\\
224	1.02903032761503e-07\\
225	9.6451225228833e-08\\
226	9.04042365412749e-08\\
227	8.47385233093689e-08\\
228	7.94293941730473e-08\\
229	7.44524037976646e-08\\
230	6.97868318688677e-08\\
231	6.54131993016449e-08\\
232	6.13132176141562e-08\\
233	5.74724761115419e-08\\
234	5.3873107308533e-08\\
235	5.04994388528246e-08\\
236	4.73371639664322e-08\\
237	4.4372879148824e-08\\
238	4.15940015763283e-08\\
239	3.89887682139545e-08\\
240	3.6546137671678e-08\\
241	3.42557804344779e-08\\
242	3.21080206866498e-08\\
243	3.00937967878667e-08\\
244	2.82115588667864e-08\\
245	2.64467421295933e-08\\
246	2.47913178874626e-08\\
247	2.3238331703368e-08\\
248	2.17812683445118e-08\\
249	2.04153920435601e-08\\
250	1.913975467005e-08\\
251	1.79425057034166e-08\\
252	1.68186686799743e-08\\
253	1.57635997588557e-08\\
254	1.4777456147641e-08\\
255	1.38539872907018e-08\\
256	1.29866339904083e-08\\
257	1.21718586321151e-08\\
258	1.14110436655324e-08\\
259	1.06979936020934e-08\\
260	1.00278096937245e-08\\
261	9.3989735994171e-09\\
262	8.81262707252972e-09\\
263	8.26119439523154e-09\\
264	7.74252262303321e-09\\
265	7.25985493943426e-09\\
266	6.80587364243479e-09\\
267	6.37858699192861e-09\\
268	5.98109739513575e-09\\
269	5.60706547858558e-09\\
270	5.2552415752416e-09\\
271	4.92792784356766e-09\\
272	4.61945992569213e-09\\
273	4.33041202896334e-09\\
274	4.06037470312981e-09\\
275	3.80568332403186e-09\\
276	3.5685698840382e-09\\
277	3.34548833080817e-09\\
278	3.13611003832648e-09\\
279	2.94072544093638e-09\\
280	2.75617484390978e-09\\
281	2.58482391046755e-09\\
282	2.42306441577966e-09\\
283	2.27196794710949e-09\\
284	2.1301627128878e-09\\
285	1.99700789238477e-09\\
286	1.87267312767858e-09\\
287	1.75539716096296e-09\\
288	1.64635416410874e-09\\
289	1.54312207456542e-09\\
290	1.44745682106873e-09\\
291	1.35674405044028e-09\\
292	1.2726837361754e-09\\
293	1.19300169743042e-09\\
294	1.11911502287398e-09\\
295	1.04985398152735e-09\\
296	9.85168835399008e-10\\
297	9.24421428294409e-10\\
298	8.67591332109896e-10\\
299	8.14338818599936e-10\\
300	7.64320162716103e-10\\
301	7.17335080224757e-10\\
302	6.73197941836179e-10\\
303	6.3184879550704e-10\\
304	5.93188609343542e-10\\
305	5.56858559264128e-10\\
306	5.22710763561918e-10\\
307	4.90630647220769e-10\\
308	4.60752769271267e-10\\
309	4.32658797677732e-10\\
310	4.0624437147585e-10\\
311	3.81489062561968e-10\\
312	3.58389762311617e-10\\
313	3.36654704113926e-10\\
314	3.16211501427688e-10\\
315	2.97223134992919e-10\\
316	2.79354317456182e-10\\
317	2.62548205398616e-10\\
318	2.4695756550841e-10\\
319	2.3226798262499e-10\\
320	2.18530527007488e-10\\
321	2.05726990998301e-10\\
322	1.93661975345094e-10\\
323	1.82516668445487e-10\\
324	1.72008185472805e-10\\
325	1.622786349742e-10\\
326	1.53292489812884e-10\\
327	1.45063072665153e-10\\
328	1.37402533795239e-10\\
329	1.30421451416396e-10\\
330	1.23938637131005e-10\\
331	1.1793899190593e-10\\
332	1.12407860797248e-10\\
333	1.07349240607846e-10\\
334	1.02792885314784e-10\\
335	9.86490888976732e-11\\
336	9.49076373046864e-11\\
337	9.15574283055776e-11\\
338	8.85895801161496e-11\\
339	8.5993878684576e-11\\
340	8.37623304050794e-11\\
341	8.18860534934629e-11\\
342	8.03579425223688e-11\\
343	7.91700038860199e-11\\
344	7.83164644246881e-11\\
345	7.77906628002256e-11\\
346	7.75868258529044e-11\\
347	7.76996245122064e-11\\
348	7.81246178860329e-11\\
349	7.88573650822855e-11\\
350	7.98938692980755e-11\\
351	8.12301337305144e-11\\
352	8.2863049755133e-11\\
353	8.47921732827217e-11\\
354	8.70445937550812e-11\\
355	8.95949980872501e-11\\
356	9.24420540115989e-11\\
357	9.55844292604979e-11\\
358	9.90212356555276e-11\\
359	1.02750252750639e-10\\
360	1.06770148278201e-10\\
361	1.11131548408139e-10\\
362	1.15858433957783e-10\\
363	1.20889520616174e-10\\
};
\addlegendentry{a = 0.019};

\addplot [color=mycolor6,dashed,line width=2.0pt,forget plot]
  table[row sep=crcr]{%
1	0\\
2	5.55111512312578e-24\\
3	1.11022302462516e-23\\
4	2.22044604925031e-23\\
5	1.66533453693773e-23\\
6	6.08252531697804e-23\\
7	5.15536935946823e-23\\
8	3.5108334685767e-23\\
9	3.55512785535919e-23\\
10	4.50974724488293e-23\\
11	6.68586823151846e-23\\
12	6.08925005680628e-23\\
13	7.81141955492397e-23\\
14	6.50333926369705e-23\\
15	1.55040419361846e-22\\
16	1.27199596944319e-22\\
17	5.83042598998135e-23\\
18	1.14035079493655e-22\\
19	1.18026485535039e-22\\
20	8.72867622265593e-23\\
21	1.28202520043097e-22\\
22	9.26907410797403e-23\\
23	1.18579992089607e-22\\
24	8.7792776681443e-23\\
25	2.04355137737572e-22\\
26	1.15095732501173e-22\\
27	2.16383373576817e-22\\
28	1.79630259945989e-22\\
29	1.98723912777653e-22\\
30	1.60244887433313e-22\\
31	2.39658185269121e-22\\
32	2.21084092730806e-22\\
33	1.70052810872375e-22\\
34	1.74092962335707e-22\\
35	1.88439283171782e-22\\
36	1.84575494929958e-22\\
37	1.26752057638441e-22\\
38	2.96625337744126e-22\\
39	2.6459395475125e-22\\
40	3.13973256322807e-22\\
41	1.84643692981994e-22\\
42	2.71153649352036e-22\\
43	2.82408722966927e-22\\
44	2.44569741789983e-22\\
45	2.95155132327798e-22\\
46	2.13540588965258e-22\\
47	3.26600107118609e-22\\
48	2.34551229806782e-22\\
49	2.55758254277683e-22\\
50	3.36959932581211e-22\\
51	2.78526587623202e-22\\
52	2.26299448046673e-22\\
53	2.7969940261555e-22\\
54	3.19729878737785e-22\\
55	2.79081803870453e-22\\
56	3.39103965163383e-22\\
57	4.46476710438357e-22\\
58	3.34783474198027e-22\\
59	2.82540872804308e-22\\
60	3.19987790179514e-22\\
61	3.97838662118291e-22\\
62	4.11448324330193e-22\\
63	3.30500652970613e-22\\
64	4.0971912134019e-22\\
65	3.79156814688744e-22\\
66	4.32953201218562e-22\\
67	3.80111706843148e-22\\
68	4.05420278487414e-22\\
69	3.82974301077019e-22\\
70	5.57935615813538e-22\\
71	4.44529177989436e-22\\
72	4.88536940728055e-22\\
73	4.42242769680873e-22\\
74	5.6401558730843e-22\\
75	4.86500376695134e-22\\
76	4.50062104325074e-22\\
77	7.75629998216863e-22\\
78	6.14093899120498e-22\\
79	4.48025846976314e-22\\
80	3.67146017164189e-22\\
81	7.63849391784848e-22\\
82	4.55504957438678e-22\\
83	6.27269796308053e-22\\
84	6.00190256677513e-22\\
85	4.62488210056998e-22\\
86	8.04523847590174e-22\\
87	8.4006939212893e-22\\
88	5.80005230338015e-22\\
89	3.90144441032631e-22\\
90	1.09133212090433e-21\\
91	8.2565646787484e-22\\
92	4.20108052358185e-22\\
93	1.01954986545185e-21\\
94	1.54466499623729e-21\\
95	1.21921250874318e-21\\
96	1.65038586616569e-21\\
97	6.35188466619568e-22\\
98	1.47935385822535e-21\\
99	1.06262027898176e-21\\
100	2.87545623352657e-21\\
101	5.86749669237389e-22\\
102	5.31604391097706e-21\\
103	2.12104088394267e-21\\
104	3.23554752229912e-21\\
105	1.54978394397526e-21\\
106	2.59685425375564e-21\\
107	6.22490866243085e-21\\
108	6.61100845527705e-21\\
109	7.31866361296222e-21\\
110	2.2241777177035e-21\\
111	5.64548371567252e-21\\
112	4.57809189483531e-21\\
113	6.6889888160021e-21\\
114	9.62499817189009e-21\\
115	2.31035388266249e-21\\
116	8.25993142438213e-21\\
117	6.84271986072624e-21\\
118	1.96122332903662e-20\\
119	1.02299533503917e-20\\
120	1.2112411958214e-20\\
121	6.04351007125723e-21\\
122	8.62082077093727e-21\\
123	1.15677592595617e-20\\
124	6.6484433938015e-21\\
125	6.91266384573919e-21\\
126	2.0020260577067e-21\\
127	3.27183448540433e-20\\
128	3.80395735819995e-20\\
129	4.25881465110668e-20\\
130	2.33367216432903e-20\\
131	1.18617561337617e-20\\
132	2.35665654681535e-20\\
133	1.9598217948991e-21\\
134	1.31473649260057e-19\\
135	2.70378667293786e-20\\
136	4.43445258137469e-21\\
137	1.14774455998615e-21\\
138	1.3348552011795e-19\\
139	5.08111773210143e-20\\
140	1.39910700783734e-20\\
141	9.992997023843e-20\\
142	2.46896005746552e-19\\
143	2.11263514047514e-19\\
144	3.77763985155054e-20\\
145	8.30249690646707e-20\\
146	4.38929729665076e-20\\
147	2.1008719510146e-19\\
148	8.51497740917423e-20\\
149	5.05454329310487e-20\\
150	8.76981656820621e-20\\
151	4.05321319544387e-19\\
152	7.94335433267524e-19\\
153	6.64213285682087e-20\\
154	1.34537205966276e-19\\
155	7.57035613245726e-20\\
156	7.58531757438499e-19\\
157	5.60605685479779e-19\\
158	8.40904644097367e-19\\
159	4.80123524049692e-19\\
160	9.83186685833887e-19\\
161	1.25379954356894e-19\\
162	4.24644571910014e-19\\
163	1.0502719536545e-18\\
164	1.55854056678575e-18\\
165	6.56098170723119e-19\\
166	1.02340381421431e-18\\
167	1.03716788842689e-18\\
168	1.73782155483111e-18\\
169	2.53282142871082e-18\\
170	4.95987318045698e-18\\
171	1.31384080105422e-18\\
172	3.58226932994459e-18\\
173	1.22738009762676e-18\\
174	3.78787957101781e-18\\
175	5.32535235332105e-18\\
176	4.81359111716142e-19\\
177	9.77663269934834e-18\\
178	1.112622688806e-17\\
179	5.82626520820507e-18\\
180	6.33098654021429e-18\\
181	8.3094546262595e-18\\
182	8.89578980465264e-18\\
183	1.96144751808575e-19\\
184	3.84761230638549e-18\\
185	2.03788604269779e-17\\
186	1.28821985451399e-17\\
187	1.11295658191776e-17\\
188	8.80645222962467e-18\\
189	1.85085987872868e-17\\
190	1.70571046321886e-17\\
191	1.54689668169341e-17\\
192	5.07290715231136e-17\\
193	6.54418569339783e-17\\
194	5.62751873724641e-18\\
195	5.45675406558499e-18\\
196	5.83301956215076e-19\\
197	3.21008456956668e-17\\
198	1.12465344935073e-16\\
199	4.44091562988504e-17\\
200	7.28613638688718e-18\\
201	2.32416940816284e-17\\
202	1.5081585613821e-16\\
203	9.01504992544136e-17\\
204	7.12918421758612e-18\\
205	1.50729755641226e-17\\
206	3.61387338077427e-17\\
207	2.40843791896203e-16\\
208	8.42546251532611e-17\\
209	1.4408879451267e-16\\
210	1.87424678698718e-16\\
211	3.42126917426037e-18\\
212	3.05930513473141e-16\\
213	1.42330773829314e-16\\
214	1.1482302473568e-16\\
215	1.54468703219965e-16\\
216	6.88849775326052e-17\\
217	6.01801211089127e-16\\
218	4.09913828698939e-16\\
219	9.22535895588999e-16\\
220	3.84503180800944e-16\\
221	2.36181061797799e-16\\
222	4.23585085766801e-16\\
223	9.58941087217378e-16\\
224	2.33364065757324e-16\\
225	5.63249102315676e-16\\
226	3.91067353447156e-16\\
227	2.79675151667248e-16\\
228	1.55028891168173e-15\\
229	1.93762893352168e-16\\
230	2.71924161425715e-15\\
231	6.81690420170388e-17\\
232	3.12771040257182e-15\\
233	4.63557630060093e-15\\
234	3.02263973096914e-15\\
235	2.44422531822508e-15\\
236	7.7927466081633e-15\\
237	3.17657610012191e-15\\
238	5.42487619288498e-15\\
239	2.09479746540679e-16\\
240	1.08921492113361e-14\\
241	8.54989176562964e-15\\
242	2.35896408256298e-15\\
243	1.50587506823914e-15\\
244	9.331124003471e-15\\
245	1.96638341003399e-14\\
246	1.53567285010828e-14\\
247	4.28419806283311e-16\\
248	5.71356580844364e-15\\
249	2.1991293099689e-14\\
250	3.94643014792083e-16\\
251	1.80396197425397e-14\\
252	2.17213588912885e-14\\
253	1.90968821838954e-14\\
254	2.1174585624333e-14\\
255	2.80229560045069e-15\\
256	9.10197717877444e-15\\
257	4.14310263832303e-14\\
258	5.94612730961128e-14\\
259	3.95654141182006e-14\\
260	3.80549687168073e-14\\
261	3.61745280858072e-14\\
262	9.44833468768694e-15\\
263	7.66406031285973e-14\\
264	1.3054538528702e-14\\
265	2.33837201698644e-14\\
266	8.56468032447642e-15\\
267	5.30788035688915e-14\\
268	4.02640626815132e-14\\
269	1.36050514545344e-13\\
270	7.99340289983508e-14\\
271	3.24858499819685e-14\\
272	5.96906485914288e-14\\
273	3.74893861353803e-13\\
274	4.41046315618132e-14\\
275	1.23049266493015e-13\\
276	2.3505410533153e-13\\
277	3.16609190358731e-13\\
278	1.18452518266271e-14\\
279	3.59155209346224e-13\\
280	4.84405043349385e-13\\
281	3.30031543950388e-13\\
282	4.64157632313323e-13\\
283	1.48762033131921e-13\\
284	1.09970229462702e-12\\
285	6.95680117123801e-13\\
286	9.16461792245637e-13\\
287	1.23456672256777e-12\\
288	5.56509515368914e-14\\
289	6.54890364408749e-13\\
290	4.10672519172713e-13\\
291	1.15184670796669e-12\\
292	2.10749093898956e-13\\
293	1.15256684261371e-12\\
294	2.38574196921836e-12\\
295	1.55492171610484e-12\\
296	3.32892514770566e-12\\
297	2.6940866536086e-12\\
298	5.60210416319902e-12\\
299	8.1086175359402e-13\\
300	2.05918479264934e-12\\
301	1.54950132352858e-12\\
302	1.18435086960586e-12\\
303	7.78881906767969e-12\\
304	4.23154112354536e-12\\
305	1.35512294358955e-11\\
306	2.01168199798464e-11\\
307	2.44592616763689e-11\\
308	1.32255299330702e-11\\
309	2.50639326996105e-12\\
310	3.80997320964496e-12\\
311	3.04797942231668e-12\\
312	3.84140634448784e-11\\
313	8.8782156940841e-12\\
314	8.27027507207514e-12\\
315	3.19050181670007e-11\\
316	8.88506011201183e-12\\
317	1.97083736492022e-11\\
318	3.88429652521043e-11\\
319	1.35074722145375e-11\\
320	3.90668521474074e-11\\
321	2.49983043921527e-11\\
322	2.81322885004961e-11\\
323	5.20674354680796e-11\\
324	7.6236292667969e-11\\
325	6.52211175334447e-11\\
326	8.58198029349678e-11\\
327	1.62038311798682e-11\\
328	8.54047276928017e-13\\
329	1.58305680219187e-10\\
330	6.41331417254323e-12\\
331	3.02399071838865e-11\\
332	1.09047296016962e-10\\
333	1.35901472665253e-10\\
334	8.83057696659869e-11\\
335	2.82084681497344e-11\\
336	7.16138800624845e-11\\
337	1.98332615822168e-10\\
338	9.20110546699994e-10\\
339	4.31807977136837e-11\\
340	1.06227321988514e-09\\
341	3.33889588225462e-10\\
342	6.34595586221659e-10\\
343	3.74809500328892e-10\\
344	1.26627318335234e-09\\
345	5.88362667956299e-10\\
346	2.74566542458038e-10\\
347	1.21001476481083e-09\\
348	1.77911292690883e-09\\
349	3.97391237660399e-10\\
350	1.36980366887031e-09\\
351	1.08242186907312e-09\\
352	1.29648235785925e-09\\
353	3.48251116065525e-09\\
354	9.74754873532285e-10\\
355	4.82317882975633e-11\\
356	3.94560277378439e-10\\
357	1.5475780468129e-09\\
358	5.47944921315905e-09\\
359	6.30354304905086e-10\\
360	1.38573354358326e-09\\
361	1.16201565491309e-09\\
362	2.14762573284026e-09\\
363	1.33903171658549e-09\\
};
\addplot [color=mycolor7,solid,line width=2.0pt]
  table[row sep=crcr]{%
1	2.99378407781835\\
2	2.42854977585588\\
3	2.06871774304329\\
4	1.80606963017849\\
5	1.60024344094618\\
6	1.43183502505362\\
7	1.29002911446315\\
8	1.16817812487286\\
9	1.06190005640807\\
10	0.968149764872001\\
11	0.884721550998725\\
12	0.809963219660444\\
13	0.742602242193583\\
14	0.681635164338727\\
15	0.626254558082342\\
16	0.575799224991735\\
17	0.529719324873609\\
18	0.487551383555629\\
19	0.448900015447294\\
20	0.41342431684877\\
21	0.38082757479731\\
22	0.350849371990064\\
23	0.323259451044861\\
24	0.297852889002241\\
25	0.274446260077527\\
26	0.252874552376358\\
27	0.232988665828607\\
28	0.214653362443425\\
29	0.197745571667603\\
30	0.182152976809983\\
31	0.1677728256538\\
32	0.154510921215215\\
33	0.14228075830102\\
34	0.131002778903971\\
35	0.120603725144571\\
36	0.111016072852014\\
37	0.102177532287736\\
38	0.0940306051833293\\
39	0.0865221893625021\\
40	0.079603223873618\\
41	0.0732283688729574\\
42	0.067355715544136\\
43	0.0619465221735966\\
44	0.0569649731703148\\
45	0.0523779583545352\\
46	0.0485026101885517\\
47	0.0464495573370538\\
48	0.0444644080633321\\
49	0.0425467644176241\\
50	0.0406957372774985\\
51	0.0389105133239203\\
52	0.0371900888043477\\
53	0.0355332456892086\\
54	0.0339387223168695\\
55	0.0324052838748377\\
56	0.0309315970922941\\
57	0.0295159975192658\\
58	0.0281571344548874\\
59	0.0268534942829999\\
60	0.0256033067180401\\
61	0.024405198439978\\
62	0.0232575327972206\\
63	0.0221586153129976\\
64	0.0211069111760658\\
65	0.0201007771086728\\
66	0.0191386124278186\\
67	0.0182190006003751\\
68	0.0173401841693437\\
69	0.0165008325084175\\
70	0.0156993919890809\\
71	0.014934470724862\\
72	0.0142044929193776\\
73	0.0135081714438758\\
74	0.0128440578077873\\
75	0.0122109187259172\\
76	0.0116075011124188\\
77	0.01103254284864\\
78	0.0104847819708844\\
79	0.0099629644508763\\
80	0.00946625056204553\\
81	0.00899325585217614\\
82	0.00854301966121973\\
83	0.00811464722732858\\
84	0.00770703714706844\\
85	0.00731925595683602\\
86	0.00695044627249253\\
87	0.00659976322243505\\
88	0.00626627509291477\\
89	0.00594919082188827\\
90	0.00564787597493321\\
91	0.00536141741783958\\
92	0.00508923384306037\\
93	0.00483062968433678\\
94	0.00458489597206935\\
95	0.00435144156574285\\
96	0.00412969248139561\\
97	0.00391909393018608\\
98	0.00371910423701927\\
99	0.00352920838076809\\
100	0.00334890498499529\\
101	0.00317771297335945\\
102	0.00301517019178377\\
103	0.00286083325250353\\
104	0.00271438476427388\\
105	0.00257538030752702\\
106	0.00244337606947465\\
107	0.00231820424939233\\
108	0.00219932204813844\\
109	0.00208658615792512\\
110	0.00197951893285087\\
111	0.00187802451695873\\
112	0.00178166515968581\\
113	0.00169022732323576\\
114	0.00160352887712367\\
115	0.00152124191625802\\
116	0.00144315286976759\\
117	0.00136909261603169\\
118	0.0012988589458347\\
119	0.00123222088380359\\
120	0.001168999698268\\
121	0.00110902448237837\\
122	0.00105213192586895\\
123	0.00099816608063108\\
124	0.000946978122050801\\
125	0.000898426107832861\\
126	0.000852374735822004\\
127	0.000808695102126222\\
128	0.000767264460616701\\
129	0.000727965984903101\\
130	0.000690688533485861\\
131	0.000655326418865343\\
132	0.000621779181204651\\
133	0.00058995136698492\\
134	0.000559787497190101\\
135	0.000531179440451446\\
136	0.000504034262622355\\
137	0.000478274134286139\\
138	0.000453830345825246\\
139	0.000430681534039579\\
140	0.000408708305094763\\
141	0.000387847578737777\\
142	0.000368071862535402\\
143	0.000349324431571318\\
144	0.000331519400766034\\
145	0.000314635052455436\\
146	0.000298632000718335\\
147	0.000283426155134769\\
148	0.000269021299315497\\
149	0.000255350652660802\\
150	0.000242364398522987\\
151	0.000230073161553435\\
152	0.000218383178716586\\
153	0.000207316421723114\\
154	0.000196800306879297\\
155	0.000186829655296705\\
156	0.000177367342804935\\
157	0.000168386244257412\\
158	0.000159869614995145\\
159	0.00015178157451512\\
160	0.000144113344942021\\
161	0.000136831041273489\\
162	0.000129923741948978\\
163	0.000123368209307806\\
164	0.000117143248226093\\
165	0.000111651579310035\\
166	0.000106429412440434\\
167	0.000101450692485017\\
168	9.67049077704729e-05\\
169	9.21798942385976e-05\\
170	8.78651428646293e-05\\
171	8.37505863016119e-05\\
172	7.98265879939919e-05\\
173	7.60866660858284e-05\\
174	7.25232265912368e-05\\
175	6.91241553765209e-05\\
176	6.5881608308338e-05\\
177	6.27953224165978e-05\\
178	5.98503742565981e-05\\
179	5.7042659065587e-05\\
180	5.4368290379081e-05\\
181	5.1815321635118e-05\\
182	4.93864419111034e-05\\
183	4.70668368350857e-05\\
184	4.48598652438292e-05\\
185	4.27521896577687e-05\\
186	4.07471557677397e-05\\
187	3.88331839236145e-05\\
188	3.7010291528361e-05\\
189	3.5272723705404e-05\\
190	3.3614516619096e-05\\
191	3.20373826872e-05\\
192	3.05326001681649e-05\\
193	2.90970495351317e-05\\
194	2.77312874290558e-05\\
195	2.64288842286575e-05\\
196	2.51867293625807e-05\\
197	2.40024568904929e-05\\
198	2.28756480860071e-05\\
199	2.18011624664172e-05\\
200	2.0776689200197e-05\\
201	1.98000044155933e-05\\
202	1.88689693301214e-05\\
203	1.79824852475363e-05\\
204	1.71378621178775e-05\\
205	1.63327447593176e-05\\
206	1.55653343174578e-05\\
207	1.48339053787083e-05\\
208	1.41368037755996e-05\\
209	1.34724445217671e-05\\
210	1.28393095630841e-05\\
211	1.22359456273813e-05\\
212	1.16609620803843e-05\\
213	1.11130287612227e-05\\
214	1.05908738365912e-05\\
215	1.00932817748145e-05\\
216	9.61909115648751e-06\\
217	9.16719274890454e-06\\
218	8.736527468578e-06\\
219	8.32608444945038e-06\\
220	7.93489911554701e-06\\
221	7.56205134244681e-06\\
222	7.20666365605638e-06\\
223	6.86789945536503e-06\\
224	6.54554006374042e-06\\
225	6.23836352886542e-06\\
226	5.94553674826415e-06\\
227	5.66636660792597e-06\\
228	5.40019325168117e-06\\
229	5.14638853488236e-06\\
230	4.90482419301941e-06\\
231	4.67481596277963e-06\\
232	4.45545529270674e-06\\
233	4.24622563688359e-06\\
234	4.04663583442044e-06\\
235	3.85698059979234e-06\\
236	3.67614387997151e-06\\
237	3.50359083700624e-06\\
238	3.33891967407851e-06\\
239	3.18254835107723e-06\\
240	3.03336368645546e-06\\
241	2.89094377414756e-06\\
242	2.75530715576622e-06\\
243	2.6262991239534e-06\\
244	2.50309140348293e-06\\
245	2.38556465870232e-06\\
246	2.27396260665813e-06\\
247	2.16733001590086e-06\\
248	2.06559918947136e-06\\
249	1.96900290205804e-06\\
250	1.87666175577306e-06\\
251	1.78869208244947e-06\\
252	1.70502607232947e-06\\
253	1.62500212930183e-06\\
254	1.54900831272897e-06\\
255	1.47648075365936e-06\\
256	1.40722077990318e-06\\
257	1.3414886161911e-06\\
258	1.27855479714611e-06\\
259	1.2188167390903e-06\\
260	1.16175267272922e-06\\
261	1.10735858882549e-06\\
262	1.0556119418581e-06\\
263	1.00610229658571e-06\\
264	9.59171110537227e-07\\
265	9.1416527858712e-07\\
266	8.7155188221999e-07\\
267	8.30705715770819e-07\\
268	7.9195257018938e-07\\
269	7.54872495178915e-07\\
270	7.19641990443165e-07\\
271	6.85971074609881e-07\\
272	6.53953731166723e-07\\
273	6.23368443086747e-07\\
274	5.94280791688107e-07\\
275	5.66488079556393e-07\\
276	5.40070659305059e-07\\
277	5.14805146956121e-07\\
278	4.90820634801992e-07\\
279	4.67842162343857e-07\\
280	4.46073614490672e-07\\
281	4.25164925488275e-07\\
282	4.05414142257143e-07\\
283	3.86437952748508e-07\\
284	3.68464770694743e-07\\
285	3.5129493269892e-07\\
286	3.35308381949062e-07\\
287	3.20047092827735e-07\\
288	3.05492195984414e-07\\
289	2.91600107615864e-07\\
290	2.78339492787438e-07\\
291	2.6568116417991e-07\\
292	2.53597310262421e-07\\
293	2.42061313215913e-07\\
294	2.31047824428288e-07\\
295	2.20532634820358e-07\\
296	2.10492632213288e-07\\
297	2.00905777347771e-07\\
298	1.91752412170842e-07\\
299	1.83038922685341e-07\\
300	1.74717964540605e-07\\
301	1.66771241261188e-07\\
302	1.59181320569246e-07\\
303	1.5193157754112e-07\\
304	1.45006177731943e-07\\
305	1.38414973704926e-07\\
306	1.32124211127405e-07\\
307	1.26113693710295e-07\\
308	1.20370367007183e-07\\
309	1.14881815171941e-07\\
310	1.09661699276842e-07\\
311	1.04677841505918e-07\\
312	9.99137697021979e-08\\
313	9.5359297347386e-08\\
314	9.10205564252919e-08\\
315	8.68860130509574e-08\\
316	8.29320816464474e-08\\
317	7.91504106700813e-08\\
318	7.55523705819883e-08\\
319	7.21196968811455e-08\\
320	6.88353498645711e-08\\
321	6.56960166267595e-08\\
322	6.27153955434778e-08\\
323	5.98623737246839e-08\\
324	5.71311264963015e-08\\
325	5.4537575344682e-08\\
326	5.2058413579914e-08\\
327	4.96839946961813e-08\\
328	4.74271821815364e-08\\
329	4.52718582621969e-08\\
330	4.3206612510005e-08\\
331	4.12450171793921e-08\\
332	3.93701471423924e-08\\
333	3.75727049473085e-08\\
334	3.58696032876082e-08\\
335	3.42375541251272e-08\\
336	3.26767146674456e-08\\
337	3.11950305587061e-08\\
338	2.97731821419234e-08\\
339	2.84203540701355e-08\\
340	2.71290909736877e-08\\
341	2.58906434069672e-08\\
342	2.47180373946776e-08\\
343	2.3591592679395e-08\\
344	2.252000363967e-08\\
345	2.14965876210726e-08\\
346	2.05166061917339e-08\\
347	1.95867571051167e-08\\
348	1.86925239731295e-08\\
349	1.78460357602717e-08\\
350	1.70330762827575e-08\\
351	1.62597073583015e-08\\
352	1.55205297502903e-08\\
353	1.481425293548e-08\\
354	1.41420386512436e-08\\
355	1.34972903964581e-08\\
356	1.28858781422991e-08\\
357	1.22975229999156e-08\\
358	1.17412595201927e-08\\
359	1.12047846556607e-08\\
360	1.06983399916771e-08\\
361	1.02098436371989e-08\\
362	9.74812497389621e-09\\
363	9.30317778369272e-09\\
364	8.88239348739717e-09\\
365	8.47696579597823e-09\\
366	8.09362710185724e-09\\
367	7.7240551732416e-09\\
368	7.3749584217353e-09\\
369	7.03797020662478e-09\\
370	6.7201719744503e-09\\
371	6.41325481609556e-09\\
372	6.12352835105412e-09\\
373	5.84424597604993e-09\\
374	5.57985213589518e-09\\
375	5.32580468615151e-09\\
376	5.08439423896334e-09\\
377	4.85338347289144e-09\\
378	4.63282923135466e-09\\
379	4.4228434248339e-09\\
380	4.22125268073614e-09\\
381	4.03045063990248e-09\\
382	3.84630727268132e-09\\
383	3.67274743950929e-09\\
384	3.50548212679769e-09\\
385	3.34665273271639e-09\\
386	3.19477599930451e-09\\
387	3.049327013116e-09\\
388	2.91147728148644e-09\\
389	2.7784885503479e-09\\
390	2.65311950187197e-09\\
391	2.53244980541467e-09\\
392	2.41747688534133e-09\\
393	2.30803021139536e-09\\
394	2.20250395699395e-09\\
395	2.10328199301557e-09\\
396	2.00757277468711e-09\\
397	1.91645899150217e-09\\
398	1.82971504614216e-09\\
399	1.74607350800216e-09\\
400	1.66737823548146e-09\\
401	1.59159174728529e-09\\
402	1.51916257351559e-09\\
403	1.45052769795484e-09\\
404	1.38435396479508e-09\\
405	1.32171429356731e-09\\
406	1.2618004419096e-09\\
407	1.20424203942093e-09\\
408	1.15015019730436e-09\\
409	1.09858611097025e-09\\
410	1.04943964629456e-09\\
411	1.00216812626286e-09\\
412	9.57400381196294e-10\\
413	9.14441855570658e-10\\
414	8.73320971095382e-10\\
415	8.3428997044166e-10\\
416	7.96696930649432e-10\\
417	7.61052554310027e-10\\
418	7.26873672363126e-10\\
419	6.94162061165571e-10\\
420	6.63090915509201e-10\\
421	6.33145091910592e-10\\
422	6.04831740247391e-10\\
423	5.77590419936769e-10\\
424	5.51628076550514e-10\\
425	5.26844345927202e-10\\
426	5.03043828814498e-10\\
427	4.80484096954115e-10\\
428	4.58720172957783e-10\\
429	4.38153513471207e-10\\
430	4.18325818429821e-10\\
431	3.99481336899044e-10\\
432	3.81428222340219e-10\\
433	3.64165586574927e-10\\
434	3.47726292204698e-10\\
435	3.31913163620357e-10\\
436	3.16926040966337e-10\\
437	3.02440739119447e-10\\
438	2.88791213165496e-10\\
439	2.75583111886135e-10\\
440	2.63081112450436e-10\\
441	2.51050735755598e-10\\
442	2.39595898676725e-10\\
443	2.28596697127159e-10\\
444	2.18098428206304e-10\\
445	2.08072670204729e-10\\
446	1.98447480670438e-10\\
447	1.89307236553304e-10\\
448	1.80476966704646e-10\\
449	1.72126313202625e-10\\
450	1.64036784155996e-10\\
451	1.56401114281834e-10\\
452	1.49020351614126e-10\\
453	1.42002853920076e-10\\
454	1.35250033395096e-10\\
455	1.28792976283876e-10\\
456	1.22652110690069e-10\\
457	1.16703091634918e-10\\
458	1.11088027665573e-10\\
459	1.05653263915428e-10\\
460	1.00741637254487e-10\\
461	9.60804769079004e-11\\
462	9.16244857762649e-11\\
463	8.73825456437771e-11\\
464	8.33253466225869e-11\\
465	7.94528887126944e-11\\
466	7.57705009846177e-11\\
467	7.22577553347037e-11\\
468	6.89208690118903e-11\\
469	6.57509602319806e-11\\
470	6.29816199193556e-11\\
471	6.04702954376535e-11\\
472	5.80926418081162e-11\\
473	5.5904614271185e-11\\
474	5.38511457648383e-11\\
475	5.19655429798149e-11\\
476	5.02371477750785e-11\\
477	4.87392348702542e-11\\
478	4.75357531115606e-11\\
479	4.64797089705371e-11\\
480	4.55728788040233e-11\\
481	4.4830361645154e-11\\
482	4.42357261931647e-11\\
483	4.38040714811905e-11\\
484	4.35282920818736e-11\\
485	4.34132729765224e-11\\
486	4.34581259867173e-11\\
487	4.36655156477173e-11\\
488	4.40296687997943e-11\\
489	4.4548809086109e-11\\
490	4.52229365066614e-11\\
491	4.60538274182909e-11\\
492	4.70317118583807e-11\\
493	4.81601425406097e-11\\
494	4.9427129056312e-11\\
495	5.09974285023418e-11\\
496	5.27684562712238e-11\\
497	5.46584999483457e-11\\
498	5.67235147741485e-11\\
499	5.88995519024138e-11\\
500	6.12079276152144e-11\\
501	6.36370955930943e-11\\
502	6.62030430476079e-11\\
503	6.89031054434963e-11\\
504	7.17825798801641e-11\\
};
\addlegendentry{a = 0.014};

\addplot [color=mycolor7,dashed,line width=2.0pt,forget plot]
  table[row sep=crcr]{%
1	0\\
2	5.55111512312578e-24\\
3	2.22044604925031e-23\\
4	2.28878750797074e-23\\
5	2.22044604925031e-23\\
6	3.33066907387547e-23\\
7	2.48641008715035e-23\\
8	4.57756679852224e-23\\
9	3.51151911128777e-23\\
10	4.19100001107273e-23\\
11	4.83935774203429e-23\\
12	4.15407418105522e-23\\
13	3.20393611103342e-23\\
14	7.96855760263093e-23\\
15	6.95033364883247e-23\\
16	9.98429436272205e-23\\
17	5.836100673094e-23\\
18	8.7154275286203e-23\\
19	9.27244992126176e-23\\
20	1.24817181520151e-22\\
21	1.03564008517885e-22\\
22	9.09215225734425e-23\\
23	1.62625306662115e-22\\
24	1.58304139037382e-22\\
25	1.68242064872787e-22\\
26	1.98952742377512e-22\\
27	1.62453125633022e-22\\
28	1.85298478321625e-22\\
29	1.74661759750364e-22\\
30	1.92407658042754e-22\\
31	1.22983452977954e-22\\
32	1.82808251360897e-22\\
33	2.14204003219084e-22\\
34	1.40165683695611e-22\\
35	2.37124452996462e-22\\
36	2.2016313233677e-22\\
37	2.16450395426461e-22\\
38	2.84340598755092e-22\\
39	2.1626656150816e-22\\
40	2.65022148398692e-22\\
41	2.92608310015565e-22\\
42	1.79089343195978e-22\\
43	4.03192131304466e-22\\
44	2.2334670857395e-22\\
45	1.79753516452348e-22\\
46	3.63950111670708e-22\\
47	3.74459259484317e-22\\
48	2.63559564005542e-22\\
49	3.54995233705913e-22\\
50	3.42583569320807e-22\\
51	2.30019426970025e-22\\
52	3.35174433793119e-22\\
53	4.2226119900047e-22\\
54	2.72856728395689e-22\\
55	3.41201906753325e-22\\
56	4.47869219199982e-22\\
57	2.3750613464554e-22\\
58	3.18709494729645e-22\\
59	3.28869945334902e-22\\
60	4.60488677463971e-22\\
61	3.84064839352756e-22\\
62	2.37802576373603e-22\\
63	3.22944203150417e-22\\
64	3.04161885007098e-22\\
65	3.02605493307546e-22\\
66	3.78397550418772e-22\\
67	3.24279214200376e-22\\
68	4.26890634743499e-22\\
69	5.13375285893492e-22\\
70	3.71188020737438e-22\\
71	3.95370386883395e-22\\
72	4.24503956822422e-22\\
73	4.80724803136046e-22\\
74	4.03492093835501e-22\\
75	5.87642717623154e-22\\
76	5.74197348768979e-22\\
77	5.38889104580016e-22\\
78	4.94945518909672e-22\\
79	4.60495045411927e-22\\
80	3.58523331770749e-22\\
81	4.28279864076629e-22\\
82	5.25538649929544e-22\\
83	4.43824738234573e-22\\
84	4.83667908293568e-22\\
85	3.67295869831439e-22\\
86	4.74819895722144e-22\\
87	5.20429648297636e-22\\
88	4.53817586634369e-22\\
89	6.04450548870563e-22\\
90	6.02670080569061e-22\\
91	5.35274951543272e-22\\
92	4.5737746932646e-22\\
93	5.2147161136799e-22\\
94	4.94050045404374e-22\\
95	6.47634383509201e-22\\
96	5.48778887400872e-22\\
97	7.60868562562041e-22\\
98	8.31653631789483e-22\\
99	7.73063338408924e-22\\
100	7.31796773593189e-22\\
101	5.51375600938443e-22\\
102	6.84027929067817e-22\\
103	4.89692742261015e-22\\
104	5.9426684195574e-22\\
105	6.66151884566218e-22\\
106	6.18708686508519e-22\\
107	5.86125353287843e-22\\
108	5.42214178416841e-22\\
109	6.12481630741686e-22\\
110	5.468596829805e-22\\
111	8.35425968497673e-22\\
112	8.20801648805763e-22\\
113	8.03676146728828e-22\\
114	6.70716783338377e-22\\
115	8.22274536316388e-22\\
116	7.73699972310963e-22\\
117	9.30654505965258e-22\\
118	6.34003649866838e-22\\
119	5.82717655387721e-22\\
120	6.71685288312065e-22\\
121	8.54342294831199e-22\\
122	8.63875170051766e-22\\
123	1.02711545804477e-21\\
124	6.13921285078561e-22\\
125	1.062702294685e-21\\
126	8.04985433916257e-22\\
127	1.07984361334884e-21\\
128	1.08533107016175e-21\\
129	1.36360859957554e-21\\
130	8.16732245822464e-22\\
131	2.60021920866359e-21\\
132	1.08746802153846e-21\\
133	9.45699112646519e-22\\
134	8.20564959270186e-22\\
135	1.5603655128849e-21\\
136	1.36353378221037e-21\\
137	2.50195097559444e-21\\
138	9.91023871398991e-22\\
139	2.04538716733607e-21\\
140	1.615217414279e-21\\
141	2.84122181649424e-21\\
142	1.63328842662358e-21\\
143	2.06595591452206e-21\\
144	2.51026870699819e-21\\
145	1.01325796666574e-21\\
146	8.24158638729304e-22\\
147	4.03991395513153e-21\\
148	4.53550549545766e-21\\
149	3.81697150538528e-21\\
150	9.53714272196626e-21\\
151	1.12316919041269e-21\\
152	2.43188879486579e-21\\
153	1.00928454007586e-21\\
154	6.72247018407794e-21\\
155	2.12495424301127e-21\\
156	1.35115897257461e-21\\
157	1.13612810614831e-20\\
158	9.02508294097152e-22\\
159	1.10755170262657e-20\\
160	3.66689157261759e-21\\
161	1.52646496217094e-21\\
162	1.06268490939608e-20\\
163	8.08185963574969e-21\\
164	1.45358926604501e-20\\
165	1.68034990368398e-21\\
166	3.35272294227133e-21\\
167	8.09686180396363e-21\\
168	1.80960557150111e-21\\
169	1.36926333659783e-20\\
170	1.85468403216771e-20\\
171	1.08437876849641e-20\\
172	8.7453669658732e-21\\
173	3.0299786059488e-20\\
174	1.53958373437325e-20\\
175	1.80175952322251e-20\\
176	4.98157239615015e-21\\
177	2.94126487514315e-20\\
178	1.34592506923931e-20\\
179	3.33939439233958e-20\\
180	9.53364125447195e-22\\
181	3.12333548068833e-20\\
182	6.04033983926035e-20\\
183	1.13368379696223e-20\\
184	3.0536102712631e-20\\
185	4.99525543046267e-20\\
186	1.38360030510957e-20\\
187	4.0421292906941e-21\\
188	9.87867980739749e-21\\
189	2.69952209647356e-20\\
190	3.64402205794285e-20\\
191	5.14610918270231e-20\\
192	6.82940978025941e-21\\
193	1.14249047401869e-19\\
194	1.34707050129477e-19\\
195	1.73474308750601e-19\\
196	3.53804667149649e-20\\
197	1.39540435718953e-19\\
198	2.26329975777226e-19\\
199	1.04808236228546e-19\\
200	3.01325386140737e-20\\
201	2.1061702933026e-19\\
202	2.09901268942293e-19\\
203	7.3599755925483e-20\\
204	1.95819418903389e-19\\
205	6.80552428368457e-20\\
206	1.76243419518247e-19\\
207	1.99801074179592e-19\\
208	2.01285909622281e-19\\
209	4.71473035198246e-19\\
210	1.33732588048182e-19\\
211	1.33835383323206e-19\\
212	2.97609752445905e-19\\
213	3.23226587405809e-19\\
214	2.84325533059275e-19\\
215	3.1877114863671e-20\\
216	2.20461468015168e-19\\
217	9.8122057028179e-20\\
218	1.01720346530163e-18\\
219	2.39206450058866e-19\\
220	5.28617274300827e-19\\
221	4.01829691168241e-19\\
222	1.30602608679051e-18\\
223	1.69317696751631e-19\\
224	1.91593922681705e-18\\
225	6.0356914842661e-19\\
226	1.40801436540227e-18\\
227	5.92655353867657e-19\\
228	7.27557081052306e-21\\
229	7.625688236148e-19\\
230	7.45441762050991e-19\\
231	1.606571842782e-18\\
232	1.66328464935691e-19\\
233	2.9346624850883e-18\\
234	7.59409275104433e-19\\
235	8.75161269462743e-19\\
236	4.44167849496609e-19\\
237	1.34062861744252e-18\\
238	3.9998740933918e-18\\
239	1.70380489298349e-18\\
240	4.18472813456423e-18\\
241	3.80018403937446e-18\\
242	2.23518413441442e-18\\
243	2.15123691459724e-18\\
244	2.32601279451765e-18\\
245	7.76052438236544e-19\\
246	4.21675759113209e-19\\
247	2.98664431853065e-18\\
248	1.35642406519964e-17\\
249	5.01325715081204e-18\\
250	7.33200509882655e-18\\
251	9.99083781758917e-19\\
252	1.06886705394555e-17\\
253	1.97375962050573e-17\\
254	8.0046480576368e-19\\
255	1.2020970801336e-17\\
256	6.94671819994516e-18\\
257	4.14945571585104e-18\\
258	2.61825293697108e-17\\
259	8.66573831935776e-18\\
260	1.30808904074484e-17\\
261	4.81331116789228e-18\\
262	1.45078631059192e-17\\
263	6.78414656790891e-18\\
264	2.21948238029292e-17\\
265	3.77322071716751e-18\\
266	3.92188199439551e-18\\
267	4.00654912531148e-17\\
268	1.29696681600808e-17\\
269	8.01892268187825e-19\\
270	2.45128915377919e-17\\
271	9.16555940750256e-18\\
272	5.43217702162299e-17\\
273	4.094140712284e-17\\
274	2.45696242699281e-17\\
275	1.95819823701523e-17\\
276	5.70245113218982e-17\\
277	2.73748498077769e-17\\
278	3.84712154349597e-17\\
279	7.23734222220694e-17\\
280	3.84641347858533e-17\\
281	1.39714200946653e-18\\
282	8.4168530729573e-17\\
283	3.13346580795108e-17\\
284	5.90391917773348e-17\\
285	2.00119482950702e-16\\
286	3.1371613074813e-18\\
287	1.90195257292592e-16\\
288	1.00886218459871e-16\\
289	1.50633742072383e-16\\
290	8.98094666929184e-17\\
291	1.42090739937894e-16\\
292	1.72425456848405e-16\\
293	1.17375528091405e-17\\
294	5.89696073794252e-18\\
295	3.18867588533992e-16\\
296	1.46459512823163e-16\\
297	1.93320768621853e-16\\
298	3.30988483719413e-16\\
299	1.07354866908923e-16\\
300	8.81642233168107e-16\\
301	2.33621418644542e-16\\
302	2.32999224977962e-16\\
303	3.6798010334552e-16\\
304	7.37907652464916e-16\\
305	1.96483270245833e-16\\
306	8.01114077742814e-16\\
307	4.59812721124059e-16\\
308	6.40760393382022e-16\\
309	3.52224220188008e-16\\
310	9.35856224592295e-16\\
311	1.7254828020295e-15\\
312	8.13851988924949e-16\\
313	8.24692941656494e-16\\
314	1.16927599750023e-15\\
315	1.79958681021439e-15\\
316	1.49871544362219e-16\\
317	1.74215014360318e-15\\
318	4.43013077922296e-16\\
319	3.51255875168306e-16\\
320	5.95529202186528e-16\\
321	1.5300021183315e-15\\
322	3.57414616350424e-15\\
323	2.58823512376222e-15\\
324	3.76780551411824e-15\\
325	1.90925553578542e-16\\
326	1.65912623052954e-15\\
327	3.33512285278789e-16\\
328	2.01924944319513e-15\\
329	1.95300966009398e-15\\
330	3.85515824671129e-15\\
331	8.28772266084796e-16\\
332	1.861219877943e-15\\
333	1.39339750746401e-14\\
334	1.12977295404635e-14\\
335	1.56365100051534e-15\\
336	1.50524736285434e-15\\
337	8.12945642740195e-15\\
338	4.83593966532057e-15\\
339	2.4971271358154e-15\\
340	1.93441851180615e-15\\
341	7.29832635069192e-15\\
342	1.56470612934453e-14\\
343	1.62359900209352e-14\\
344	1.44710340260198e-14\\
345	7.48285183810505e-16\\
346	1.21970043233603e-14\\
347	1.14536446898579e-14\\
348	2.20561236172887e-14\\
349	3.56124549601351e-16\\
350	8.79834578693064e-15\\
351	6.46222080214313e-15\\
352	1.20415875520078e-14\\
353	6.87923501015562e-15\\
354	1.23977000552595e-14\\
355	3.15508372924533e-15\\
356	3.68569500382061e-14\\
357	1.13797114766517e-14\\
358	3.89134219631951e-14\\
359	6.24851362462561e-14\\
360	4.68804761443288e-14\\
361	2.94981202880455e-14\\
362	5.77371198645677e-14\\
363	2.23187709553927e-14\\
364	5.66717391234156e-15\\
365	3.26173616401931e-14\\
366	7.19686891910847e-14\\
367	9.98454897444684e-14\\
368	6.84069541045262e-14\\
369	1.35737625301022e-13\\
370	2.24055405921717e-13\\
371	1.03369802374946e-13\\
372	1.15221061374693e-13\\
373	1.44978630514016e-13\\
374	2.83532724925734e-14\\
375	2.46301743314928e-14\\
376	3.47367066708506e-13\\
377	2.67088863884869e-13\\
378	3.17025384119597e-13\\
379	2.54983973055536e-13\\
380	9.42189202405855e-14\\
381	1.73172964101684e-13\\
382	1.29131138184542e-13\\
383	3.01474837117883e-13\\
384	6.01610770127129e-13\\
385	8.46370115879735e-13\\
386	8.57966780730837e-13\\
387	3.47761925418172e-13\\
388	9.42581570727414e-15\\
389	1.67955377839725e-13\\
390	3.65048706549594e-13\\
391	1.7158919221093e-13\\
392	5.53344436204934e-13\\
393	1.33203273427991e-13\\
394	1.94606140020114e-13\\
395	9.65782372758845e-13\\
396	5.46948956169578e-13\\
397	2.51431933868659e-14\\
398	1.35456748463447e-12\\
399	9.26409693558631e-13\\
400	1.52350419971107e-12\\
401	1.25102622360284e-12\\
402	1.53370150669572e-12\\
403	1.38868466708702e-12\\
404	1.48372619767546e-12\\
405	1.97124301035336e-12\\
406	8.2538021986499e-13\\
407	1.55209085715052e-12\\
408	5.27579045249944e-12\\
409	7.938439024982e-13\\
410	2.10910918360801e-12\\
411	3.51752377927775e-12\\
412	2.84870355236607e-12\\
413	4.90601315574075e-13\\
414	3.63875185798918e-14\\
415	1.51533080674587e-12\\
416	3.18751246630997e-12\\
417	6.97497566362725e-12\\
418	1.11271294986841e-12\\
419	1.77187107248795e-12\\
420	6.87625018521712e-12\\
421	6.8635296689359e-12\\
422	1.22304474009318e-11\\
423	1.45134346847651e-12\\
424	4.64836837977168e-12\\
425	1.49309315044731e-11\\
426	1.20305214845648e-11\\
427	8.45335798717864e-12\\
428	9.33355934143246e-12\\
429	1.00388508941161e-11\\
430	1.5529656499753e-11\\
431	1.44894182891405e-11\\
432	2.03699361415383e-11\\
433	2.78157638494174e-11\\
434	1.81612203125188e-11\\
435	1.52334491975035e-11\\
436	3.13085917643813e-11\\
437	2.95333048402314e-11\\
438	1.15647180896725e-11\\
439	4.70413525473495e-12\\
440	3.10333542614718e-11\\
441	5.8654035943825e-11\\
442	5.83992720405893e-11\\
443	5.40406259190628e-12\\
444	5.29485511404658e-11\\
445	5.99119513527259e-11\\
446	1.2177280932357e-11\\
447	6.3978560341211e-11\\
448	2.31383374993819e-11\\
449	1.83000536618652e-10\\
450	1.02919894985777e-10\\
451	4.52667461693796e-11\\
452	6.62742564183266e-11\\
453	7.50691393668682e-13\\
454	1.30077670574144e-10\\
455	5.29933826329292e-11\\
456	5.41623317558757e-11\\
457	9.96523455158995e-11\\
458	4.13728916690597e-11\\
459	1.82156699662122e-10\\
460	1.6517762875365e-10\\
461	5.83328608024647e-10\\
462	1.0935751419122e-10\\
463	2.70169216171684e-10\\
464	6.56170116784242e-11\\
465	6.94349728306299e-11\\
466	1.39605632305963e-10\\
467	2.57141286841846e-10\\
468	4.90098621280291e-10\\
469	7.81419061258243e-13\\
470	2.65171252307569e-10\\
471	2.09021013164196e-10\\
472	5.23750565732208e-10\\
473	5.20314035403379e-10\\
474	5.00145146354504e-10\\
475	6.00542701651935e-10\\
476	9.20983363072698e-10\\
477	2.76184758005527e-10\\
478	1.29505010542659e-10\\
479	4.54884340532972e-10\\
480	5.05085103095501e-11\\
481	4.91089091941462e-10\\
482	3.5935134665929e-10\\
483	2.05625529949378e-09\\
484	1.18674331101834e-10\\
485	4.42923404643265e-09\\
486	7.89150061336216e-10\\
487	1.65279967442932e-10\\
488	2.00400418703027e-10\\
489	2.37318980998858e-09\\
490	2.14218922167178e-09\\
491	2.60817589969734e-09\\
492	4.49075754609217e-09\\
493	6.19440587299058e-10\\
494	4.09003805533485e-09\\
495	9.85132910316246e-10\\
496	4.36716310242716e-09\\
497	4.18864236983216e-09\\
498	2.51999204729137e-09\\
499	3.7670121784117e-09\\
500	6.63702330502011e-09\\
501	1.03758164744387e-09\\
502	3.59270241057109e-09\\
503	6.56226853552478e-09\\
504	7.28570850193873e-09\\
};
\addplot [color=mycolor8,solid,line width=2.0pt]
  table[row sep=crcr]{%
1	3.3223796846225\\
2	2.75634382076456\\
3	2.39525743436197\\
4	2.13090886100636\\
5	1.92294517532823\\
6	1.75197354347843\\
7	1.60719212789209\\
8	1.48196874075294\\
9	1.37193857920247\\
10	1.27407531591165\\
11	1.18619349545047\\
12	1.10666239163264\\
13	1.03423196228491\\
14	0.967922047176277\\
15	0.906949110511888\\
16	0.850676238508858\\
17	0.798578068565029\\
18	0.750215606212103\\
19	0.705217767581327\\
20	0.66326760508217\\
21	0.624091862466115\\
22	0.587452940848749\\
23	0.553142639647965\\
24	0.520977223717431\\
25	0.490793494763525\\
26	0.462445632570907\\
27	0.435802632867681\\
28	0.410746212307318\\
29	0.387169082556889\\
30	0.364973518529085\\
31	0.344070162854974\\
32	0.324377021462181\\
33	0.305818614777642\\
34	0.288325256444868\\
35	0.271832437123438\\
36	0.256280295348948\\
37	0.2416131608846\\
38	0.227779158718766\\
39	0.21472986402537\\
40	0.20242000013214\\
41	0.190807172931377\\
42	0.179851636291429\\
43	0.169516083940283\\
44	0.15976546403867\\
45	0.150566813272092\\
46	0.141889107795569\\
47	0.133703128782114\\
48	0.125981340672339\\
49	0.118697780511174\\
50	0.111827956998798\\
51	0.10534875808497\\
52	0.0992383661056984\\
53	0.0934761796042096\\
54	0.0880427410988416\\
55	0.0829196701626032\\
56	0.0780896012656402\\
57	0.0735361259053011\\
58	0.0692437386109477\\
59	0.0651977864638535\\
60	0.061384421817908\\
61	0.0577905579456095\\
62	0.0544038273670004\\
63	0.0512125426476135\\
64	0.0485237347296197\\
65	0.0470389701648886\\
66	0.0455894109387334\\
67	0.0441749771723972\\
68	0.0427954860187576\\
69	0.0414504644872373\\
70	0.0401397893164255\\
71	0.03886294932067\\
72	0.0376197060828432\\
73	0.0364094508259161\\
74	0.035231943818455\\
75	0.0340865330042188\\
76	0.0329729491283883\\
77	0.0318903666762003\\
78	0.0308385518702128\\
79	0.0298167708937083\\
80	0.0288244777498434\\
81	0.0278612038476602\\
82	0.0269263058828333\\
83	0.0260192101268979\\
84	0.0251392803384487\\
85	0.024285982523633\\
86	0.0234587194779694\\
87	0.0226568399243678\\
88	0.0218797499469394\\
89	0.0211269712037003\\
90	0.0203977828371702\\
91	0.0196915909352224\\
92	0.0190078673732819\\
93	0.0183460526353043\\
94	0.0177055612452719\\
95	0.0170857866613678\\
96	0.0164861056740175\\
97	0.0159061402468899\\
98	0.0153451648845477\\
99	0.0148028173192722\\
100	0.0142783937474302\\
101	0.0137714341384552\\
102	0.0132815355409255\\
103	0.0128081068606551\\
104	0.0123506471379109\\
105	0.0119087050930444\\
106	0.0114818703971631\\
107	0.0110695958506559\\
108	0.0106713935406759\\
109	0.0102869387426514\\
110	0.00991576756646317\\
111	0.00955740630656443\\
112	0.00921153027894395\\
113	0.00887771518986114\\
114	0.00855548632084968\\
115	0.00824462988829211\\
116	0.00794470274745684\\
117	0.00765531370221151\\
118	0.00737613288010941\\
119	0.00710683392810907\\
120	0.00684709464654887\\
121	0.00659659754203368\\
122	0.00635503030581308\\
123	0.0061220862241429\\
124	0.00589746452643425\\
125	0.00568087067663736\\
126	0.00547208548641631\\
127	0.00527086168463864\\
128	0.00507684103971418\\
129	0.00488978250436567\\
130	0.00470959972752638\\
131	0.00453585056732919\\
132	0.00436845990117263\\
133	0.00420711888567915\\
134	0.00405166554881387\\
135	0.00390184863017584\\
136	0.00375754167429587\\
137	0.00361841213407565\\
138	0.00348449679404794\\
139	0.00335541448094112\\
140	0.00323102399452768\\
141	0.00311129128395304\\
142	0.00299591305394742\\
143	0.00288474393611082\\
144	0.00277766820452641\\
145	0.00267459153787009\\
146	0.00257529477424523\\
147	0.00247964737104756\\
148	0.0023875222696379\\
149	0.00229879588665405\\
150	0.00221334809682183\\
151	0.0021310622081181\\
152	0.00205182980747853\\
153	0.00197554388293231\\
154	0.00190208154372051\\
155	0.00183134001730245\\
156	0.00176321972335103\\
157	0.00169763545288237\\
158	0.00163449584292152\\
159	0.00157370328337514\\
160	0.00151516996232104\\
161	0.00145881098528555\\
162	0.00140454430885395\\
163	0.00135229067311737\\
164	0.00130197353340744\\
165	0.0012535189914562\\
166	0.00120690355758502\\
167	0.00116203195298414\\
168	0.00111881916046741\\
169	0.00107720092370389\\
170	0.00103711530352957\\
171	0.000998584147287929\\
172	0.000961471360430899\\
173	0.000925720083503556\\
174	0.000891296467758984\\
175	0.000858201746122056\\
176	0.000826314743974876\\
177	0.000795587583166224\\
178	0.000766067034590812\\
179	0.000737621164618396\\
180	0.000710203347551008\\
181	0.000683872102572813\\
182	0.00065848735455809\\
183	0.000634042077522601\\
184	0.000610541504370232\\
185	0.000587877322216102\\
186	0.000566096010694572\\
187	0.000545109492377627\\
188	0.000524899749353658\\
189	0.000505465749133016\\
190	0.000486718363263527\\
191	0.000468720492062147\\
192	0.00045134612461073\\
193	0.000434663426259085\\
194	0.000418565965611073\\
195	0.000403098674390812\\
196	0.000388181533671705\\
197	0.000373843918988648\\
198	0.000360017488220343\\
199	0.000346729577591631\\
200	0.000333910772361889\\
201	0.000321598011984392\\
202	0.000309709844331429\\
203	0.00029830277288756\\
204	0.000287281742341605\\
205	0.000276707880193072\\
206	0.000266499648024165\\
207	0.000256687139142286\\
208	0.000247233422531856\\
209	0.000238123491999254\\
210	0.000229370088150382\\
211	0.000220914265513628\\
212	0.000212804926380095\\
213	0.000204976899728315\\
214	0.000197440912864444\\
215	0.000190195398337423\\
216	0.00018319681137946\\
217	0.000176483128057292\\
218	0.00017000642702758\\
219	0.00016375989950479\\
220	0.000157767348357929\\
221	0.000151979268179225\\
222	0.000146408092026817\\
223	0.000141053597068286\\
224	0.000135882069717042\\
225	0.000130910256725514\\
226	0.000126126943865046\\
227	0.00012150725726201\\
228	0.000117067200256393\\
229	0.000113092493895905\\
230	0.000109257882082048\\
231	0.000105558945715067\\
232	0.000101980821797909\\
233	9.85228229399837e-05\\
234	9.51846409087409e-05\\
235	9.19556426790535e-05\\
236	8.88364042195278e-05\\
237	8.58247359989406e-05\\
238	8.29122962375806e-05\\
239	8.00965081158012e-05\\
240	7.73805853748755e-05\\
241	7.47546698312007e-05\\
242	7.22160528248139e-05\\
243	6.97620828624679e-05\\
244	6.73948689957626e-05\\
245	6.5106901941725e-05\\
246	6.28954909762669e-05\\
247	6.07582471010559e-05\\
248	5.86928385502006e-05\\
249	5.66982529415228e-05\\
250	5.47725027892909e-05\\
251	5.29115718181927e-05\\
252	5.11133786140761e-05\\
253	4.93758964950075e-05\\
254	4.76971529286274e-05\\
255	4.60752290365463e-05\\
256	4.45082588731438e-05\\
257	4.29944287754225e-05\\
258	4.1531976529896e-05\\
259	4.01191906576059e-05\\
260	3.87544095028503e-05\\
261	3.74360203743151e-05\\
262	3.61624586551201e-05\\
263	3.49322068498026e-05\\
264	3.3743793669494e-05\\
265	3.25957930149556e-05\\
266	3.1486823017346e-05\\
267	3.04155451038568e-05\\
268	2.93806628937077e-05\\
269	2.83809212762165e-05\\
270	2.74151054382443e-05\\
271	2.64820397557486e-05\\
272	2.55805869233683e-05\\
273	2.47096468761754e-05\\
274	2.38681558268894e-05\\
275	2.30550853537181e-05\\
276	2.22704994135015e-05\\
277	2.15128972165246e-05\\
278	2.07808587306957e-05\\
279	2.00734822435322e-05\\
280	1.93898970097806e-05\\
281	1.87292622912949e-05\\
282	1.80912755407547e-05\\
283	1.74760107061189e-05\\
284	1.68813413887747e-05\\
285	1.63065349738645e-05\\
286	1.57508847422605e-05\\
287	1.5214318070278e-05\\
288	1.46969714371892e-05\\
289	1.41967864877302e-05\\
290	1.37131501904975e-05\\
291	1.32455846273416e-05\\
292	1.27953748201293e-05\\
293	1.23599719694667e-05\\
294	1.19388461055081e-05\\
295	1.15320439482147e-05\\
296	1.11400529316796e-05\\
297	1.07608227493117e-05\\
298	1.03938966837802e-05\\
299	1.00405048124586e-05\\
300	9.69894870284094e-06\\
301	9.3683805708622e-06\\
302	9.04958971492675e-06\\
303	8.74188774346862e-06\\
304	8.44399119870332e-06\\
305	8.15663802367084e-06\\
306	7.87933935519902e-06\\
307	7.61078533706439e-06\\
308	7.35196810008887e-06\\
309	7.10196010000885e-06\\
310	6.85974663472422e-06\\
311	6.62680877994148e-06\\
312	6.40128774342941e-06\\
313	6.18319224887642e-06\\
314	5.97322518114396e-06\\
315	5.76966895238229e-06\\
316	5.57361353603625e-06\\
317	5.38404844441231e-06\\
318	5.20062026776458e-06\\
319	5.02408328806325e-06\\
320	4.85280947071232e-06\\
321	4.68811103360167e-06\\
322	4.5285523668781e-06\\
323	4.3745686104657e-06\\
324	4.22591313942178e-06\\
325	4.08198743162558e-06\\
326	3.94347676380136e-06\\
327	3.80898867824442e-06\\
328	3.67991489813591e-06\\
329	3.55449873357117e-06\\
330	3.43398137925277e-06\\
331	3.31704922107434e-06\\
332	3.20450786261972e-06\\
333	3.09546590138154e-06\\
334	2.99039961948466e-06\\
335	2.88869440190354e-06\\
336	2.79063140951052e-06\\
337	2.69574745903611e-06\\
338	2.60424368914158e-06\\
339	2.51570096931886e-06\\
340	2.43033870184206e-06\\
341	2.34769047668948e-06\\
342	2.26807699910125e-06\\
343	2.19090765263275e-06\\
344	2.11667395699777e-06\\
345	2.04459695041237e-06\\
346	1.97539647839307e-06\\
347	1.90806581823466e-06\\
348	1.84355989674145e-06\\
349	1.78081204094127e-06\\
350	1.72052501490327e-06\\
351	1.66206207996566e-06\\
352	1.6056952727439e-06\\
353	1.55123796741208e-06\\
354	1.49851405950585e-06\\
355	1.44780061894778e-06\\
356	1.39847490121525e-06\\
357	1.35124739220061e-06\\
358	1.3053241190164e-06\\
359	1.26110976150784e-06\\
360	1.21836515631912e-06\\
361	1.17695104240312e-06\\
362	1.13717563099414e-06\\
363	1.09849671314777e-06\\
364	1.061362142174e-06\\
365	1.02537845947381e-06\\
366	9.9055830116157e-07\\
367	9.57091388364972e-07\\
368	9.24552445269455e-07\\
369	8.93305907467834e-07\\
370	8.63049788080161e-07\\
371	8.33715267489765e-07\\
372	8.05590248731392e-07\\
373	7.78248210764332e-07\\
374	7.51898265427542e-07\\
375	7.26488016411508e-07\\
376	7.01789121926311e-07\\
377	6.78110266605358e-07\\
378	6.55161750628963e-07\\
379	6.32884941786926e-07\\
380	6.1156999375811e-07\\
381	5.90852523352225e-07\\
382	5.70807686450792e-07\\
383	5.51569725892875e-07\\
384	5.32873196235073e-07\\
385	5.14825816644304e-07\\
386	4.97468855442662e-07\\
387	4.80602081687209e-07\\
388	4.64342156725195e-07\\
389	4.48687920595603e-07\\
390	4.33477082673051e-07\\
391	4.18816695102464e-07\\
392	4.0470315720853e-07\\
393	3.90990341436748e-07\\
394	3.77761429248835e-07\\
395	3.6504143796634e-07\\
396	3.52683288795674e-07\\
397	3.41079536170241e-07\\
398	3.29843508950489e-07\\
399	3.18947216193521e-07\\
400	3.08454162656346e-07\\
401	2.98298367340522e-07\\
402	2.88446416085719e-07\\
403	2.78950045284887e-07\\
404	2.69767628680029e-07\\
405	2.60856857003944e-07\\
406	2.52269663825189e-07\\
407	2.43964019652765e-07\\
408	2.35901163492258e-07\\
409	2.28142508440499e-07\\
410	2.20626430014192e-07\\
411	2.13328429587989e-07\\
412	2.06323011120446e-07\\
413	1.99517817378592e-07\\
414	1.92932450460148e-07\\
415	1.86588422934619e-07\\
416	1.80423234574789e-07\\
417	1.74485756510023e-07\\
418	1.68736992378626e-07\\
419	1.63159266541868e-07\\
420	1.57799076028198e-07\\
421	1.52586110857555e-07\\
422	1.47563467223222e-07\\
423	1.42701308547544e-07\\
424	1.37986658543809e-07\\
425	1.33451695027986e-07\\
426	1.29037800888909e-07\\
427	1.24797718825675e-07\\
428	1.20679233717169e-07\\
429	1.16702112862299e-07\\
430	1.12859050993563e-07\\
431	1.09129768333105e-07\\
432	1.05543392692198e-07\\
433	1.02048615779893e-07\\
434	9.87004469266139e-08\\
435	9.54374943518133e-08\\
436	9.23001941544044e-08\\
437	8.92533087437641e-08\\
438	8.63144657969883e-08\\
439	8.34689020123847e-08\\
440	8.07167568339651e-08\\
441	7.80587621207474e-08\\
442	7.54821822823715e-08\\
443	7.29989189096614e-08\\
444	7.05873723916284e-08\\
445	6.82668730433988e-08\\
446	6.60104033656239e-08\\
447	6.38414894282846e-08\\
448	6.17306907813031e-08\\
449	5.97029812254846e-08\\
450	5.77289744896348e-08\\
451	5.58327446498197e-08\\
452	5.39871516380686e-08\\
453	5.2213386503297e-08\\
454	5.04883059804229e-08\\
455	4.88285953892387e-08\\
456	4.72165817555492e-08\\
457	4.56630910861122e-08\\
458	4.41571295084486e-08\\
459	4.27025783622526e-08\\
460	4.12960634577075e-08\\
461	3.99336839151943e-08\\
462	3.86203984348299e-08\\
463	3.73438613365806e-08\\
464	3.61179566255032e-08\\
465	3.49213991057695e-08\\
466	3.37773746750258e-08\\
467	3.26596767408205e-08\\
468	3.15880503976018e-08\\
469	3.05452907412018e-08\\
470	2.95400424121794e-08\\
471	2.85674648381473e-08\\
472	2.76240514907045e-08\\
473	2.67171866852323e-08\\
474	2.58315289158872e-08\\
475	2.49860896417431e-08\\
476	2.41604771744619e-08\\
477	2.33663230986281e-08\\
478	2.25969083444966e-08\\
479	2.18505649129952e-08\\
480	2.11337374267373e-08\\
481	2.04338510556568e-08\\
482	1.97643696964178e-08\\
483	1.91124707171753e-08\\
484	1.84826118854176e-08\\
485	1.78756014435066e-08\\
486	1.72830789679779e-08\\
487	1.67177089949178e-08\\
488	1.61661120046119e-08\\
489	1.56336028567239e-08\\
490	1.51202748099877e-08\\
491	1.46192906669285e-08\\
492	1.41408813547628e-08\\
493	1.36747599910336e-08\\
494	1.32235955518922e-08\\
495	1.27900667834524e-08\\
496	1.23670149676514e-08\\
497	1.19612950655323e-08\\
498	1.15679155143766e-08\\
499	1.11847802131138e-08\\
500	1.08191322567563e-08\\
501	1.04623474328491e-08\\
502	1.01174126854175e-08\\
503	9.78586012223559e-09\\
504	9.4624077462413e-09\\
505	9.15170161874812e-09\\
506	8.85119710858362e-09\\
507	8.55808757194154e-09\\
508	8.27806712067058e-09\\
509	8.00582089510726e-09\\
510	7.74057173913434e-09\\
511	7.48781836534818e-09\\
512	7.24127602325098e-09\\
513	7.00187552382658e-09\\
514	6.7730479003103e-09\\
515	6.54987264425699e-09\\
516	6.33365360158678e-09\\
517	6.12657213849843e-09\\
518	5.92463056392489e-09\\
519	5.72920644259511e-09\\
520	5.541882508453e-09\\
521	5.35922595190641e-09\\
522	5.18244824831982e-09\\
523	5.01306551825564e-09\\
524	4.8479176228966e-09\\
525	4.68790517516027e-09\\
526	4.53480897277814e-09\\
527	4.38555058934753e-09\\
528	4.24054835690413e-09\\
529	4.10222877889055e-09\\
530	3.96738553121168e-09\\
531	3.83597242858968e-09\\
532	3.71097641505003e-09\\
533	3.58920448917388e-09\\
534	3.47053497051775e-09\\
535	3.35707639464999e-09\\
536	3.24715454524949e-09\\
537	3.14003667511997e-09\\
538	3.03693159509066e-09\\
539	2.93774959914117e-09\\
540	2.8410980235094e-09\\
541	2.74730993510275e-09\\
542	2.65785793374107e-09\\
543	2.57068855091802e-09\\
544	2.48576981221049e-09\\
545	2.40464270717666e-09\\
546	2.32606112149369e-09\\
547	2.24950635896448e-09\\
548	2.17552997838766e-09\\
549	2.10472350659074e-09\\
550	2.03574312962473e-09\\
551	1.96856131395862e-09\\
552	1.90442595027207e-09\\
553	1.84230142252773e-09\\
554	1.78179337950723e-09\\
555	1.72316116930915e-09\\
556	1.66724056782641e-09\\
557	1.6127730262383e-09\\
558	1.55973722826275e-09\\
559	1.50880019589295e-09\\
560	1.45979939247809e-09\\
561	1.41208289505812e-09\\
562	1.36563205188622e-09\\
563	1.32130750785109e-09\\
564	1.27840404928747e-09\\
565	1.23663657092266e-09\\
566	1.19598642100982e-09\\
567	1.15738352235439e-09\\
568	1.11985265505155e-09\\
569	1.08332365300612e-09\\
570	1.04779829257495e-09\\
571	1.01410080333153e-09\\
572	9.81300374292005e-10\\
573	9.49383682780081e-10\\
574	9.18360498758375e-10\\
575	8.89041729124074e-10\\
576	8.60336690777785e-10\\
577	8.32821811513895e-10\\
578	8.06093858329859e-10\\
579	7.80161713009875e-10\\
580	7.55287388187753e-10\\
581	7.3094952313113e-10\\
582	7.07716552028614e-10\\
583	6.85078660467298e-10\\
584	6.63167298853296e-10\\
585	6.4212368755534e-10\\
586	6.21548146284567e-10\\
587	6.01879435180308e-10\\
588	5.82765835588361e-10\\
589	5.6419935390295e-10\\
590	5.46456213612601e-10\\
591	5.29119859038474e-10\\
592	5.12419440212852e-10\\
593	4.96339858102601e-10\\
594	4.80642192712821e-10\\
595	4.65648852809863e-10\\
596	4.51103154830435e-10\\
597	4.36913616397305e-10\\
598	4.23444390662553e-10\\
599	4.10476097556511e-10\\
600	3.97901267490397e-10\\
601	3.85714571393692e-10\\
602	3.74036801531474e-10\\
603	3.62874175152683e-10\\
604	3.52061491071254e-10\\
605	3.41591643859829e-10\\
606	3.3145841626947e-10\\
607	3.21658255586499e-10\\
608	3.12337711250166e-10\\
609	3.03389313671687e-10\\
610	2.94741120399067e-10\\
611	2.86388690540207e-10\\
612	2.78326695024589e-10\\
613	2.70551581138534e-10\\
614	2.63087329699374e-10\\
615	2.56023646727499e-10\\
616	2.49221088211016e-10\\
617	2.42674325079406e-10\\
618	2.36380692797411e-10\\
619	2.30336638651352e-10\\
620	2.2453772174913e-10\\
621	2.18980389377066e-10\\
622	2.13661088821482e-10\\
623	2.08668637924347e-10\\
624	2.03900007988977e-10\\
625	1.99351646301693e-10\\
626	1.95020000148816e-10\\
627	1.90902404995086e-10\\
628	1.86996196305245e-10\\
629	1.83298709544033e-10\\
630	1.79807280176192e-10\\
631	1.76519243666462e-10\\
632	1.73432823658004e-10\\
633	1.70545355615559e-10\\
634	1.67853286825448e-10\\
635	1.65356617287671e-10\\
636	1.63050906110129e-10\\
637	1.60935265114404e-10\\
638	1.59007917943654e-10\\
639	1.57266200062622e-10\\
640	1.55708335114468e-10\\
641	1.54331658563933e-10\\
642	1.53139723124696e-10\\
643	1.52127199726237e-10\\
644	1.5128964747646e-10\\
645	1.50627066375364e-10\\
646	1.50136791887689e-10\\
647	1.49817935835017e-10\\
648	1.49670498217347e-10\\
649	1.49692702677839e-10\\
650	1.49883661038075e-10\\
651	1.50243373298053e-10\\
652	1.50770063100936e-10\\
653	1.51463730446721e-10\\
654	1.52322598978571e-10\\
655	1.53347556874905e-10\\
656	1.54536827778884e-10\\
657	1.55889523512087e-10\\
658	1.57407420431355e-10\\
659	1.59087854001427e-10\\
660	1.60930824222305e-10\\
661	1.62937219272408e-10\\
662	1.65105262794896e-10\\
663	1.67435842968189e-10\\
664	1.69928071613867e-10\\
665	1.72581060553512e-10\\
666	1.75396586143961e-10\\
667	1.78371983849956e-10\\
668	1.81509918206757e-10\\
669	1.84807724679104e-10\\
670	1.88266291445416e-10\\
671	1.91885618505694e-10\\
672	1.95665705859938e-10\\
673	1.99605665329727e-10\\
674	2.03762340333924e-10\\
675	2.08124184553071e-10\\
676	2.12655670850381e-10\\
677	2.17355022869015e-10\\
678	2.2222490514423e-10\\
679	2.27263541319189e-10\\
680	2.32470931393891e-10\\
681	2.37848851725175e-10\\
682	2.43395525956203e-10\\
683	2.4923885177941e-10\\
684	2.55295340423345e-10\\
685	2.61533017464899e-10\\
686	2.67952771082491e-10\\
687	2.74554601276122e-10\\
688	2.81337619867372e-10\\
689	2.88301826856241e-10\\
690	2.95563573615709e-10\\
691	3.03123748324197e-10\\
692	3.10879322285018e-10\\
693	3.18831183676593e-10\\
694	3.26989102461539e-10\\
695	3.35656835659393e-10\\
696	3.44505757254865e-10\\
697	3.53812978914902e-10\\
698	3.63290730831523e-10\\
699	3.72934572112626e-10\\
700	3.82915921193216e-10\\
};
\addlegendentry{a = 0.010};

\addplot [color=mycolor8,dashed,line width=2.0pt,forget plot]
  table[row sep=crcr]{%
1	0\\
2	6.93889390390723e-25\\
3	1.11022302462516e-23\\
4	5.55111512312578e-24\\
5	2.28884913772994e-23\\
6	3.33066907387547e-23\\
7	4.74287484026755e-23\\
8	2.48253415324727e-23\\
9	8.62239587807288e-23\\
10	4.97475625386571e-23\\
11	4.62777927105281e-23\\
12	5.41055774894659e-23\\
13	5.21479921135491e-23\\
14	6.75322301446426e-23\\
15	1.25248476555686e-22\\
16	1.27570012268057e-22\\
17	4.93686575128598e-23\\
18	1.3612263586542e-22\\
19	8.49787896895541e-23\\
20	8.87771750785001e-23\\
21	1.18462181896914e-22\\
22	1.6887762462497e-22\\
23	9.30665806957336e-23\\
24	1.99941311028126e-22\\
25	1.76176862793457e-22\\
26	1.53467251522634e-22\\
27	1.04526878280103e-22\\
28	1.8371696240518e-22\\
29	1.06283104123546e-22\\
30	1.49115300298434e-22\\
31	3.0473774284058e-22\\
32	3.10527761316497e-22\\
33	1.81016046550664e-22\\
34	1.96164629256508e-22\\
35	1.96258567390534e-22\\
36	2.05714507675129e-22\\
37	1.92716457645847e-22\\
38	1.97307847750613e-22\\
39	1.98723912777653e-22\\
40	3.89142672637273e-22\\
41	2.4681792899e-22\\
42	1.90329761669286e-22\\
43	2.35211424139527e-22\\
44	3.45453967186125e-22\\
45	2.20100864822863e-22\\
46	2.90607967933353e-22\\
47	2.40824685188467e-22\\
48	1.92112146952708e-22\\
49	3.12140726150206e-22\\
50	3.01122812906858e-22\\
51	4.03057764081609e-22\\
52	3.50076663945669e-22\\
53	2.68885496284751e-22\\
54	2.84772915936188e-22\\
55	3.00546437079365e-22\\
56	3.0195374954928e-22\\
57	3.64620594399057e-22\\
58	3.57980522406862e-22\\
59	2.71552222315703e-22\\
60	2.37647745984887e-22\\
61	2.52070841163296e-22\\
62	3.75003122694571e-22\\
63	4.20146525263784e-22\\
64	3.66786676268056e-22\\
65	3.43623992667434e-22\\
66	3.93295173598808e-22\\
67	2.63972535611988e-22\\
68	3.13710196337115e-22\\
69	4.35676897495677e-22\\
70	4.02992057133879e-22\\
71	4.93577331717451e-22\\
72	3.91324351440705e-22\\
73	3.84844896638343e-22\\
74	3.96146582058441e-22\\
75	4.97574820266671e-22\\
76	4.20561230587547e-22\\
77	4.54728509849664e-22\\
78	4.81385858352391e-22\\
79	3.98457261990286e-22\\
80	4.21568643062805e-22\\
81	5.09712704448107e-22\\
82	6.00618487482504e-22\\
83	6.54724825039372e-22\\
84	5.68253576725461e-22\\
85	4.46846101256592e-22\\
86	4.5926360924537e-22\\
87	4.77106356017176e-22\\
88	3.38423971176423e-22\\
89	6.03054255216092e-22\\
90	4.54178223639944e-22\\
91	4.60698597215138e-22\\
92	5.75519167400953e-22\\
93	5.3136321560281e-22\\
94	4.45154921581816e-22\\
95	4.72023503473814e-22\\
96	5.71049573670532e-22\\
97	6.06272453408541e-22\\
98	6.312392723008e-22\\
99	5.06473218312753e-22\\
100	5.00441596059339e-22\\
101	5.8985463180375e-22\\
102	5.69809271836754e-22\\
103	5.07717883843323e-22\\
104	5.97155420211561e-22\\
105	6.47111290179805e-22\\
106	5.95279276986066e-22\\
107	5.83533052965519e-22\\
108	5.61574333535697e-22\\
109	5.50327100005776e-22\\
110	7.02092375997825e-22\\
111	5.53940667336864e-22\\
112	5.72379329093204e-22\\
113	5.6823115628724e-22\\
114	6.62066935850085e-22\\
115	8.45638889218615e-22\\
116	6.48596405113776e-22\\
117	6.32390036345417e-22\\
118	6.2918715213364e-22\\
119	6.43948794855671e-22\\
120	8.2823895941474e-22\\
121	6.57618025700539e-22\\
122	6.40634859335056e-22\\
123	7.60928226943695e-22\\
124	7.93088790219894e-22\\
125	9.28438435950124e-22\\
126	7.06606619316611e-22\\
127	7.35504690375033e-22\\
128	7.55269393315821e-22\\
129	6.50373465122527e-22\\
130	6.85480629469103e-22\\
131	7.28846767642348e-22\\
132	6.95593500049346e-22\\
133	5.64419338578261e-22\\
134	7.93564157133452e-22\\
135	7.14349233138232e-22\\
136	8.33170818468909e-22\\
137	6.14417857574734e-22\\
138	6.78487076388405e-22\\
139	7.988771439568e-22\\
140	6.41757122468341e-22\\
141	6.88809044109244e-22\\
142	7.01366485996445e-22\\
143	8.56038808134629e-22\\
144	1.01647952499542e-21\\
145	7.82582308460354e-22\\
146	1.11619307018052e-21\\
147	9.03834880682849e-22\\
148	7.66111964243171e-22\\
149	9.22426093845832e-22\\
150	1.06218363585945e-21\\
151	1.15815980220714e-21\\
152	9.96309912378883e-22\\
153	1.05552440176413e-21\\
154	1.02573360079979e-21\\
155	1.25567376152473e-21\\
156	9.8346490622568e-22\\
157	1.35347177310978e-21\\
158	8.97634667889539e-22\\
159	8.96346106724481e-22\\
160	9.31024946164156e-22\\
161	1.08900733869061e-21\\
162	1.23494348973473e-21\\
163	1.03091033486367e-21\\
164	1.02701150269699e-21\\
165	1.14409709654914e-21\\
166	9.56391298528098e-22\\
167	1.22858668195229e-21\\
168	1.1868990826417e-21\\
169	9.71376542684188e-22\\
170	1.18732699392966e-21\\
171	9.79648541067587e-22\\
172	1.09981101731294e-21\\
173	1.59984422373201e-21\\
174	1.22172993091306e-21\\
175	1.35285765579923e-21\\
176	1.34976330265567e-21\\
177	1.13203295813194e-21\\
178	1.96306046551326e-21\\
179	2.24528108911353e-21\\
180	1.43041655187636e-21\\
181	1.21977766749726e-21\\
182	1.16745497055393e-21\\
183	1.42752075545823e-21\\
184	1.45127119766081e-21\\
185	1.94052893720365e-21\\
186	1.13871096791738e-21\\
187	2.70340252666897e-21\\
188	1.97450812170944e-21\\
189	1.38741720243997e-21\\
190	2.2313072409359e-21\\
191	1.37678640974051e-21\\
192	8.90452811041627e-22\\
193	1.55999977803237e-21\\
194	1.83075075742757e-21\\
195	1.30525387847538e-21\\
196	1.25178316739559e-21\\
197	3.56150898034831e-21\\
198	1.99901808705214e-21\\
199	2.26451283609039e-21\\
200	1.27420564258405e-21\\
201	1.81464666821431e-21\\
202	3.91770613854265e-21\\
203	3.31512394016442e-21\\
204	1.80638998702084e-21\\
205	3.22157632077012e-21\\
206	3.02152657071169e-21\\
207	1.92110342438919e-21\\
208	3.21263295292631e-21\\
209	1.93839572293033e-21\\
210	6.72827923162071e-21\\
211	1.67677962003731e-21\\
212	6.03977820514661e-21\\
213	2.75929745823698e-21\\
214	3.21617488500519e-21\\
215	2.71044114298712e-21\\
216	3.48955308020515e-21\\
217	1.58807422484068e-21\\
218	2.82130149864056e-21\\
219	1.76155934255732e-21\\
220	7.60390271351966e-21\\
221	1.32085326595372e-20\\
222	9.74963279327774e-21\\
223	9.62654951670745e-21\\
224	1.99176580293282e-20\\
225	1.21685200207728e-20\\
226	4.59006730658213e-21\\
227	1.39683525894811e-21\\
228	9.8556333438076e-21\\
229	2.82082252140343e-21\\
230	1.8073834420799e-21\\
231	1.56733472739277e-20\\
232	2.27307834920557e-20\\
233	2.33007567296994e-20\\
234	1.16714715697333e-20\\
235	6.54234428052291e-21\\
236	2.15777233845865e-20\\
237	7.58604187378385e-21\\
238	7.14656093374905e-21\\
239	1.0021082316923e-20\\
240	3.21367318049146e-20\\
241	5.20078897275157e-20\\
242	1.14528533612881e-20\\
243	3.37713839665206e-20\\
244	1.0107686299938e-20\\
245	5.83462852317445e-21\\
246	1.6903186985085e-20\\
247	9.44431973498298e-21\\
248	2.13297615994817e-20\\
249	3.89633144648173e-20\\
250	2.52871032048613e-20\\
251	6.34997206589585e-21\\
252	1.457429957482e-19\\
253	2.55152125609681e-20\\
254	3.03480182597668e-20\\
255	3.24881001082641e-20\\
256	1.15441483763402e-20\\
257	7.97212063907462e-20\\
258	7.98291945639838e-20\\
259	1.96500042060327e-20\\
260	7.37424071894331e-20\\
261	5.69193047536907e-20\\
262	5.34107084810892e-20\\
263	6.73287214572266e-20\\
264	1.52623588949853e-20\\
265	8.77476978475007e-20\\
266	5.87210229236893e-20\\
267	5.06978274278089e-20\\
268	1.76567766681896e-20\\
269	4.73117080884889e-20\\
270	5.42863863953533e-20\\
271	6.27769455271445e-20\\
272	6.53839095409702e-20\\
273	8.2679127140403e-20\\
274	9.25022791362272e-20\\
275	1.10559983973239e-19\\
276	1.84794679423938e-20\\
277	1.06520790890729e-19\\
278	3.21292183926365e-20\\
279	5.30425906004951e-20\\
280	2.81823418877288e-19\\
281	2.3940373066298e-19\\
282	2.34835266223759e-19\\
283	3.3396850041107e-19\\
284	1.37649511747592e-19\\
285	6.53643868905712e-20\\
286	1.40951833011099e-19\\
287	2.9207666686308e-19\\
288	9.47284997616556e-20\\
289	1.31535070522751e-19\\
290	1.85713408092061e-19\\
291	1.15683617432674e-19\\
292	9.60176725500235e-20\\
293	1.12032183183003e-19\\
294	2.53871542123771e-19\\
295	4.17735881960733e-19\\
296	2.64764866611334e-20\\
297	3.32931582710505e-20\\
298	1.00515388459496e-19\\
299	3.15526201872675e-19\\
300	1.19066762615325e-18\\
301	6.58793696504153e-19\\
302	8.04656603548468e-19\\
303	8.24760153594434e-19\\
304	1.16203991020779e-20\\
305	5.7571647677299e-19\\
306	1.1572814466916e-19\\
307	4.29673409222203e-19\\
308	5.47271687968429e-19\\
309	5.74044228035463e-19\\
310	1.85792301948438e-19\\
311	5.12135379932272e-19\\
312	1.08031151761634e-18\\
313	1.43527736410223e-18\\
314	1.01526106031447e-18\\
315	4.57835415125394e-19\\
316	3.59824215974981e-18\\
317	8.554322627438e-19\\
318	1.16139243345817e-18\\
319	3.61399401981852e-19\\
320	2.28468955013154e-18\\
321	2.91539369027591e-19\\
322	1.48088305105803e-18\\
323	2.62130236433968e-18\\
324	1.91333010516743e-19\\
325	4.35475021554675e-18\\
326	4.64802218494918e-19\\
327	8.67656681615923e-19\\
328	1.66608632793299e-18\\
329	4.59750297523726e-18\\
330	1.37468700277132e-18\\
331	1.12473561433369e-19\\
332	6.96372712635378e-19\\
333	3.92771730169631e-18\\
334	3.60640789090649e-18\\
335	9.11654312927391e-19\\
336	8.52811361194146e-19\\
337	1.80162243791969e-18\\
338	5.90950912487235e-18\\
339	2.18308869276674e-19\\
340	6.97551221626523e-19\\
341	1.04820582053031e-17\\
342	1.49543317954709e-18\\
343	5.06445574701788e-18\\
344	1.25526961689187e-17\\
345	3.15557210280868e-18\\
346	7.52328782525952e-18\\
347	1.51337770573565e-18\\
348	5.9884816039709e-18\\
349	4.99390627850674e-19\\
350	7.89420742947731e-18\\
351	2.1003608587066e-18\\
352	8.23494061230087e-18\\
353	3.83111307864242e-18\\
354	1.49022918959808e-18\\
355	8.17307475185861e-18\\
356	2.20356995832694e-18\\
357	1.42628062175469e-17\\
358	1.4240633173697e-17\\
359	1.30140422702377e-17\\
360	1.17646448529029e-17\\
361	2.63212898691055e-18\\
362	5.58057028366679e-18\\
363	9.72779270817087e-18\\
364	2.49495563990338e-17\\
365	1.32662258833144e-17\\
366	5.00221144451425e-17\\
367	2.33084670316349e-18\\
368	1.92203810637383e-17\\
369	3.53859303322884e-17\\
370	9.49484796681477e-18\\
371	6.3746874503379e-19\\
372	2.61045888578806e-17\\
373	3.88029680023692e-19\\
374	2.6515920759292e-17\\
375	4.52746708526721e-17\\
376	2.75915753981886e-17\\
377	6.8484856376293e-17\\
378	8.10682812178957e-17\\
379	3.02751377018789e-17\\
380	2.25290862639849e-17\\
381	7.48765073974649e-17\\
382	7.1345480969707e-17\\
383	3.14604891686487e-17\\
384	5.04916413087708e-17\\
385	1.76734551904884e-17\\
386	3.09331475518559e-17\\
387	2.95978203627233e-17\\
388	6.80045033773734e-17\\
389	2.40536973716883e-17\\
390	7.03545940269834e-17\\
391	7.63300683261711e-17\\
392	9.48130818757791e-17\\
393	5.80098810396367e-17\\
394	1.10810045949228e-16\\
395	6.80054260033837e-17\\
396	4.61836291546301e-17\\
397	1.00912094620786e-16\\
398	1.04186344332984e-16\\
399	1.70678495695608e-16\\
400	8.82968059978557e-17\\
401	1.97334106437351e-16\\
402	1.63550287982696e-16\\
403	4.09167628247833e-17\\
404	1.81851020165954e-16\\
405	1.99395391984685e-16\\
406	1.14406508011898e-16\\
407	2.50507732413725e-16\\
408	5.23426032072092e-16\\
409	4.22813775354337e-16\\
410	5.73171335307054e-16\\
411	2.37783236298821e-16\\
412	3.7511941122055e-16\\
413	1.38416661598835e-16\\
414	7.95541041772158e-17\\
415	1.69156754763869e-16\\
416	5.25113058247936e-16\\
417	1.37200514767566e-16\\
418	3.19703469112122e-16\\
419	7.75576889717786e-17\\
420	3.16422689470839e-16\\
421	2.93850143099527e-16\\
422	7.90493190910932e-16\\
423	4.52409196841304e-16\\
424	1.45816323021016e-16\\
425	6.18029155064165e-16\\
426	5.12127675831641e-16\\
427	5.75577347601422e-16\\
428	1.15825387751038e-15\\
429	1.1446542007651e-15\\
430	1.10349395439468e-15\\
431	2.01387346046335e-16\\
432	5.2463190067692e-17\\
433	5.74493958636962e-16\\
434	1.80668643301516e-15\\
435	1.56057537956032e-15\\
436	1.93474289319223e-15\\
437	9.71450287094653e-16\\
438	4.20275149799944e-16\\
439	2.88423435640958e-15\\
440	1.27608322413742e-15\\
441	1.06206583720424e-15\\
442	1.60994630822641e-15\\
443	2.22018421960081e-16\\
444	6.37998878539194e-16\\
445	3.08597973450703e-15\\
446	1.13301301767156e-15\\
447	1.75269090176524e-15\\
448	2.5490224032526e-15\\
449	3.64232098493702e-16\\
450	2.25563153323267e-15\\
451	1.48003402604527e-15\\
452	7.66746649946528e-18\\
453	8.38125411475093e-16\\
454	1.03926308124266e-15\\
455	1.24124339969494e-15\\
456	2.36996331774917e-15\\
457	2.12504368482885e-16\\
458	2.5827795042717e-15\\
459	4.32653944121128e-15\\
460	7.92744879792282e-15\\
461	1.84171360580166e-16\\
462	1.16543142630816e-15\\
463	3.20593193940374e-15\\
464	1.44233186368814e-15\\
465	1.06564823409347e-14\\
466	7.8942756775377e-15\\
467	1.04869669721033e-15\\
468	4.33354825559286e-16\\
469	5.5529025002239e-15\\
470	8.24371281827681e-15\\
471	7.28749167144441e-15\\
472	8.79340830044837e-15\\
473	8.43140461744553e-15\\
474	4.80992192089574e-15\\
475	9.69405607288444e-15\\
476	2.90451852839674e-14\\
477	1.47201954099536e-15\\
478	1.76648909319907e-14\\
479	1.86782257197783e-14\\
480	4.04767092030808e-14\\
481	1.82128002817906e-14\\
482	1.74922635155381e-14\\
483	1.53930360373296e-14\\
484	3.67347264674362e-14\\
485	4.4193904906824e-15\\
486	2.73298747148981e-15\\
487	1.8181799226411e-14\\
488	1.9419333280323e-15\\
489	6.30183004179998e-16\\
490	3.82612136037958e-14\\
491	2.70692994864624e-14\\
492	7.89319672422599e-15\\
493	1.96623681039391e-14\\
494	7.76141419123672e-15\\
495	1.60108630601148e-14\\
496	5.39493890992775e-14\\
497	1.97273034465642e-14\\
498	4.34847982594742e-14\\
499	3.65235212281644e-17\\
500	1.38166814825224e-14\\
501	5.38795408227558e-15\\
502	1.64578869312495e-14\\
503	2.01798430370559e-14\\
504	1.08708715237454e-13\\
505	9.70964242747402e-15\\
506	6.19425005053625e-14\\
507	5.88887434472157e-14\\
508	1.65812277702519e-14\\
509	4.79719186331162e-14\\
510	1.8742542469249e-13\\
511	3.06080041187089e-13\\
512	8.91845616554372e-14\\
513	1.60362032442558e-13\\
514	9.45182658631303e-14\\
515	1.91217868108812e-13\\
516	1.05228504902352e-13\\
517	3.18234420018261e-14\\
518	1.55317678816886e-13\\
519	3.05976422411083e-13\\
520	3.54209068261351e-14\\
521	3.79855975661461e-13\\
522	4.09576406414548e-14\\
523	1.77717639567545e-13\\
524	1.68591862047139e-13\\
525	2.33095261425354e-13\\
526	2.27038534138917e-13\\
527	7.89399536782563e-14\\
528	1.91937485120852e-13\\
529	5.13986244718572e-13\\
530	3.21237317665585e-13\\
531	1.58622267866278e-13\\
532	3.46877966882723e-13\\
533	1.34278552361362e-13\\
534	2.64076760762629e-13\\
535	1.16958453043731e-12\\
536	2.36491950597975e-13\\
537	1.305354465808e-12\\
538	7.82685490155936e-14\\
539	3.14660002888485e-13\\
540	5.01121010677233e-13\\
541	8.98867383587211e-13\\
542	1.20444805384212e-13\\
543	6.70210587503125e-13\\
544	4.68295519917736e-13\\
545	2.87309071433617e-13\\
546	5.71436924079982e-14\\
547	1.10468553718513e-12\\
548	1.62245210417816e-13\\
549	8.26865837960605e-13\\
550	1.16400549709112e-12\\
551	1.94951998721649e-14\\
552	1.73592726514475e-12\\
553	5.67152495839189e-13\\
554	1.36798300593163e-13\\
555	1.28571419161941e-13\\
556	6.94108990822817e-13\\
557	6.2454302209225e-13\\
558	1.67953184455722e-13\\
559	2.70198666090955e-12\\
560	1.51669996666595e-13\\
561	5.02697419953823e-13\\
562	5.23875722072405e-13\\
563	1.04546261666056e-12\\
564	3.45544976941315e-12\\
565	1.4628781424253e-12\\
566	5.8189154656954e-13\\
567	1.60693915938932e-12\\
568	1.12442319457981e-12\\
569	2.08417093711623e-12\\
570	2.19569581480222e-12\\
571	3.94426161615727e-12\\
572	5.3810043474108e-12\\
573	1.17626730342074e-12\\
574	1.63498981102189e-12\\
575	2.60318383331926e-12\\
576	2.62514698251241e-12\\
577	3.27319104562265e-12\\
578	5.07640945011341e-12\\
579	4.65668487146437e-12\\
580	4.92927092811389e-13\\
581	4.64931316858873e-12\\
582	4.12533501240204e-12\\
583	5.69674955106008e-12\\
584	5.20628739355216e-12\\
585	4.41410232526265e-12\\
586	1.01220787633982e-11\\
587	8.90536200704789e-13\\
588	6.18385028536106e-12\\
589	5.68226080015714e-12\\
590	3.66546372521317e-12\\
591	4.83478229732771e-12\\
592	1.41018982469878e-11\\
593	4.81480568249262e-12\\
594	7.60897465535919e-12\\
595	3.27584157192348e-12\\
596	7.91927930711041e-12\\
597	1.24947674929338e-11\\
598	6.77077937634198e-12\\
599	2.80857446196463e-11\\
600	4.46170414433678e-11\\
601	8.74607305233933e-12\\
602	7.231925410702e-12\\
603	3.83215279038526e-12\\
604	4.24328782583175e-11\\
605	3.83153953246875e-13\\
606	3.67420978075401e-11\\
607	1.7190653002814e-11\\
608	5.86407625071606e-11\\
609	3.55730917534135e-11\\
610	1.0031866880957e-11\\
611	1.04638416525775e-11\\
612	4.50498911147711e-11\\
613	5.63166500489564e-11\\
614	6.2181435632911e-11\\
615	4.00015192655286e-11\\
616	5.73755419300572e-11\\
617	1.0394005202907e-11\\
618	3.75680865671744e-11\\
619	2.93601545033442e-11\\
620	2.09647416498089e-11\\
621	2.09298029389751e-11\\
622	2.66187268345698e-11\\
623	1.58055930527787e-10\\
624	6.09648707623752e-11\\
625	8.68341512959339e-11\\
626	8.48702367941803e-11\\
627	1.6699671637197e-10\\
628	4.09322788928438e-12\\
629	1.03995043758541e-10\\
630	4.3057361119424e-11\\
631	1.36452696315247e-10\\
632	7.77123343658789e-11\\
633	2.44148583823963e-10\\
634	7.74806284795395e-11\\
635	2.69355133889155e-12\\
636	5.61237020370489e-11\\
637	1.30734581513398e-11\\
638	2.3715564517147e-10\\
639	3.07637584907284e-10\\
640	1.47543546471611e-11\\
641	1.46385131850913e-10\\
642	2.8059815018332e-10\\
643	4.31694606114023e-10\\
644	3.99355150533975e-10\\
645	1.86772938008166e-10\\
646	5.06508658323691e-10\\
647	3.99361329189857e-10\\
648	5.59683716696506e-10\\
649	1.30967057507419e-10\\
650	6.01564238522018e-10\\
651	1.38310950537991e-09\\
652	2.50799689955422e-10\\
653	1.00223358254025e-10\\
654	2.25258263801376e-10\\
655	4.88766687662017e-10\\
656	4.66448552342151e-10\\
657	7.15085277561872e-10\\
658	2.82871365549809e-10\\
659	2.00300814978296e-10\\
660	4.61439166064345e-10\\
661	7.778886692933e-10\\
662	5.55431791104001e-10\\
663	2.75384803378526e-10\\
664	1.56968320464369e-09\\
665	5.19746434941684e-10\\
666	2.97678448716574e-10\\
667	3.46556093974236e-10\\
668	1.82440258358168e-09\\
669	6.4348517534704e-10\\
670	1.51460644340803e-10\\
671	8.22571665689423e-10\\
672	7.82434403472426e-10\\
673	2.73795605432486e-09\\
674	2.11874076879673e-09\\
675	2.258849276005e-09\\
676	5.14802595360098e-10\\
677	2.37100106875797e-09\\
678	2.77160986721874e-09\\
679	3.48516000834128e-10\\
680	9.51618700497978e-10\\
681	1.93335962789667e-09\\
682	2.34135931623221e-09\\
683	4.76115847519475e-10\\
684	5.28103147528664e-09\\
685	3.62261891179478e-09\\
686	2.4247503989422e-09\\
687	2.05057239340731e-09\\
688	2.78146709271626e-09\\
689	2.35283368361232e-09\\
690	9.76555698022431e-10\\
691	7.68401915366982e-09\\
692	1.02688655087667e-08\\
693	9.61056428787515e-09\\
694	5.15769819177698e-10\\
695	1.32662946442934e-09\\
696	6.44865300072516e-09\\
697	7.63221819618019e-09\\
698	4.33232102738364e-09\\
699	6.25140142217547e-10\\
700	1.0816956273229e-08\\
};
\addplot [color=red,dashed,line width=3.0pt,forget plot]
  table[row sep=crcr]{%
1	1e-10\\
2	1e-10\\
3	1e-10\\
4	1e-10\\
5	1e-10\\
6	1e-10\\
7	1e-10\\
8	1e-10\\
9	1e-10\\
10	1e-10\\
11	1e-10\\
12	1e-10\\
13	1e-10\\
14	1e-10\\
15	1e-10\\
16	1e-10\\
17	1e-10\\
18	1e-10\\
19	1e-10\\
20	1e-10\\
21	1e-10\\
22	1e-10\\
23	1e-10\\
24	1e-10\\
25	1e-10\\
26	1e-10\\
27	1e-10\\
28	1e-10\\
29	1e-10\\
30	1e-10\\
31	1e-10\\
32	1e-10\\
33	1e-10\\
34	1e-10\\
35	1e-10\\
36	1e-10\\
37	1e-10\\
38	1e-10\\
39	1e-10\\
40	1e-10\\
41	1e-10\\
42	1e-10\\
43	1e-10\\
44	1e-10\\
45	1e-10\\
46	1e-10\\
47	1e-10\\
48	1e-10\\
49	1e-10\\
50	1e-10\\
51	1e-10\\
52	1e-10\\
53	1e-10\\
54	1e-10\\
55	1e-10\\
56	1e-10\\
57	1e-10\\
58	1e-10\\
59	1e-10\\
60	1e-10\\
61	1e-10\\
62	1e-10\\
63	1e-10\\
64	1e-10\\
65	1e-10\\
66	1e-10\\
67	1e-10\\
68	1e-10\\
69	1e-10\\
70	1e-10\\
71	1e-10\\
72	1e-10\\
73	1e-10\\
74	1e-10\\
75	1e-10\\
76	1e-10\\
77	1e-10\\
78	1e-10\\
79	1e-10\\
80	1e-10\\
81	1e-10\\
82	1e-10\\
83	1e-10\\
84	1e-10\\
85	1e-10\\
86	1e-10\\
87	1e-10\\
88	1e-10\\
89	1e-10\\
90	1e-10\\
91	1e-10\\
92	1e-10\\
93	1e-10\\
94	1e-10\\
95	1e-10\\
96	1e-10\\
97	1e-10\\
98	1e-10\\
99	1e-10\\
100	1e-10\\
101	1e-10\\
102	1e-10\\
103	1e-10\\
104	1e-10\\
105	1e-10\\
106	1e-10\\
107	1e-10\\
108	1e-10\\
109	1e-10\\
110	1e-10\\
111	1e-10\\
112	1e-10\\
113	1e-10\\
114	1e-10\\
115	1e-10\\
116	1e-10\\
117	1e-10\\
118	1e-10\\
119	1e-10\\
120	1e-10\\
121	1e-10\\
122	1e-10\\
123	1e-10\\
124	1e-10\\
125	1e-10\\
126	1e-10\\
127	1e-10\\
128	1e-10\\
129	1e-10\\
130	1e-10\\
131	1e-10\\
132	1e-10\\
133	1e-10\\
134	1e-10\\
135	1e-10\\
136	1e-10\\
137	1e-10\\
138	1e-10\\
139	1e-10\\
140	1e-10\\
141	1e-10\\
142	1e-10\\
143	1e-10\\
144	1e-10\\
145	1e-10\\
146	1e-10\\
147	1e-10\\
148	1e-10\\
149	1e-10\\
150	1e-10\\
151	1e-10\\
152	1e-10\\
153	1e-10\\
154	1e-10\\
155	1e-10\\
156	1e-10\\
157	1e-10\\
158	1e-10\\
159	1e-10\\
160	1e-10\\
161	1e-10\\
162	1e-10\\
163	1e-10\\
164	1e-10\\
165	1e-10\\
166	1e-10\\
167	1e-10\\
168	1e-10\\
169	1e-10\\
170	1e-10\\
171	1e-10\\
172	1e-10\\
173	1e-10\\
174	1e-10\\
175	1e-10\\
176	1e-10\\
177	1e-10\\
178	1e-10\\
179	1e-10\\
180	1e-10\\
181	1e-10\\
182	1e-10\\
183	1e-10\\
184	1e-10\\
185	1e-10\\
186	1e-10\\
187	1e-10\\
188	1e-10\\
189	1e-10\\
190	1e-10\\
191	1e-10\\
192	1e-10\\
193	1e-10\\
194	1e-10\\
195	1e-10\\
196	1e-10\\
197	1e-10\\
198	1e-10\\
199	1e-10\\
200	1e-10\\
201	1e-10\\
202	1e-10\\
203	1e-10\\
204	1e-10\\
205	1e-10\\
206	1e-10\\
207	1e-10\\
208	1e-10\\
209	1e-10\\
210	1e-10\\
211	1e-10\\
212	1e-10\\
213	1e-10\\
214	1e-10\\
215	1e-10\\
216	1e-10\\
217	1e-10\\
218	1e-10\\
219	1e-10\\
220	1e-10\\
221	1e-10\\
222	1e-10\\
223	1e-10\\
224	1e-10\\
225	1e-10\\
226	1e-10\\
227	1e-10\\
228	1e-10\\
229	1e-10\\
230	1e-10\\
231	1e-10\\
232	1e-10\\
233	1e-10\\
234	1e-10\\
235	1e-10\\
236	1e-10\\
237	1e-10\\
238	1e-10\\
239	1e-10\\
240	1e-10\\
241	1e-10\\
242	1e-10\\
243	1e-10\\
244	1e-10\\
245	1e-10\\
246	1e-10\\
247	1e-10\\
248	1e-10\\
249	1e-10\\
250	1e-10\\
251	1e-10\\
252	1e-10\\
253	1e-10\\
254	1e-10\\
255	1e-10\\
256	1e-10\\
257	1e-10\\
258	1e-10\\
259	1e-10\\
260	1e-10\\
261	1e-10\\
262	1e-10\\
263	1e-10\\
264	1e-10\\
265	1e-10\\
266	1e-10\\
267	1e-10\\
268	1e-10\\
269	1e-10\\
270	1e-10\\
271	1e-10\\
272	1e-10\\
273	1e-10\\
274	1e-10\\
275	1e-10\\
276	1e-10\\
277	1e-10\\
278	1e-10\\
279	1e-10\\
280	1e-10\\
281	1e-10\\
282	1e-10\\
283	1e-10\\
284	1e-10\\
285	1e-10\\
286	1e-10\\
287	1e-10\\
288	1e-10\\
289	1e-10\\
290	1e-10\\
291	1e-10\\
292	1e-10\\
293	1e-10\\
294	1e-10\\
295	1e-10\\
296	1e-10\\
297	1e-10\\
298	1e-10\\
299	1e-10\\
300	1e-10\\
301	1e-10\\
302	1e-10\\
303	1e-10\\
304	1e-10\\
305	1e-10\\
306	1e-10\\
307	1e-10\\
308	1e-10\\
309	1e-10\\
310	1e-10\\
311	1e-10\\
312	1e-10\\
313	1e-10\\
314	1e-10\\
315	1e-10\\
316	1e-10\\
317	1e-10\\
318	1e-10\\
319	1e-10\\
320	1e-10\\
321	1e-10\\
322	1e-10\\
323	1e-10\\
324	1e-10\\
325	1e-10\\
326	1e-10\\
327	1e-10\\
328	1e-10\\
329	1e-10\\
330	1e-10\\
331	1e-10\\
332	1e-10\\
333	1e-10\\
334	1e-10\\
335	1e-10\\
336	1e-10\\
337	1e-10\\
338	1e-10\\
339	1e-10\\
340	1e-10\\
341	1e-10\\
342	1e-10\\
343	1e-10\\
344	1e-10\\
345	1e-10\\
346	1e-10\\
347	1e-10\\
348	1e-10\\
349	1e-10\\
350	1e-10\\
351	1e-10\\
352	1e-10\\
353	1e-10\\
354	1e-10\\
355	1e-10\\
356	1e-10\\
357	1e-10\\
358	1e-10\\
359	1e-10\\
360	1e-10\\
361	1e-10\\
362	1e-10\\
363	1e-10\\
364	1e-10\\
365	1e-10\\
366	1e-10\\
367	1e-10\\
368	1e-10\\
369	1e-10\\
370	1e-10\\
371	1e-10\\
372	1e-10\\
373	1e-10\\
374	1e-10\\
375	1e-10\\
376	1e-10\\
377	1e-10\\
378	1e-10\\
379	1e-10\\
380	1e-10\\
381	1e-10\\
382	1e-10\\
383	1e-10\\
384	1e-10\\
385	1e-10\\
386	1e-10\\
387	1e-10\\
388	1e-10\\
389	1e-10\\
390	1e-10\\
391	1e-10\\
392	1e-10\\
393	1e-10\\
394	1e-10\\
395	1e-10\\
396	1e-10\\
397	1e-10\\
398	1e-10\\
399	1e-10\\
400	1e-10\\
401	1e-10\\
402	1e-10\\
403	1e-10\\
404	1e-10\\
405	1e-10\\
406	1e-10\\
407	1e-10\\
408	1e-10\\
409	1e-10\\
410	1e-10\\
411	1e-10\\
412	1e-10\\
413	1e-10\\
414	1e-10\\
415	1e-10\\
416	1e-10\\
417	1e-10\\
418	1e-10\\
419	1e-10\\
420	1e-10\\
421	1e-10\\
422	1e-10\\
423	1e-10\\
424	1e-10\\
425	1e-10\\
426	1e-10\\
427	1e-10\\
428	1e-10\\
429	1e-10\\
430	1e-10\\
431	1e-10\\
432	1e-10\\
433	1e-10\\
434	1e-10\\
435	1e-10\\
436	1e-10\\
437	1e-10\\
438	1e-10\\
439	1e-10\\
440	1e-10\\
441	1e-10\\
442	1e-10\\
443	1e-10\\
444	1e-10\\
445	1e-10\\
446	1e-10\\
447	1e-10\\
448	1e-10\\
449	1e-10\\
450	1e-10\\
451	1e-10\\
452	1e-10\\
453	1e-10\\
454	1e-10\\
455	1e-10\\
456	1e-10\\
457	1e-10\\
458	1e-10\\
459	1e-10\\
460	1e-10\\
461	1e-10\\
462	1e-10\\
463	1e-10\\
464	1e-10\\
465	1e-10\\
466	1e-10\\
467	1e-10\\
468	1e-10\\
469	1e-10\\
470	1e-10\\
471	1e-10\\
472	1e-10\\
473	1e-10\\
474	1e-10\\
475	1e-10\\
476	1e-10\\
477	1e-10\\
478	1e-10\\
479	1e-10\\
480	1e-10\\
481	1e-10\\
482	1e-10\\
483	1e-10\\
484	1e-10\\
485	1e-10\\
486	1e-10\\
487	1e-10\\
488	1e-10\\
489	1e-10\\
490	1e-10\\
491	1e-10\\
492	1e-10\\
493	1e-10\\
494	1e-10\\
495	1e-10\\
496	1e-10\\
497	1e-10\\
498	1e-10\\
499	1e-10\\
500	1e-10\\
501	1e-10\\
502	1e-10\\
503	1e-10\\
504	1e-10\\
505	1e-10\\
506	1e-10\\
507	1e-10\\
508	1e-10\\
509	1e-10\\
510	1e-10\\
511	1e-10\\
512	1e-10\\
513	1e-10\\
514	1e-10\\
515	1e-10\\
516	1e-10\\
517	1e-10\\
518	1e-10\\
519	1e-10\\
520	1e-10\\
521	1e-10\\
522	1e-10\\
523	1e-10\\
524	1e-10\\
525	1e-10\\
526	1e-10\\
527	1e-10\\
528	1e-10\\
529	1e-10\\
530	1e-10\\
531	1e-10\\
532	1e-10\\
533	1e-10\\
534	1e-10\\
535	1e-10\\
536	1e-10\\
537	1e-10\\
538	1e-10\\
539	1e-10\\
540	1e-10\\
541	1e-10\\
542	1e-10\\
543	1e-10\\
544	1e-10\\
545	1e-10\\
546	1e-10\\
547	1e-10\\
548	1e-10\\
549	1e-10\\
550	1e-10\\
551	1e-10\\
552	1e-10\\
553	1e-10\\
554	1e-10\\
555	1e-10\\
556	1e-10\\
557	1e-10\\
558	1e-10\\
559	1e-10\\
560	1e-10\\
561	1e-10\\
562	1e-10\\
563	1e-10\\
564	1e-10\\
565	1e-10\\
566	1e-10\\
567	1e-10\\
568	1e-10\\
569	1e-10\\
570	1e-10\\
571	1e-10\\
572	1e-10\\
573	1e-10\\
574	1e-10\\
575	1e-10\\
576	1e-10\\
577	1e-10\\
578	1e-10\\
579	1e-10\\
580	1e-10\\
581	1e-10\\
582	1e-10\\
583	1e-10\\
584	1e-10\\
585	1e-10\\
586	1e-10\\
587	1e-10\\
588	1e-10\\
589	1e-10\\
590	1e-10\\
591	1e-10\\
592	1e-10\\
593	1e-10\\
594	1e-10\\
595	1e-10\\
596	1e-10\\
597	1e-10\\
598	1e-10\\
599	1e-10\\
600	1e-10\\
601	1e-10\\
602	1e-10\\
603	1e-10\\
604	1e-10\\
605	1e-10\\
606	1e-10\\
607	1e-10\\
608	1e-10\\
609	1e-10\\
610	1e-10\\
611	1e-10\\
612	1e-10\\
613	1e-10\\
614	1e-10\\
615	1e-10\\
616	1e-10\\
617	1e-10\\
618	1e-10\\
619	1e-10\\
620	1e-10\\
621	1e-10\\
622	1e-10\\
623	1e-10\\
624	1e-10\\
625	1e-10\\
626	1e-10\\
627	1e-10\\
628	1e-10\\
629	1e-10\\
630	1e-10\\
631	1e-10\\
632	1e-10\\
633	1e-10\\
634	1e-10\\
635	1e-10\\
636	1e-10\\
637	1e-10\\
638	1e-10\\
639	1e-10\\
640	1e-10\\
641	1e-10\\
642	1e-10\\
643	1e-10\\
644	1e-10\\
645	1e-10\\
646	1e-10\\
647	1e-10\\
648	1e-10\\
649	1e-10\\
650	1e-10\\
651	1e-10\\
652	1e-10\\
653	1e-10\\
654	1e-10\\
655	1e-10\\
656	1e-10\\
657	1e-10\\
658	1e-10\\
659	1e-10\\
660	1e-10\\
661	1e-10\\
662	1e-10\\
663	1e-10\\
664	1e-10\\
665	1e-10\\
666	1e-10\\
667	1e-10\\
668	1e-10\\
669	1e-10\\
670	1e-10\\
671	1e-10\\
672	1e-10\\
673	1e-10\\
674	1e-10\\
675	1e-10\\
676	1e-10\\
677	1e-10\\
678	1e-10\\
679	1e-10\\
680	1e-10\\
681	1e-10\\
682	1e-10\\
683	1e-10\\
684	1e-10\\
685	1e-10\\
686	1e-10\\
687	1e-10\\
688	1e-10\\
689	1e-10\\
690	1e-10\\
691	1e-10\\
692	1e-10\\
693	1e-10\\
694	1e-10\\
695	1e-10\\
696	1e-10\\
697	1e-10\\
698	1e-10\\
699	1e-10\\
700	1e-10\\
701	1e-10\\
702	1e-10\\
703	1e-10\\
704	1e-10\\
705	1e-10\\
706	1e-10\\
707	1e-10\\
708	1e-10\\
709	1e-10\\
710	1e-10\\
711	1e-10\\
712	1e-10\\
713	1e-10\\
714	1e-10\\
715	1e-10\\
716	1e-10\\
717	1e-10\\
718	1e-10\\
719	1e-10\\
720	1e-10\\
721	1e-10\\
722	1e-10\\
723	1e-10\\
724	1e-10\\
725	1e-10\\
726	1e-10\\
727	1e-10\\
728	1e-10\\
729	1e-10\\
730	1e-10\\
731	1e-10\\
732	1e-10\\
733	1e-10\\
734	1e-10\\
735	1e-10\\
736	1e-10\\
737	1e-10\\
738	1e-10\\
739	1e-10\\
740	1e-10\\
741	1e-10\\
742	1e-10\\
743	1e-10\\
744	1e-10\\
745	1e-10\\
746	1e-10\\
747	1e-10\\
748	1e-10\\
749	1e-10\\
750	1e-10\\
751	1e-10\\
752	1e-10\\
753	1e-10\\
754	1e-10\\
755	1e-10\\
756	1e-10\\
757	1e-10\\
758	1e-10\\
759	1e-10\\
760	1e-10\\
761	1e-10\\
762	1e-10\\
763	1e-10\\
764	1e-10\\
765	1e-10\\
766	1e-10\\
767	1e-10\\
768	1e-10\\
769	1e-10\\
770	1e-10\\
771	1e-10\\
772	1e-10\\
773	1e-10\\
774	1e-10\\
775	1e-10\\
776	1e-10\\
777	1e-10\\
778	1e-10\\
779	1e-10\\
780	1e-10\\
781	1e-10\\
782	1e-10\\
783	1e-10\\
784	1e-10\\
785	1e-10\\
786	1e-10\\
787	1e-10\\
788	1e-10\\
789	1e-10\\
790	1e-10\\
791	1e-10\\
792	1e-10\\
793	1e-10\\
794	1e-10\\
795	1e-10\\
796	1e-10\\
797	1e-10\\
798	1e-10\\
799	1e-10\\
800	1e-10\\
801	1e-10\\
802	1e-10\\
803	1e-10\\
804	1e-10\\
805	1e-10\\
806	1e-10\\
807	1e-10\\
808	1e-10\\
809	1e-10\\
810	1e-10\\
811	1e-10\\
812	1e-10\\
813	1e-10\\
814	1e-10\\
815	1e-10\\
816	1e-10\\
817	1e-10\\
818	1e-10\\
819	1e-10\\
820	1e-10\\
821	1e-10\\
822	1e-10\\
823	1e-10\\
824	1e-10\\
825	1e-10\\
826	1e-10\\
827	1e-10\\
828	1e-10\\
829	1e-10\\
830	1e-10\\
831	1e-10\\
832	1e-10\\
833	1e-10\\
834	1e-10\\
835	1e-10\\
836	1e-10\\
837	1e-10\\
838	1e-10\\
839	1e-10\\
840	1e-10\\
841	1e-10\\
842	1e-10\\
843	1e-10\\
844	1e-10\\
845	1e-10\\
846	1e-10\\
847	1e-10\\
848	1e-10\\
849	1e-10\\
850	1e-10\\
851	1e-10\\
852	1e-10\\
853	1e-10\\
854	1e-10\\
855	1e-10\\
856	1e-10\\
857	1e-10\\
858	1e-10\\
859	1e-10\\
860	1e-10\\
861	1e-10\\
862	1e-10\\
863	1e-10\\
864	1e-10\\
865	1e-10\\
866	1e-10\\
867	1e-10\\
868	1e-10\\
869	1e-10\\
870	1e-10\\
871	1e-10\\
872	1e-10\\
873	1e-10\\
874	1e-10\\
875	1e-10\\
876	1e-10\\
877	1e-10\\
878	1e-10\\
879	1e-10\\
880	1e-10\\
881	1e-10\\
882	1e-10\\
883	1e-10\\
884	1e-10\\
885	1e-10\\
886	1e-10\\
887	1e-10\\
888	1e-10\\
889	1e-10\\
890	1e-10\\
891	1e-10\\
892	1e-10\\
893	1e-10\\
894	1e-10\\
895	1e-10\\
896	1e-10\\
897	1e-10\\
898	1e-10\\
899	1e-10\\
900	1e-10\\
901	1e-10\\
902	1e-10\\
903	1e-10\\
904	1e-10\\
905	1e-10\\
906	1e-10\\
907	1e-10\\
908	1e-10\\
909	1e-10\\
910	1e-10\\
911	1e-10\\
912	1e-10\\
913	1e-10\\
914	1e-10\\
915	1e-10\\
916	1e-10\\
917	1e-10\\
918	1e-10\\
919	1e-10\\
920	1e-10\\
921	1e-10\\
922	1e-10\\
923	1e-10\\
924	1e-10\\
925	1e-10\\
926	1e-10\\
927	1e-10\\
928	1e-10\\
929	1e-10\\
930	1e-10\\
931	1e-10\\
932	1e-10\\
933	1e-10\\
934	1e-10\\
935	1e-10\\
936	1e-10\\
937	1e-10\\
938	1e-10\\
939	1e-10\\
940	1e-10\\
941	1e-10\\
942	1e-10\\
943	1e-10\\
944	1e-10\\
945	1e-10\\
946	1e-10\\
947	1e-10\\
948	1e-10\\
949	1e-10\\
950	1e-10\\
951	1e-10\\
952	1e-10\\
953	1e-10\\
954	1e-10\\
955	1e-10\\
956	1e-10\\
957	1e-10\\
958	1e-10\\
959	1e-10\\
960	1e-10\\
961	1e-10\\
962	1e-10\\
963	1e-10\\
964	1e-10\\
965	1e-10\\
966	1e-10\\
967	1e-10\\
968	1e-10\\
969	1e-10\\
970	1e-10\\
971	1e-10\\
972	1e-10\\
973	1e-10\\
974	1e-10\\
975	1e-10\\
976	1e-10\\
977	1e-10\\
978	1e-10\\
979	1e-10\\
980	1e-10\\
981	1e-10\\
982	1e-10\\
983	1e-10\\
984	1e-10\\
985	1e-10\\
986	1e-10\\
987	1e-10\\
988	1e-10\\
989	1e-10\\
990	1e-10\\
991	1e-10\\
992	1e-10\\
993	1e-10\\
994	1e-10\\
995	1e-10\\
996	1e-10\\
997	1e-10\\
998	1e-10\\
999	1e-10\\
1000	1e-10\\
1001	1e-10\\
1002	1e-10\\
1003	1e-10\\
1004	1e-10\\
1005	1e-10\\
1006	1e-10\\
1007	1e-10\\
1008	1e-10\\
1009	1e-10\\
1010	1e-10\\
1011	1e-10\\
1012	1e-10\\
1013	1e-10\\
1014	1e-10\\
1015	1e-10\\
1016	1e-10\\
1017	1e-10\\
1018	1e-10\\
1019	1e-10\\
1020	1e-10\\
1021	1e-10\\
1022	1e-10\\
1023	1e-10\\
1024	1e-10\\
1025	1e-10\\
1026	1e-10\\
1027	1e-10\\
1028	1e-10\\
1029	1e-10\\
1030	1e-10\\
1031	1e-10\\
1032	1e-10\\
1033	1e-10\\
1034	1e-10\\
1035	1e-10\\
1036	1e-10\\
1037	1e-10\\
1038	1e-10\\
1039	1e-10\\
1040	1e-10\\
1041	1e-10\\
1042	1e-10\\
1043	1e-10\\
1044	1e-10\\
1045	1e-10\\
1046	1e-10\\
1047	1e-10\\
1048	1e-10\\
1049	1e-10\\
1050	1e-10\\
1051	1e-10\\
1052	1e-10\\
1053	1e-10\\
1054	1e-10\\
1055	1e-10\\
1056	1e-10\\
1057	1e-10\\
1058	1e-10\\
1059	1e-10\\
1060	1e-10\\
1061	1e-10\\
1062	1e-10\\
1063	1e-10\\
1064	1e-10\\
1065	1e-10\\
1066	1e-10\\
1067	1e-10\\
1068	1e-10\\
1069	1e-10\\
1070	1e-10\\
1071	1e-10\\
1072	1e-10\\
1073	1e-10\\
1074	1e-10\\
1075	1e-10\\
1076	1e-10\\
1077	1e-10\\
1078	1e-10\\
1079	1e-10\\
1080	1e-10\\
1081	1e-10\\
1082	1e-10\\
1083	1e-10\\
1084	1e-10\\
1085	1e-10\\
1086	1e-10\\
1087	1e-10\\
1088	1e-10\\
1089	1e-10\\
1090	1e-10\\
1091	1e-10\\
1092	1e-10\\
1093	1e-10\\
1094	1e-10\\
1095	1e-10\\
1096	1e-10\\
1097	1e-10\\
1098	1e-10\\
1099	1e-10\\
1100	1e-10\\
1101	1e-10\\
1102	1e-10\\
1103	1e-10\\
1104	1e-10\\
1105	1e-10\\
1106	1e-10\\
1107	1e-10\\
1108	1e-10\\
1109	1e-10\\
1110	1e-10\\
1111	1e-10\\
1112	1e-10\\
1113	1e-10\\
1114	1e-10\\
1115	1e-10\\
1116	1e-10\\
1117	1e-10\\
1118	1e-10\\
1119	1e-10\\
1120	1e-10\\
1121	1e-10\\
1122	1e-10\\
1123	1e-10\\
1124	1e-10\\
1125	1e-10\\
1126	1e-10\\
1127	1e-10\\
1128	1e-10\\
1129	1e-10\\
1130	1e-10\\
1131	1e-10\\
1132	1e-10\\
1133	1e-10\\
1134	1e-10\\
1135	1e-10\\
1136	1e-10\\
1137	1e-10\\
1138	1e-10\\
1139	1e-10\\
1140	1e-10\\
1141	1e-10\\
1142	1e-10\\
1143	1e-10\\
1144	1e-10\\
1145	1e-10\\
1146	1e-10\\
1147	1e-10\\
1148	1e-10\\
1149	1e-10\\
1150	1e-10\\
1151	1e-10\\
1152	1e-10\\
1153	1e-10\\
1154	1e-10\\
1155	1e-10\\
1156	1e-10\\
1157	1e-10\\
1158	1e-10\\
1159	1e-10\\
1160	1e-10\\
1161	1e-10\\
1162	1e-10\\
1163	1e-10\\
1164	1e-10\\
1165	1e-10\\
1166	1e-10\\
1167	1e-10\\
1168	1e-10\\
1169	1e-10\\
1170	1e-10\\
1171	1e-10\\
1172	1e-10\\
1173	1e-10\\
1174	1e-10\\
1175	1e-10\\
1176	1e-10\\
1177	1e-10\\
1178	1e-10\\
1179	1e-10\\
1180	1e-10\\
1181	1e-10\\
1182	1e-10\\
1183	1e-10\\
1184	1e-10\\
1185	1e-10\\
1186	1e-10\\
1187	1e-10\\
1188	1e-10\\
1189	1e-10\\
1190	1e-10\\
1191	1e-10\\
1192	1e-10\\
1193	1e-10\\
1194	1e-10\\
1195	1e-10\\
1196	1e-10\\
1197	1e-10\\
1198	1e-10\\
1199	1e-10\\
1200	1e-10\\
1201	1e-10\\
1202	1e-10\\
1203	1e-10\\
1204	1e-10\\
1205	1e-10\\
1206	1e-10\\
1207	1e-10\\
1208	1e-10\\
1209	1e-10\\
1210	1e-10\\
1211	1e-10\\
1212	1e-10\\
1213	1e-10\\
1214	1e-10\\
1215	1e-10\\
1216	1e-10\\
1217	1e-10\\
1218	1e-10\\
1219	1e-10\\
1220	1e-10\\
1221	1e-10\\
1222	1e-10\\
1223	1e-10\\
1224	1e-10\\
1225	1e-10\\
1226	1e-10\\
1227	1e-10\\
1228	1e-10\\
1229	1e-10\\
1230	1e-10\\
1231	1e-10\\
1232	1e-10\\
1233	1e-10\\
1234	1e-10\\
1235	1e-10\\
1236	1e-10\\
1237	1e-10\\
1238	1e-10\\
1239	1e-10\\
1240	1e-10\\
1241	1e-10\\
1242	1e-10\\
1243	1e-10\\
1244	1e-10\\
1245	1e-10\\
1246	1e-10\\
1247	1e-10\\
1248	1e-10\\
1249	1e-10\\
1250	1e-10\\
1251	1e-10\\
1252	1e-10\\
1253	1e-10\\
1254	1e-10\\
1255	1e-10\\
1256	1e-10\\
1257	1e-10\\
1258	1e-10\\
1259	1e-10\\
1260	1e-10\\
1261	1e-10\\
1262	1e-10\\
1263	1e-10\\
1264	1e-10\\
1265	1e-10\\
1266	1e-10\\
1267	1e-10\\
1268	1e-10\\
1269	1e-10\\
1270	1e-10\\
1271	1e-10\\
1272	1e-10\\
1273	1e-10\\
1274	1e-10\\
1275	1e-10\\
1276	1e-10\\
1277	1e-10\\
1278	1e-10\\
1279	1e-10\\
1280	1e-10\\
1281	1e-10\\
1282	1e-10\\
1283	1e-10\\
1284	1e-10\\
1285	1e-10\\
1286	1e-10\\
1287	1e-10\\
1288	1e-10\\
1289	1e-10\\
1290	1e-10\\
1291	1e-10\\
1292	1e-10\\
1293	1e-10\\
1294	1e-10\\
1295	1e-10\\
1296	1e-10\\
1297	1e-10\\
1298	1e-10\\
1299	1e-10\\
1300	1e-10\\
1301	1e-10\\
1302	1e-10\\
1303	1e-10\\
1304	1e-10\\
1305	1e-10\\
1306	1e-10\\
1307	1e-10\\
1308	1e-10\\
1309	1e-10\\
1310	1e-10\\
1311	1e-10\\
1312	1e-10\\
1313	1e-10\\
1314	1e-10\\
1315	1e-10\\
1316	1e-10\\
1317	1e-10\\
1318	1e-10\\
1319	1e-10\\
1320	1e-10\\
1321	1e-10\\
1322	1e-10\\
1323	1e-10\\
1324	1e-10\\
1325	1e-10\\
1326	1e-10\\
1327	1e-10\\
1328	1e-10\\
1329	1e-10\\
1330	1e-10\\
1331	1e-10\\
1332	1e-10\\
1333	1e-10\\
1334	1e-10\\
1335	1e-10\\
1336	1e-10\\
1337	1e-10\\
1338	1e-10\\
1339	1e-10\\
1340	1e-10\\
1341	1e-10\\
1342	1e-10\\
1343	1e-10\\
1344	1e-10\\
1345	1e-10\\
1346	1e-10\\
1347	1e-10\\
1348	1e-10\\
1349	1e-10\\
1350	1e-10\\
1351	1e-10\\
1352	1e-10\\
1353	1e-10\\
1354	1e-10\\
1355	1e-10\\
1356	1e-10\\
1357	1e-10\\
1358	1e-10\\
1359	1e-10\\
1360	1e-10\\
1361	1e-10\\
1362	1e-10\\
1363	1e-10\\
1364	1e-10\\
1365	1e-10\\
1366	1e-10\\
1367	1e-10\\
1368	1e-10\\
1369	1e-10\\
1370	1e-10\\
1371	1e-10\\
1372	1e-10\\
1373	1e-10\\
1374	1e-10\\
1375	1e-10\\
1376	1e-10\\
1377	1e-10\\
1378	1e-10\\
1379	1e-10\\
1380	1e-10\\
1381	1e-10\\
1382	1e-10\\
1383	1e-10\\
1384	1e-10\\
1385	1e-10\\
1386	1e-10\\
1387	1e-10\\
1388	1e-10\\
1389	1e-10\\
1390	1e-10\\
1391	1e-10\\
1392	1e-10\\
1393	1e-10\\
1394	1e-10\\
1395	1e-10\\
1396	1e-10\\
1397	1e-10\\
1398	1e-10\\
1399	1e-10\\
1400	1e-10\\
1401	1e-10\\
1402	1e-10\\
1403	1e-10\\
1404	1e-10\\
1405	1e-10\\
1406	1e-10\\
1407	1e-10\\
1408	1e-10\\
1409	1e-10\\
1410	1e-10\\
1411	1e-10\\
1412	1e-10\\
1413	1e-10\\
1414	1e-10\\
1415	1e-10\\
1416	1e-10\\
1417	1e-10\\
1418	1e-10\\
1419	1e-10\\
1420	1e-10\\
1421	1e-10\\
1422	1e-10\\
1423	1e-10\\
1424	1e-10\\
1425	1e-10\\
1426	1e-10\\
1427	1e-10\\
1428	1e-10\\
1429	1e-10\\
1430	1e-10\\
1431	1e-10\\
1432	1e-10\\
1433	1e-10\\
1434	1e-10\\
1435	1e-10\\
1436	1e-10\\
1437	1e-10\\
1438	1e-10\\
1439	1e-10\\
1440	1e-10\\
1441	1e-10\\
1442	1e-10\\
1443	1e-10\\
1444	1e-10\\
1445	1e-10\\
1446	1e-10\\
1447	1e-10\\
1448	1e-10\\
1449	1e-10\\
1450	1e-10\\
1451	1e-10\\
1452	1e-10\\
1453	1e-10\\
1454	1e-10\\
1455	1e-10\\
1456	1e-10\\
1457	1e-10\\
1458	1e-10\\
1459	1e-10\\
1460	1e-10\\
1461	1e-10\\
1462	1e-10\\
1463	1e-10\\
1464	1e-10\\
1465	1e-10\\
1466	1e-10\\
1467	1e-10\\
1468	1e-10\\
1469	1e-10\\
1470	1e-10\\
1471	1e-10\\
1472	1e-10\\
1473	1e-10\\
1474	1e-10\\
1475	1e-10\\
1476	1e-10\\
1477	1e-10\\
1478	1e-10\\
1479	1e-10\\
1480	1e-10\\
1481	1e-10\\
1482	1e-10\\
1483	1e-10\\
1484	1e-10\\
1485	1e-10\\
1486	1e-10\\
1487	1e-10\\
1488	1e-10\\
1489	1e-10\\
1490	1e-10\\
1491	1e-10\\
1492	1e-10\\
1493	1e-10\\
1494	1e-10\\
1495	1e-10\\
1496	1e-10\\
1497	1e-10\\
1498	1e-10\\
1499	1e-10\\
1500	1e-10\\
1501	1e-10\\
1502	1e-10\\
1503	1e-10\\
1504	1e-10\\
1505	1e-10\\
1506	1e-10\\
1507	1e-10\\
1508	1e-10\\
1509	1e-10\\
1510	1e-10\\
1511	1e-10\\
1512	1e-10\\
1513	1e-10\\
1514	1e-10\\
1515	1e-10\\
1516	1e-10\\
1517	1e-10\\
1518	1e-10\\
1519	1e-10\\
1520	1e-10\\
1521	1e-10\\
1522	1e-10\\
1523	1e-10\\
1524	1e-10\\
1525	1e-10\\
1526	1e-10\\
1527	1e-10\\
1528	1e-10\\
1529	1e-10\\
1530	1e-10\\
1531	1e-10\\
1532	1e-10\\
1533	1e-10\\
1534	1e-10\\
1535	1e-10\\
1536	1e-10\\
1537	1e-10\\
1538	1e-10\\
1539	1e-10\\
1540	1e-10\\
1541	1e-10\\
1542	1e-10\\
1543	1e-10\\
1544	1e-10\\
1545	1e-10\\
1546	1e-10\\
1547	1e-10\\
1548	1e-10\\
1549	1e-10\\
1550	1e-10\\
1551	1e-10\\
1552	1e-10\\
1553	1e-10\\
1554	1e-10\\
1555	1e-10\\
1556	1e-10\\
1557	1e-10\\
1558	1e-10\\
1559	1e-10\\
1560	1e-10\\
1561	1e-10\\
1562	1e-10\\
1563	1e-10\\
1564	1e-10\\
1565	1e-10\\
1566	1e-10\\
1567	1e-10\\
1568	1e-10\\
1569	1e-10\\
1570	1e-10\\
1571	1e-10\\
1572	1e-10\\
1573	1e-10\\
1574	1e-10\\
1575	1e-10\\
1576	1e-10\\
1577	1e-10\\
1578	1e-10\\
1579	1e-10\\
1580	1e-10\\
1581	1e-10\\
1582	1e-10\\
1583	1e-10\\
1584	1e-10\\
1585	1e-10\\
1586	1e-10\\
1587	1e-10\\
1588	1e-10\\
1589	1e-10\\
1590	1e-10\\
1591	1e-10\\
1592	1e-10\\
1593	1e-10\\
1594	1e-10\\
1595	1e-10\\
1596	1e-10\\
1597	1e-10\\
1598	1e-10\\
1599	1e-10\\
1600	1e-10\\
1601	1e-10\\
1602	1e-10\\
1603	1e-10\\
1604	1e-10\\
1605	1e-10\\
1606	1e-10\\
1607	1e-10\\
1608	1e-10\\
1609	1e-10\\
1610	1e-10\\
1611	1e-10\\
1612	1e-10\\
1613	1e-10\\
1614	1e-10\\
1615	1e-10\\
1616	1e-10\\
1617	1e-10\\
1618	1e-10\\
1619	1e-10\\
1620	1e-10\\
1621	1e-10\\
1622	1e-10\\
1623	1e-10\\
1624	1e-10\\
1625	1e-10\\
1626	1e-10\\
1627	1e-10\\
1628	1e-10\\
1629	1e-10\\
1630	1e-10\\
1631	1e-10\\
1632	1e-10\\
1633	1e-10\\
1634	1e-10\\
1635	1e-10\\
1636	1e-10\\
1637	1e-10\\
1638	1e-10\\
1639	1e-10\\
1640	1e-10\\
1641	1e-10\\
1642	1e-10\\
1643	1e-10\\
1644	1e-10\\
1645	1e-10\\
1646	1e-10\\
1647	1e-10\\
1648	1e-10\\
1649	1e-10\\
1650	1e-10\\
1651	1e-10\\
1652	1e-10\\
1653	1e-10\\
1654	1e-10\\
1655	1e-10\\
1656	1e-10\\
1657	1e-10\\
1658	1e-10\\
1659	1e-10\\
1660	1e-10\\
1661	1e-10\\
1662	1e-10\\
1663	1e-10\\
1664	1e-10\\
1665	1e-10\\
1666	1e-10\\
1667	1e-10\\
1668	1e-10\\
1669	1e-10\\
1670	1e-10\\
1671	1e-10\\
1672	1e-10\\
1673	1e-10\\
1674	1e-10\\
1675	1e-10\\
1676	1e-10\\
1677	1e-10\\
1678	1e-10\\
1679	1e-10\\
1680	1e-10\\
1681	1e-10\\
1682	1e-10\\
1683	1e-10\\
1684	1e-10\\
1685	1e-10\\
1686	1e-10\\
1687	1e-10\\
1688	1e-10\\
1689	1e-10\\
1690	1e-10\\
1691	1e-10\\
1692	1e-10\\
1693	1e-10\\
1694	1e-10\\
1695	1e-10\\
1696	1e-10\\
1697	1e-10\\
1698	1e-10\\
1699	1e-10\\
1700	1e-10\\
1701	1e-10\\
1702	1e-10\\
1703	1e-10\\
1704	1e-10\\
1705	1e-10\\
1706	1e-10\\
1707	1e-10\\
1708	1e-10\\
1709	1e-10\\
1710	1e-10\\
1711	1e-10\\
1712	1e-10\\
1713	1e-10\\
1714	1e-10\\
1715	1e-10\\
1716	1e-10\\
1717	1e-10\\
1718	1e-10\\
1719	1e-10\\
1720	1e-10\\
1721	1e-10\\
1722	1e-10\\
1723	1e-10\\
1724	1e-10\\
1725	1e-10\\
1726	1e-10\\
1727	1e-10\\
1728	1e-10\\
1729	1e-10\\
1730	1e-10\\
1731	1e-10\\
1732	1e-10\\
1733	1e-10\\
1734	1e-10\\
1735	1e-10\\
1736	1e-10\\
1737	1e-10\\
1738	1e-10\\
1739	1e-10\\
1740	1e-10\\
1741	1e-10\\
1742	1e-10\\
1743	1e-10\\
1744	1e-10\\
1745	1e-10\\
1746	1e-10\\
1747	1e-10\\
1748	1e-10\\
1749	1e-10\\
1750	1e-10\\
1751	1e-10\\
1752	1e-10\\
1753	1e-10\\
1754	1e-10\\
1755	1e-10\\
1756	1e-10\\
1757	1e-10\\
1758	1e-10\\
1759	1e-10\\
1760	1e-10\\
1761	1e-10\\
1762	1e-10\\
1763	1e-10\\
1764	1e-10\\
1765	1e-10\\
1766	1e-10\\
1767	1e-10\\
1768	1e-10\\
1769	1e-10\\
1770	1e-10\\
1771	1e-10\\
1772	1e-10\\
1773	1e-10\\
1774	1e-10\\
1775	1e-10\\
1776	1e-10\\
1777	1e-10\\
1778	1e-10\\
1779	1e-10\\
1780	1e-10\\
1781	1e-10\\
1782	1e-10\\
1783	1e-10\\
1784	1e-10\\
1785	1e-10\\
1786	1e-10\\
1787	1e-10\\
1788	1e-10\\
1789	1e-10\\
1790	1e-10\\
1791	1e-10\\
1792	1e-10\\
1793	1e-10\\
1794	1e-10\\
1795	1e-10\\
1796	1e-10\\
1797	1e-10\\
1798	1e-10\\
1799	1e-10\\
1800	1e-10\\
1801	1e-10\\
1802	1e-10\\
1803	1e-10\\
1804	1e-10\\
1805	1e-10\\
1806	1e-10\\
1807	1e-10\\
1808	1e-10\\
1809	1e-10\\
1810	1e-10\\
1811	1e-10\\
1812	1e-10\\
1813	1e-10\\
1814	1e-10\\
1815	1e-10\\
1816	1e-10\\
1817	1e-10\\
1818	1e-10\\
1819	1e-10\\
1820	1e-10\\
1821	1e-10\\
1822	1e-10\\
1823	1e-10\\
1824	1e-10\\
1825	1e-10\\
1826	1e-10\\
1827	1e-10\\
1828	1e-10\\
1829	1e-10\\
1830	1e-10\\
1831	1e-10\\
1832	1e-10\\
1833	1e-10\\
1834	1e-10\\
1835	1e-10\\
1836	1e-10\\
1837	1e-10\\
1838	1e-10\\
1839	1e-10\\
1840	1e-10\\
1841	1e-10\\
1842	1e-10\\
1843	1e-10\\
1844	1e-10\\
1845	1e-10\\
1846	1e-10\\
1847	1e-10\\
1848	1e-10\\
1849	1e-10\\
1850	1e-10\\
1851	1e-10\\
1852	1e-10\\
1853	1e-10\\
1854	1e-10\\
1855	1e-10\\
1856	1e-10\\
1857	1e-10\\
1858	1e-10\\
1859	1e-10\\
1860	1e-10\\
1861	1e-10\\
1862	1e-10\\
1863	1e-10\\
1864	1e-10\\
1865	1e-10\\
1866	1e-10\\
1867	1e-10\\
1868	1e-10\\
1869	1e-10\\
1870	1e-10\\
1871	1e-10\\
1872	1e-10\\
1873	1e-10\\
1874	1e-10\\
1875	1e-10\\
1876	1e-10\\
1877	1e-10\\
1878	1e-10\\
1879	1e-10\\
1880	1e-10\\
1881	1e-10\\
1882	1e-10\\
1883	1e-10\\
1884	1e-10\\
1885	1e-10\\
1886	1e-10\\
1887	1e-10\\
1888	1e-10\\
1889	1e-10\\
1890	1e-10\\
1891	1e-10\\
1892	1e-10\\
1893	1e-10\\
1894	1e-10\\
1895	1e-10\\
1896	1e-10\\
1897	1e-10\\
1898	1e-10\\
1899	1e-10\\
1900	1e-10\\
1901	1e-10\\
1902	1e-10\\
1903	1e-10\\
1904	1e-10\\
1905	1e-10\\
1906	1e-10\\
1907	1e-10\\
1908	1e-10\\
1909	1e-10\\
1910	1e-10\\
1911	1e-10\\
1912	1e-10\\
1913	1e-10\\
1914	1e-10\\
1915	1e-10\\
1916	1e-10\\
1917	1e-10\\
1918	1e-10\\
1919	1e-10\\
1920	1e-10\\
1921	1e-10\\
1922	1e-10\\
1923	1e-10\\
1924	1e-10\\
1925	1e-10\\
1926	1e-10\\
1927	1e-10\\
1928	1e-10\\
1929	1e-10\\
1930	1e-10\\
1931	1e-10\\
1932	1e-10\\
1933	1e-10\\
1934	1e-10\\
1935	1e-10\\
1936	1e-10\\
1937	1e-10\\
1938	1e-10\\
1939	1e-10\\
1940	1e-10\\
1941	1e-10\\
1942	1e-10\\
1943	1e-10\\
1944	1e-10\\
1945	1e-10\\
1946	1e-10\\
1947	1e-10\\
1948	1e-10\\
1949	1e-10\\
1950	1e-10\\
1951	1e-10\\
1952	1e-10\\
1953	1e-10\\
1954	1e-10\\
1955	1e-10\\
1956	1e-10\\
1957	1e-10\\
1958	1e-10\\
1959	1e-10\\
1960	1e-10\\
1961	1e-10\\
1962	1e-10\\
1963	1e-10\\
1964	1e-10\\
1965	1e-10\\
1966	1e-10\\
1967	1e-10\\
1968	1e-10\\
1969	1e-10\\
1970	1e-10\\
1971	1e-10\\
1972	1e-10\\
1973	1e-10\\
1974	1e-10\\
1975	1e-10\\
1976	1e-10\\
1977	1e-10\\
1978	1e-10\\
1979	1e-10\\
1980	1e-10\\
1981	1e-10\\
1982	1e-10\\
1983	1e-10\\
1984	1e-10\\
1985	1e-10\\
1986	1e-10\\
1987	1e-10\\
1988	1e-10\\
1989	1e-10\\
1990	1e-10\\
1991	1e-10\\
1992	1e-10\\
1993	1e-10\\
1994	1e-10\\
1995	1e-10\\
1996	1e-10\\
1997	1e-10\\
1998	1e-10\\
1999	1e-10\\
2000	1e-10\\
2001	1e-10\\
2002	1e-10\\
2003	1e-10\\
2004	1e-10\\
2005	1e-10\\
2006	1e-10\\
2007	1e-10\\
2008	1e-10\\
2009	1e-10\\
2010	1e-10\\
2011	1e-10\\
2012	1e-10\\
2013	1e-10\\
2014	1e-10\\
2015	1e-10\\
2016	1e-10\\
2017	1e-10\\
2018	1e-10\\
2019	1e-10\\
2020	1e-10\\
2021	1e-10\\
2022	1e-10\\
2023	1e-10\\
2024	1e-10\\
2025	1e-10\\
2026	1e-10\\
2027	1e-10\\
2028	1e-10\\
2029	1e-10\\
2030	1e-10\\
2031	1e-10\\
2032	1e-10\\
2033	1e-10\\
2034	1e-10\\
2035	1e-10\\
2036	1e-10\\
2037	1e-10\\
2038	1e-10\\
2039	1e-10\\
2040	1e-10\\
2041	1e-10\\
2042	1e-10\\
2043	1e-10\\
2044	1e-10\\
2045	1e-10\\
2046	1e-10\\
2047	1e-10\\
2048	1e-10\\
2049	1e-10\\
2050	1e-10\\
2051	1e-10\\
2052	1e-10\\
2053	1e-10\\
2054	1e-10\\
2055	1e-10\\
2056	1e-10\\
2057	1e-10\\
2058	1e-10\\
2059	1e-10\\
2060	1e-10\\
2061	1e-10\\
2062	1e-10\\
2063	1e-10\\
2064	1e-10\\
2065	1e-10\\
2066	1e-10\\
2067	1e-10\\
2068	1e-10\\
2069	1e-10\\
2070	1e-10\\
2071	1e-10\\
2072	1e-10\\
2073	1e-10\\
2074	1e-10\\
2075	1e-10\\
2076	1e-10\\
2077	1e-10\\
2078	1e-10\\
2079	1e-10\\
2080	1e-10\\
2081	1e-10\\
2082	1e-10\\
2083	1e-10\\
2084	1e-10\\
2085	1e-10\\
2086	1e-10\\
2087	1e-10\\
2088	1e-10\\
2089	1e-10\\
2090	1e-10\\
2091	1e-10\\
2092	1e-10\\
2093	1e-10\\
2094	1e-10\\
2095	1e-10\\
2096	1e-10\\
2097	1e-10\\
2098	1e-10\\
2099	1e-10\\
2100	1e-10\\
2101	1e-10\\
2102	1e-10\\
2103	1e-10\\
2104	1e-10\\
2105	1e-10\\
2106	1e-10\\
2107	1e-10\\
2108	1e-10\\
2109	1e-10\\
2110	1e-10\\
2111	1e-10\\
2112	1e-10\\
2113	1e-10\\
2114	1e-10\\
2115	1e-10\\
2116	1e-10\\
2117	1e-10\\
2118	1e-10\\
2119	1e-10\\
2120	1e-10\\
2121	1e-10\\
2122	1e-10\\
2123	1e-10\\
2124	1e-10\\
2125	1e-10\\
2126	1e-10\\
2127	1e-10\\
2128	1e-10\\
2129	1e-10\\
2130	1e-10\\
2131	1e-10\\
2132	1e-10\\
2133	1e-10\\
2134	1e-10\\
2135	1e-10\\
2136	1e-10\\
2137	1e-10\\
2138	1e-10\\
2139	1e-10\\
2140	1e-10\\
2141	1e-10\\
2142	1e-10\\
2143	1e-10\\
2144	1e-10\\
2145	1e-10\\
2146	1e-10\\
2147	1e-10\\
2148	1e-10\\
2149	1e-10\\
2150	1e-10\\
2151	1e-10\\
2152	1e-10\\
2153	1e-10\\
2154	1e-10\\
2155	1e-10\\
2156	1e-10\\
2157	1e-10\\
2158	1e-10\\
2159	1e-10\\
2160	1e-10\\
2161	1e-10\\
2162	1e-10\\
2163	1e-10\\
2164	1e-10\\
2165	1e-10\\
2166	1e-10\\
2167	1e-10\\
2168	1e-10\\
2169	1e-10\\
2170	1e-10\\
2171	1e-10\\
2172	1e-10\\
2173	1e-10\\
2174	1e-10\\
2175	1e-10\\
2176	1e-10\\
2177	1e-10\\
2178	1e-10\\
2179	1e-10\\
2180	1e-10\\
2181	1e-10\\
2182	1e-10\\
2183	1e-10\\
2184	1e-10\\
2185	1e-10\\
2186	1e-10\\
2187	1e-10\\
2188	1e-10\\
2189	1e-10\\
2190	1e-10\\
2191	1e-10\\
2192	1e-10\\
2193	1e-10\\
2194	1e-10\\
2195	1e-10\\
2196	1e-10\\
2197	1e-10\\
2198	1e-10\\
2199	1e-10\\
2200	1e-10\\
2201	1e-10\\
2202	1e-10\\
2203	1e-10\\
2204	1e-10\\
2205	1e-10\\
2206	1e-10\\
2207	1e-10\\
2208	1e-10\\
2209	1e-10\\
2210	1e-10\\
2211	1e-10\\
2212	1e-10\\
2213	1e-10\\
2214	1e-10\\
2215	1e-10\\
2216	1e-10\\
2217	1e-10\\
2218	1e-10\\
2219	1e-10\\
2220	1e-10\\
2221	1e-10\\
2222	1e-10\\
2223	1e-10\\
2224	1e-10\\
2225	1e-10\\
2226	1e-10\\
2227	1e-10\\
2228	1e-10\\
2229	1e-10\\
2230	1e-10\\
2231	1e-10\\
2232	1e-10\\
2233	1e-10\\
2234	1e-10\\
2235	1e-10\\
2236	1e-10\\
2237	1e-10\\
2238	1e-10\\
2239	1e-10\\
2240	1e-10\\
2241	1e-10\\
2242	1e-10\\
2243	1e-10\\
2244	1e-10\\
2245	1e-10\\
2246	1e-10\\
2247	1e-10\\
2248	1e-10\\
2249	1e-10\\
2250	1e-10\\
2251	1e-10\\
2252	1e-10\\
2253	1e-10\\
2254	1e-10\\
2255	1e-10\\
2256	1e-10\\
2257	1e-10\\
2258	1e-10\\
2259	1e-10\\
2260	1e-10\\
2261	1e-10\\
2262	1e-10\\
2263	1e-10\\
2264	1e-10\\
2265	1e-10\\
2266	1e-10\\
2267	1e-10\\
2268	1e-10\\
2269	1e-10\\
2270	1e-10\\
2271	1e-10\\
2272	1e-10\\
2273	1e-10\\
2274	1e-10\\
2275	1e-10\\
2276	1e-10\\
2277	1e-10\\
2278	1e-10\\
2279	1e-10\\
2280	1e-10\\
2281	1e-10\\
2282	1e-10\\
2283	1e-10\\
2284	1e-10\\
2285	1e-10\\
2286	1e-10\\
2287	1e-10\\
2288	1e-10\\
2289	1e-10\\
2290	1e-10\\
2291	1e-10\\
2292	1e-10\\
2293	1e-10\\
2294	1e-10\\
2295	1e-10\\
2296	1e-10\\
2297	1e-10\\
2298	1e-10\\
2299	1e-10\\
2300	1e-10\\
2301	1e-10\\
2302	1e-10\\
2303	1e-10\\
2304	1e-10\\
2305	1e-10\\
2306	1e-10\\
2307	1e-10\\
2308	1e-10\\
2309	1e-10\\
2310	1e-10\\
2311	1e-10\\
2312	1e-10\\
2313	1e-10\\
2314	1e-10\\
2315	1e-10\\
2316	1e-10\\
2317	1e-10\\
2318	1e-10\\
2319	1e-10\\
2320	1e-10\\
2321	1e-10\\
2322	1e-10\\
2323	1e-10\\
2324	1e-10\\
2325	1e-10\\
2326	1e-10\\
2327	1e-10\\
2328	1e-10\\
2329	1e-10\\
2330	1e-10\\
2331	1e-10\\
2332	1e-10\\
2333	1e-10\\
2334	1e-10\\
2335	1e-10\\
2336	1e-10\\
2337	1e-10\\
2338	1e-10\\
2339	1e-10\\
2340	1e-10\\
2341	1e-10\\
2342	1e-10\\
2343	1e-10\\
2344	1e-10\\
2345	1e-10\\
2346	1e-10\\
2347	1e-10\\
2348	1e-10\\
2349	1e-10\\
2350	1e-10\\
2351	1e-10\\
2352	1e-10\\
2353	1e-10\\
2354	1e-10\\
2355	1e-10\\
2356	1e-10\\
2357	1e-10\\
2358	1e-10\\
2359	1e-10\\
2360	1e-10\\
2361	1e-10\\
2362	1e-10\\
2363	1e-10\\
2364	1e-10\\
2365	1e-10\\
2366	1e-10\\
2367	1e-10\\
2368	1e-10\\
2369	1e-10\\
2370	1e-10\\
2371	1e-10\\
2372	1e-10\\
2373	1e-10\\
2374	1e-10\\
2375	1e-10\\
2376	1e-10\\
2377	1e-10\\
2378	1e-10\\
2379	1e-10\\
2380	1e-10\\
2381	1e-10\\
2382	1e-10\\
2383	1e-10\\
2384	1e-10\\
2385	1e-10\\
2386	1e-10\\
2387	1e-10\\
2388	1e-10\\
2389	1e-10\\
2390	1e-10\\
2391	1e-10\\
2392	1e-10\\
2393	1e-10\\
2394	1e-10\\
2395	1e-10\\
2396	1e-10\\
2397	1e-10\\
2398	1e-10\\
2399	1e-10\\
2400	1e-10\\
2401	1e-10\\
2402	1e-10\\
2403	1e-10\\
2404	1e-10\\
2405	1e-10\\
2406	1e-10\\
2407	1e-10\\
2408	1e-10\\
2409	1e-10\\
2410	1e-10\\
2411	1e-10\\
2412	1e-10\\
2413	1e-10\\
2414	1e-10\\
2415	1e-10\\
2416	1e-10\\
2417	1e-10\\
2418	1e-10\\
2419	1e-10\\
2420	1e-10\\
2421	1e-10\\
2422	1e-10\\
2423	1e-10\\
2424	1e-10\\
2425	1e-10\\
2426	1e-10\\
2427	1e-10\\
2428	1e-10\\
2429	1e-10\\
2430	1e-10\\
2431	1e-10\\
2432	1e-10\\
2433	1e-10\\
2434	1e-10\\
2435	1e-10\\
2436	1e-10\\
2437	1e-10\\
2438	1e-10\\
2439	1e-10\\
2440	1e-10\\
2441	1e-10\\
2442	1e-10\\
2443	1e-10\\
2444	1e-10\\
2445	1e-10\\
2446	1e-10\\
2447	1e-10\\
2448	1e-10\\
2449	1e-10\\
2450	1e-10\\
2451	1e-10\\
2452	1e-10\\
2453	1e-10\\
2454	1e-10\\
2455	1e-10\\
2456	1e-10\\
2457	1e-10\\
2458	1e-10\\
2459	1e-10\\
2460	1e-10\\
2461	1e-10\\
2462	1e-10\\
2463	1e-10\\
2464	1e-10\\
2465	1e-10\\
2466	1e-10\\
2467	1e-10\\
2468	1e-10\\
2469	1e-10\\
2470	1e-10\\
2471	1e-10\\
2472	1e-10\\
2473	1e-10\\
2474	1e-10\\
2475	1e-10\\
2476	1e-10\\
2477	1e-10\\
2478	1e-10\\
2479	1e-10\\
2480	1e-10\\
2481	1e-10\\
2482	1e-10\\
2483	1e-10\\
2484	1e-10\\
2485	1e-10\\
2486	1e-10\\
2487	1e-10\\
2488	1e-10\\
2489	1e-10\\
2490	1e-10\\
2491	1e-10\\
2492	1e-10\\
2493	1e-10\\
2494	1e-10\\
2495	1e-10\\
2496	1e-10\\
2497	1e-10\\
2498	1e-10\\
2499	1e-10\\
2500	1e-10\\
2501	1e-10\\
2502	1e-10\\
2503	1e-10\\
2504	1e-10\\
2505	1e-10\\
2506	1e-10\\
2507	1e-10\\
2508	1e-10\\
2509	1e-10\\
2510	1e-10\\
2511	1e-10\\
2512	1e-10\\
2513	1e-10\\
2514	1e-10\\
2515	1e-10\\
2516	1e-10\\
2517	1e-10\\
2518	1e-10\\
2519	1e-10\\
2520	1e-10\\
2521	1e-10\\
2522	1e-10\\
2523	1e-10\\
2524	1e-10\\
2525	1e-10\\
2526	1e-10\\
2527	1e-10\\
2528	1e-10\\
2529	1e-10\\
2530	1e-10\\
2531	1e-10\\
2532	1e-10\\
2533	1e-10\\
2534	1e-10\\
2535	1e-10\\
2536	1e-10\\
2537	1e-10\\
2538	1e-10\\
2539	1e-10\\
2540	1e-10\\
2541	1e-10\\
2542	1e-10\\
2543	1e-10\\
2544	1e-10\\
2545	1e-10\\
2546	1e-10\\
2547	1e-10\\
2548	1e-10\\
2549	1e-10\\
2550	1e-10\\
2551	1e-10\\
2552	1e-10\\
2553	1e-10\\
2554	1e-10\\
2555	1e-10\\
2556	1e-10\\
2557	1e-10\\
2558	1e-10\\
2559	1e-10\\
2560	1e-10\\
2561	1e-10\\
2562	1e-10\\
2563	1e-10\\
2564	1e-10\\
2565	1e-10\\
2566	1e-10\\
2567	1e-10\\
2568	1e-10\\
2569	1e-10\\
2570	1e-10\\
2571	1e-10\\
2572	1e-10\\
2573	1e-10\\
2574	1e-10\\
2575	1e-10\\
2576	1e-10\\
2577	1e-10\\
2578	1e-10\\
2579	1e-10\\
2580	1e-10\\
2581	1e-10\\
2582	1e-10\\
2583	1e-10\\
2584	1e-10\\
2585	1e-10\\
2586	1e-10\\
2587	1e-10\\
2588	1e-10\\
2589	1e-10\\
2590	1e-10\\
2591	1e-10\\
2592	1e-10\\
2593	1e-10\\
2594	1e-10\\
2595	1e-10\\
2596	1e-10\\
2597	1e-10\\
2598	1e-10\\
2599	1e-10\\
2600	1e-10\\
2601	1e-10\\
2602	1e-10\\
2603	1e-10\\
2604	1e-10\\
2605	1e-10\\
2606	1e-10\\
2607	1e-10\\
2608	1e-10\\
2609	1e-10\\
2610	1e-10\\
2611	1e-10\\
2612	1e-10\\
2613	1e-10\\
2614	1e-10\\
2615	1e-10\\
2616	1e-10\\
2617	1e-10\\
2618	1e-10\\
2619	1e-10\\
2620	1e-10\\
2621	1e-10\\
2622	1e-10\\
2623	1e-10\\
2624	1e-10\\
2625	1e-10\\
2626	1e-10\\
2627	1e-10\\
2628	1e-10\\
2629	1e-10\\
2630	1e-10\\
2631	1e-10\\
2632	1e-10\\
2633	1e-10\\
2634	1e-10\\
2635	1e-10\\
2636	1e-10\\
2637	1e-10\\
2638	1e-10\\
2639	1e-10\\
2640	1e-10\\
2641	1e-10\\
2642	1e-10\\
2643	1e-10\\
2644	1e-10\\
2645	1e-10\\
2646	1e-10\\
2647	1e-10\\
2648	1e-10\\
2649	1e-10\\
2650	1e-10\\
2651	1e-10\\
2652	1e-10\\
2653	1e-10\\
2654	1e-10\\
2655	1e-10\\
2656	1e-10\\
2657	1e-10\\
2658	1e-10\\
2659	1e-10\\
2660	1e-10\\
2661	1e-10\\
2662	1e-10\\
2663	1e-10\\
2664	1e-10\\
2665	1e-10\\
2666	1e-10\\
2667	1e-10\\
2668	1e-10\\
2669	1e-10\\
2670	1e-10\\
2671	1e-10\\
2672	1e-10\\
2673	1e-10\\
2674	1e-10\\
2675	1e-10\\
2676	1e-10\\
2677	1e-10\\
2678	1e-10\\
2679	1e-10\\
2680	1e-10\\
2681	1e-10\\
2682	1e-10\\
2683	1e-10\\
2684	1e-10\\
2685	1e-10\\
2686	1e-10\\
2687	1e-10\\
2688	1e-10\\
2689	1e-10\\
2690	1e-10\\
2691	1e-10\\
2692	1e-10\\
2693	1e-10\\
2694	1e-10\\
2695	1e-10\\
2696	1e-10\\
2697	1e-10\\
2698	1e-10\\
2699	1e-10\\
2700	1e-10\\
2701	1e-10\\
2702	1e-10\\
2703	1e-10\\
2704	1e-10\\
2705	1e-10\\
2706	1e-10\\
2707	1e-10\\
2708	1e-10\\
2709	1e-10\\
2710	1e-10\\
2711	1e-10\\
2712	1e-10\\
2713	1e-10\\
2714	1e-10\\
2715	1e-10\\
2716	1e-10\\
2717	1e-10\\
2718	1e-10\\
2719	1e-10\\
2720	1e-10\\
2721	1e-10\\
2722	1e-10\\
2723	1e-10\\
2724	1e-10\\
2725	1e-10\\
2726	1e-10\\
2727	1e-10\\
2728	1e-10\\
2729	1e-10\\
2730	1e-10\\
2731	1e-10\\
2732	1e-10\\
2733	1e-10\\
2734	1e-10\\
2735	1e-10\\
2736	1e-10\\
2737	1e-10\\
2738	1e-10\\
2739	1e-10\\
2740	1e-10\\
2741	1e-10\\
2742	1e-10\\
2743	1e-10\\
2744	1e-10\\
2745	1e-10\\
2746	1e-10\\
2747	1e-10\\
2748	1e-10\\
2749	1e-10\\
2750	1e-10\\
2751	1e-10\\
2752	1e-10\\
2753	1e-10\\
2754	1e-10\\
2755	1e-10\\
2756	1e-10\\
2757	1e-10\\
2758	1e-10\\
2759	1e-10\\
2760	1e-10\\
2761	1e-10\\
2762	1e-10\\
2763	1e-10\\
2764	1e-10\\
2765	1e-10\\
2766	1e-10\\
2767	1e-10\\
2768	1e-10\\
2769	1e-10\\
2770	1e-10\\
2771	1e-10\\
2772	1e-10\\
2773	1e-10\\
2774	1e-10\\
2775	1e-10\\
2776	1e-10\\
2777	1e-10\\
2778	1e-10\\
2779	1e-10\\
2780	1e-10\\
2781	1e-10\\
2782	1e-10\\
2783	1e-10\\
2784	1e-10\\
2785	1e-10\\
2786	1e-10\\
2787	1e-10\\
2788	1e-10\\
2789	1e-10\\
2790	1e-10\\
2791	1e-10\\
2792	1e-10\\
2793	1e-10\\
2794	1e-10\\
2795	1e-10\\
2796	1e-10\\
2797	1e-10\\
2798	1e-10\\
2799	1e-10\\
2800	1e-10\\
2801	1e-10\\
2802	1e-10\\
2803	1e-10\\
2804	1e-10\\
2805	1e-10\\
2806	1e-10\\
2807	1e-10\\
2808	1e-10\\
2809	1e-10\\
2810	1e-10\\
2811	1e-10\\
2812	1e-10\\
2813	1e-10\\
2814	1e-10\\
2815	1e-10\\
2816	1e-10\\
2817	1e-10\\
2818	1e-10\\
2819	1e-10\\
2820	1e-10\\
2821	1e-10\\
2822	1e-10\\
2823	1e-10\\
2824	1e-10\\
2825	1e-10\\
2826	1e-10\\
2827	1e-10\\
2828	1e-10\\
2829	1e-10\\
2830	1e-10\\
2831	1e-10\\
2832	1e-10\\
2833	1e-10\\
2834	1e-10\\
2835	1e-10\\
2836	1e-10\\
2837	1e-10\\
2838	1e-10\\
2839	1e-10\\
2840	1e-10\\
2841	1e-10\\
2842	1e-10\\
2843	1e-10\\
2844	1e-10\\
2845	1e-10\\
2846	1e-10\\
2847	1e-10\\
2848	1e-10\\
2849	1e-10\\
2850	1e-10\\
2851	1e-10\\
2852	1e-10\\
2853	1e-10\\
2854	1e-10\\
2855	1e-10\\
2856	1e-10\\
2857	1e-10\\
2858	1e-10\\
2859	1e-10\\
2860	1e-10\\
2861	1e-10\\
2862	1e-10\\
2863	1e-10\\
2864	1e-10\\
2865	1e-10\\
2866	1e-10\\
2867	1e-10\\
2868	1e-10\\
2869	1e-10\\
2870	1e-10\\
2871	1e-10\\
2872	1e-10\\
2873	1e-10\\
2874	1e-10\\
2875	1e-10\\
2876	1e-10\\
2877	1e-10\\
2878	1e-10\\
2879	1e-10\\
2880	1e-10\\
2881	1e-10\\
2882	1e-10\\
2883	1e-10\\
2884	1e-10\\
2885	1e-10\\
2886	1e-10\\
2887	1e-10\\
2888	1e-10\\
2889	1e-10\\
2890	1e-10\\
2891	1e-10\\
2892	1e-10\\
2893	1e-10\\
2894	1e-10\\
2895	1e-10\\
2896	1e-10\\
2897	1e-10\\
2898	1e-10\\
2899	1e-10\\
2900	1e-10\\
2901	1e-10\\
2902	1e-10\\
2903	1e-10\\
2904	1e-10\\
2905	1e-10\\
2906	1e-10\\
2907	1e-10\\
2908	1e-10\\
2909	1e-10\\
2910	1e-10\\
2911	1e-10\\
2912	1e-10\\
2913	1e-10\\
2914	1e-10\\
2915	1e-10\\
2916	1e-10\\
2917	1e-10\\
2918	1e-10\\
2919	1e-10\\
2920	1e-10\\
2921	1e-10\\
2922	1e-10\\
2923	1e-10\\
2924	1e-10\\
2925	1e-10\\
2926	1e-10\\
2927	1e-10\\
2928	1e-10\\
2929	1e-10\\
2930	1e-10\\
2931	1e-10\\
2932	1e-10\\
2933	1e-10\\
2934	1e-10\\
2935	1e-10\\
2936	1e-10\\
2937	1e-10\\
2938	1e-10\\
2939	1e-10\\
2940	1e-10\\
2941	1e-10\\
2942	1e-10\\
2943	1e-10\\
2944	1e-10\\
2945	1e-10\\
2946	1e-10\\
2947	1e-10\\
2948	1e-10\\
2949	1e-10\\
2950	1e-10\\
2951	1e-10\\
2952	1e-10\\
2953	1e-10\\
2954	1e-10\\
2955	1e-10\\
2956	1e-10\\
2957	1e-10\\
2958	1e-10\\
2959	1e-10\\
2960	1e-10\\
2961	1e-10\\
2962	1e-10\\
2963	1e-10\\
2964	1e-10\\
2965	1e-10\\
2966	1e-10\\
2967	1e-10\\
2968	1e-10\\
2969	1e-10\\
2970	1e-10\\
2971	1e-10\\
2972	1e-10\\
2973	1e-10\\
2974	1e-10\\
2975	1e-10\\
2976	1e-10\\
2977	1e-10\\
2978	1e-10\\
2979	1e-10\\
2980	1e-10\\
2981	1e-10\\
2982	1e-10\\
2983	1e-10\\
2984	1e-10\\
2985	1e-10\\
2986	1e-10\\
2987	1e-10\\
2988	1e-10\\
2989	1e-10\\
2990	1e-10\\
2991	1e-10\\
2992	1e-10\\
2993	1e-10\\
2994	1e-10\\
2995	1e-10\\
2996	1e-10\\
2997	1e-10\\
2998	1e-10\\
2999	1e-10\\
3000	1e-10\\
3001	1e-10\\
3002	1e-10\\
3003	1e-10\\
3004	1e-10\\
3005	1e-10\\
3006	1e-10\\
3007	1e-10\\
3008	1e-10\\
3009	1e-10\\
3010	1e-10\\
3011	1e-10\\
3012	1e-10\\
3013	1e-10\\
3014	1e-10\\
3015	1e-10\\
3016	1e-10\\
3017	1e-10\\
3018	1e-10\\
3019	1e-10\\
3020	1e-10\\
3021	1e-10\\
3022	1e-10\\
3023	1e-10\\
3024	1e-10\\
3025	1e-10\\
3026	1e-10\\
3027	1e-10\\
3028	1e-10\\
3029	1e-10\\
3030	1e-10\\
3031	1e-10\\
3032	1e-10\\
3033	1e-10\\
3034	1e-10\\
3035	1e-10\\
3036	1e-10\\
3037	1e-10\\
3038	1e-10\\
3039	1e-10\\
3040	1e-10\\
3041	1e-10\\
3042	1e-10\\
3043	1e-10\\
3044	1e-10\\
3045	1e-10\\
3046	1e-10\\
3047	1e-10\\
3048	1e-10\\
3049	1e-10\\
3050	1e-10\\
3051	1e-10\\
3052	1e-10\\
3053	1e-10\\
3054	1e-10\\
3055	1e-10\\
3056	1e-10\\
3057	1e-10\\
3058	1e-10\\
3059	1e-10\\
3060	1e-10\\
3061	1e-10\\
3062	1e-10\\
3063	1e-10\\
3064	1e-10\\
3065	1e-10\\
3066	1e-10\\
3067	1e-10\\
3068	1e-10\\
3069	1e-10\\
3070	1e-10\\
3071	1e-10\\
3072	1e-10\\
3073	1e-10\\
3074	1e-10\\
3075	1e-10\\
3076	1e-10\\
3077	1e-10\\
3078	1e-10\\
3079	1e-10\\
3080	1e-10\\
3081	1e-10\\
3082	1e-10\\
3083	1e-10\\
3084	1e-10\\
3085	1e-10\\
3086	1e-10\\
3087	1e-10\\
3088	1e-10\\
3089	1e-10\\
3090	1e-10\\
3091	1e-10\\
3092	1e-10\\
3093	1e-10\\
3094	1e-10\\
3095	1e-10\\
3096	1e-10\\
3097	1e-10\\
3098	1e-10\\
3099	1e-10\\
3100	1e-10\\
3101	1e-10\\
3102	1e-10\\
3103	1e-10\\
3104	1e-10\\
3105	1e-10\\
3106	1e-10\\
3107	1e-10\\
3108	1e-10\\
3109	1e-10\\
3110	1e-10\\
3111	1e-10\\
3112	1e-10\\
3113	1e-10\\
3114	1e-10\\
3115	1e-10\\
3116	1e-10\\
3117	1e-10\\
3118	1e-10\\
3119	1e-10\\
3120	1e-10\\
3121	1e-10\\
3122	1e-10\\
3123	1e-10\\
3124	1e-10\\
3125	1e-10\\
3126	1e-10\\
3127	1e-10\\
3128	1e-10\\
3129	1e-10\\
3130	1e-10\\
3131	1e-10\\
3132	1e-10\\
3133	1e-10\\
3134	1e-10\\
3135	1e-10\\
3136	1e-10\\
3137	1e-10\\
3138	1e-10\\
3139	1e-10\\
3140	1e-10\\
3141	1e-10\\
3142	1e-10\\
3143	1e-10\\
3144	1e-10\\
3145	1e-10\\
3146	1e-10\\
3147	1e-10\\
3148	1e-10\\
3149	1e-10\\
3150	1e-10\\
3151	1e-10\\
3152	1e-10\\
3153	1e-10\\
3154	1e-10\\
3155	1e-10\\
3156	1e-10\\
3157	1e-10\\
3158	1e-10\\
3159	1e-10\\
3160	1e-10\\
3161	1e-10\\
3162	1e-10\\
3163	1e-10\\
3164	1e-10\\
3165	1e-10\\
3166	1e-10\\
3167	1e-10\\
3168	1e-10\\
3169	1e-10\\
3170	1e-10\\
3171	1e-10\\
3172	1e-10\\
3173	1e-10\\
3174	1e-10\\
3175	1e-10\\
3176	1e-10\\
3177	1e-10\\
3178	1e-10\\
3179	1e-10\\
3180	1e-10\\
3181	1e-10\\
3182	1e-10\\
3183	1e-10\\
3184	1e-10\\
3185	1e-10\\
3186	1e-10\\
3187	1e-10\\
3188	1e-10\\
3189	1e-10\\
3190	1e-10\\
3191	1e-10\\
3192	1e-10\\
3193	1e-10\\
3194	1e-10\\
3195	1e-10\\
3196	1e-10\\
3197	1e-10\\
3198	1e-10\\
3199	1e-10\\
3200	1e-10\\
3201	1e-10\\
3202	1e-10\\
3203	1e-10\\
3204	1e-10\\
3205	1e-10\\
3206	1e-10\\
3207	1e-10\\
3208	1e-10\\
3209	1e-10\\
3210	1e-10\\
3211	1e-10\\
3212	1e-10\\
3213	1e-10\\
3214	1e-10\\
3215	1e-10\\
3216	1e-10\\
3217	1e-10\\
3218	1e-10\\
3219	1e-10\\
3220	1e-10\\
3221	1e-10\\
3222	1e-10\\
3223	1e-10\\
3224	1e-10\\
3225	1e-10\\
3226	1e-10\\
3227	1e-10\\
3228	1e-10\\
3229	1e-10\\
3230	1e-10\\
3231	1e-10\\
3232	1e-10\\
3233	1e-10\\
3234	1e-10\\
3235	1e-10\\
3236	1e-10\\
3237	1e-10\\
3238	1e-10\\
3239	1e-10\\
3240	1e-10\\
3241	1e-10\\
3242	1e-10\\
3243	1e-10\\
3244	1e-10\\
3245	1e-10\\
3246	1e-10\\
3247	1e-10\\
3248	1e-10\\
3249	1e-10\\
3250	1e-10\\
3251	1e-10\\
3252	1e-10\\
3253	1e-10\\
3254	1e-10\\
3255	1e-10\\
3256	1e-10\\
3257	1e-10\\
3258	1e-10\\
3259	1e-10\\
3260	1e-10\\
3261	1e-10\\
3262	1e-10\\
3263	1e-10\\
3264	1e-10\\
3265	1e-10\\
3266	1e-10\\
3267	1e-10\\
3268	1e-10\\
3269	1e-10\\
3270	1e-10\\
3271	1e-10\\
3272	1e-10\\
3273	1e-10\\
3274	1e-10\\
3275	1e-10\\
3276	1e-10\\
3277	1e-10\\
3278	1e-10\\
3279	1e-10\\
3280	1e-10\\
3281	1e-10\\
3282	1e-10\\
3283	1e-10\\
3284	1e-10\\
3285	1e-10\\
3286	1e-10\\
3287	1e-10\\
3288	1e-10\\
3289	1e-10\\
3290	1e-10\\
3291	1e-10\\
3292	1e-10\\
3293	1e-10\\
3294	1e-10\\
3295	1e-10\\
3296	1e-10\\
3297	1e-10\\
3298	1e-10\\
3299	1e-10\\
3300	1e-10\\
3301	1e-10\\
3302	1e-10\\
3303	1e-10\\
3304	1e-10\\
3305	1e-10\\
3306	1e-10\\
3307	1e-10\\
3308	1e-10\\
3309	1e-10\\
3310	1e-10\\
3311	1e-10\\
3312	1e-10\\
3313	1e-10\\
3314	1e-10\\
3315	1e-10\\
3316	1e-10\\
3317	1e-10\\
3318	1e-10\\
3319	1e-10\\
3320	1e-10\\
3321	1e-10\\
3322	1e-10\\
3323	1e-10\\
3324	1e-10\\
3325	1e-10\\
3326	1e-10\\
3327	1e-10\\
3328	1e-10\\
3329	1e-10\\
3330	1e-10\\
3331	1e-10\\
3332	1e-10\\
3333	1e-10\\
3334	1e-10\\
3335	1e-10\\
3336	1e-10\\
3337	1e-10\\
3338	1e-10\\
3339	1e-10\\
3340	1e-10\\
3341	1e-10\\
3342	1e-10\\
3343	1e-10\\
3344	1e-10\\
3345	1e-10\\
3346	1e-10\\
3347	1e-10\\
3348	1e-10\\
3349	1e-10\\
3350	1e-10\\
3351	1e-10\\
3352	1e-10\\
3353	1e-10\\
3354	1e-10\\
3355	1e-10\\
3356	1e-10\\
3357	1e-10\\
3358	1e-10\\
3359	1e-10\\
3360	1e-10\\
3361	1e-10\\
3362	1e-10\\
3363	1e-10\\
3364	1e-10\\
3365	1e-10\\
3366	1e-10\\
3367	1e-10\\
3368	1e-10\\
3369	1e-10\\
3370	1e-10\\
3371	1e-10\\
3372	1e-10\\
3373	1e-10\\
3374	1e-10\\
3375	1e-10\\
3376	1e-10\\
3377	1e-10\\
3378	1e-10\\
3379	1e-10\\
3380	1e-10\\
3381	1e-10\\
3382	1e-10\\
3383	1e-10\\
3384	1e-10\\
3385	1e-10\\
3386	1e-10\\
3387	1e-10\\
3388	1e-10\\
3389	1e-10\\
3390	1e-10\\
3391	1e-10\\
3392	1e-10\\
3393	1e-10\\
3394	1e-10\\
3395	1e-10\\
3396	1e-10\\
3397	1e-10\\
3398	1e-10\\
3399	1e-10\\
3400	1e-10\\
3401	1e-10\\
3402	1e-10\\
3403	1e-10\\
3404	1e-10\\
3405	1e-10\\
3406	1e-10\\
3407	1e-10\\
3408	1e-10\\
3409	1e-10\\
3410	1e-10\\
3411	1e-10\\
3412	1e-10\\
3413	1e-10\\
3414	1e-10\\
3415	1e-10\\
3416	1e-10\\
3417	1e-10\\
3418	1e-10\\
3419	1e-10\\
3420	1e-10\\
3421	1e-10\\
3422	1e-10\\
3423	1e-10\\
3424	1e-10\\
3425	1e-10\\
3426	1e-10\\
3427	1e-10\\
3428	1e-10\\
3429	1e-10\\
3430	1e-10\\
3431	1e-10\\
3432	1e-10\\
3433	1e-10\\
3434	1e-10\\
3435	1e-10\\
3436	1e-10\\
3437	1e-10\\
3438	1e-10\\
3439	1e-10\\
3440	1e-10\\
3441	1e-10\\
3442	1e-10\\
3443	1e-10\\
3444	1e-10\\
3445	1e-10\\
3446	1e-10\\
3447	1e-10\\
3448	1e-10\\
3449	1e-10\\
3450	1e-10\\
3451	1e-10\\
3452	1e-10\\
3453	1e-10\\
3454	1e-10\\
3455	1e-10\\
3456	1e-10\\
3457	1e-10\\
3458	1e-10\\
3459	1e-10\\
3460	1e-10\\
3461	1e-10\\
3462	1e-10\\
3463	1e-10\\
3464	1e-10\\
3465	1e-10\\
3466	1e-10\\
3467	1e-10\\
3468	1e-10\\
3469	1e-10\\
3470	1e-10\\
3471	1e-10\\
3472	1e-10\\
3473	1e-10\\
3474	1e-10\\
3475	1e-10\\
3476	1e-10\\
3477	1e-10\\
3478	1e-10\\
3479	1e-10\\
3480	1e-10\\
3481	1e-10\\
3482	1e-10\\
3483	1e-10\\
3484	1e-10\\
3485	1e-10\\
3486	1e-10\\
3487	1e-10\\
3488	1e-10\\
3489	1e-10\\
3490	1e-10\\
3491	1e-10\\
3492	1e-10\\
3493	1e-10\\
3494	1e-10\\
3495	1e-10\\
3496	1e-10\\
3497	1e-10\\
3498	1e-10\\
3499	1e-10\\
3500	1e-10\\
3501	1e-10\\
3502	1e-10\\
3503	1e-10\\
3504	1e-10\\
3505	1e-10\\
3506	1e-10\\
3507	1e-10\\
3508	1e-10\\
3509	1e-10\\
3510	1e-10\\
3511	1e-10\\
3512	1e-10\\
3513	1e-10\\
3514	1e-10\\
3515	1e-10\\
3516	1e-10\\
3517	1e-10\\
3518	1e-10\\
3519	1e-10\\
3520	1e-10\\
3521	1e-10\\
3522	1e-10\\
3523	1e-10\\
3524	1e-10\\
3525	1e-10\\
3526	1e-10\\
3527	1e-10\\
3528	1e-10\\
3529	1e-10\\
3530	1e-10\\
3531	1e-10\\
3532	1e-10\\
3533	1e-10\\
3534	1e-10\\
3535	1e-10\\
3536	1e-10\\
3537	1e-10\\
3538	1e-10\\
3539	1e-10\\
3540	1e-10\\
3541	1e-10\\
3542	1e-10\\
3543	1e-10\\
3544	1e-10\\
3545	1e-10\\
3546	1e-10\\
3547	1e-10\\
3548	1e-10\\
3549	1e-10\\
3550	1e-10\\
3551	1e-10\\
3552	1e-10\\
3553	1e-10\\
3554	1e-10\\
3555	1e-10\\
3556	1e-10\\
3557	1e-10\\
3558	1e-10\\
3559	1e-10\\
3560	1e-10\\
3561	1e-10\\
3562	1e-10\\
3563	1e-10\\
3564	1e-10\\
3565	1e-10\\
3566	1e-10\\
3567	1e-10\\
3568	1e-10\\
3569	1e-10\\
3570	1e-10\\
3571	1e-10\\
3572	1e-10\\
3573	1e-10\\
3574	1e-10\\
3575	1e-10\\
3576	1e-10\\
3577	1e-10\\
3578	1e-10\\
3579	1e-10\\
3580	1e-10\\
3581	1e-10\\
3582	1e-10\\
3583	1e-10\\
3584	1e-10\\
3585	1e-10\\
3586	1e-10\\
3587	1e-10\\
3588	1e-10\\
3589	1e-10\\
3590	1e-10\\
3591	1e-10\\
3592	1e-10\\
3593	1e-10\\
3594	1e-10\\
3595	1e-10\\
3596	1e-10\\
3597	1e-10\\
3598	1e-10\\
3599	1e-10\\
3600	1e-10\\
3601	1e-10\\
3602	1e-10\\
3603	1e-10\\
3604	1e-10\\
3605	1e-10\\
3606	1e-10\\
3607	1e-10\\
3608	1e-10\\
3609	1e-10\\
3610	1e-10\\
3611	1e-10\\
3612	1e-10\\
3613	1e-10\\
3614	1e-10\\
3615	1e-10\\
3616	1e-10\\
3617	1e-10\\
3618	1e-10\\
3619	1e-10\\
3620	1e-10\\
3621	1e-10\\
3622	1e-10\\
3623	1e-10\\
3624	1e-10\\
3625	1e-10\\
3626	1e-10\\
3627	1e-10\\
3628	1e-10\\
3629	1e-10\\
3630	1e-10\\
3631	1e-10\\
3632	1e-10\\
3633	1e-10\\
3634	1e-10\\
3635	1e-10\\
3636	1e-10\\
3637	1e-10\\
3638	1e-10\\
3639	1e-10\\
3640	1e-10\\
3641	1e-10\\
3642	1e-10\\
3643	1e-10\\
3644	1e-10\\
3645	1e-10\\
3646	1e-10\\
3647	1e-10\\
3648	1e-10\\
3649	1e-10\\
3650	1e-10\\
3651	1e-10\\
3652	1e-10\\
3653	1e-10\\
3654	1e-10\\
3655	1e-10\\
3656	1e-10\\
3657	1e-10\\
3658	1e-10\\
3659	1e-10\\
3660	1e-10\\
3661	1e-10\\
3662	1e-10\\
3663	1e-10\\
3664	1e-10\\
3665	1e-10\\
3666	1e-10\\
3667	1e-10\\
3668	1e-10\\
3669	1e-10\\
3670	1e-10\\
3671	1e-10\\
3672	1e-10\\
3673	1e-10\\
3674	1e-10\\
3675	1e-10\\
3676	1e-10\\
3677	1e-10\\
3678	1e-10\\
3679	1e-10\\
3680	1e-10\\
3681	1e-10\\
3682	1e-10\\
3683	1e-10\\
3684	1e-10\\
3685	1e-10\\
3686	1e-10\\
3687	1e-10\\
3688	1e-10\\
3689	1e-10\\
3690	1e-10\\
3691	1e-10\\
3692	1e-10\\
3693	1e-10\\
3694	1e-10\\
3695	1e-10\\
3696	1e-10\\
3697	1e-10\\
3698	1e-10\\
3699	1e-10\\
3700	1e-10\\
3701	1e-10\\
3702	1e-10\\
3703	1e-10\\
3704	1e-10\\
3705	1e-10\\
3706	1e-10\\
3707	1e-10\\
3708	1e-10\\
3709	1e-10\\
3710	1e-10\\
3711	1e-10\\
3712	1e-10\\
3713	1e-10\\
3714	1e-10\\
3715	1e-10\\
3716	1e-10\\
3717	1e-10\\
3718	1e-10\\
3719	1e-10\\
3720	1e-10\\
3721	1e-10\\
3722	1e-10\\
3723	1e-10\\
3724	1e-10\\
3725	1e-10\\
3726	1e-10\\
3727	1e-10\\
3728	1e-10\\
3729	1e-10\\
3730	1e-10\\
3731	1e-10\\
3732	1e-10\\
3733	1e-10\\
3734	1e-10\\
3735	1e-10\\
3736	1e-10\\
3737	1e-10\\
3738	1e-10\\
3739	1e-10\\
3740	1e-10\\
3741	1e-10\\
3742	1e-10\\
3743	1e-10\\
3744	1e-10\\
3745	1e-10\\
3746	1e-10\\
3747	1e-10\\
3748	1e-10\\
3749	1e-10\\
3750	1e-10\\
3751	1e-10\\
3752	1e-10\\
3753	1e-10\\
3754	1e-10\\
3755	1e-10\\
3756	1e-10\\
3757	1e-10\\
3758	1e-10\\
3759	1e-10\\
3760	1e-10\\
3761	1e-10\\
3762	1e-10\\
3763	1e-10\\
3764	1e-10\\
3765	1e-10\\
3766	1e-10\\
3767	1e-10\\
3768	1e-10\\
3769	1e-10\\
3770	1e-10\\
3771	1e-10\\
3772	1e-10\\
3773	1e-10\\
3774	1e-10\\
3775	1e-10\\
3776	1e-10\\
3777	1e-10\\
3778	1e-10\\
3779	1e-10\\
3780	1e-10\\
3781	1e-10\\
3782	1e-10\\
3783	1e-10\\
3784	1e-10\\
3785	1e-10\\
3786	1e-10\\
3787	1e-10\\
3788	1e-10\\
3789	1e-10\\
3790	1e-10\\
3791	1e-10\\
3792	1e-10\\
3793	1e-10\\
3794	1e-10\\
3795	1e-10\\
3796	1e-10\\
3797	1e-10\\
3798	1e-10\\
3799	1e-10\\
3800	1e-10\\
3801	1e-10\\
3802	1e-10\\
3803	1e-10\\
3804	1e-10\\
3805	1e-10\\
3806	1e-10\\
3807	1e-10\\
3808	1e-10\\
3809	1e-10\\
3810	1e-10\\
3811	1e-10\\
3812	1e-10\\
3813	1e-10\\
3814	1e-10\\
3815	1e-10\\
3816	1e-10\\
3817	1e-10\\
3818	1e-10\\
3819	1e-10\\
3820	1e-10\\
3821	1e-10\\
3822	1e-10\\
3823	1e-10\\
3824	1e-10\\
3825	1e-10\\
3826	1e-10\\
3827	1e-10\\
3828	1e-10\\
3829	1e-10\\
3830	1e-10\\
3831	1e-10\\
3832	1e-10\\
3833	1e-10\\
3834	1e-10\\
3835	1e-10\\
3836	1e-10\\
3837	1e-10\\
3838	1e-10\\
3839	1e-10\\
3840	1e-10\\
3841	1e-10\\
3842	1e-10\\
3843	1e-10\\
3844	1e-10\\
3845	1e-10\\
3846	1e-10\\
3847	1e-10\\
3848	1e-10\\
3849	1e-10\\
3850	1e-10\\
3851	1e-10\\
3852	1e-10\\
3853	1e-10\\
3854	1e-10\\
3855	1e-10\\
3856	1e-10\\
3857	1e-10\\
3858	1e-10\\
3859	1e-10\\
3860	1e-10\\
3861	1e-10\\
3862	1e-10\\
3863	1e-10\\
3864	1e-10\\
3865	1e-10\\
3866	1e-10\\
3867	1e-10\\
3868	1e-10\\
3869	1e-10\\
3870	1e-10\\
3871	1e-10\\
3872	1e-10\\
3873	1e-10\\
3874	1e-10\\
3875	1e-10\\
3876	1e-10\\
3877	1e-10\\
3878	1e-10\\
3879	1e-10\\
3880	1e-10\\
3881	1e-10\\
3882	1e-10\\
3883	1e-10\\
3884	1e-10\\
3885	1e-10\\
3886	1e-10\\
3887	1e-10\\
3888	1e-10\\
3889	1e-10\\
3890	1e-10\\
3891	1e-10\\
3892	1e-10\\
3893	1e-10\\
3894	1e-10\\
3895	1e-10\\
3896	1e-10\\
3897	1e-10\\
3898	1e-10\\
3899	1e-10\\
3900	1e-10\\
3901	1e-10\\
3902	1e-10\\
3903	1e-10\\
3904	1e-10\\
3905	1e-10\\
3906	1e-10\\
3907	1e-10\\
3908	1e-10\\
3909	1e-10\\
3910	1e-10\\
3911	1e-10\\
3912	1e-10\\
3913	1e-10\\
3914	1e-10\\
3915	1e-10\\
3916	1e-10\\
3917	1e-10\\
3918	1e-10\\
3919	1e-10\\
3920	1e-10\\
3921	1e-10\\
3922	1e-10\\
3923	1e-10\\
3924	1e-10\\
3925	1e-10\\
3926	1e-10\\
3927	1e-10\\
3928	1e-10\\
3929	1e-10\\
3930	1e-10\\
3931	1e-10\\
3932	1e-10\\
3933	1e-10\\
3934	1e-10\\
3935	1e-10\\
3936	1e-10\\
3937	1e-10\\
3938	1e-10\\
3939	1e-10\\
3940	1e-10\\
3941	1e-10\\
3942	1e-10\\
3943	1e-10\\
3944	1e-10\\
3945	1e-10\\
3946	1e-10\\
3947	1e-10\\
3948	1e-10\\
3949	1e-10\\
3950	1e-10\\
3951	1e-10\\
3952	1e-10\\
3953	1e-10\\
3954	1e-10\\
3955	1e-10\\
3956	1e-10\\
3957	1e-10\\
3958	1e-10\\
3959	1e-10\\
3960	1e-10\\
3961	1e-10\\
3962	1e-10\\
3963	1e-10\\
3964	1e-10\\
3965	1e-10\\
3966	1e-10\\
3967	1e-10\\
3968	1e-10\\
3969	1e-10\\
3970	1e-10\\
3971	1e-10\\
3972	1e-10\\
3973	1e-10\\
3974	1e-10\\
3975	1e-10\\
3976	1e-10\\
3977	1e-10\\
3978	1e-10\\
3979	1e-10\\
3980	1e-10\\
3981	1e-10\\
3982	1e-10\\
3983	1e-10\\
3984	1e-10\\
3985	1e-10\\
3986	1e-10\\
3987	1e-10\\
3988	1e-10\\
3989	1e-10\\
3990	1e-10\\
3991	1e-10\\
3992	1e-10\\
3993	1e-10\\
3994	1e-10\\
3995	1e-10\\
3996	1e-10\\
3997	1e-10\\
3998	1e-10\\
3999	1e-10\\
4000	1e-10\\
4001	1e-10\\
};
\addplot [color=red,dashed,line width=3.0pt,forget plot]
  table[row sep=crcr]{%
4001	1e-10\\
4002	1e-10\\
4003	1e-10\\
4004	1e-10\\
4005	1e-10\\
4006	1e-10\\
4007	1e-10\\
4008	1e-10\\
4009	1e-10\\
4010	1e-10\\
4011	1e-10\\
4012	1e-10\\
4013	1e-10\\
4014	1e-10\\
4015	1e-10\\
4016	1e-10\\
4017	1e-10\\
4018	1e-10\\
4019	1e-10\\
4020	1e-10\\
4021	1e-10\\
4022	1e-10\\
4023	1e-10\\
4024	1e-10\\
4025	1e-10\\
4026	1e-10\\
4027	1e-10\\
4028	1e-10\\
4029	1e-10\\
4030	1e-10\\
4031	1e-10\\
4032	1e-10\\
4033	1e-10\\
4034	1e-10\\
4035	1e-10\\
4036	1e-10\\
4037	1e-10\\
4038	1e-10\\
4039	1e-10\\
4040	1e-10\\
4041	1e-10\\
4042	1e-10\\
4043	1e-10\\
4044	1e-10\\
4045	1e-10\\
4046	1e-10\\
4047	1e-10\\
4048	1e-10\\
4049	1e-10\\
4050	1e-10\\
4051	1e-10\\
4052	1e-10\\
4053	1e-10\\
4054	1e-10\\
4055	1e-10\\
4056	1e-10\\
4057	1e-10\\
4058	1e-10\\
4059	1e-10\\
4060	1e-10\\
4061	1e-10\\
4062	1e-10\\
4063	1e-10\\
4064	1e-10\\
4065	1e-10\\
4066	1e-10\\
4067	1e-10\\
4068	1e-10\\
4069	1e-10\\
4070	1e-10\\
4071	1e-10\\
4072	1e-10\\
4073	1e-10\\
4074	1e-10\\
4075	1e-10\\
4076	1e-10\\
4077	1e-10\\
4078	1e-10\\
4079	1e-10\\
4080	1e-10\\
4081	1e-10\\
4082	1e-10\\
4083	1e-10\\
4084	1e-10\\
4085	1e-10\\
4086	1e-10\\
4087	1e-10\\
4088	1e-10\\
4089	1e-10\\
4090	1e-10\\
4091	1e-10\\
4092	1e-10\\
4093	1e-10\\
4094	1e-10\\
4095	1e-10\\
4096	1e-10\\
4097	1e-10\\
4098	1e-10\\
4099	1e-10\\
4100	1e-10\\
4101	1e-10\\
4102	1e-10\\
4103	1e-10\\
4104	1e-10\\
4105	1e-10\\
4106	1e-10\\
4107	1e-10\\
4108	1e-10\\
4109	1e-10\\
4110	1e-10\\
4111	1e-10\\
4112	1e-10\\
4113	1e-10\\
4114	1e-10\\
4115	1e-10\\
4116	1e-10\\
4117	1e-10\\
4118	1e-10\\
4119	1e-10\\
4120	1e-10\\
4121	1e-10\\
4122	1e-10\\
4123	1e-10\\
4124	1e-10\\
4125	1e-10\\
4126	1e-10\\
4127	1e-10\\
4128	1e-10\\
4129	1e-10\\
4130	1e-10\\
4131	1e-10\\
4132	1e-10\\
4133	1e-10\\
4134	1e-10\\
4135	1e-10\\
4136	1e-10\\
4137	1e-10\\
4138	1e-10\\
4139	1e-10\\
4140	1e-10\\
4141	1e-10\\
4142	1e-10\\
4143	1e-10\\
4144	1e-10\\
4145	1e-10\\
4146	1e-10\\
4147	1e-10\\
4148	1e-10\\
4149	1e-10\\
4150	1e-10\\
4151	1e-10\\
4152	1e-10\\
4153	1e-10\\
4154	1e-10\\
4155	1e-10\\
4156	1e-10\\
4157	1e-10\\
4158	1e-10\\
4159	1e-10\\
4160	1e-10\\
4161	1e-10\\
4162	1e-10\\
4163	1e-10\\
4164	1e-10\\
4165	1e-10\\
4166	1e-10\\
4167	1e-10\\
4168	1e-10\\
4169	1e-10\\
4170	1e-10\\
4171	1e-10\\
4172	1e-10\\
4173	1e-10\\
4174	1e-10\\
4175	1e-10\\
4176	1e-10\\
4177	1e-10\\
4178	1e-10\\
4179	1e-10\\
4180	1e-10\\
4181	1e-10\\
4182	1e-10\\
4183	1e-10\\
4184	1e-10\\
4185	1e-10\\
4186	1e-10\\
4187	1e-10\\
4188	1e-10\\
4189	1e-10\\
4190	1e-10\\
4191	1e-10\\
4192	1e-10\\
4193	1e-10\\
4194	1e-10\\
4195	1e-10\\
4196	1e-10\\
4197	1e-10\\
4198	1e-10\\
4199	1e-10\\
4200	1e-10\\
};
\end{axis}
\end{tikzpicture}%
%	\caption{Solid line: residual $L^{\infty}$ error of the numerical approximation $G(x) \approx \sum_{p=1}^P\alpha_p J_0(\rho_p |x|)$. Dashed line: estimated contribution of the level of error in the numerical inversion of matrix $A$. The latter is computed as $C ||A (\hat{X}x) - x||$ where $\hat{X}$ is the numerical inversion of $A$ and $C$ is a constant is estimated graphically.}
%	\label{emina}
%\end{figure}

\section{Application to the Laplace kernel}
\label{sec:ApplicationLaplace}
Solving PDE's involving the Laplace operator (for example heat conduction of electrostatics), one is led to \eqref{discreteConv} with the Laplace kernel $G(r) = \log(r)$ (we have dropped the $\frac{1}{2\pi}$ constant for simplicity). Here we show that its SBD converges exponentially fast:
\begin{The} 
	\label{theRadialQuadLaplaceErreur}
	There exist two positive constants $D_1$ and $D_2$ such that
	\[ \forall a \in (0,1), \forall P \in \N^*, \forall r \in (a,1), \quad \abs{G(r) - \sum_{p=1}^P \alpha_p e_p(r)} \leq D_1 e^{-D_2 a P} \]
	where $\alpha_1,\cdots,\alpha_P$ are the SBD coefficients of $G$ of order $P$.  
\end{The}

Following the ideas of the previous paragraph, we will show this by exhibiting an extension $\tilde{G}$ for which we are able to estimate the remainder of the Fourier-Bessel series. Observe that for all $s \in \N$:
\[(-\Delta)^s G \text{ vanishes on } \mathcal{C}.\]
This allows to choose $b=1$ and we will thus note $\mathcal{A}(a) \isdef \mathcal{A}(a,1)$. 

For any $n \in \N^{*}$, let us define extensions $\tilde{G}_n$ of $G$ for $r \leq a$, as
\[\tilde{G}_n(r) = \sum_{k=0}^{2n} \dfrac{a_{k,n}}{k!}(r-a)^k r^{2n},\]
where the coefficients $a_{k,n}$ are chosen so that $\tilde{G}_n$ has continuous derivatives up to the order $2n$:
\[a_{k,n} = {\dfrac{d^k}{dr^k}\left(\dfrac{\log(r)}{r^{2n}}\right)\bigg|}_{r=a}.\]
Notice that the $r^{2n}$ term ensures the boundedness of $(-\Delta)^n \tilde{G}_n$. We now go into some tedious computations to provide a crude bound for $\norm{(-\Delta)^n \tilde{G}_n}_{L^2(B)}$ in terms of the coefficients $a_{k,n}$.

\begin{Lem} 
	\label{LemmeDegueu}
	There exists a constant $C$ independent of $n$ and $a$ such that for $r<a$
	\begin{equation}
		\left|\Delta^n \tilde{G}_n(r)\right| \leq  C \left( \frac{16n}{e}\right)^{2n}\!\!\!\!\!\max_{k\in \left\{1,\cdots,2n\right\}}\left(\dfrac{|a_{k,n}|}{k!}a^k\right).
		\label{bigBadEq1Reduced}
	\end{equation}
	\label{LemAkDeltanf}
\end{Lem}

\begin{proof} For $r \leq a$, we have
	\[\Delta^n \tilde{G}_n(r) = \sum_{k=0}^{2n}\sum_{l=0}^k \dbinom{k}{l}\dfrac{a_{k,n}}{k!}(-a)^{k-l}(2n+l)^2 (2(n-1)+l)^2\times ... \times (2+l)^2 r^{l}.\]
	This result is obtained by expanding the sum in the definition of $\tilde{G}_n$ and using the fact that $\Delta r^k = k^2r^{k-2}$. Hence, using triangular inequality
	\[|(-\Delta)^n \tilde{G}_n(r)| \leq \sum_{k=0}^{2n}\sum_{l=0}^k \dbinom{k}{l}\dfrac{|a_{k,n}|}{k!}a^{k-l}(2n+l)^2(2(n-1)+l)^2\times ... \times (2+l)^2r^{l}.\]	
	For $l\in \{1,\cdots,2n\}$, we apply the following (crude) inequality:
	\begin{equation}
		(2n+l)^2(2(n-1)+l)^2\times ... \times (2+l)^2 \leq (4n)^2(4n-2)^2 \times ... \times (2n+2)^2
		\label{estimationTresGrossiere}
	\end{equation}
	to obtain: 
	\begin{equation*}
		\begin{split}
			|(-\Delta)^n \tilde{G}_n(r)| &\leq (4n)^2(4n-2)^2 \times ... \times (2n+2)^2\max_{k\in\llbracket 0,2n\rrbracket}\left(\dfrac{|a_{k,n}|}{k!}a^k\right)\sum_{k=0}^{2n}\sum_{l=0}^k \dbinom{k}{l}a^{-l}r^l\\
			&\leq (4n)^2(4n-2)^2 \times ... \times (2n+2)^2\max_{k\in\llbracket 0,2n\rrbracket}\left(\dfrac{|a_{k,n}|}{k!}a^k\right)\sum_{k=0}^{2n}\left(1+\frac{r}{a}\right)^k.		
		\end{split}
	\end{equation*}
	Since $r<a$, the last sum is bounded by $\displaystyle\sum_{k=0}^{2n}2^k = 2^{2n+1}-1 < 2^{2n+1}$,
	while 
	\[(4n)^2(4n-2)^2\times...\times (2n+2)^2 \sim 2\left(\dfrac{8n}{e}\right)^{2n}\]
	follows from Stirling formula. \qedhere
	%for large $n$, we get
	%	\[\dfrac{(2n)!}{(n)!} \sim \dfrac{\sqrt{2\pi\times 2n}}{\sqrt{2\pi n}} \dfrac{\left(\dfrac{2n}{e}\right)^{2n}}{\left(\dfrac{n}{e}\right)^{n}}.\]
	%	This leads to 
	%	\[\dfrac{(4n)!!}{(2n)!!} \sim \sqrt{2} \left(\dfrac{8n}{e}\right)^n \]
	%	which  
\end{proof}
We are now able to prove the following, which implies \autoref{theRadialQuadLaplaceErreur}.
\begin{The}
	There exists a constant $C$ such that, for any $P \in \N^*$ and $a \in (0,1)$, there exists a radial function $\tilde{G}$ which coincides with $G$ on $\mathcal{A}(a)$ satisfying:
	\[\norm{\tilde{G} - \sum_{p=1}^{P}c_p(\tilde{G})e_p}_{H^1_0(B)} \hspace{-0.7cm}\leq C \sqrt{P} \exp\left(-\frac{aP\pi}{32}\right).\]
	\begin{proof}
		Let $n \in \N^*$. We may compute the coefficients $a_{k,n}$ using Leibniz formula: 
		\begin{eqnarray*}						
			\dfrac{d^k }{dr^k}\left(r^{-2n}\log(r)\right) & = & \displaystyle\sum_{j=0}^k\dbinom{k}{j}\dfrac{d^j}{dx^j}\left(r^{-2n}\right)\dfrac{d^{k-j}}{dx^{k-j}}\left(\log(r)\right)           \\
			& = & \displaystyle\sum_{j=0}^{k-1} \dbinom{k}{j}(-1)^j \dfrac{(2n+j-1)!}{(2n-1)!}r^{-2n-j}(-1)^{k-j-1}\left(k-j-1\right)!r^{-k+j}       \\ 
			& &+ (-1)^k \dfrac{(2n+k-1)!}{(2n-1)!}r^{-2n-k}\log(r)                                                                                  \\
			& = & \dfrac{(-1)^k k!}{r^{2n+k}}  \left(-\displaystyle\sum_{j=0}^{k-1}\dbinom{2n+j-1}{j}\dfrac{1}{k-j}+\dbinom{2n+k-1}{k}\log(r)\right). \\
		\end{eqnarray*}
		This leads to \[\dfrac{|a_{k,n}|}{k!}a^k \leq a^{-2n} \dbinom{2n + k -1}{k}\left(\frac{k}{2n}-\log(a)\right),\]
		where we used the identity
		\begin{equation*}
			\sum_{j=0}^{k-1}\dbinom{j+2n-1}{j} = \dfrac{k}{2n}\dbinom{k+2n-1}{k}.
		\end{equation*}
		Observe that
		\begin{equation*}
			\dbinom{2n+k-1}{k}\leq \dbinom{4n-1}{2n} = \frac{1}{2}\dbinom{4n}{2n} \leq \dfrac{4^{2n}}{2\sqrt{2\pi n}} \quad k \in \{1,\cdots,2n\},
		\end{equation*}
		and thus,
		\begin{equation}
			\max_{0\leq k \leq 2n}\left(\dfrac{|a_{k,n}|}{k!}a^k\right) \leq \left(\frac{4}{a}\right)^{2n}\dfrac{1}{2\sqrt{2\pi n}}\left(\log\left(\frac{e}{a}\right)\right).
			\label{majorAkLog} 
		\end{equation}							
		Combining (\ref{majorAkLog}) with estimation (\ref{bigBadEq1Reduced}), we find that there exists a constant $C$ such that, for $r<a$
		\[|(-\Delta)^n \tilde{G}_n (r)|\leq \dfrac{C}{\sqrt{n}}\left( \frac{16n}{e}\right)^{2n}\left(\frac{4}{a}\right)^{2n}\log\left(\dfrac{e}{a}\right).\]
		Therefore, integrating on $B(0,a)$, we get
		\[ \norm{(-\Delta)^n \tilde{G}_n}_{L^2(B(0,a))} \leq \dfrac{C a^2}{\sqrt{n}}\log\left(\frac{e}{a}\right)\left( \frac{64n}{ae}\right)^{2n},\]
		and since 
		\[(-\Delta)^n \tilde{G}_n(x) = (-\Delta)^n G(x) = 0\]
		for $|x|>a$, the same bound applies to $\norm{(-\Delta)^n \tilde{G}_n(x)}_{L^2(B)}$. 
		We now plug this estimate into the inequality of corollary \ref{EstimationRest}, to get
		\[ \norm{\tilde{G}_n - \sum_{p=1}^{P}c_p(\tilde{G}_n)e_p}_{H^1_0(B)} \!\!\!\!\!\!\!\!\!\!\leq~~ C \dfrac{P^\frac{3}{2}}{n} a^2 \log\left(\dfrac{e}{a}\right)\left( \frac{64 n}{ae P \pi}\right)^{2n}.\] 
		The previous inequality holds true for any integer $n$ such that $n>1$ and any $P \in \mathbb{N}$. Without loss of generality, one can assume that $\frac{aP\pi}{64} >1$. In this case, let $n_P = \lfloor \frac{aP\pi}{64}\rfloor $, and $\tilde{G} = \tilde{G}_{n_P}$. Using the fact that $x\mapsto x \log\left(\dfrac{e}{x}\right)$ is bounded on $(0,1]$, we get 
		\[ \norm{\tilde{G} - \sum_{p=1}^{P}c_p(\tilde{G})e_p}_{H^1_0(B)} \leq C \sqrt{P} e^{-\frac{aP\pi}{32}}. 	\qedhere\]
	\end{proof}
\end{The}

\subsection{Numerical results}

\section{Circular quadrature}
\label{sec:circular}
In this section, we study an approximation of the form
\[ J_0(\rho_p|x|) \approx \dfrac{1}{M_p}\sum_{m=0}^{M_p-1}e^{i \rho_p \xi^p_m \cdot x}, \]
for some integer $M_p$ and some quadrature points $(\xi_m^p)_{1 \leq m \leq M_p}$. 

\subsection{Theoretical bound}
\begin{The} There exists a constant $C$ such that for any $r>0$, $M\in \N^*$, $\varphi \in \R$ 
	\[\left|J_0(r) -  \dfrac{1}{M}\sum_{m=0}^{M-1}e^{ir\sin\left(\frac{2m\pi}{M}-\varphi\right)} \right| \leq C \left(\dfrac{er}{2M}\right)^M\]
	\label{QuadratureCirc}
\end{The}
\noindent In order to prove this proposition, we first prove a result on Fourier series
\begin{Lem} For any $\mathcal{C}^2$ function $f$ defined on $\mathbb{R}$ and complex-valued, that is $2\pi-$periodic, one has \[\dfrac{1}{2\pi}\int_{0}^{2\pi}f - \dfrac{1}{M}\sum_{m=0}^{M-1}f\left(\frac{2m\pi}{M} \right) = - \sum\limits_{k \in \Z^*}c_{kM}(f),\]
	where $c_n(f)$ denotes the Fourier coefficient of $f$ defined as \[c_n(f) = \dfrac{1}{2\pi}\int_{0}^{2\pi}f(x)e^{-inx}dx\]
	\begin{proof}
		Since $f$ is $\mathcal{C}^2$, it is equal to its Fourier Series, which converges normally: \[\forall x \in \mathbb{R}, f(x) = \sum_{k\in\Z} c_k(f)e^{ikx}\] Using this expression, we obtain \[\dfrac{1}{M}\sum_{m=0}^{M-1}f\left(\frac{2m\pi}{M}\right) = \sum\limits_{k\in \Z^*}c_k(f)\left(\frac{1}{M}\sum_{m=0}^{M-1}e^{ik\frac{2m\pi}{M}}\right)\] Now observe that if $k\notin M\Z$, \[\dfrac{1}{M}\sum_{m=0}^{M-1}e^{ik\frac{2m\pi}{M}} = 0\] and if $k\in M\Z$ then \[\dfrac{1}{M}\sum_{m=0}^{M-1}e^{ik\frac{2m\pi}{M}} = 1\] Therefore \[\int_{0}^{2\pi}f(x)dx - \dfrac{1}{M}\sum_{m=0}^{M-1}f\left(\frac{2m\pi}{M} \right) = c_0(f) - \sum\limits_{k \in M\Z}c_{k}(f) = - \sum\limits_{k \in \Z^*}c_{kM}(f)\qedhere\]
	\end{proof}
\end{Lem}
\noindent Let us now prove the proposition: 
\begin{proof}
	The result is based on the fact that 
	\[J_0(r) =  \int_0^{2\pi} e^{ir\sin(x)}dx = \int_0^{2\pi} e^{ir\sin(x - \varphi)}dx.\] 
	Let $f : x \mapsto e^{ir\sin(x - \varphi)}$. Let us recall the integral representation of the Bessel function of the first kind and of order $k$ where $k$ is a relative integer: \[J_k(r) =  \int_{0}^{2\pi}e^{ir\sin(x)}e^{-ikx}dx =  e^{-ik\varphi}\int_{0}^{2\pi}e^{ir\sin(x - \varphi)}e^{-ikx}dx.\] Thus, one has $c_k(f) = e^{ik\varphi}J_k(r)$. Consequently, the former Lemma yields 
	\[J_0(r) -  \dfrac{1}{M} \sum_{j=0}^{M-1} e^{ir \sin \left( \frac{2j\pi}{M}-\varphi \right)} = -\sum_{k\in \Z^*}e^{iNk\varphi}J_{Nk}(r).\] 
	For large $\abs{k}$,
	\[J_k(r) \sim \left(\dfrac{er}{2\abs{k}}\right)^{\abs{k}}.\]
	Therefore, there exists a constant $C'$ such that: 
	\begin{eqnarray*}
		\abs{J_0(r) -  \dfrac{1}{M}\sum_{m=0}^{M-1}e^{ir\sin\left(\frac{2m\pi}{M}-\varphi\right)}} &\leq& C' \sum_{k\in \Z^*} \left(\dfrac{er}{2M|k|}\right)^{M|k|}\\
		&\leq& 2 C' \left(\dfrac{er}{2M}\right)^M
	\end{eqnarray*}
	As announced.\qedhere	
\end{proof}
We conclude with the following result
\begin{Prop} Let $\varepsilon >0$, $r>0$, and assume $M > \dfrac{e}{2}r + \log\left(\dfrac{C}{\varepsilon}\right)$. Then 
	\[\left|J_0(r) -  \dfrac{1}{M}\sum_{m=0}^{M-1}e^{ir\sin\left(\frac{2m\pi}{M}-\varphi\right)} \right| \leq \varepsilon \]
	\label{suboptCirc}
	\begin{proof}
		This result is a direct consequence of the previous proposition together with the following inequality: for any $(A,B) \in \left(\mathbb{R}_+^*\right)^2$ one has
		\[ \left( \dfrac{A}{A+B}\right)^{A+B} \leq e^{-B}\]
		To prove it, we take the logarithm of this quantity, $f(A,B) = -B\left(1+\dfrac{A}{B}\right)\log\left(1+\dfrac{B}{A}\right)$ and observe that for any positive $x$, \[\left(1+\dfrac{1}{x}\right)\log(1+x) \geq 1.\qedhere\]
	\end{proof}
\end{Prop}

Hence, we can approximate very efficiently the functions $e_p$ of the previous paragraph as a finite sum as follows. We define the quadrature points $\xi_0^p, \xi_1^p, ..., \xi^p_{M_p-1}$  by
\begin{equation}
	\label{defXimp}
	\xi_m^p := \displaystyle e^{i\frac{2\pi m}{M_p}} \quad \text{for any } 1\leq p \leq P  \text{ and } 0 \leq m \leq M_p -1
\end{equation}
With this definition, for any $x \in \mathbb{R}^2$
\[ e_p(|x|) = C_p J_0(\rho_p |x|)\approx \dfrac{C_p}{M_p}\sum_{m=0}^{M_p-1}{e^{i \rho_px \cdot \xi_m^p}}\]
Where the approximation is valid at a precision $\varepsilon$ as soon as $M > \frac{e}{2}\rho_p|x| + \log\left(\dfrac{C}{\varepsilon}\right)$.



%% BIBLIO
																																							
\IfFileExists{biblio.bib}{\bibliography{biblio}}{\bibliography{/home/martin/Documents/These/Biblio/biblio}}
\bibliographystyle{plain}
																																									
\end{document} 