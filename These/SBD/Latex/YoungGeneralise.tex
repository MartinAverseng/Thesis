\documentclass[11pt,a4paper]{article}
\usepackage[utf8]{inputenc}
\usepackage[english]{babel}
\usepackage{amsmath}
\usepackage{bbm}
\usepackage{amsthm}
\usepackage{amsfonts}
\usepackage{amssymb}
\usepackage{graphicx}
\usepackage{lmodern}
\usepackage[left=2cm,right=2cm,top=2cm,bottom=2cm]{geometry}
\author{Martin}
\title{Young Généralisé}
\begin{document}
\maketitle
\theoremstyle{plain}
\newtheorem{The}{Theorem}[section]
\newtheorem{Prop}{Proposition}[section]
\newtheorem{Lem}{Lemma}[section]
\theoremstyle{definition}
\newtheorem{Def}{Definition}[section]
\newcommand{\enstq}[2]{\left\{#1\mathrel{}\middle|\mathrel{}#2\right\}}
\newcommand{\Lp}[2]{L^#1(#2)}
\newcommand{\Sob}[3]{W^{#1,#2}(#3)}
\newcommand{\RN}[0]{\mathbb{R}^N}
\newcommand{\norm}[1]{\left\|#1\right\|}
\newcommand{\sinc}[0]{\textup{sinc}}
\newcommand{\N}[0]{\mathbb{N}}
\newcommand{\R}[0]{\mathbb{R}}
\newcommand{\Z}[0]{\mathbb{Z}}

Soient $\Sigma_x$ et $\Sigma_y$ deux ensembles finis de points inclus dans $\RN$. On se donne deux fonctions $f$ et $g$ qui sont définies sur $\RN$
On note \[\norm{f}_p^{\Sigma_x} =\left( \sum_{x \in \Sigma_x} f(x)^p\right)^{1/p}\]
\[f \ast^{\Sigma_y} g (x) = \sum_{y\in\Sigma_y} f(x-y) g(y)\]

\end{document}

