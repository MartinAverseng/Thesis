\documentclass[11pt,a4paper]{article}

\usepackage{adjustbox}
\usepackage{algorithm}
\usepackage{algorithmic}
\usepackage{amsmath}
\usepackage{amssymb}
\usepackage{amsthm}
\usepackage{amsfonts}
\usepackage{afterpage}
\usepackage{blindtext}
\usepackage[font=footnotesize,labelfont=bf]{caption}
\usepackage{hyperref}
\usepackage[english]{babel}
\usepackage{bbm}
\usepackage{bigints}
\usepackage{bm}
\usepackage{cite}
\usepackage{color}
\usepackage{float}
\usepackage[left=2cm,right=2cm,top=2cm,bottom=2cm]{geometry}
\usepackage{graphicx}
\usepackage[utf8]{inputenc}
\usepackage{mathtools}
\usepackage{mdframed}
\usepackage{pgfplots} 
\usepackage{subfigure}
\usepackage{stmaryrd}
\usepackage{textcomp}
\usepackage{tikz}
\usepackage{url}
\renewcommand{\proofname}{Proof}
\theoremstyle{plain}
\newtheorem{monTheoNumrote}{Théorème}[section] % Environnement numéroté en fonction de la section
\newtheorem*{monTheoNonNumerote}{Théorème}  % Environnement non numéroté
\newtheorem{The}{Theorem}[section]
\newtheorem*{The*}{Theorem}
\newtheorem{Prop}{Proposition}[section]
\newtheorem*{Prop*}{Proposition} 
\newtheorem{Cor}{Corollary}[section]
\newtheorem*{Cor*}{Corollary}
\newtheorem{Conj}{Conjecture}[section]
\newtheorem{Lem}{Lemma}[section]
\renewcommand{\qed}{\unskip\nobreak\quad\qedsymbol}%
\numberwithin{equation}{section} % Numérote les équations section.numéro.
\theoremstyle{definition}
\newtheorem{Def}{Definition}[section]
\newtheorem{Rem}{Remark}[section]
\newtheorem*{Rem*}{Remark}
\newtheorem*{Lem*}{Lemma}
\newtheorem{Que}{Question}
\newcommand{\enstq}[2]{\left\{#1\mathrel{}\middle|\mathrel{}#2\right\}}
\newcommand{\Lp}[2]{L^#1(#2)}
\newcommand{\Sob}[3]{W^{#1,#2}(#3)}
\newcommand{\Rd}[0]{\mathbb{R}^d}
\newcommand{\RN}[0]{\mathbb{R}^N}
\newcommand{\Rn}[0]{\mathbb{R}^n}
\newcommand{\norm}[1]{\left\|#1\right\|}
\newcommand{\sinc}[0]{\textup{sinc}}
\newcommand{\functionDef}[5]{\begin{array}{lllll}
#1 & : & #2 & \longrightarrow & #3 \\
 & & #4 & \longmapsto &\displaystyle #5 \\
\end{array}}
\newcommand{\Theautorefname}{Theorem}
\newcommand{\Propautorefname}{Proposition}
\newcommand{\Corautorefname}{Corollary}
\newcommand{\Lemautorefname}{Lemma}
\newcommand{\Defautorefname}{Definition}
\newcommand{\N}{\mathbb{N}}
\newcommand{\Z}{\mathbb{Z}}
\newcommand{\D}{\mathbb{D}}
\newcommand{\R}{\mathbb{R}}
\newcommand{\A}{\mathcal{A}_{a,b}}
\newcommand{\Crad}{C^\infty_{c,rad}(B)}
\newcommand{\Lrad}{L^2_{rad}(B)}
\newcommand{\Lradab}{L^2_{rad}(\mathcal{A}_{a,b})}
\newcommand{\duality}[2]{\left\langle #1,#2\right\rangle}
\newcommand{\Hrad}{H^1_{rad}(B)}
\newcommand{\Hzrad}{H^1_{0,rad}(B)}
\newcommand{\rmin}{\delta_{\min}}
\newcommand{\rmax}{\delta_{\max}}
\newcommand{\corr}{\gamma}
\newcommand{\question}[1]{\begin{Que} \ 
#1
\end{Que}}
\newcommand{\abs}[1]{\left\lvert #1 \right\rvert}
\newcommand{\CL}[2]{\textup{CL}\left(\enstq{#1}{#2}\right)}
\newcommand{\Script}[1]{`\texttt{#1}`}
\newcommand{\espace}{\text{ }\qquad} 
\newcommand{\loc}{\text{loc}}
\newcommand{\SL}{\textup{SL}\hspace{1.5pt}}
\newcommand{\DL}{\textup{DL}\hspace{1.5pt}}
\newcommand{\fp}{\underset{\varepsilon \to 0}{\textup{f.p.}}}
\newcommand{\scalProd}[2]{\left(#1|#2\right)}
\newcommand{\toDo}[1]{{\color{red}#1}}
\newcommand{\bs}[1]{\boldsymbol{#1}}
\newcommand{\varInRange}[4]{(#1_{#2})_{#3 \leq #2 \leq #4}}
\newcommand{\from}{\colon}
\newcommand{\Cinf}{C^{\infty}}
\newcommand{\isdef}{\mathrel{\mathop:}=}
\newcommand{\defis}{=\mathrel{\mathop:}}

\renewcommand{\algorithmicrequire}{\textbf{Inputs:}}
\renewcommand{\algorithmicensure}{\textbf{Outputs:}}

\pgfplotsset{compat=1.13}
\author{Martin AVERSENG}
\title{Link between Sobolev norms after cosine change}
\begin{document}	
	\maketitle
	\section{Notations and preliminary results}
	In this note, when $u$ refers to a function defined on $\Gamma_x = (-1,1)$, we will denote by $\alpha$ the function such that 
	\begin{equation}
	u = \dfrac{\alpha}{\omega}, \quad \omega(x) = \sqrt{1-x^2},
	\end{equation}
	and by $\tilde{\alpha}(\theta) = \alpha(\cos\theta)$. 
	Let $\Gamma_\theta = (0,\pi)$. Let $H^s$ stand for the usual Sobolev space of order $s$. For $\Gamma \subset \tilde{\Gamma}$, where $\tilde{\Gamma}$ is any closed Lipschitz curve, recall that $\tilde{H}^s(\Gamma)$ is defined by  
	\[\tilde{H}^s(\Gamma) = \enstq{u \in H^s(\tilde{\Gamma})}{\tilde{u} \in H^s(\tilde{\Gamma})}, \tilde{u}(x) = \left\{
		\begin{array}{cl}
			u(x) &  x \in \Gamma\\
			0 & x \in \tilde{\Gamma}\setminus\overline{\Gamma}
		\end{array} \right.\] 
	We denote by $\norm{\cdot}_s$ the $H^s$ norm. Also, recall that for integer $s$, the norms $\tilde{H}^s$ and $H^s$ are equivalent. Given that functions in $H^1(\mathbb{R})$ are continuous, the elements of $\tilde{H}^1(-1,1)$ must vanish at $x=-1,1$, so we have simply 
	\[ \tilde{H}^1(\Gamma_x) = H_0^1(\Gamma_x)\]
	with equivalent norms. 
	
	We first state the following lemma which is a particular case of a weighted Hardy inequality (see for example the introduction of \cite{edmunds2005weighted}). 
	\begin{Lem}
		\label{HardyWeighted}
		Let $\alpha \in C^{\infty}_0(\Gamma_x)$. There holds 
		\[ \int_{\Gamma_x} \frac{\alpha^2(x)}{\omega^3(x)} \leq \int_{\Gamma_x} \alpha'^2(x)\omega(x)dx\]		
	\end{Lem}
	\begin{Rem}
		Observe that after cosine change of variable, this result is equivalent to 
		\[ \int_{0}^{\pi} \frac{\tilde{\alpha}^2(\theta) }{\sin^2\theta}d\theta \leq \int_{0}^\pi \tilde{\alpha}'^2(\theta)d\theta,\]
		which, taking into account $\sin\theta \underset{\theta\to 0}{\sim}\theta$, is under the form of a classical Hardy inequality. 
	\end{Rem}
	
	We also introduce $S$ and $N$ the usual single layer operator and hypersingular operator on $\Gamma_x$.   The kernel of $S$ is chosen so that it is positive definite and bounded below on  $\tilde{H}^{-\frac{1}{2}}(\Gamma_x)$. For example, one can choose 
	\begin{equation}
		Su(x) = -\frac{1}{2\pi}\int_{-1}^{1} \ln |x-y|u(y)dy,\quad x \in \Gamma
		\label{SimpleCouche_x}
	\end{equation}
	In this case, we have
	\begin{Lem}
		\text{ }
		\begin{enumerate}
			\item For any $u \in \tilde{H}^{-\frac{1}{2}}(\Gamma_x)$, one has 
			\[ \norm{u}_{\tilde{H}^{-\frac{1}{2}}(\Gamma_x)} \sim \sqrt{\duality{Su}{u}}_{H^{\frac{1}{2}}(\Gamma_x),\tilde{H}^{-\frac{1}{2}}(\Gamma_x)}\]		
			\item For any $u \in \tilde{H}^{\frac{1}{2}}(\Gamma_x)$, one has
			\[\norm{u}_{\tilde{H}^\frac{1}{2}(\Gamma_x)} \sim \sqrt{\duality{Nu}{u}}_{H^{-\frac{1}{2}}(\Gamma_x),\tilde{H}^{\frac{1}{2}}(\Gamma_x)}\]
		\end{enumerate}		
	\end{Lem}
	By $a \sim b$, we imply that there exist two constants $c$ and $C$ such that $ca \leq b \leq Ca$. 
	
	As was shown in \cite{yan1988integral}, we have the following result: (the proof will be reproduced here for convenience)
	\begin{Prop} \label{FormuleSImpleCoucheConvol}
		Let $x = \cos\theta$, we have the identity
		\begin{equation}
			Su(x) = -\frac{1}{2\pi}\int_{-\pi}^{\pi} \ln\left|\sqrt{2}\sin \frac{\theta - \theta'}{2}\right| \tilde{\alpha}(\theta')d\theta'
		\end{equation}
		\begin{proof}
			In (\ref{SimpleCouche_x}), do the variable change 
			\begin{equation}
			x = \cos\theta, y = \cos\theta', \frac{-dy}{\omega} = d\theta'
			\end{equation}
			leading to 
			\begin{equation*}
				Su(x) = -\frac{1}{2\pi}\int_{0}^\pi \ln\abs{\cos\theta - \cos\theta'} \tilde{\alpha}(\theta)d\theta
			\end{equation*}
			The result is then obtained with the help of the formula $\cos\theta - \cos\theta' = -2\sin\frac{\theta - \theta'}{2} \sin\frac{\theta + \theta'}{2}$. 
		\end{proof}
	\end{Prop}
	\begin{Rem}
		This shows that, in the variable $\theta$, the single layer potential is actually a convolution by the kernel $A(\theta) = -\frac{1}{2\pi}\ln \abs{\sqrt{2}\sin\frac{\theta}{2}}$. 
	\end{Rem}
	
	We easily deduce the following formula	
	\begin{Lem} For smooth $u$ and $v$ in $C^{\infty}_0(\Gamma_x)$, 
		\begin{equation}
			\duality{Su}{v} = \frac{-1}{4\pi} \int_{-\pi}^{\pi}\int_{-\pi}^{\pi} \ln\abs{\sqrt{2}\sin\frac{\theta - \theta'}{2}}\tilde{\alpha}(\theta)\tilde{\beta}(\theta')d\theta d\theta'
		\end{equation}
		
	\end{Lem}
	
	\section{Norm estimates}
	
	\begin{The}
		We have the following four estimates : 
		\begin{enumerate}
			\item[(i)] $\norm{u}_{\tilde{H}^{-\frac{1}{2}}(\Gamma_x)}  \leq C \norm{\tilde{\alpha}}_{-1/2}$ 
			\item[(ii)] $\norm{\sqrt{\omega}u}_{L^2(\Gamma)} \leq C \norm{\tilde{\alpha}}_{0}$
			\item[(iii)] $\norm{\omega u}_{\tilde{H}^{\frac{1}{2}}(\Gamma_x)} \leq C \norm{\tilde{\alpha}}_{1/2}$
			\item[(iv)] $\norm{\omega^{\frac{3}{2}} u}_{1} \leq C \norm{\tilde{\alpha}}_{1}$
		\end{enumerate}
	\end{The}
	\begin{proof}
		\textbf{Proof of (ii):}\\
			By the change of variables $t=\cos\theta$, we can see that
				\[ \int_{-1}^{1} \frac{\alpha^2(x)}{\omega(x)}dx = \int_{0}^\pi \tilde{\alpha}^2(\theta) d\theta\]
		\textbf{Proof of (iv):} \\
			The same change of variables also yields
				\[ \int_{-1}^{1} \omega \alpha'(x)^2dx = \int_{0}^\pi \tilde{\alpha'}^2(\theta) d\theta\] 
			Morever, observe that 
			\begin{align*}
				\left(\omega^{\frac{3}{2}}u\right)' &= \left(\sqrt{\omega}\alpha\right)'\\
				&=-\frac{x\alpha}{2\omega^{\frac{3}{2}}} + \alpha' \sqrt{\omega}. 
			\end{align*}		
			The second term has its $L^2$ norm controlled by the $H^1$ norm of $\tilde{\alpha}$. It remains to show that this also holds for the first one that is, 
				\[ \int_{-1}^1 \frac{\alpha^2}{\omega^3} \leq C \norm{\tilde{\alpha}}_{1},\]
			which is a simple consequence of Lemma \ref{HardyWeighted}\\
		\textbf{Proof of (i):}\\
			Since $\tilde{\alpha}$ can be extended as an even $2\pi-periodic$ function, its Sobolev norm of order $s$ can be expressed as 
			\begin{equation*}
				\norm{\tilde{\alpha}}_{s}^2 = \abs{\alpha_0}^2 + \sum_{n=1}^{+\infty} \abs{\alpha_n}^2 n^{2s}, 
			\end{equation*}				
			where $\alpha_k$ are the usual Fourier coefficients. Simple calculations show that 
			\begin{equation*}
				\norm{\tilde{\alpha}}_{-\frac{1}{2}} = \int_{-\pi}^{\pi}\int_{-\pi}^{\pi}\tilde{\alpha}(\theta)\tilde{\alpha}(\theta') \left(\frac{1}{2} + \sum_{n=1}^{\infty} \frac{\cos \left(n \left(\theta-\theta'\right)\right)}{n}\right)
			\end{equation*}					
			We aim to compute the function $G$ in parenthesis:
				\[G(\theta) = \frac{1}{2} + \sum_{n=1}^{+\infty} \frac{\cos n\theta}{n}\]
			To achieve this, we will need the following well-known property of Chebyshev's polynomials $T_n(x)$.
			\begin{Lem} For any $t \in (-1,1)$, 
				\[\sum_{n=0}^{+\infty} t^n T_n(x) = \frac{1-tx}{1-2tx + t^2}\]
			\end{Lem}
			Integrating in $t$ and taking the value at $t=1$ leads to the following identity: 
			\begin{equation*}
				\sum_{n=1}^{+\infty} \frac{T_n(x)}{n} = -\ln\sqrt{2 - 2x}
			\end{equation*}		
			Therefore, taking $x = \cos\theta$, we find:
			\begin{align*}
				G(\theta) &= \frac{1}{2} - \ln\sqrt{2 - 2\cos\theta}\\
				&= -\ln \left[2e^{-1/2}\sin\frac{\abs{\theta}}{2}\right]
			\end{align*}
			By Proposition \ref{FormuleSImpleCoucheConvol}, we see that there exists a constant $C$ such that 
			\begin{equation}
				\norm{\tilde{\alpha}}_{-1/2} = C \abs{\alpha_0}^2 + \sqrt{\duality{Su}{u}}
			\end{equation}
			which implies the first inequality. 
	\end{proof}
	
	\bibliographystyle{plain}
	\bibliography{../Biblio/biblio}
	
\end{document}
