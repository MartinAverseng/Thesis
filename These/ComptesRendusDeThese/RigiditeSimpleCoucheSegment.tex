\documentclass[11pt,a4paper]{article}

\usepackage{adjustbox}
\usepackage{algorithm}
\usepackage{algorithmic}
\usepackage{amsmath}
\usepackage{amssymb}
\usepackage{amsthm}
\usepackage{amsfonts}
\usepackage{afterpage}
\usepackage{blindtext}
\usepackage[font=footnotesize,labelfont=bf]{caption}
\usepackage{hyperref}
\usepackage[english]{babel}
\usepackage{bbm}
\usepackage{bigints}
\usepackage{bm}
\usepackage{cite}
\usepackage{color}
\usepackage{float}
\usepackage[left=2cm,right=2cm,top=2cm,bottom=2cm]{geometry}
\usepackage{graphicx}
\usepackage[utf8]{inputenc}
\usepackage{mathtools}
\usepackage{mdframed}
\usepackage{pgfplots} 
\usepackage{subfigure}
\usepackage{stmaryrd}
\usepackage{textcomp}
\usepackage{tikz}
\usepackage{url}
\renewcommand{\proofname}{Proof}
\theoremstyle{plain}
\newtheorem{monTheoNumrote}{Théorème}[section] % Environnement numéroté en fonction de la section
\newtheorem*{monTheoNonNumerote}{Théorème}  % Environnement non numéroté
\newtheorem{The}{Theorem}[section]
\newtheorem*{The*}{Theorem}
\newtheorem{Prop}{Proposition}[section]
\newtheorem*{Prop*}{Proposition} 
\newtheorem{Cor}{Corollary}[section]
\newtheorem*{Cor*}{Corollary}
\newtheorem{Conj}{Conjecture}[section]
\newtheorem{Lem}{Lemma}[section]
\renewcommand{\qed}{\unskip\nobreak\quad\qedsymbol}%
\numberwithin{equation}{section} % Numérote les équations section.numéro.
\theoremstyle{definition}
\newtheorem{Def}{Definition}[section]
\newtheorem{Rem}{Remark}[section]
\newtheorem*{Rem*}{Remark}
\newtheorem*{Lem*}{Lemma}
\newtheorem{Que}{Question}
\newcommand{\enstq}[2]{\left\{#1\mathrel{}\middle|\mathrel{}#2\right\}}
\newcommand{\Lp}[2]{L^#1(#2)}
\newcommand{\Sob}[3]{W^{#1,#2}(#3)}
\newcommand{\Rd}[0]{\mathbb{R}^d}
\newcommand{\RN}[0]{\mathbb{R}^N}
\newcommand{\Rn}[0]{\mathbb{R}^n}
\newcommand{\norm}[1]{\left\|#1\right\|}
\newcommand{\sinc}[0]{\textup{sinc}}
\newcommand{\functionDef}[5]{\begin{array}{lllll}
#1 & : & #2 & \longrightarrow & #3 \\
 & & #4 & \longmapsto &\displaystyle #5 \\
\end{array}}
\newcommand{\Theautorefname}{Theorem}
\newcommand{\Propautorefname}{Proposition}
\newcommand{\Corautorefname}{Corollary}
\newcommand{\Lemautorefname}{Lemma}
\newcommand{\Defautorefname}{Definition}
\newcommand{\N}{\mathbb{N}}
\newcommand{\Z}{\mathbb{Z}}
\newcommand{\D}{\mathbb{D}}
\newcommand{\R}{\mathbb{R}}
\newcommand{\A}{\mathcal{A}_{a,b}}
\newcommand{\Crad}{C^\infty_{c,rad}(B)}
\newcommand{\Lrad}{L^2_{rad}(B)}
\newcommand{\Lradab}{L^2_{rad}(\mathcal{A}_{a,b})}
\newcommand{\duality}[2]{\left\langle #1,#2\right\rangle}
\newcommand{\Hrad}{H^1_{rad}(B)}
\newcommand{\Hzrad}{H^1_{0,rad}(B)}
\newcommand{\rmin}{\delta_{\min}}
\newcommand{\rmax}{\delta_{\max}}
\newcommand{\corr}{\gamma}
\newcommand{\question}[1]{\begin{Que} \ 
#1
\end{Que}}
\newcommand{\abs}[1]{\left\lvert #1 \right\rvert}
\newcommand{\CL}[2]{\textup{CL}\left(\enstq{#1}{#2}\right)}
\newcommand{\Script}[1]{`\texttt{#1}`}
\newcommand{\espace}{\text{ }\qquad} 
\newcommand{\loc}{\text{loc}}
\newcommand{\SL}{\textup{SL}\hspace{1.5pt}}
\newcommand{\DL}{\textup{DL}\hspace{1.5pt}}
\newcommand{\fp}{\underset{\varepsilon \to 0}{\textup{f.p.}}}
\newcommand{\scalProd}[2]{\left(#1|#2\right)}
\newcommand{\toDo}[1]{{\color{red}#1}}
\newcommand{\bs}[1]{\boldsymbol{#1}}
\newcommand{\varInRange}[4]{(#1_{#2})_{#3 \leq #2 \leq #4}}
\newcommand{\from}{\colon}
\newcommand{\Cinf}{C^{\infty}}
\newcommand{\isdef}{\mathrel{\mathop:}=}
\newcommand{\defis}{=\mathrel{\mathop:}}

\renewcommand{\algorithmicrequire}{\textbf{Inputs:}}
\renewcommand{\algorithmicensure}{\textbf{Outputs:}}

\pgfplotsset{compat=1.13}
\author{Martin AVERSENG}
\title{Matrice de rigidité du simple couche sur le segment $(-1,1)$ avec les points de Tchebitchev}
\begin{document}
	\maketitle
	On se propose dans cette note de donner une valeur approchée calculable numériquement de l'intégrale double
	\[I_{ij} = \int_{x_i}^{x_{i+1}} \int_{x_j}^{x_{j+1}} \dfrac{\ln(|x - x'|)}{\sqrt{1-x^2}\sqrt{1- x'^2}}dx dx' \]
	Où les points $x_0,...,x_{N-1}$ sont les points de Tchebitchev, définit par $x_i = \cos\left(i\frac{\pi}{N-1}\right)$
	Dans la suite, on pose $\Delta = \pi/(N-1)$, $\theta_i = i \Delta$, $i = 0.. N-1$ $\phi_i = \frac{\theta_i + \theta_{i+1}}{2}$. 
	
	On commence par le changement de variable $x = \cos{\theta}$, $x' = \cos\theta'$, d'où la transformation 
	\[I_{ij} = \int_{\theta_i}^{\theta_{i+1}} \int_{\theta_j}^{\theta_{j+1}} \ln(|\cos \theta - \cos \theta'|) d\theta d\theta'\]
	On pose $f(\theta) = \int_{\theta_j}^{\theta{j+1}} \ln(|\cos \theta - \cos \theta'|) d\theta'$, et on fait l'approximation suivante : 
	\[ I_{ij} \approx \Delta f(\phi_i).\]
	Nous sommes donc ramenés à calculer $I _i = f(\phi_i)$. 
	En utilisant l'identité $\cos p - \cos q = -2 \sin \frac{p+q}{2} \sin \frac{p-q}{2}$, on obtient 
	\[I_i = \int_{\theta_i}^{\theta_{i+1}}\ln\left(2 \sin \frac{|\phi_i - \theta'|}{2}\right) + \int_{\theta_i}^{\theta_{i+1}}\ln\left(\sin\dfrac{\phi_i+\theta'}{2}\right)d\theta'  \]
	On fait l'approximation $\sin u \approx u$ pour les petits arguments, en l'occurrence $|\phi_i - \theta|$ ici. On trouve donc : 
	\[I_i \approx \int_{\theta_i}^{\theta_{i+1}}\ln\left(|\theta'-\phi_i|\right) + \int_{\theta_i}^{\theta_{i+1}}\ln\left(\sin\dfrac{\phi_i+\theta'}{2}\right)d\theta' \]
	Pour la deuxième intégrale, on utilise une formule d'intégration à un point : 
	\[ \int_{\theta_i}^{\theta_{i+1}}\ln\left(\sin\dfrac{\phi_i+\theta'}{2}\right)d\theta' \approx \Delta \ln(\sin \phi_i) \]
	La première intégrale se calcule exactement : 
	\[\int_{\theta_i}^{\theta_{i+1}}\ln\left(|\theta'-\phi_i|\right) = 2\int_{0}^{\Delta /2} \ln(u)du = \Delta\ln\left(\frac{\Delta}{2}\right)- \Delta\]
	On obtient donc in fine 
	\[ I_{ii} \approx \Delta^2 \left( \ln \left( \frac{\Delta}{2}\sin \phi_i\right) - 1\right)\]
	Ce calcul est faux quand $i$ ou $j$ s'approchent de $0$ ou $\pi$.
	\section{Meilleure approche}
	On décompose l'intégrale en quatre parties distinctes : 
	\[ I_{ij} = A_{ij} + B_{ij} + C_{ij} + D_{ij}, \]
	Où on pose 
	\[A_{ij} = \int_{\theta_i}^{\theta_{i+1}}\int_{\theta_j}^{\theta_{j+1}} \ln|\theta - \theta'|d\theta d\theta'\]
	\[B_{ij} = \int_{\theta_i}^{\theta_{i+1}}\int_{\theta_j}^{\theta_{j+1}} \ln \frac{\theta + \theta'}{2} d\theta d\theta'\]
	\[C_{ij} = \int_{\theta_i}^{\theta_{i+1}}\int_{\theta_j}^{\theta_{j+1}} \ln \left(\pi - \frac{\theta + \theta'}{2}\right)d\theta d\theta'\]
	\[D_{ij} = \int_{\theta_i}^{\theta_{i+1}}\int_{\theta_j}^{\theta_{j+1}} \ln \left(\frac{\sin\frac{\theta + \theta'}{2}}{\frac{\theta + \theta'}{2} \left(\pi-\frac{\theta + \theta'}{2}\right)}\sinc \frac{\theta-\theta'}{2}\right)d\theta d\theta'\]
	
	Les trois premiers termes de la décomposition isolent les différentes singularités de l'intégrande et le quatrième est très régulier. 
	
	\subsection*{Valeur de $A$}
	\subsubsection*{Si $i=j$}
	Dans ce cas, on obtient $A_{ii} = \Delta^2(\ln\Delta - \frac{3}{2})$
	\subsubsection*{Si $|i-j| = 1$}
	Dans ce cas, on obtient $A_{ij} = \Delta^2(\ln\Delta + 2\ln2 - \frac{3}{2})$
	\subsubsection*{Sinon ($|i-j| > 1$)}
	Dans ce cas, on obtient $A_{ij} = \Delta^2(\ln\Delta + \ln|i-j|) + O(\Delta^4)$
	\subsection*{Valeur de $B$}
	\subsubsection*{Si $i = j = 0$}
	Dans ce cas on obtient $B_{00} = \Delta^2(\ln(\Delta) + \ln(2) - \frac{3}{2})$
	\subsubsection*{Sinon}
	On obtient $B_{ij} = \Delta^2(\ln\Delta + \ln\frac{i+j+1}{2}) + O(\Delta^4)$
	\subsection*{Valeur de $C$}
	Elle se déduit de celle de $B$ par changement de variable dans l'intégrale.
	\subsubsection*{Si $i = j = N-1$}
	Dans ce cas on obtient $C_{N-1,N-1} = B_{0,0} = \Delta^2(\ln(\Delta) + \ln(2) - \frac{3}{2})$
	\subsubsection*{Sinon}
	On obtient $C_{i,j} = B_{N-1-i,N-1-j} = \Delta^2\left(\ln(\Delta) + \ln\left(N-\frac{i+j+1}{2}\right)\right) + O(\Delta^4)$
	
	\subsection*{Valeur de $D$}
	On peut montrer que l'intégrande dans $D$ est infiniment régulier, donc l'intégrale est facilement approchée par n'importe quelle méthode de quadrature numérique. 
	
	\begin{Rem}
		Ces résultats peuvent être vérifiés à l'aide de Maple. 	
	\end{Rem}
	
	\section{Le cas $\mathbbm{P}_1$}
	Le but est de calculer des intégrales de la forme 
	\[ \int_{\theta_i}^{\theta_{i+1}} \int_{\theta_j}^{\theta_{j+1}}\ln{|\cos\theta- \cos\theta'|}(a\theta + b)(b\theta'+c)\] 
	
		
\end{document}