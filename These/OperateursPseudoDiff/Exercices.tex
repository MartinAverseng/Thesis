\documentclass[11pt,a4paper]{article}

\usepackage{adjustbox}
\usepackage{algorithm}
\usepackage{algorithmic}
\usepackage{amsmath}
\usepackage{amssymb}
\usepackage{amsthm}
\usepackage{amsfonts}
\usepackage{afterpage}
\usepackage{blindtext}
\usepackage[font=footnotesize,labelfont=bf]{caption}
\usepackage{hyperref}
\usepackage[english]{babel}
\usepackage{bbm}
\usepackage{bigints}
\usepackage{bm}
\usepackage{cite}
\usepackage{color}
\usepackage{float}
\usepackage[left=2cm,right=2cm,top=2cm,bottom=2cm]{geometry}
\usepackage{graphicx}
\usepackage[utf8]{inputenc}
\usepackage{mathtools}
\usepackage{mdframed}
\usepackage{pgfplots} 
\usepackage{subfigure}
\usepackage{stmaryrd}
\usepackage{textcomp}
\usepackage{tikz}
\usepackage{url}
\renewcommand{\proofname}{Proof}
\theoremstyle{plain}
\newtheorem{monTheoNumrote}{Théorème}[section] % Environnement numéroté en fonction de la section
\newtheorem*{monTheoNonNumerote}{Théorème}  % Environnement non numéroté
\newtheorem{The}{Theorem}[section]
\newtheorem*{The*}{Theorem}
\newtheorem{Prop}{Proposition}[section]
\newtheorem*{Prop*}{Proposition} 
\newtheorem{Cor}{Corollary}[section]
\newtheorem*{Cor*}{Corollary}
\newtheorem{Conj}{Conjecture}[section]
\newtheorem{Lem}{Lemma}[section]
\renewcommand{\qed}{\unskip\nobreak\quad\qedsymbol}%
\numberwithin{equation}{section} % Numérote les équations section.numéro.
\theoremstyle{definition}
\newtheorem{Def}{Definition}[section]
\newtheorem{Rem}{Remark}[section]
\newtheorem*{Rem*}{Remark}
\newtheorem*{Lem*}{Lemma}
\newtheorem{Que}{Question}
\newcommand{\enstq}[2]{\left\{#1\mathrel{}\middle|\mathrel{}#2\right\}}
\newcommand{\Lp}[2]{L^#1(#2)}
\newcommand{\Sob}[3]{W^{#1,#2}(#3)}
\newcommand{\Rd}[0]{\mathbb{R}^d}
\newcommand{\RN}[0]{\mathbb{R}^N}
\newcommand{\Rn}[0]{\mathbb{R}^n}
\newcommand{\norm}[1]{\left\|#1\right\|}
\newcommand{\sinc}[0]{\textup{sinc}}
\newcommand{\functionDef}[5]{\begin{array}{lllll}
#1 & : & #2 & \longrightarrow & #3 \\
 & & #4 & \longmapsto &\displaystyle #5 \\
\end{array}}
\newcommand{\Theautorefname}{Theorem}
\newcommand{\Propautorefname}{Proposition}
\newcommand{\Corautorefname}{Corollary}
\newcommand{\Lemautorefname}{Lemma}
\newcommand{\Defautorefname}{Definition}
\newcommand{\N}{\mathbb{N}}
\newcommand{\Z}{\mathbb{Z}}
\newcommand{\D}{\mathbb{D}}
\newcommand{\R}{\mathbb{R}}
\newcommand{\A}{\mathcal{A}_{a,b}}
\newcommand{\Crad}{C^\infty_{c,rad}(B)}
\newcommand{\Lrad}{L^2_{rad}(B)}
\newcommand{\Lradab}{L^2_{rad}(\mathcal{A}_{a,b})}
\newcommand{\duality}[2]{\left\langle #1,#2\right\rangle}
\newcommand{\Hrad}{H^1_{rad}(B)}
\newcommand{\Hzrad}{H^1_{0,rad}(B)}
\newcommand{\rmin}{\delta_{\min}}
\newcommand{\rmax}{\delta_{\max}}
\newcommand{\corr}{\gamma}
\newcommand{\question}[1]{\begin{Que} \ 
#1
\end{Que}}
\newcommand{\abs}[1]{\left\lvert #1 \right\rvert}
\newcommand{\CL}[2]{\textup{CL}\left(\enstq{#1}{#2}\right)}
\newcommand{\Script}[1]{`\texttt{#1}`}
\newcommand{\espace}{\text{ }\qquad} 
\newcommand{\loc}{\text{loc}}
\newcommand{\SL}{\textup{SL}\hspace{1.5pt}}
\newcommand{\DL}{\textup{DL}\hspace{1.5pt}}
\newcommand{\fp}{\underset{\varepsilon \to 0}{\textup{f.p.}}}
\newcommand{\scalProd}[2]{\left(#1|#2\right)}
\newcommand{\toDo}[1]{{\color{red}#1}}
\newcommand{\bs}[1]{\boldsymbol{#1}}
\newcommand{\varInRange}[4]{(#1_{#2})_{#3 \leq #2 \leq #4}}
\newcommand{\from}{\colon}
\newcommand{\Cinf}{C^{\infty}}
\newcommand{\isdef}{\mathrel{\mathop:}=}
\newcommand{\defis}{=\mathrel{\mathop:}}

\renewcommand{\algorithmicrequire}{\textbf{Inputs:}}
\renewcommand{\algorithmicensure}{\textbf{Outputs:}}

\pgfplotsset{compat=1.13}
\author{Martin AVERSENG}
\title{Exercices opérateurs pseudo-différentiels}
\begin{document}
\maketitle

\section{Introduction}
\paragraph{Exercice 1.1}

a) La fonction $p_m$ est continue (c'est un polynôme) et non nulle (on la suppose elliptique) sur la sphère unité. Sa valeur absolue est donc également continue et atteint par conséquent un minimum $\nu >0$ sur la sphère. En appliquant le fait que $p_m$ est homogène de degré $m$, on en déduit que pour tout $\xi \in \mathbb{R}^n$
\[p_m(\xi) > \nu \abs{\xi}^m \]
Comme le polynôme $p - p_m$ est de degré au plus $m-1$, il existe $R > 0$ pour lequel 
\[ \forall \abs{\xi} > R, \quad \abs{(p - p_m)(\xi)} < \dfrac{\abs{p_m(\xi)}}{2}\]
D'où l'on tire que 
\[\forall \abs{\xi} > R,\quad \abs{p(\xi)} > \abs{ \abs{p_m(\xi)} - \abs{ (p-p_m)(\xi)}} = \abs{p_m(\xi)} - \abs{(p - p_m)(\xi)} > \dfrac{\abs{p_m(\xi)}}{2} > \dfrac{\nu}{2^m}\abs{\xi}^m\]
On considère donc une fonction $\chi \in C^{\infty}_0(\mathbb{R}^n)$ qui est identiquement égale à $1$ dans un voisinnage de $B(0,R)$, et dans ce cas, la fonction 
\[ \hat{E}(\xi) = \dfrac{1 - \chi(\xi)}{p(\xi)},\]
est bien définie. Comme c'est une fonction de classe $C^\infty$ et bornée, c'est une distribution tempérée. 
On a alors 
\[ \widehat{p(D) E} = 1 - \chi(\xi),\]
ce qui implique que $p(D)E = \delta + w$ où $w$ est la transformée de Fourier de $-\chi$. Comme $\chi$ est dans l'espace de Schwartz (puisqu'elle est $C^{\infty}_0$), $w$ l'est aussi. \newline 
\\
b) Nous allons montrer le résultat suivant 
\begin{Prop} Pour tout $\alpha$, il existe une constante $C_\alpha > 0$ telle que  
\[ |\partial^\alpha_\xi \hat{E}| \leq C_\alpha \left(1 + |\xi|\right)^{-m-|\alpha|}\]
\begin{proof}
Etant donné que $\hat{E}$ est nulle dans un voisinnage de $0$, il suffit de montrer que pour de grands $|\xi|$, 
\begin{equation}
|\partial^\alpha_\xi \hat{E}| \leq C_\alpha |\xi|^{-m-|\alpha|}
\label{estimHomogene}
\end{equation}
D'après la formule de Leibniz, la fonction $\partial_{\xi}^\alpha \hat{E}$ s'exprime comme une combinaison linaire de termes de la forme 
\[\partial^{\beta_1}{\xi} (1 - \chi) \partial_{\xi}^{\beta_2} \left(\dfrac{1}{p}\right)\]
Où $\beta_1 + \beta_2 = \alpha$, ce que nous notons par la suite 
\[ \partial^\alpha_\xi \hat{E} = \CL{\partial^{\beta_1}{\xi} (1 - \chi) \partial_{\xi}^{\beta_2}\left(\dfrac{1}{p}\right)}{\beta_1 + \beta_2 = \alpha}\]
Lorsque $\beta_1 \neq 0$, le terme correspondant est dans $C^{\infty}_c(\Rn)$ et vérifie donc évidemment l'estimation (\ref{estimHomogene}). Le seul terme qui n'obéit pas à cette condition est de la forme 
\[ (1 - \chi) \partial_{\xi}^{\alpha} \left( \dfrac{1}{p}\right)\]
Exprimons $\partial_{\xi}^{\alpha} \left( \dfrac{1}{p}\right)$ à l'aide de la formule de Fàa di Bruno : 
\[ \dfrac{1}{p} = \CL{\dfrac{1}{p^{K+1}}\prod_{i=1}^K \partial^{\alpha_i}_{\xi}p}{\alpha_1 + \alpha_2 + ... + \alpha_K = \alpha}\]
Or puisque $p$ est un polynôme de degré $m$, pour tout $\beta$, il existe une constante $C_\beta$ vérifiant 
\[ \partial^{\beta} p \leq C_\beta |\xi|^{m-|\beta|} \]
D'autre part, on a montré en a) que \[\frac{1}{p} > \nu \dfrac{1}{|\xi|^m}\]
On en déduit que tous les termes dans la formule de Fàa di Bruno sont majorés par une quantité de la forme $\dfrac{C}{|\xi|^{m + |\alpha|}}$
\end{proof}
\end{Prop} 
\begin{Cor} Pour tout $\alpha$ tel que $|\alpha|\geq n - m + 1$ et pour tout $\beta$, on a 
\[ D^{\beta}(x^{\alpha+\beta} E) \in L^{\infty}\]
\begin{proof}
La transformée de Fourier de $D^{\beta}(x^{\alpha+\beta} E)$ est proportionnelle à $\xi^{\beta} \partial^{\alpha+\beta}_\xi\hat{E}$ qui est intégrable sous les conditions de l'énoncé, grâce au résultat de la proposition précédente. 
\end{proof}
\end{Cor}
\begin{Cor} $E$ est $C^\infty$ sur $\Rn \setminus\{0\}$. 
\begin{proof}
Soit $k \in \mathbb{N}^*$, montrons que $E$ est de  classe $C^k$ en dehors de $0$. Soit $\alpha_k$ tel que
\[|\beta| \leq k  \implies \beta \leq \alpha_k \textup{ et } |\alpha_k - \beta| \geq n + 1 - m\]
Nous allons montrer que $x^{\alpha_k} E$ est de classe $C^{k-1}$. Pour cela, il suffit de montrer que pour tout multi-indice $\beta$ de longueur inférieure à $k$, $\partial^{\beta} (x^{\alpha_k} E)$ est bornée. Soit $\beta$ un tel multi-indice, on peut vérifier que le couple $(\alpha_k - \beta,\beta)$ vérifie les hypothèses du corollaire précédent ce qui fournit le résultat. 
\end{proof}
\end{Cor} 
\text{ }\\
c) La transformée de Fourier de $D^{\beta} E$ s'écrit $\xi^{\beta}  \hat{E}$. Les réponses aux questions précédentes nous ont permis de voir que $\hat{E} \in S^{-m}$ et $\xi^{\beta} \in S^{|\beta|}$ donc $\xi^{\beta}  \hat{E} \in S^{|\beta| - m}$. Sous l'hypothèse $|\beta| \leq m - n - 1$,  $\xi^{\beta}  \hat{E} \in S^{-n-1}$. C'est donc une fonction intégrable, ce qui prouve que $D^{\beta} E$ est bornée, donc en particulier intégrable en $0$. D'autre part, on a 
\[ x^{\alpha} D^{\beta} E \propto \int e^{i x\xi} \partial^\alpha_\xi(\xi^{\beta}  \hat{E}) d\xi\]
Le second membre est toujours intégrable (la dérivation a même accéléré la décroissance à l'infini du spectre). On en déduit que $D^\beta E$ est intégrable.  
Soit $\alpha$ tel que $|\alpha| \leq m - n - 1$. On sait que, étant donné que $p(D)$ et $D^\alpha$ commutent, 
\[ D^\alpha u = \left( D^\alpha E * p(D) u\right) + S^{-\infty}\]
Or la quantité du membre de droite est bornée à cause de l'inégalité de Young et du fait que $D^\alpha E \in L^1$ et $p(D)u \in L^{\infty}$
\text{ }\\
\end{document}