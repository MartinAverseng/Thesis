\documentclass[11pt]{beamer}
\usetheme{Warsaw}

\usepackage[utf8]{inputenc}
\usepackage[french]{babel}
\usepackage[T1]{fontenc}
\usepackage{amsmath}
\usepackage{amsfonts}
\usepackage{amssymb}
\usepackage{afterpage}
\usepackage{bbm}
\usepackage[]{algorithm2e}
\usepackage{amsthm}
\usepackage{amsfonts}
\usepackage{subfigure}
\usepackage{tikz}
\usepackage{pgfplots} 
\usepackage{amssymb}
\usepackage{graphicx}
\usepackage{lmodern}
\usepackage{stmaryrd}
\usepackage{float}
\usepackage{url}
\usepackage{bigints}
\pgfplotsset{compat=1.13}
\author{Martin Averseng}
\title{Convolution rapide par des noyaux radiaux en deux dimensions}
%\setbeamercovered{transparent} 
%\setbeamertemplate{navigation symbols}{} 
\logo{logo-cmap.jpg} 
\institute{Ecole Polytechnique, Centre de Mathématiques Appliquées} 
\date{15-07-2016} 
\subject{Soutenance de stage de fin de Master} 
\theoremstyle{plain}
\newtheorem{monTheoNumrote}{Théorème}[section] % Environnement numéroté en fonction de la section
\newtheorem*{monTheoNonNumerote}{Théorème}  % Environnement non numéroté
\newtheorem{The}{Theorem}[section]
\newtheorem{Prop}{Proposition}[section]
\newtheorem*{Prop*}{Proposition} 
\newtheorem{Cor}{Corollary}[section]
\newtheorem{Conj}{Conjecture}[section]
\newtheorem{Lem}{Lemma}[section]
\theoremstyle{definition}
\newtheorem{Def}{Definition}[section]
\newtheorem{Rem}{Remark}[section]
\newcommand{\enstq}[2]{\left\{#1\mathrel{}\middle|\mathrel{}#2\right\}}
\newcommand{\Lp}[2]{L^#1(#2)}
\newcommand{\Sob}[3]{W^{#1,#2}(#3)}
\newcommand{\RN}[0]{\mathbb{R}^N}
\newcommand{\norm}[1]{\left\|#1\right\|}
\newcommand{\sinc}[0]{\textup{sinc}}
\newcommand{\functionDef}[5]{\begin{array}{lllll}
#1 & : & #2 & \longrightarrow & #3 \\
 & & #4 & \longmapsto &\displaystyle #5 \\
\end{array}}
\newcommand{\N}{\mathbb{N}}
\newcommand{\Z}{\mathbb{Z}}
\newcommand{\D}{\mathbb{D}}
\newcommand{\R}{\mathbb{R}}
\newcommand{\A}{\mathcal{A}_{a,b}}
\newcommand{\Crad}{C^\infty_{c,rad}(B)}
\newcommand{\Lrad}{L^2_{rad}(B)}
\newcommand{\Lradab}{L^2_{rad}(\mathcal{A}_{a,b})}
\newcommand{\duality}[2]{\left\langle #1,#2\right\rangle}
\newcommand{\Hrad}{H^1_{rad}(B)}
\newcommand{\Hzrad}{H^1_{0,rad}(B)}
\addtobeamertemplate{navigation symbols}{}{%
    \usebeamerfont{footline}%
    \usebeamercolor[fg]{footline}%
    \hspace{1em}%
    \insertframenumber/\inserttotalframenumber
}
\begin{document}

\begin{frame}
\titlepage
\end{frame}

\begin{frame}
\tableofcontents
\end{frame}

\section{Introduction}

\begin{frame}{Introduction}
Le problème : calculer 
\begin{equation*}
 q_k = \sum_{\underset{l\neq k}{l=1}}^N G(|x_k - x_l|)f_l 
\label{LaSommeACalculerDansLArticle}
\end{equation*}

\begin{itemize}

\item[-] Discretisation dans le domaine de Fourier $\iff$ périodisation de l'espace. 
\item[-] Convolution en espace $\iff$ multiplication en Fourier. 
\item[-] Fonctions de base 
\[e_\lambda(x) = \int_{\mathbbm{S}^{N-1}} e^{i\lambda x \cdot  \xi} d\sigma(u)\]
\item[-] On obtient une base avec un ensemble discret de valeurs de $\lambda$. 
\item[-] En dimension $3$ : sinc. En dimension $2$ : fonctions de Bessel. 
\end{itemize}
\end{frame}

\section{Présentation de la méthode}
\begin{frame}{Présentation de la méthode}
\begin{itemize}
\item[-] Champ lointain : \begin{itemize}
\item[-] Partie radiale
\item[-] Partie circulaire
\end{itemize}
\item[-] Champ proche 
\end{itemize}

Partie radiale : 
\begin{equation*}
G \approx \sum_{p=1}^P \alpha_p J_0(\rho_p |x|)
\label{GasASumOfJ0}
\end{equation*}

Partie circulaire : 
\[J_0(|x|) \approx \frac{1}{M}\sum_{m=1}^M e^{i x \cdot \xi_m}\] 


\end{frame}

\section{Approximation radiale}
\subsection{L'opérateur Laplacien radial}
\begin{frame}{Opérateur Laplacien radial}
\begin{itemize}
\item[-] Espace de Hilbert des fonctions radiales $L^2$ sur la boule unité : $L^2_{rad}(B)$, $\Hrad$ et $\Hzrad$ 
\item[-] Densité de $\mathcal{C}^{\infty}_{c,rad}(B)$ dans $L^2$, inégalité de Poincaré. 
\item[-] Le Laplacien est auto-adjoint positif à résolvente compacte - diagonalisable en base orthonormée de vecteurs propres. 
\end{itemize} 
\end{frame}

\subsection{Décroissance rapide des coefficients}


\begin{frame}{Décroissance rapide des coefficients}
Proposition fondamentale
\begin{Prop} Soit $n\in \mathbb{N}^*$ et $f \in H^{2n}(B)$. On suppose que pour tout entier $s\leq n-1$, l'itéré numéro $s$ de l'opérateur $-\Delta$ sur $f$ noté $(-\Delta)^s f$ vérifie la condition de Dirichlet sur $\partial B$. Alors :

\[ c_k(f) = \dfrac{c_k\left(\left(-\Delta\right)^n f\right)}{\rho_{k}^{2n}}\] for any $k \in \mathbb{N}^*$
\label{PropDecrCond}
\end{Prop}
\begin{center}
$\rho_k \sim k\pi$ 
\end{center}
\end{frame}

\begin{frame}
Exploitation du résultat 
\begin{itemize}
\item[-] Restriction à un anneau et majoration du reste à l'aide d'un prolongement (polynomial, interpolation par fonctions de Bessel ?)
\item[-] Condition de Dirichlet à tout ordre $s$ : les noyaux de Laplace et Helmholtz sont des cas favorables... (mais bien choisir l'anneau pour $Y_0$).
\end{itemize}
\end{frame}

\begin{frame}{Importance de la fréquence ajustée}
\begingroup
\centering
\tikzset{every picture/.style={scale=0.88}}%
% This file was created by matlab2tikz.
%
%The latest updates can be retrieved from
%  http://www.mathworks.com/matlabcentral/fileexchange/22022-matlab2tikz-matlab2tikz
%where you can also make suggestions and rate matlab2tikz.
%
\definecolor{mycolor1}{rgb}{0.00000,0.44700,0.74100}%
\definecolor{mycolor2}{rgb}{0.85000,0.32500,0.09800}%
%
\begin{tikzpicture}

\begin{axis}[%
width=4.236in,
height=3.091in,
at={(1.043in,0.956in)},
scale only axis,
unbounded coords=jump,
xmin=0,
xmax=1,
xlabel={$|x|$},
ymin=-20,
ymax=0,
ylabel={Error (dB)},
axis background/.style={fill=white},
legend style={legend cell align=left,align=left,fill=none,draw=none}
]
\addplot [color=mycolor1,solid]
  table[row sep=crcr]{%
0	inf\\
0.00200400801603206	-0.467198430045094\\
0.00400801603206413	-1.37887448541621\\
0.00601202404809619	-2.37970505441492\\
0.00801603206412826	-3.65768413469392\\
0.0100200400801603	-5.76960589156241\\
0.0120240480961924	-6.81046188492003\\
0.0140280561122244	-8.77312417677475\\
0.0160320641282565	-7.38391941620846\\
0.0180360721442886	-7.98089912598749\\
0.0200400801603206	-8.39241659812629\\
0.0220440881763527	-7.82413072965104\\
0.0240480961923848	-8.89050251217323\\
0.0260521042084168	-8.55510151906915\\
0.0280561122244489	-8.19997582961138\\
0.030060120240481	-9.27511683236344\\
0.032064128256513	-8.90274860199646\\
0.0340681362725451	-8.48321138312791\\
0.0360721442885772	-9.33977127752456\\
0.0380761523046092	-9.3448148556653\\
0.0400801603206413	-8.72625174425234\\
0.0420841683366733	-9.32121470831298\\
0.0440881763527054	-9.9302933553443\\
0.0460921843687375	-8.96017706310585\\
0.0480961923847695	-9.30013071896324\\
0.0501002004008016	-10.9804425741698\\
0.0521042084168337	-9.22179506780176\\
0.0541082164328657	-9.31775340190162\\
0.0561122244488978	-11.805830511284\\
0.0581162324649299	-9.52862712981902\\
0.0601202404809619	-9.36983717311928\\
0.062124248496994	-10.5339625367943\\
0.064128256513026	-9.93845489293213\\
0.0661322645290581	-9.47663919855471\\
0.0681362725450902	-10.1410223117691\\
0.0701402805611222	-10.5142091941489\\
0.0721442885771543	-9.62272179925308\\
0.0741482965931864	-9.94171853292683\\
0.0761523046092184	-11.7708046525942\\
0.0781563126252505	-9.84939786104206\\
0.0801603206412826	-9.88225060909756\\
0.0821643286573146	-11.8028677526945\\
0.0841683366733467	-10.1426803026169\\
0.0861723446893788	-9.8840055812382\\
0.0881763527054108	-10.8043483918448\\
0.0901803607214429	-10.6054133179076\\
0.0921843687374749	-9.9749335083143\\
0.094188376753507	-10.4568980265759\\
0.0961923847695391	-11.3657854019347\\
0.0981963927855711	-10.1236145822159\\
0.100200400801603	-10.2814389774498\\
0.102204408817635	-14.0141242147197\\
0.104208416833667	-10.3714413121284\\
0.106212424849699	-10.2323187579011\\
0.108216432865731	-11.42359352394\\
0.110220440881764	-10.7363428959581\\
0.112224448897796	-10.262404547217\\
0.114228456913828	-10.8786446599301\\
0.11623246492986	-11.3451694331386\\
0.118236472945892	-10.3710686703881\\
0.120240480961924	-10.6258103587648\\
0.122244488977956	-13.0144957379605\\
0.124248496993988	-10.5739431833523\\
0.12625250501002	-10.5269876611675\\
0.128256513026052	-12.036181134923\\
0.130260521042084	-10.8716060038645\\
0.132264529058116	-10.5047964493788\\
0.134268537074148	-11.2361714857931\\
0.13627254509018	-11.4152824842293\\
0.138276553106212	-10.5969718609957\\
0.140280561122244	-10.940848245825\\
0.142284569138277	-12.4426053706536\\
0.144288577154309	-10.7391006928748\\
0.146292585170341	-10.7647958712549\\
0.148296593186373	-12.5661143794309\\
0.150300601202405	-11.0356170484356\\
0.152304609218437	-10.7452767634629\\
0.154308617234469	-11.6308330783964\\
0.156312625250501	-11.4338053043054\\
0.158316633266533	-10.7614115776014\\
0.160320641282565	-11.1677582307848\\
0.162324649298597	-12.4429430148724\\
0.164328657314629	-10.9380478023194\\
0.166332665330661	-11.0340631386178\\
0.168336673346693	-13.9055844798892\\
0.170340681362725	-11.108251832466\\
0.172344689378758	-10.8962548707626\\
0.17434869739479	-11.8363354202969\\
0.176352705410822	-11.6045160966371\\
0.178356713426854	-10.9836153464864\\
0.180360721442886	-11.495936204351\\
0.182364729458918	-12.1621914601413\\
0.18436873747495	-11.0247295234194\\
0.186372745490982	-11.170674934517\\
0.188376753507014	-14.0340677530735\\
0.190380761523046	-11.3044174092172\\
0.192384769539078	-11.138591053244\\
0.19438877755511	-12.337059693343\\
0.196392785571142	-11.5722581534056\\
0.198396793587174	-11.0790121411687\\
0.200400801603206	-11.6267768243174\\
0.202404809619238	-12.3316923719171\\
0.204408817635271	-11.2173504739632\\
0.206412825651303	-11.4281054776391\\
0.208416833667335	-14.0158768856234\\
0.210420841683367	-11.3475902574116\\
0.212424849699399	-11.246892736385\\
0.214428857715431	-12.4816230529511\\
0.216432865731463	-11.7247359231783\\
0.218436873747495	-11.2661067026969\\
0.220440881763527	-11.9169356214279\\
0.222444889779559	-12.218789754291\\
0.224448897795591	-11.3035758927768\\
0.226452905811623	-11.5635015146959\\
0.228456913827655	-13.8374341289259\\
0.230460921843687	-11.4788609126888\\
0.232464929859719	-11.4215148493577\\
0.234468937875752	-12.8773536303925\\
0.236472945891784	-11.7736812982228\\
0.238476953907816	-11.3889588014296\\
0.240480961923848	-12.1185669422371\\
0.24248496993988	-12.2351470174053\\
0.244488977955912	-11.4124117768771\\
0.246492985971944	-11.7156033235713\\
0.248496993987976	-13.629500165725\\
0.250501002004008	-11.5997051499503\\
0.25250501002004	-11.5988873056149\\
0.254509018036072	-13.5071936491031\\
0.256513026052104	-11.7717248114245\\
0.258517034068136	-11.4526692409404\\
0.260521042084168	-12.177048384393\\
0.2625250501002	-12.4385672576535\\
0.264529058116232	-11.608482176463\\
0.266533066132265	-12.0137002467857\\
0.268537074148297	-12.9050652307101\\
0.270541082164329	-11.5724604656878\\
0.272545090180361	-11.6084641986614\\
0.274549098196393	-13.1642960209815\\
0.276553106212425	-12.0155821809536\\
0.278557114228457	-11.7056378585868\\
0.280561122244489	-12.7176348426383\\
0.282565130260521	-12.164634773684\\
0.284569138276553	-11.5592739515276\\
0.286573146292585	-11.9436052741129\\
0.288577154308617	-13.5614753510017\\
0.290581162324649	-11.823513628414\\
0.292585170340681	-11.9142587943808\\
0.294589178356713	-16.969948682493\\
0.296593186372746	-11.8656581863206\\
0.298597194388778	-11.6538454899609\\
0.30060120240481	-12.5155178391671\\
0.302605210420842	-12.4901368779458\\
0.304609218436874	-11.8077809621196\\
0.306613226452906	-12.330019510668\\
0.308617234468938	-12.8214426373336\\
0.31062124248497	-11.7401269987749\\
0.312625250501002	-11.8599341601863\\
0.314629258517034	-13.9843002495865\\
0.316633266533066	-12.111525264217\\
0.318637274549098	-11.9060968538531\\
0.32064128256513	-13.1223393169394\\
0.322645290581162	-12.2680423250625\\
0.324649298597194	-11.7670491091232\\
0.326653306613226	-12.2745124726501\\
0.328657314629259	-13.1428735454622\\
0.330661322645291	-11.919928012889\\
0.332665330661323	-12.0869956035978\\
0.334669338677355	-15.8810263103806\\
0.336673346693387	-12.0658584495671\\
0.338677354709419	-11.9328162453041\\
0.340681362725451	-13.1480615201661\\
0.342685370741483	-12.3679943112389\\
0.344689378757515	-11.8786645362397\\
0.346693386773547	-12.4312379559767\\
0.348697394789579	-13.1146115609707\\
0.350701402805611	-12.0185625058832\\
0.352705410821643	-12.2639121454152\\
0.354709418837675	-14.0196235560857\\
0.356713426853707	-12.0229690404252\\
0.35871743486974	-11.9227677978693\\
0.360721442885772	-12.972920856414\\
0.362725450901804	-12.6365166563076\\
0.364729458917836	-12.1256656341544\\
0.366733466933868	-12.9385114294288\\
0.3687374749499	-12.6336699919543\\
0.370741482965932	-11.8864394311792\\
0.372745490981964	-12.1076934893827\\
0.374749498997996	-14.8308258823698\\
0.376753507014028	-12.3612072408269\\
0.37875751503006	-12.2884414629863\\
0.380761523046092	-14.6871264787627\\
0.382765531062124	-12.2582928556982\\
0.384769539078156	-11.92798816087\\
0.386773547094188	-12.5171574158063\\
0.38877755511022	-13.3572966360659\\
0.390781563126253	-12.2535124897301\\
0.392785571142285	-12.6240196162458\\
0.394789579158317	-13.3638571572783\\
0.396793587174349	-12.0634911852204\\
0.398797595190381	-12.0500791232308\\
0.400801603206413	-13.2302666691001\\
0.402805611222445	-12.7208064865377\\
0.404809619238477	-12.2982170327141\\
0.406813627254509	-13.2962146285669\\
0.408817635270541	-12.6546773842579\\
0.410821643286573	-12.0262389936994\\
0.412825651302605	-12.3371331996761\\
0.414829659318637	-15.1310240081604\\
0.416833667334669	-12.3939009680282\\
0.418837675350701	-12.4031243475662\\
0.420841683366733	-15.0359585386197\\
0.422845691382766	-12.4213841976133\\
0.424849699398798	-12.1609823421584\\
0.42685370741483	-12.9887608873505\\
0.428857715430862	-12.9462229910387\\
0.430861723446894	-12.2124630048243\\
0.432865731462926	-12.5870329816964\\
0.434869739478958	-13.9386848536487\\
0.43687374749499	-12.3570321459043\\
0.438877755511022	-12.4347766681112\\
0.440881763527054	-15.4644737379525\\
0.442885771543086	-12.4483915170738\\
0.444889779559118	-12.2059717319172\\
0.44689378757515	-13.0193630916741\\
0.448897795591182	-13.098764568276\\
0.450901803607214	-12.3765047233378\\
0.452905811623247	-12.9069735488778\\
0.454909819639279	-13.2694262833714\\
0.456913827655311	-12.2162223740327\\
0.458917835671343	-12.2928135581791\\
0.460921843687375	-13.799093328474\\
0.462925851703407	-12.8089697448016\\
0.464929859719439	-12.5496110812544\\
0.466933867735471	-14.0864400964221\\
0.468937875751503	-12.6275138353585\\
0.470941883767535	-12.1618364821642\\
0.472945891783567	-12.5683996582531\\
0.474949899799599	-14.5313837894637\\
0.476953907815631	-12.5933580669442\\
0.478957915831663	-12.7631137024917\\
0.480961923847695	-14.3068922212199\\
0.482965931863727	-12.3996664224351\\
0.48496993987976	-12.2562983065043\\
0.486973947895792	-13.111628151699\\
0.488977955911824	-13.2372552320506\\
0.490981963927856	-12.5639646153436\\
0.492985971943888	-13.2780782269051\\
0.49498997995992	-13.0862226403785\\
0.496993987975952	-12.2621709612477\\
0.498997995991984	-12.4087303057475\\
0.501002004008016	-14.1473039752491\\
0.503006012024048	-12.8777557558933\\
0.50501002004008	-12.7064409444103\\
0.507014028056112	-14.8447165596068\\
0.509018036072144	-12.6493999577131\\
0.511022044088176	-12.2708175302944\\
0.513026052104208	-12.7718053385639\\
0.51503006012024	-14.0205152213984\\
0.517034068136273	-12.624646761323\\
0.519038076152305	-12.8557017308466\\
0.521042084168337	-14.3888336863684\\
0.523046092184369	-12.5584813842635\\
0.525050100200401	-12.4987719972668\\
0.527054108216433	-13.8300386019534\\
0.529058116232465	-12.9010243601667\\
0.531062124248497	-12.4564227998988\\
0.533066132264529	-13.0607110319526\\
0.535070140280561	-13.5805063801435\\
0.537074148296593	-12.5838868626424\\
0.539078156312625	-12.8783906591377\\
0.541082164328657	-14.2740588904322\\
0.543086172344689	-12.5635771582847\\
0.545090180360721	-12.5068949772987\\
0.547094188376754	-13.6958549518153\\
0.549098196392786	-13.1025552376917\\
0.551102204408818	-12.6710958589662\\
0.55310621242485	-13.5975257308832\\
0.555110220440882	-13.0745129576612\\
0.557114228456914	-12.3921036531476\\
0.559118236472946	-12.630397142625\\
0.561122244488978	-15.2066856917757\\
0.56312625250501	-12.9545855524859\\
0.565130260521042	-12.9453334921588\\
0.567134268537074	-15.4387643438649\\
0.569138276553106	-12.639956103866\\
0.571142284569138	-12.374512514749\\
0.57314629258517	-12.9595039496822\\
0.575150300601202	-14.0300988361101\\
0.577154308617234	-12.8455340960897\\
0.579158316633267	-13.3210803566027\\
0.581162324649299	-13.5227653064335\\
0.583166332665331	-12.4732608442536\\
0.585170340681363	-12.484939687718\\
0.587174348697395	-13.5664411766739\\
0.589178356713427	-13.3685104931813\\
0.591182364729459	-12.9500030832281\\
0.593186372745491	-14.5543099283114\\
0.595190380761523	-12.876205254947\\
0.597194388777555	-12.3745463669233\\
0.599198396793587	-12.6643276078042\\
0.601202404809619	-15.1200101435198\\
0.603206412825651	-13.0986895732475\\
0.605210420841683	-13.1759349574838\\
0.607214428857715	-14.3943888830758\\
0.609218436873748	-12.6401323743057\\
0.61122244488978	-12.4593642007571\\
0.613226452905812	-13.1641436440234\\
0.615230460921844	-13.7490953639257\\
0.617234468937876	-12.8327064919856\\
0.619238476953908	-13.312803579215\\
0.62124248496994	-13.7718625464844\\
0.623246492985972	-12.6851221780829\\
0.625250501002004	-12.8098464578862\\
0.627254509018036	-15.4112552972167\\
0.629258517034068	-12.9198780548914\\
0.6312625250501	-12.690551560357\\
0.633266533066132	-13.6105825764112\\
0.635270541082164	-13.394703709199\\
0.637274549098196	-12.7639039129176\\
0.639278557114228	-13.3023336332766\\
0.641282565130261	-13.7519635248542\\
0.643286573146293	-12.6877745489258\\
0.645290581162325	-12.8041720134137\\
0.647294589178357	-14.810569591752\\
0.649298597194389	-13.105193808524\\
0.651302605210421	-12.9262332946271\\
0.653306613226453	-14.5887132955426\\
0.655310621242485	-12.9877155096213\\
0.657314629258517	-12.536198438034\\
0.659318637274549	-12.9217745028254\\
0.661322645290581	-15.7965395258086\\
0.663326653306613	-13.1441749311752\\
0.665330661322645	-13.4293685660618\\
0.667334669338677	-13.8831577481279\\
0.669338677354709	-12.6267021280018\\
0.671342685370741	-12.5201030600185\\
0.673346693386774	-13.2452060131783\\
0.675350701402806	-14.0698348369492\\
0.677354709418838	-13.2003109056569\\
0.67935871743487	-14.4088608994657\\
0.681362725450902	-13.0961638247141\\
0.683366733466934	-12.4729734603746\\
0.685370741482966	-12.6100937308698\\
0.687374749498998	-13.7861087173399\\
0.68937875751503	-13.6783545124543\\
0.691382765531062	-13.5033222650983\\
0.693386773547094	-14.3908811655299\\
0.695390781563126	-12.7016570494879\\
0.697394789579158	-12.4206344540131\\
0.69939879759519	-12.815101078395\\
0.701402805611222	-15.4547748806467\\
0.703406813627255	-13.4344777940935\\
0.705410821643287	-13.9007670612799\\
0.707414829659319	-13.5345471406575\\
0.709418837675351	-12.6074981177851\\
0.711422845691383	-12.5866316061709\\
0.713426853707415	-13.4619598869741\\
0.715430861723447	-13.77551192905\\
0.717434869739479	-13.1185970662833\\
0.719438877755511	-14.0900104640364\\
0.721442885771543	-13.3576463932868\\
0.723446893787575	-12.7179781170391\\
0.725450901803607	-13.0035483132591\\
0.727454909819639	-16.2759143307907\\
0.729458917835671	-13.0624519229101\\
0.731462925851703	-13.0195849596497\\
0.733466933867735	-14.9321056771691\\
0.735470941883767	-13.1482228137025\\
0.7374749498998	-12.817893011928\\
0.739478957915832	-13.5337258287519\\
0.741482965931864	-13.7514146588396\\
0.743486973947896	-12.8988384327534\\
0.745490981963928	-13.2110341087486\\
0.74749498997996	-14.916925286747\\
0.749498997995992	-13.0597091142155\\
0.751503006012024	-13.1200269481854\\
0.753507014028056	-18.9171078129007\\
0.755511022044088	-12.9778460107334\\
0.75751503006012	-12.6890940727558\\
0.759519038076152	-13.2399976845363\\
0.761523046092184	-14.6465574948025\\
0.763527054108216	-13.3187416425909\\
0.765531062124248	-14.0616722399141\\
0.767535070140281	-13.4192594178803\\
0.769539078156313	-12.6129325660905\\
0.771543086172345	-12.631334236948\\
0.773547094188377	-13.4759995614173\\
0.775551102204409	-14.2108901747254\\
0.777555110220441	-13.6266590781465\\
0.779559118236473	-15.1156555345338\\
0.781563126252505	-12.8713074257507\\
0.783567134268537	-12.4749517577388\\
0.785571142284569	-12.7048840312109\\
0.787575150300601	-13.9010582749693\\
0.789579158316633	-14.1238167418069\\
0.791583166332665	-14.4611136985532\\
0.793587174348697	-13.4884541803637\\
0.795591182364729	-12.5867541012113\\
0.797595190380762	-12.4596017007306\\
0.799599198396794	-12.9465385473535\\
0.801603206412826	-15.7089040312678\\
0.803607214428858	-13.8112170138485\\
0.80561122244489	-15.5669801761913\\
0.807615230460922	-13.1773101591347\\
0.809619238476954	-12.5863791005448\\
0.811623246492986	-12.6922891574734\\
0.813627254509018	-13.7351264439262\\
0.81563126252505	-13.8426649892804\\
0.817635270541082	-13.4322312377882\\
0.819639278557114	-16.5054364419529\\
0.821643286573146	-13.1268519330429\\
0.823647294589178	-12.7357855026912\\
0.82565130260521	-13.1432006341094\\
0.827655310621243	-15.5481718340997\\
0.829659318637275	-13.2754495979335\\
0.831663326653307	-13.4747535467613\\
0.833667334669339	-14.5611965312615\\
0.835671342685371	-12.961795058303\\
0.837675350701403	-12.841083756943\\
0.839679358717435	-13.7049211643892\\
0.841683366733467	-13.8174845408541\\
0.843687374749499	-13.1343960464043\\
0.845691382765531	-13.7582955971986\\
0.847695390781563	-13.8933303364193\\
0.849699398797595	-13.0033114589468\\
0.851703406813627	-13.2624737299433\\
0.853707414829659	-15.0771204571184\\
0.855711422845691	-13.0109353253434\\
0.857715430861723	-12.8577664000027\\
0.859719438877755	-13.6407275709889\\
0.861723446893788	-14.1135756696066\\
0.86372745490982	-13.3993630788504\\
0.865731462925852	-14.6513652287517\\
0.867735470941884	-13.3150061833543\\
0.869739478957916	-12.7030935030931\\
0.871743486973948	-12.8508624408222\\
0.87374749498998	-14.0673879296266\\
0.875751503006012	-13.8823772939763\\
0.877755511022044	-13.7496605403174\\
0.879759519038076	-14.4871281656123\\
0.881763527054108	-12.893839690136\\
0.88376753507014	-12.6207735319203\\
0.885771543086172	-13.0050444034364\\
0.887775551102204	-15.2091000747582\\
0.889779559118236	-13.7905955600591\\
0.891783567134269	-14.4937928785454\\
0.893787575150301	-13.4966729393693\\
0.895791583166333	-12.6953619039024\\
0.897795591182365	-12.6636891660653\\
0.899799599198397	-13.3519239876255\\
0.901803607214429	-14.8352830292584\\
0.903807615230461	-13.7282031576279\\
0.905811623246493	-17.3856604062041\\
0.907815631262525	-13.1683686075844\\
0.909819639278557	-12.6903398677696\\
0.911823647294589	-12.9064201649017\\
0.913827655310621	-14.3561020272319\\
0.915831663326653	-13.7464199473011\\
0.917835671342685	-13.6852990960235\\
0.919839679358717	-14.7210455648507\\
0.92184368737475	-13.0044234010042\\
0.923847695390782	-12.7697706601367\\
0.925851703406814	-13.2971711258295\\
0.927855711422846	-15.0376908632744\\
0.929859719438878	-13.4439835530866\\
0.93186372745491	-13.9945752327374\\
0.933867735470942	-13.7800789798029\\
0.935871743486974	-12.8536430430944\\
0.937875751503006	-12.861495125051\\
0.939879759519038	-13.8342642126912\\
0.94188376753507	-13.8988731437279\\
0.943887775551102	-13.3526837191452\\
0.945891783567134	-14.4143611492975\\
0.947895791583166	-13.5991199648846\\
0.949899799599198	-13.0275191845606\\
0.95190380761523	-13.4600195719267\\
0.953907815631262	-14.4385489547393\\
0.955911823647295	-13.0441205553209\\
0.957915831663327	-13.0279543941306\\
0.959919839679359	-14.1783444474784\\
0.961923847695391	-13.7860723249799\\
0.963927855711423	-13.4159365031671\\
0.965931863727455	-15.1826110419623\\
0.967935871743487	-13.2831714291208\\
0.969939879759519	-12.7852484821774\\
0.971943887775551	-13.0468332892693\\
0.973947895791583	-15.0099780821279\\
0.975951903807615	-13.5949569311709\\
0.977955911823647	-13.6043293457341\\
0.979959919839679	-14.9448294889398\\
0.981963927855711	-13.0398835673145\\
0.983967935871743	-12.7697863480538\\
0.985971943887776	-13.2009134736091\\
0.987975951903808	-17.4078535904261\\
0.98997995991984	-13.6781129250151\\
0.991983967935872	-14.0926067314705\\
0.993987975951904	-13.9325556857704\\
0.995991983967936	-12.8495169207173\\
0.997995991983968	-12.5726082792283\\
1	-12.5903489322427\\
};
\addlegendentry{Dirichlet condition satisfied};

\addplot [color=mycolor2,solid]
  table[row sep=crcr]{%
0	inf\\
0.00200400801603206	-0.235507561936102\\
0.00400801603206413	-0.974660544164625\\
0.00601202404809619	-1.77002331860044\\
0.00801603206412826	-2.79392023574145\\
0.0100200400801603	-4.48349890639722\\
0.0120240480961924	-5.60708622896124\\
0.0140280561122244	-5.97850110709912\\
0.0160320641282565	-6.38747264552747\\
0.0180360721442886	-5.86150901459141\\
0.0200400801603206	-7.13320622309916\\
0.0220440881763527	-6.39062246572884\\
0.0240480961923848	-6.11683885389976\\
0.0260521042084168	-7.34847127288276\\
0.0280561122244489	-6.62895895851204\\
0.030060120240481	-6.25502228038654\\
0.032064128256513	-7.14885003144352\\
0.0340681362725451	-6.97197761783674\\
0.0360721442885772	-6.36224726971161\\
0.0380761523046092	-6.92507798754622\\
0.0400801603206413	-7.50805153961835\\
0.0420841683366733	-6.48726604003434\\
0.0440881763527054	-6.76409908731904\\
0.0460921843687375	-8.69219445804395\\
0.0480961923847695	-6.65826669135874\\
0.0501002004008016	-6.66946754347287\\
0.0521042084168337	-8.55711720887994\\
0.0541082164328657	-6.90859608436787\\
0.0561122244488978	-6.6405207578471\\
0.0581162324649299	-7.58046864871192\\
0.0601202404809619	-7.29007472609408\\
0.062124248496994	-6.67335520147907\\
0.064128256513026	-7.16430130688034\\
0.0661322645290581	-7.97478468911095\\
0.0681362725450902	-6.77702407703154\\
0.0701402805611222	-6.94543256291524\\
0.0721442885771543	-11.6117185448902\\
0.0741482965931864	-6.96294185400542\\
0.0761523046092184	-6.83775203642071\\
0.0781563126252505	-8.09236887355307\\
0.0801603206412826	-7.27319932705634\\
0.0821643286573146	-6.81956887506905\\
0.0841683366733467	-7.46188507718639\\
0.0861723446893788	-7.80596926749453\\
0.0881763527054108	-6.87921909413854\\
0.0901803607214429	-7.15090730518415\\
0.0921843687374749	-9.16746174758324\\
0.094188376753507	-7.02820559954922\\
0.0961923847695391	-6.9942944999918\\
0.0981963927855711	-8.58486868766222\\
0.100200400801603	-7.28804249034827\\
0.102204408817635	-6.93710899491715\\
0.104208416833667	-7.7086204007833\\
0.106212424849699	-7.74045870575765\\
0.108216432865731	-6.96755210580501\\
0.110220440881764	-7.32398848975706\\
0.112224448897796	-8.71313521859645\\
0.114228456913828	-7.08411245497241\\
0.11623246492986	-7.12227330444252\\
0.118236472945892	-9.13327847827494\\
0.120240480961924	-7.30960952593989\\
0.122244488977956	-7.0355778767\\
0.124248496993988	-7.93164587114055\\
0.12625250501002	-7.70400467839592\\
0.128256513026052	-7.04043183995602\\
0.130260521042084	-7.47115567893755\\
0.132264529058116	-8.49681874502016\\
0.134268537074148	-7.13369398089708\\
0.13627254509018	-7.23202392289957\\
0.138276553106212	-9.88012522381496\\
0.140280561122244	-7.33293889545072\\
0.142284569138277	-7.12130712341721\\
0.144288577154309	-8.14287136952019\\
0.146292585170341	-7.68306868857817\\
0.148296593186373	-7.10431418510988\\
0.150300601202405	-7.60420496948906\\
0.152304609218437	-8.36433866068698\\
0.154308617234469	-7.17929291576212\\
0.156312625250501	-7.33315079502073\\
0.158316633266533	-12.3431703150996\\
0.160320641282565	-7.3514498775414\\
0.162324649298597	-7.19575352156355\\
0.164328657314629	-8.35003105991183\\
0.166332665330661	-7.6698581017521\\
0.168336673346693	-7.16223208547208\\
0.170340681362725	-7.73309181298296\\
0.172344689378758	-8.25792150721424\\
0.17434869739479	-7.21483114680104\\
0.176352705410822	-7.42046368095047\\
0.178356713426854	-10.3566077255583\\
0.180360721442886	-7.37062245319343\\
0.182364729458918	-7.26486780867599\\
0.18436873747495	-8.57232393322109\\
0.186372745490982	-7.6547431656158\\
0.188376753507014	-7.21006586540565\\
0.190380761523046	-7.84782604655091\\
0.192384769539078	-8.18899948675902\\
0.19438877755511	-7.25081181762434\\
0.196392785571142	-7.50639771959401\\
0.198396793587174	-9.64445913324924\\
0.200400801603206	-7.38454765076742\\
0.202404809619238	-7.32573253381569\\
0.204408817635271	-8.80049320546016\\
0.206412825651303	-7.64698745827929\\
0.208416833667335	-7.25761218926345\\
0.210420841683367	-7.9687996696227\\
0.212424849699399	-8.12192263168023\\
0.214428857715431	-7.27962724438643\\
0.216432865731463	-7.58511391705009\\
0.218436873747495	-9.28207226491415\\
0.220440881763527	-7.39751144409735\\
0.222444889779559	-7.38395988935008\\
0.224448897795591	-9.07291709578473\\
0.226452905811623	-7.63469133672107\\
0.228456913827655	-7.2978267331702\\
0.230460921843687	-8.08448645341669\\
0.232464929859719	-8.06697237982979\\
0.234468937875752	-7.30506327758667\\
0.236472945891784	-7.65926226687407\\
0.238476953907816	-9.04413601712081\\
0.240480961923848	-7.40857690966607\\
0.24248496993988	-7.437966585414\\
0.244488977955912	-9.3969906396522\\
0.246492985971944	-7.62367692118157\\
0.248496993987976	-7.33464542416921\\
0.250501002004008	-8.19856009393996\\
0.25250501002004	-8.02323824386963\\
0.254509018036072	-7.33093834758469\\
0.256513026052104	-7.73602907975017\\
0.258517034068136	-8.85270362856958\\
0.260521042084168	-7.41538302535552\\
0.2625250501002	-7.48731548931675\\
0.264529058116232	-9.81172500131905\\
0.266533066132265	-7.6152981905127\\
0.268537074148297	-7.37210425840019\\
0.270541082164329	-8.32795679278352\\
0.272545090180361	-7.97303731619871\\
0.274549098196393	-7.34886929134864\\
0.276553106212425	-7.80430136154493\\
0.278557114228457	-8.7155957867865\\
0.280561122244489	-7.42360183477127\\
0.282565130260521	-7.537760237157\\
0.284569138276553	-10.5478654597989\\
0.286573146292585	-7.60008890374609\\
0.288577154308617	-7.40102261194059\\
0.290581162324649	-8.4464510625235\\
0.292585170340681	-7.93714158199652\\
0.294589178356713	-7.36946457866855\\
0.296593186372746	-7.87766420784752\\
0.298597194388778	-8.59081931825068\\
0.30060120240481	-7.4271518458461\\
0.302605210420842	-7.58230249379553\\
0.304609218436874	-13.1423419175599\\
0.306613226452906	-7.59076042684053\\
0.308617234468938	-7.4333009222776\\
0.31062124248497	-8.58751938953806\\
0.312625250501002	-7.89682295696712\\
0.314629258517034	-7.38567148960471\\
0.316633266533066	-7.94893295756395\\
0.318637274549098	-8.48665438105574\\
0.32064128256513	-7.43048479567711\\
0.322645290581162	-7.62779189638747\\
0.324649298597194	-10.6852385405541\\
0.326653306613226	-7.57705166446402\\
0.328657314629259	-7.46116065041501\\
0.330661322645291	-8.73757993330585\\
0.332665330661323	-7.85827481356524\\
0.334669338677355	-7.39896959152614\\
0.336673346693387	-8.01821847732575\\
0.338677354709419	-8.39783170476813\\
0.340681362725451	-7.43283229287797\\
0.342685370741483	-7.67206248471543\\
0.344689378757515	-9.97048696277751\\
0.346693386773547	-7.56206006031153\\
0.348697394789579	-7.4856322259303\\
0.350701402805611	-8.89367344808721\\
0.352705410821643	-7.82637584237826\\
0.354709418837675	-7.41427072559613\\
0.356713426853707	-8.09516021697197\\
0.35871743486974	-8.31087173030809\\
0.360721442885772	-7.43143878858668\\
0.362725450901804	-7.71242752995744\\
0.364729458917836	-9.59183678273895\\
0.366733466933868	-7.55064040286645\\
0.3687374749499	-7.51370606650773\\
0.370741482965932	-9.09903527281099\\
0.372745490981964	-7.78743122051934\\
0.374749498997996	-7.42333302644899\\
0.376753507014028	-8.16660835924223\\
0.37875751503006	-8.23888791941226\\
0.380761523046092	-7.4318730325035\\
0.382765531062124	-7.75702149794552\\
0.384769539078156	-9.29360783715511\\
0.386773547094188	-7.53253232204417\\
0.38877755511022	-7.53447138463723\\
0.390781563126253	-9.31538534312111\\
0.392785571142285	-7.75529345348398\\
0.394789579158317	-7.43418718817951\\
0.396793587174349	-8.24619927364965\\
0.398797595190381	-8.16683098025819\\
0.400801603206413	-7.42824435299161\\
0.402805611222445	-7.79655867250393\\
0.404809619238477	-9.08696958114872\\
0.406813627254509	-7.51805157469524\\
0.408817635270541	-7.5578431419796\\
0.410821643286573	-9.60609389463783\\
0.412825651302605	-7.7216062839406\\
0.414829659318637	-7.44317578816516\\
0.416833667334669	-8.32934643692533\\
0.418837675350701	-8.10049438833491\\
0.420841683366733	-7.42469352059819\\
0.422845691382766	-7.83873488926206\\
0.424849699398798	-8.90776825000702\\
0.42685370741483	-7.50117226759825\\
0.428857715430862	-7.5799138957278\\
0.430861723446894	-10.0238578716637\\
0.432865731462926	-7.68570047210357\\
0.434869739478958	-7.44885681983242\\
0.43687374749499	-8.41396836232115\\
0.438877755511022	-8.03869471750104\\
0.440881763527054	-7.41990223735026\\
0.442885771543086	-7.88104277817091\\
0.444889779559118	-8.75358137224621\\
0.44689378757515	-7.48176240994687\\
0.448897795591182	-7.59798510561273\\
0.450901803607214	-10.6613198511012\\
0.452905811623247	-7.65410093886647\\
0.454909819639279	-7.45619677992218\\
0.456913827655311	-8.51088688288286\\
0.458917835671343	-7.97699488134462\\
0.460921843687375	-7.41276114799207\\
0.462925851703407	-7.92146421188368\\
0.464929859719439	-8.62644472456235\\
0.466933867735471	-7.46482036913332\\
0.468937875751503	-7.61983327139611\\
0.470941883767535	-15.4719723521434\\
0.472945891783567	-7.61816958992972\\
0.474949899799599	-7.45999364092994\\
0.476953907815631	-8.61288401378114\\
0.478957915831663	-7.9189318411332\\
0.480961923847695	-7.40555369896418\\
0.482965931863727	-7.96594136489307\\
0.48496993987976	-8.50339277335048\\
0.486973947895792	-7.44340049358972\\
0.488977955911824	-7.63716276602092\\
0.490981963927856	-10.7315066574865\\
0.492985971943888	-7.583486190735\\
0.49498997995992	-7.46302813304178\\
0.496993987975952	-8.72651885475103\\
0.498997995991984	-7.861913703504\\
0.501002004008016	-7.39630645489479\\
0.503006012024048	-8.00931784979556\\
0.50501002004008	-8.39569467412447\\
0.507014028056112	-7.42236478658228\\
0.509018036072144	-7.65479671660765\\
0.511022044088176	-10.0267121117923\\
0.513026052104208	-7.54920524765899\\
0.51503006012024	-7.46620240647272\\
0.517034068136273	-8.86076422756294\\
0.519038076152305	-7.80474919666426\\
0.521042084168337	-7.38555918129487\\
0.523046092184369	-8.05450109152679\\
0.525050100200401	-8.29559748091533\\
0.527054108216433	-7.40082488568846\\
0.529058116232465	-7.67398540758075\\
0.531062124248497	-9.58771058072423\\
0.533066132264529	-7.51111103302662\\
0.535070140280561	-7.46562529807613\\
0.537074148296593	-9.00949495273566\\
0.539078156312625	-7.74985200751555\\
0.541082164328657	-7.37434367623095\\
0.543086172344689	-8.10433083246626\\
0.545090180360721	-8.19673498840062\\
0.547094188376754	-7.37524145845498\\
0.549098196392786	-7.68849331376692\\
0.551102204408818	-9.29647819340654\\
0.55310621242485	-7.47512332776107\\
0.555110220440882	-7.46604246436973\\
0.557114228456914	-9.19831576809339\\
0.559118236472946	-7.69310928432245\\
0.561122244488978	-7.36007576982196\\
0.56312625250501	-8.15271495817462\\
0.565130260521042	-8.10803003041033\\
0.567134268537074	-7.3509584482325\\
0.569138276553106	-7.7061646241796\\
0.571142284569138	-9.0515649099177\\
0.57314629258517	-7.43648035595085\\
0.575150300601202	-7.46452225399476\\
0.577154308617234	-9.4378037394073\\
0.579158316633267	-7.63661687571498\\
0.581162324649299	-7.344976982496\\
0.583166332665331	-8.20737779295436\\
0.585170340681363	-8.01920675139549\\
0.587174348697395	-7.32385530426077\\
0.589178356713427	-7.72308377633984\\
0.591182364729459	-8.84323020815959\\
0.593186372745491	-7.39483463031137\\
0.595190380761523	-7.45957326911938\\
0.597194388777555	-9.75047409323781\\
0.599198396793587	-7.580855285502\\
0.601202404809619	-7.32869238725691\\
0.603206412825651	-8.26792807694588\\
0.605210420841683	-7.93087824224084\\
0.607214428857715	-7.29313327052219\\
0.609218436873748	-7.73560028663538\\
0.61122244488978	-8.67524993411321\\
0.613226452905812	-7.35543967607287\\
0.615230460921844	-7.45711398507954\\
0.617234468937876	-10.294977638138\\
0.619238476953908	-7.52081935586482\\
0.62124248496994	-7.30801259555933\\
0.623246492985972	-8.32672954912365\\
0.625250501002004	-7.85035852209887\\
0.627254509018036	-7.26473090137946\\
0.629258517034068	-7.75593284967552\\
0.6312625250501	-8.50400778538737\\
0.633266533066132	-7.3088467839121\\
0.635270541082164	-7.44809409964057\\
0.637274549098196	-11.4820674106859\\
0.639278557114228	-7.46328368385187\\
0.641282565130261	-7.28857955546073\\
0.643286573146293	-8.40451523056875\\
0.645290581162325	-7.76180037882009\\
0.647294589178357	-7.22867182118149\\
0.649298597194389	-7.76754408328991\\
0.651302605210421	-8.36211969210199\\
0.653306613226453	-7.26399021007989\\
0.655310621242485	-7.44079922695789\\
0.657314629258517	-11.009351697848\\
0.659318637274549	-7.40110003345599\\
0.661322645290581	-7.26421908368735\\
0.663326653306613	-8.48405932372488\\
0.665330661322645	-7.67787406325284\\
0.667334669338677	-7.19249281624483\\
0.669338677354709	-7.78237879228958\\
0.671342685370741	-8.2241161345364\\
0.673346693386774	-7.21589907383968\\
0.675350701402806	-7.43166594478544\\
0.677354709418838	-9.98930896716567\\
0.67935871743487	-7.33601958889788\\
0.681362725450902	-7.23657403197951\\
0.683366733466934	-8.57409740110933\\
0.685370741482966	-7.59419559322339\\
0.687374749498998	-7.15468213094934\\
0.68937875751503	-7.80141576668253\\
0.691382765531062	-8.08423085350266\\
0.693386773547094	-7.16119111956198\\
0.695390781563126	-7.41564586619206\\
0.697394789579158	-9.49501455540924\\
0.69939879759519	-7.27179384441175\\
0.701402805611222	-7.20919148088981\\
0.703406813627255	-8.69775758068148\\
0.705410821643287	-7.50255581079744\\
0.707414829659319	-7.10854573325885\\
0.709418837675351	-7.810157681914\\
0.711422845691383	-7.96222201388744\\
0.713426853707415	-7.1084333249066\\
0.715430861723447	-7.40380310560291\\
0.717434869739479	-9.09993790652872\\
0.719438877755511	-7.19871735736919\\
0.721442885771543	-7.17291880622209\\
0.723446893787575	-8.82220360523195\\
0.725450901803607	-7.4167936283568\\
0.727454909819639	-7.06439098232328\\
0.729458917835671	-7.83155112483363\\
0.731462925851703	-7.82796955299247\\
0.733466933867735	-7.04597739623967\\
0.735470941883767	-7.38233761374288\\
0.7374749498998	-8.81485057093311\\
0.739478957915832	-7.12634996438994\\
0.741482965931864	-7.13739311326116\\
0.743486973947896	-9.01539325374587\\
0.745490981963928	-7.32047639999997\\
0.74749498997996	-7.01065150942582\\
0.749498997995992	-7.84424237161113\\
0.751503006012024	-7.70170999132006\\
0.753507014028056	-6.98129069544127\\
0.755511022044088	-7.36034682439751\\
0.75751503006012	-8.55663624778093\\
0.759519038076152	-7.04649535414476\\
0.761523046092184	-7.09431376720603\\
0.763527054108216	-9.26523761681341\\
0.765531062124248	-7.22249128587503\\
0.767535070140281	-6.95265318320389\\
0.769539078156313	-7.85845722981958\\
0.771543086172345	-7.57384493775225\\
0.773547094188377	-6.91135716712665\\
0.775551102204409	-7.33556030447882\\
0.777555110220441	-8.31969808065795\\
0.779559118236473	-6.95954984994899\\
0.781563126252505	-7.04419269158207\\
0.783567134268537	-9.63781072021548\\
0.785571142284569	-7.12101605092851\\
0.787575150300601	-6.8904892440374\\
0.789579158316633	-7.882046850756\\
0.791583166332665	-7.43552791216588\\
0.793587174348697	-6.83041499616987\\
0.795591182364729	-7.29969214581723\\
0.797595190380762	-8.11145523143969\\
0.799599198396794	-6.8688251074125\\
0.801603206412826	-6.99125728019988\\
0.803607214428858	-10.5471862774739\\
0.80561122244489	-7.00616774466216\\
0.807615230460922	-6.81475847982307\\
0.809619238476954	-7.89124608374609\\
0.811623246492986	-7.30149149755944\\
0.813627254509018	-6.74591072748995\\
0.81563126252505	-7.26515343381149\\
0.817635270541082	-7.89339308934662\\
0.819639278557114	-6.76396014263739\\
0.821643286573146	-6.92367881744866\\
0.823647294589178	-11.1895684788272\\
0.82565130260521	-6.88891775694175\\
0.827655310621243	-6.7349237455744\\
0.829659318637275	-7.9195506500606\\
0.831663326653307	-7.150306892975\\
0.833667334669339	-6.64590756872404\\
0.835671342685371	-7.21543691065716\\
0.837675350701403	-7.68908822622654\\
0.839679358717435	-6.65149851846314\\
0.841683366733467	-6.84995757927087\\
0.843687374749499	-9.58668979603057\\
0.845691382765531	-6.75468402831422\\
0.847695390781563	-6.63859376262279\\
0.849699398797595	-7.94075433052566\\
0.851703406813627	-6.99209130291932\\
0.853707414829659	-6.53425411557123\\
0.855711422845691	-7.1583006526836\\
0.857715430861723	-7.4743384442523\\
0.859719438877755	-6.52237979198658\\
0.861723446893788	-6.76016856366732\\
0.86372745490982	-8.90074159401312\\
0.865731462925852	-6.60582853241698\\
0.867735470941884	-6.52657984386304\\
0.869739478957916	-7.96754337012134\\
0.871743486973948	-6.81820098325661\\
0.87374749498998	-6.40518160029287\\
0.875751503006012	-7.08878385197346\\
0.877755511022044	-7.24555895568427\\
0.879759519038076	-6.3724428329072\\
0.881763527054108	-6.65001008747217\\
0.88376753507014	-8.40519023428839\\
0.885771543086172	-6.4376194727638\\
0.887775551102204	-6.39532782726281\\
0.889779559118236	-8.0191387092722\\
0.891783567134269	-6.61744822649013\\
0.893787575150301	-6.24883573926579\\
0.895791583166333	-6.99425491718211\\
0.897795591182365	-7.0023557554218\\
0.899799599198397	-6.19818089770152\\
0.901803607214429	-6.51831385333498\\
0.903807615230461	-7.9441380709213\\
0.905811623246493	-6.23491620980676\\
0.907815631262525	-6.22951959817323\\
0.909819639278557	-8.07382726985232\\
0.911823647294589	-6.38834598833847\\
0.913827655310621	-6.06112263222419\\
0.915831663326653	-6.8794351913388\\
0.917835671342685	-6.71654137727368\\
0.919839679358717	-5.98049698144288\\
0.92184368737475	-6.34365672466245\\
0.923847695390782	-7.51163167229772\\
0.925851703406814	-5.9906011833151\\
0.927855711422846	-6.02215736893183\\
0.929859719438878	-8.21530998074467\\
0.93186372745491	-6.10272292868078\\
0.933867735470942	-5.81711873466177\\
0.935871743486974	-6.71613539912436\\
0.937875751503006	-6.37663637010156\\
0.939879759519038	-5.70230384262635\\
0.94188376753507	-6.11299618335683\\
0.943887775551102	-7.01981596996412\\
0.945891783567134	-5.66837661953634\\
0.947895791583166	-5.73790827626596\\
0.949899799599198	-8.59132212167169\\
0.95190380761523	-5.72740348900664\\
0.953907815631262	-5.48219037719545\\
0.955911823647295	-6.48551186355957\\
0.957915831663327	-5.92100304270964\\
0.959919839679359	-5.30787169499227\\
0.961923847695391	-5.77289168528095\\
0.963927855711423	-6.39670787056771\\
0.965931863727455	-5.19946597425353\\
0.967935871743487	-5.30896848468158\\
0.969939879759519	-9.1609144008643\\
0.971943887775551	-5.15634561725309\\
0.973947895791583	-4.9517137785111\\
0.975951903807615	-6.13209679738126\\
0.977955911823647	-5.18859375903984\\
0.979959919839679	-4.63851916959005\\
0.981963927855711	-5.18754951098934\\
0.983967935871743	-5.34082880704451\\
0.985971943887776	-4.31628948520467\\
0.987975951903808	-4.48167947509354\\
0.98997995991984	-5.75950718279556\\
0.991983967935872	-3.87161573817541\\
0.993987975951904	-3.70906504726515\\
0.995991983967936	-6.42723210384824\\
0.997995991983968	-2.78476391908252\\
1	-1.93693533585192\\
};
\addlegendentry{Dirichlet condition not satisfied};

\end{axis}
\end{tikzpicture}%%
\endgroup
\end{frame}

\subsection{Extensions}

\begin{frame}{Conditions de Robin}
Cas des conditions de Robin pour $c \geq 0$
\begin{itemize}
\item[-] Opérateur $\Delta_c$ sur $\Hrad$ toujours auto-adjoint 
\item[-] Valeurs propres positives ou nulles (pas le cas pour $c <0$)
\item[-] Condition de Robin à ajuster sur la fonction à approximer. 
\end{itemize}
\end{frame}

\begin{frame}{Preuve d'un contrôle $L^{\infty}$}
Au tableau
\end{frame}

\subsection{Résultats numériques}

\begin{frame}{Quadrature radiale pour le $\log$}
\begingroup
\centering
\tikzset{every picture/.style={scale=0.88}}%
% This file was created by matlab2tikz.
%
%The latest updates can be retrieved from
%  http://www.mathworks.com/matlabcentral/fileexchange/22022-matlab2tikz-matlab2tikz
%where you can also make suggestions and rate matlab2tikz.
%
\definecolor{mycolor1}{rgb}{0.00000,0.44700,0.74100}%
\definecolor{mycolor2}{rgb}{0.85000,0.32500,0.09800}%
\definecolor{mycolor3}{rgb}{0.92900,0.69400,0.12500}%
%
\begin{tikzpicture}

\begin{axis}[%
width=3.945in,
height=3.088in,
at={(1.334in,0.959in)},
scale only axis,
unbounded coords=jump,
xmin=0,
xmax=1000,
xlabel={Number of components $P$},
xmajorgrids,
ymode=log,
ymin=1e-12,
ymax=51,
yminorticks=true,
ylabel={Logarithmic error $\log(\varepsilon)$},
ymajorgrids,
yminorgrids,
axis background/.style={fill=white},
legend style={legend cell align=left,align=left,fill=none,draw=none}
]
\addplot [color=mycolor1,solid,line width=2.0pt]
  table[row sep=crcr]{%
10	0.263225400198431\\
90	0.0297102534119276\\
170	0.0102576565593497\\
250	0.00659898037029327\\
330	0.0041411003149312\\
410	0.00310889454079488\\
490	0.00250207181981255\\
570	0.00183865743152056\\
650	0.00151224702639796\\
730	0.00124483332891412\\
810	0.0010059566873224\\
890	0.00102602879388058\\
970	0.000714536732337567\\
};
\addlegendentry{Truncated Fourier-Bessel of G};

\addplot [color=mycolor2,solid,line width=2.0pt]
  table[row sep=crcr]{%
10	0.473497779052663\\
90	0.00727899056456138\\
170	0.00185952657621957\\
250	0.000236952574207194\\
330	1.90122394090331e-05\\
410	1.21889799009622e-06\\
490	3.29399664877883e-07\\
570	6.28558236570598e-08\\
650	1.20776539880296e-08\\
730	1.5499921346418e-09\\
810	1.88231652487048e-10\\
890	1.27994947973775e-10\\
970	3.5145220067534e-11\\
};
\addlegendentry{Truncated Fourier-Bessel of polynomial extension};

\addplot [color=mycolor3,solid,line width=2.0pt]
  table[row sep=crcr]{%
10	0.353859623754497\\
90	2.68568085779464e-05\\
170	7.38350980356017e-09\\
250	5.79021275370906e-11\\
330	nan\\
410	nan\\
490	nan\\
570	nan\\
650	nan\\
730	nan\\
810	nan\\
890	nan\\
970	nan\\
};
\addlegendentry{Gram-Schmidt};

\end{axis}
\end{tikzpicture}%%
\endgroup
\end{frame}
\begin{frame}{Quadrature radiale pour le $\log$}
\begingroup
\centering
\tikzset{every picture/.style={scale=0.88}}%
% This file was created by matlab2tikz.
%
%The latest updates can be retrieved from
%  http://www.mathworks.com/matlabcentral/fileexchange/22022-matlab2tikz-matlab2tikz
%where you can also make suggestions and rate matlab2tikz.
%
\definecolor{mycolor1}{rgb}{0.00000,0.44700,0.74100}%
\definecolor{mycolor2}{rgb}{0.85000,0.32500,0.09800}%
\definecolor{mycolor3}{rgb}{0.92900,0.69400,0.12500}%
%
\begin{tikzpicture}

\begin{axis}[%
width=4.234in,
height=3.091in,
at={(1.045in,0.956in)},
scale only axis,
xmin=0,
xmax=96.6053053182372,
xlabel={$\rho_p$},
ymin=-20,
ymax=4,
ylabel={$\log(|\alpha_p|)$},
axis background/.style={fill=white},
legend style={legend cell align=left,align=left,fill=none,draw=none}
]
\addplot[only marks,mark=o,mark options={},mark size=1.5000pt,color=mycolor1] plot table[row sep=crcr]{%
2.40482555762315	-0.0595560186878546\\
5.52007810274485	-1.29892142374055\\
8.65372791289424	-1.97218007969845\\
11.7915344388313	-2.43589439418346\\
14.9309037381854	-2.78980664404281\\
18.0710639678808	-3.07603559603896\\
21.211626861317	-3.31634047814016\\
24.3524495360216	-3.52343229092601\\
27.4934725322526	-3.70538296001701\\
30.6345898998327	-3.86763910444304\\
33.7758159545786	-4.01404963200899\\
36.917084913022	-4.14743519676087\\
40.0584232777753	-4.26992462969729\\
43.1997802848901	-4.38316414330123\\
46.3411872641951	-4.48845299808829\\
49.4825998634098	-4.58683461918007\\
52.6240518346981	-4.67915963917424\\
55.7655017400859	-4.76613064610321\\
58.9069848173393	-4.84833465886013\\
62.0484609504762	-4.92626715959704\\
65.1899664364413	-5.00035018724746\\
68.3314616987713	-5.07094617054835\\
71.472983868065	-5.13836865040029\\
74.6144935024883	-5.20289069491361\\
77.7560284312404	-5.26475157841005\\
80.8975491323772	-5.32416213716151\\
84.0390940409322	-5.38130910449939\\
87.1806234414138	-5.4363586501388\\
90.3221763051478	-5.48945929281651\\
93.4637126646474	-5.54074431485235\\
};
\addlegendentry{Truncated Fourier-Bessel of G};

\addplot[only marks,mark=o,mark options={},mark size=1.5000pt,color=mycolor2] plot table[row sep=crcr]{%
2.40482555762315	-0.0231838178905557\\
5.52007810274485	-1.61535852964004\\
8.65372791289424	-3.2997785403795\\
11.7915344388313	-3.92163929254402\\
14.9309037381854	-3.27585161655716\\
18.0710639678808	-3.29180320065408\\
21.211626861317	-3.58350749665352\\
24.3524495360216	-4.16085366882024\\
27.4934725322526	-5.40187997600935\\
30.6345898998327	-5.70396506211945\\
33.7758159545786	-4.92756813027028\\
36.917084913022	-4.87066720372305\\
40.0584232777753	-5.16018044719195\\
43.1997802848901	-5.86791386494028\\
46.3411872641951	-8.92563252093615\\
49.4825998634098	-6.45271352136794\\
52.6240518346981	-6.18990871930147\\
55.7655017400859	-6.50606359148113\\
58.9069848173393	-7.84113308543085\\
62.0484609504762	-7.25341319951939\\
65.1899664364413	-6.5793678281535\\
68.3314616987713	-6.52570184397092\\
71.472983868065	-6.90534568022029\\
74.6144935024883	-8.31113534777972\\
77.7560284312404	-7.61375004073648\\
80.8975491323772	-6.95188808123909\\
84.0390940409322	-6.88949537982059\\
87.1806234414138	-7.24946050129589\\
90.3221763051478	-8.56759870375677\\
93.4637126646474	-8.01109815285378\\
};
\addlegendentry{Truncated Fourier-Bessel of polynomial extension};

\addplot[only marks,mark=o,mark options={},mark size=1.5000pt,color=mycolor3] plot table[row sep=crcr]{%
2.40482555762315	-0.052408399766762\\
5.52007810274485	-1.31665841112951\\
8.65372791289424	-2.03476760128999\\
11.7915344388313	-2.56344881184711\\
14.9309037381854	-3.00265144707083\\
18.0710639678808	-3.3947692499593\\
21.211626861317	-3.76190961569123\\
24.3524495360216	-4.11721108143688\\
27.4934725322526	-4.46926024598518\\
30.6345898998327	-4.82411531720457\\
33.7758159545786	-5.18634598224769\\
36.917084913022	-5.55961827025646\\
40.0584232777753	-5.94705037601962\\
43.1997802848901	-6.35144642065153\\
46.3411872641951	-6.77546347280186\\
49.4825998634098	-7.22174313111453\\
52.6240518346981	-7.69302747737406\\
55.7655017400859	-8.19227408853041\\
58.9069848173393	-8.72278354351784\\
62.0484609504762	-9.28835477543871\\
65.1899664364413	-9.89348928383497\\
68.3314616987713	-10.5436768090679\\
71.472983868065	-11.245817729079\\
74.6144935024883	-12.0088831109859\\
77.7560284312404	-12.8450115103249\\
80.8975491323772	-13.7714725961456\\
84.0390940409322	-14.8145417661211\\
87.1806234414138	-16.0182637778206\\
90.3221763051478	-17.4689726486279\\
93.4637126646474	-19.3975469826778\\
};
\addlegendentry{Gram-Schmidt};

\end{axis}
\end{tikzpicture}%%
\endgroup
\end{frame}

\begin{frame}{Importance du paramètre $a$ pour la fonction $\log$}
\begingroup
\centering
\tikzset{every picture/.style={scale=0.88}}%
% This file was created by matlab2tikz.
%
%The latest updates can be retrieved from
%  http://www.mathworks.com/matlabcentral/fileexchange/22022-matlab2tikz-matlab2tikz
%where you can also make suggestions and rate matlab2tikz.
%
\definecolor{mycolor1}{rgb}{0.00000,0.44700,0.74100}%
\definecolor{mycolor2}{rgb}{0.85000,0.32500,0.09800}%
\definecolor{mycolor3}{rgb}{0.92900,0.69400,0.12500}%
\definecolor{mycolor4}{rgb}{0.49400,0.18400,0.55600}%
\definecolor{mycolor5}{rgb}{0.46600,0.67400,0.18800}%
\definecolor{mycolor6}{rgb}{0.30100,0.74500,0.93300}%
%
\begin{tikzpicture}

\begin{axis}[%
width=4.236in,
height=3.093in,
at={(1.043in,0.954in)},
scale only axis,
unbounded coords=jump,
xmin=0,
xmax=100,
xlabel={Number of components P},
xmajorgrids,
ymin=-20,
ymax=10,
ylabel={Logarithmic error of approximation},
ymajorgrids,
axis background/.style={fill=white},
legend style={legend cell align=left,align=left,fill=none,draw=none}
]
\addplot [color=mycolor1,solid,line width=2.0pt]
  table[row sep=crcr]{%
5	0.961440878714007\\
10	0.673360866045236\\
15	0.458284921521856\\
20	0.276771407284532\\
25	0.114971366976436\\
30	-0.0337241578470986\\
35	-0.173025581920097\\
40	-0.305234259750934\\
45	-0.431886825962707\\
50	-0.554058294239057\\
55	-0.672537409047838\\
60	-0.787914903630546\\
65	-0.900651763684522\\
70	-1.01110796366996\\
75	-1.11957766093215\\
80	-1.22629642616378\\
85	-1.33146436768367\\
90	-1.43524381619336\\
95	-1.53777778467737\\
100	-1.63918251816429\\
};
\addlegendentry{a = 0.005};

\addplot [color=mycolor2,solid,line width=2.0pt]
  table[row sep=crcr]{%
5	0.0823903458467554\\
10	-0.579183389490567\\
15	-1.14207118309321\\
20	-1.6603464150768\\
25	-2.15284911879246\\
30	-2.62847159422771\\
35	-3.09215967152659\\
40	-3.54693833243047\\
45	-3.99482282262146\\
50	-4.43718723844779\\
55	-4.87506402863437\\
60	-5.30917165264094\\
65	-5.74013827103851\\
70	-6.16834527588657\\
75	-6.59426054286486\\
80	-7.01802343205212\\
85	-7.44010487519762\\
90	-7.86036716481114\\
95	-8.27948413010206\\
100	-8.69684745120324\\
};
\addlegendentry{a = 0.025};

\addplot [color=mycolor3,solid,line width=2.0pt]
  table[row sep=crcr]{%
5	-0.485253533759817\\
10	-1.47974575847842\\
15	-2.36635615189124\\
20	-3.20482884063567\\
25	-4.01575921132754\\
30	-4.80877808413405\\
35	-5.58921641578002\\
40	-6.36028652619429\\
45	-7.12418303714923\\
50	-7.882249939797\\
55	-8.63582230312543\\
60	-9.38511573218826\\
65	-10.1319463371574\\
70	-10.8741417712837\\
75	-11.617378689794\\
80	-12.349286437168\\
85	-13.0905654973904\\
90	-13.7723847791087\\
95	-14.4290600843698\\
100	-14.7060355047458\\
};
\addlegendentry{a = 0.045};

\addplot [color=mycolor4,solid,line width=2.0pt]
  table[row sep=crcr]{%
5	-1.64420614723545\\
10	-3.446838042014\\
15	-5.1319296947879\\
20	-6.76552188437996\\
25	-8.36979784182386\\
30	-9.95588776785829\\
35	-11.5266676503659\\
40	-13.1001891566826\\
45	nan\\
50	nan\\
55	nan\\
60	nan\\
65	nan\\
70	nan\\
75	nan\\
80	nan\\
85	nan\\
90	nan\\
95	nan\\
100	nan\\
};
\addlegendentry{a = 0.095};

\addplot [color=mycolor5,solid,line width=2.0pt]
  table[row sep=crcr]{%
5	-2.06800841111862\\
10	-4.19258088715578\\
15	-6.19763046385295\\
20	-8.15067431377338\\
25	-10.0732586228132\\
30	-11.9829416100451\\
35	-13.8664060738234\\
40	nan\\
45	nan\\
50	nan\\
55	nan\\
60	nan\\
65	nan\\
70	nan\\
75	nan\\
80	nan\\
85	nan\\
90	nan\\
95	nan\\
100	nan\\
};
\addlegendentry{a = 0.115};

\addplot [color=mycolor6,solid,line width=2.0pt]
  table[row sep=crcr]{%
5	-6.74285882982651\\
10	-12.7859772294086\\
15	-15.1806747936138\\
20	nan\\
25	nan\\
30	nan\\
35	nan\\
40	nan\\
45	nan\\
50	nan\\
55	nan\\
60	nan\\
65	nan\\
70	nan\\
75	nan\\
80	nan\\
85	nan\\
90	nan\\
95	nan\\
100	nan\\
};
\addlegendentry{a = 0.345};

\end{axis}
\end{tikzpicture}%%
\endgroup
\end{frame}
\begin{frame}{Importance du paramètre $a$ pour la fonction $Y_0$}
\begin{center}
\begingroup
\centering
\tikzset{every picture/.style={scale=0.88}}%
% This file was created by matlab2tikz.
%
%The latest updates can be retrieved from
%  http://www.mathworks.com/matlabcentral/fileexchange/22022-matlab2tikz-matlab2tikz
%where you can also make suggestions and rate matlab2tikz.
%
\definecolor{mycolor1}{rgb}{0.00000,0.44700,0.74100}%
\definecolor{mycolor2}{rgb}{0.85000,0.32500,0.09800}%
\definecolor{mycolor3}{rgb}{0.92900,0.69400,0.12500}%
\definecolor{mycolor4}{rgb}{0.49400,0.18400,0.55600}%
\definecolor{mycolor5}{rgb}{0.46600,0.67400,0.18800}%
\definecolor{mycolor6}{rgb}{0.30100,0.74500,0.93300}%
%
\begin{tikzpicture}

\begin{axis}[%
width=4.236in,
height=3.093in,
at={(1.043in,0.954in)},
scale only axis,
unbounded coords=jump,
xmin=0,
xmax=100,
xlabel={Number of components P},
xmajorgrids,
ymin=-15,
ymax=5,
ylabel={Logarithmic error of approximation},
ymajorgrids,
axis background/.style={fill=white},
legend style={legend cell align=left,align=left,fill=none,draw=none}
]
\addplot [color=mycolor1,solid,line width=2.0pt]
  table[row sep=crcr]{%
5	0.435979365283817\\
10	0.256092146380958\\
15	0.0601311415953336\\
20	-0.140769783158704\\
25	-0.312167561515658\\
30	-0.466482970913044\\
35	-0.609426787145836\\
40	-0.744167065513137\\
45	-0.872668008311448\\
50	-0.996243649718126\\
55	-1.11581977850619\\
60	-1.23207780575998\\
65	-1.34553307028898\\
70	-1.45658814106725\\
75	-1.56556117214105\\
80	-1.67271193200002\\
85	-1.77824962453398\\
90	-1.88235326937547\\
95	-1.98517103415603\\
100	-2.08682808589674\\
};
\addlegendentry{a = 0.005};

\addplot [color=mycolor2,solid,line width=2.0pt]
  table[row sep=crcr]{%
5	-0.596668601743884\\
10	-0.966327805568993\\
15	-1.44859648294285\\
20	-2.01194874073806\\
25	-2.52834523845899\\
30	-3.0187418829607\\
35	-3.49244047072114\\
40	-3.95445530591144\\
45	-4.40778756166549\\
50	-4.85443146401809\\
55	-5.29570817937261\\
60	-5.73265383019654\\
65	-6.16591470717832\\
70	-6.59617507118382\\
75	-7.02368861326013\\
80	-7.44908282347961\\
85	-7.87222335930877\\
90	-8.29396284771822\\
95	-8.7135981994232\\
100	-9.13261222791123\\
};
\addlegendentry{a = 0.025};

\addplot [color=mycolor3,solid,line width=2.0pt]
  table[row sep=crcr]{%
5	-1.56680711198865\\
10	-1.9399143272388\\
15	-2.58707613720019\\
20	-3.49356981832179\\
25	-4.34148392910766\\
30	-5.15781580041051\\
35	-5.95423902155605\\
40	-6.73708094312773\\
45	-7.50973813241832\\
50	-8.27513675251596\\
55	-9.03373818025513\\
60	-9.78902479342674\\
65	-10.5370994258816\\
70	-11.2882485442312\\
75	-12.0234042176922\\
80	-12.7827771742811\\
85	-13.4859051571851\\
90	-14.29735933284\\
95	-14.6623272250937\\
100	-14.6621058918815\\
};
\addlegendentry{a = 0.045};

\addplot [color=mycolor4,solid,line width=2.0pt]
  table[row sep=crcr]{%
5	-1.50050453157682\\
10	-2.64590533443851\\
15	-5.14284386092929\\
20	-6.89968242039218\\
25	-8.57248001682411\\
30	-10.2040964380133\\
35	-11.7987089717776\\
40	-13.4241061390254\\
45	-14.6574250342176\\
50	-14.6584783871581\\
55	-14.658105126748\\
60	-14.6664450399974\\
65	-14.6595304847321\\
70	-14.6738088422698\\
75	-14.6852552267502\\
80	-14.6951493333541\\
85	-14.7068841409367\\
90	-14.7115686925412\\
95	-14.7165726985756\\
100	-14.7524207967554\\
};
\addlegendentry{a = 0.095};

\addplot [color=mycolor5,solid,line width=2.0pt]
  table[row sep=crcr]{%
5	-1.41347651643263\\
10	-2.38722884182496\\
15	-6.12582944905123\\
20	-8.22381825642789\\
25	-10.2265121088053\\
30	-12.1965153953624\\
35	-14.0458071460169\\
40	-14.6552481351647\\
45	-14.6553155417244\\
50	-14.6818560846755\\
55	-14.2552956636878\\
60	-14.1696439646056\\
65	-14.170408155681\\
70	-14.2027542522096\\
75	-14.2441972197978\\
80	-14.2921245144565\\
85	-14.3297009781386\\
90	-14.3638872877249\\
95	-14.4156287608613\\
100	-14.4456050879567\\
};
\addlegendentry{a = 0.115};

\addplot [color=mycolor6,solid,line width=2.0pt]
  table[row sep=crcr]{%
5	-3.91130425213666\\
10	-9.6695755164259\\
15	-14.5481490031608\\
20	nan\\
25	nan\\
30	nan\\
35	nan\\
40	nan\\
45	nan\\
50	nan\\
55	nan\\
60	nan\\
65	nan\\
70	nan\\
75	nan\\
80	nan\\
85	nan\\
90	nan\\
95	nan\\
100	nan\\
};
\addlegendentry{a = 0.345};

\end{axis}
\end{tikzpicture}%%
\endgroup
\end{center}

\end{frame}

\section{Complexité de la méthode}

\begin{frame}{Complexité de la méthode}
\begin{itemize}
\item[-] Champ lointain  $O(P^2)$ ($O(P^3)$ en dimensions $3$).
\item[-] Champ proche $O(a^2N^2)$
\item[-] Calcul de la quadrature $O(P^3)$ (mais précalcul uniquement)
\item[-] Choix de la constante $a$ détermine la complexité globale. 
\end{itemize}
\[ P = O\left( -\dfrac{\log(\varepsilon)}{a}\right) \quad a =  O\left(\dfrac{-\log(\varepsilon)}{\sqrt{N}}\right)\]

Complexité globale : 
\begin{itemize}
\item[-] Pré-calcul : $O(N^{3/2})$
\item[-] Calcul : $O(N \log(N) )$
\end{itemize}
\end{frame}

\subsection{Premiers résultats numériques}
\begin{frame}{Tests numériques du temps de calcul}
\begingroup
\centering
\tikzset{every picture/.style={scale=0.88}}%
% This file was created by matlab2tikz.
%
%The latest updates can be retrieved from
%  http://www.mathworks.com/matlabcentral/fileexchange/22022-matlab2tikz-matlab2tikz
%where you can also make suggestions and rate matlab2tikz.
%
\definecolor{mycolor1}{rgb}{0.00000,0.44700,0.74100}%
\definecolor{mycolor2}{rgb}{0.85000,0.32500,0.09800}%
\definecolor{mycolor3}{rgb}{1.00000,1.00000,0.00000}%
\definecolor{mycolor4}{rgb}{0.92900,0.69400,0.12500}%
%
\begin{tikzpicture}

\begin{axis}[%
width=4.162in,
height=2.579in,
at={(1.118in,1.467in)},
scale only axis,
xmin=0,
xmax=60000,
xlabel={Number of charges},
ymin=0,
ymax=150,
ylabel={Computational time (s)},
axis background/.style={fill=white},
legend style={legend cell align=left,align=left,fill=none,draw=none}
]
\addplot [color=blue,line width=2.0pt,only marks,mark=o,mark options={solid},forget plot]
  table[row sep=crcr]{%
1000	0.305570494320304\\
5000	1.47242709996839\\
10000	2.97897219473946\\
20000	6.21374662851466\\
30000	9.16623095649604\\
40000	12.5695386042769\\
50000	15.717194267346\\
};
\addplot [color=mycolor1,solid,line width=2.0pt]
  table[row sep=crcr]{%
1000	0.305570494320304\\
5000	1.47242709996839\\
10000	2.97897219473946\\
20000	6.21374662851466\\
30000	9.16623095649604\\
40000	12.5695386042769\\
50000	15.717194267346\\
};
\addlegendentry{Far-field};

\addplot [color=red,line width=2.0pt,only marks,mark=o,mark options={solid},forget plot]
  table[row sep=crcr]{%
1000	0.232273068759827\\
5000	1.84296575761432\\
10000	5.03760504460419\\
20000	14.7489751095091\\
30000	27.9000160107066\\
40000	42.3953876848929\\
50000	91.9921247849844\\
};
\addplot [color=mycolor2,solid,line width=2.0pt]
  table[row sep=crcr]{%
1000	0.232273068759827\\
5000	1.84296575761432\\
10000	5.03760504460419\\
20000	14.7489751095091\\
30000	27.9000160107066\\
40000	42.3953876848929\\
50000	91.9921247849844\\
};
\addlegendentry{Close-field};

\addplot [color=mycolor3,line width=2.0pt,only marks,mark=o,mark options={solid},forget plot]
  table[row sep=crcr]{%
1000	2.83523874426796\\
5000	13.2894218492777\\
10000	25.1053738500002\\
20000	47.9341779323199\\
30000	70.8491307007353\\
40000	90.9308813688744\\
50000	117.888789631631\\
};
\addplot [color=mycolor4,solid,line width=2.0pt]
  table[row sep=crcr]{%
1000	2.83523874426796\\
5000	13.2894218492777\\
10000	25.1053738500002\\
20000	47.9341779323199\\
30000	70.8491307007353\\
40000	90.9308813688744\\
50000	117.888789631631\\
};
\addlegendentry{Assembling};

\end{axis}
\end{tikzpicture}%%
\endgroup
\end{frame}

\subsection{Quelques mandalas}
\begin{frame}{Quelques jolies figures}
\begin{center}
Champ Coulombien rayonné par des charges aléatoires disposées sur un cercle (grille 500x500, 1000 charges). 
\begingroup
\centering
\tikzset{every picture/.style={scale=0.75}}%
% This file was created by matlab2tikz.
%
%The latest updates can be retrieved from
%  http://www.mathworks.com/matlabcentral/fileexchange/22022-matlab2tikz-matlab2tikz
%where you can also make suggestions and rate matlab2tikz.
%
\begin{tikzpicture}

\begin{axis}[%
width=4.521in,
height=3.566in,
at={(0.758in,0.481in)},
scale only axis,
axis on top,
xmin=-1.00200400801603,
xmax=1.00200400801603,
xlabel={x},
y dir=reverse,
ymin=-1.00200400801603,
ymax=1.00200400801603,
ylabel={y},
axis background/.style={fill=white}
]
\addplot [forget plot] graphics [xmin=-1.00200400801603,xmax=1.00200400801603,ymin=-1.00200400801603,ymax=1.00200400801603] {CircleLaplace-1.png};
\end{axis}
\end{tikzpicture}%%
\endgroup
\end{center}

\end{frame}
\begin{frame}{Quelques jolies figures}
\begin{center}
Champ Helmholtzien rayonné par des sources aléatoires disposées sur un cercle (grille 500x500, 1000 charges) 
\begingroup
\centering
\tikzset{every picture/.style={scale=0.75}}%
% This file was created by matlab2tikz.
%
%The latest updates can be retrieved from
%  http://www.mathworks.com/matlabcentral/fileexchange/22022-matlab2tikz-matlab2tikz
%where you can also make suggestions and rate matlab2tikz.
%
\begin{tikzpicture}

\begin{axis}[%
width=4.521in,
height=3.566in,
at={(0.758in,0.481in)},
scale only axis,
axis on top,
xmin=-1.00200400801603,
xmax=1.00200400801603,
xlabel={x},
y dir=reverse,
ymin=-1.00200400801603,
ymax=1.00200400801603,
ylabel={y},
axis background/.style={fill=white}
]
\addplot [forget plot] graphics [xmin=-1.00200400801603,xmax=1.00200400801603,ymin=-1.00200400801603,ymax=1.00200400801603] {CircleHelmholtz-1.png};
\end{axis}
\end{tikzpicture}%%
\endgroup
\end{center}

\end{frame}
\begin{frame}{Quelques jolies figures}
\begin{center}
Champ Helmholtzien rayonné par des sources aléatoires disposées sur un cercle (grille 500x500, 1000 charges)
\begingroup
\centering
\tikzset{every picture/.style={scale=0.75}}%
% This file was created by matlab2tikz.
%
%The latest updates can be retrieved from
%  http://www.mathworks.com/matlabcentral/fileexchange/22022-matlab2tikz-matlab2tikz
%where you can also make suggestions and rate matlab2tikz.
%
\begin{tikzpicture}

\begin{axis}[%
width=4.521in,
height=3.566in,
at={(0.758in,0.481in)},
scale only axis,
axis on top,
xmin=-1.00200400801603,
xmax=1.00200400801603,
xlabel={x},
y dir=reverse,
ymin=-1.00200400801603,
ymax=1.00200400801603,
ylabel={y},
axis background/.style={fill=white}
]
\addplot [forget plot] graphics [xmin=-1.00200400801603,xmax=1.00200400801603,ymin=-1.00200400801603,ymax=1.00200400801603] {CircleHelmholtz2-1.png};
\end{axis}
\end{tikzpicture}%%
\endgroup
\end{center}

\end{frame}
\begin{frame}{Quelques jolies figures}
\begin{center}
Champ Helmholtzien rayonné par des sources aléatoires disposées sur un cercle (grille 500x500, 1000 charges)
\begingroup
\centering
\tikzset{every picture/.style={scale=0.75}}%
% This file was created by matlab2tikz.
%
%The latest updates can be retrieved from
%  http://www.mathworks.com/matlabcentral/fileexchange/22022-matlab2tikz-matlab2tikz
%where you can also make suggestions and rate matlab2tikz.
%
\begin{tikzpicture}

\begin{axis}[%
width=4.143in,
height=3.093in,
at={(1.136in,0.954in)},
scale only axis,
axis on top,
xmin=-1.00200400801603,
xmax=1.00200400801603,
xlabel={x},
y dir=reverse,
ymin=-1.00200400801603,
ymax=1.00200400801603,
ylabel={y},
axis background/.style={fill=white}
]
\addplot [forget plot] graphics [xmin=-1.00200400801603,xmax=1.00200400801603,ymin=-1.00200400801603,ymax=1.00200400801603] {CircleHelmholtz4-1.png};
\end{axis}
\end{tikzpicture}%%
\endgroup
\end{center}

\end{frame}

\begin{frame}{Quelques jolies figures}
\begin{center}
Champ Helmholtzien rayonné par des sources constantes disposées sur un cercle (grille 500x500, 10 charges)
\begingroup
\centering
\tikzset{every picture/.style={scale=0.75}}%
% This file was created by matlab2tikz.
%
%The latest updates can be retrieved from
%  http://www.mathworks.com/matlabcentral/fileexchange/22022-matlab2tikz-matlab2tikz
%where you can also make suggestions and rate matlab2tikz.
%
\begin{tikzpicture}

\begin{axis}[%
width=4.143in,
height=3.093in,
at={(1.136in,0.954in)},
scale only axis,
axis on top,
xmin=-1.00200400801603,
xmax=1.00200400801603,
xlabel={x},
y dir=reverse,
ymin=-1.00200400801603,
ymax=1.00200400801603,
ylabel={y},
axis background/.style={fill=white}
]
\addplot [forget plot] graphics [xmin=-1.00200400801603,xmax=1.00200400801603,ymin=-1.00200400801603,ymax=1.00200400801603] {CircleHelmholtz5-1.png};
\end{axis}
\end{tikzpicture}%%
\endgroup
\end{center}

\end{frame}



\end{document}