\documentclass[]{article}

\usepackage{adjustbox}
\usepackage{algorithm}
\usepackage{algorithmic}
\usepackage{amsmath}
\usepackage{amssymb}
\usepackage{amsthm}
\usepackage{amsfonts}
\usepackage{afterpage}
\usepackage{blindtext}
\usepackage[font=footnotesize,labelfont=bf]{caption}
\usepackage{hyperref}
\usepackage[english]{babel}
\usepackage{bbm}
\usepackage{bigints}
\usepackage{bm}
\usepackage{cite}
\usepackage{color}
\usepackage{float}
\usepackage[left=2cm,right=2cm,top=2cm,bottom=2cm]{geometry}
\usepackage{graphicx}
\usepackage[utf8]{inputenc}
\usepackage{mathtools}
\usepackage{mdframed}
\usepackage{pgfplots} 
\usepackage{subfigure}
\usepackage{stmaryrd}
\usepackage{textcomp}
\usepackage{tikz}
\usepackage{url}
\renewcommand{\proofname}{Proof}
\theoremstyle{plain}
\newtheorem{monTheoNumrote}{Théorème}[section] % Environnement numéroté en fonction de la section
\newtheorem*{monTheoNonNumerote}{Théorème}  % Environnement non numéroté
\newtheorem{The}{Theorem}[section]
\newtheorem*{The*}{Theorem}
\newtheorem{Prop}{Proposition}[section]
\newtheorem*{Prop*}{Proposition} 
\newtheorem{Cor}{Corollary}[section]
\newtheorem*{Cor*}{Corollary}
\newtheorem{Conj}{Conjecture}[section]
\newtheorem{Lem}{Lemma}[section]
\renewcommand{\qed}{\unskip\nobreak\quad\qedsymbol}%
\numberwithin{equation}{section} % Numérote les équations section.numéro.
\theoremstyle{definition}
\newtheorem{Def}{Definition}[section]
\newtheorem{Rem}{Remark}[section]
\newtheorem*{Rem*}{Remark}
\newtheorem*{Lem*}{Lemma}
\newtheorem{Que}{Question}
\newcommand{\enstq}[2]{\left\{#1\mathrel{}\middle|\mathrel{}#2\right\}}
\newcommand{\Lp}[2]{L^#1(#2)}
\newcommand{\Sob}[3]{W^{#1,#2}(#3)}
\newcommand{\Rd}[0]{\mathbb{R}^d}
\newcommand{\RN}[0]{\mathbb{R}^N}
\newcommand{\Rn}[0]{\mathbb{R}^n}
\newcommand{\norm}[1]{\left\|#1\right\|}
\newcommand{\sinc}[0]{\textup{sinc}}
\newcommand{\functionDef}[5]{\begin{array}{lllll}
#1 & : & #2 & \longrightarrow & #3 \\
 & & #4 & \longmapsto &\displaystyle #5 \\
\end{array}}
\newcommand{\Theautorefname}{Theorem}
\newcommand{\Propautorefname}{Proposition}
\newcommand{\Corautorefname}{Corollary}
\newcommand{\Lemautorefname}{Lemma}
\newcommand{\Defautorefname}{Definition}
\newcommand{\N}{\mathbb{N}}
\newcommand{\Z}{\mathbb{Z}}
\newcommand{\D}{\mathbb{D}}
\newcommand{\R}{\mathbb{R}}
\newcommand{\A}{\mathcal{A}_{a,b}}
\newcommand{\Crad}{C^\infty_{c,rad}(B)}
\newcommand{\Lrad}{L^2_{rad}(B)}
\newcommand{\Lradab}{L^2_{rad}(\mathcal{A}_{a,b})}
\newcommand{\duality}[2]{\left\langle #1,#2\right\rangle}
\newcommand{\Hrad}{H^1_{rad}(B)}
\newcommand{\Hzrad}{H^1_{0,rad}(B)}
\newcommand{\rmin}{\delta_{\min}}
\newcommand{\rmax}{\delta_{\max}}
\newcommand{\corr}{\gamma}
\newcommand{\question}[1]{\begin{Que} \ 
#1
\end{Que}}
\newcommand{\abs}[1]{\left\lvert #1 \right\rvert}
\newcommand{\CL}[2]{\textup{CL}\left(\enstq{#1}{#2}\right)}
\newcommand{\Script}[1]{`\texttt{#1}`}
\newcommand{\espace}{\text{ }\qquad} 
\newcommand{\loc}{\text{loc}}
\newcommand{\SL}{\textup{SL}\hspace{1.5pt}}
\newcommand{\DL}{\textup{DL}\hspace{1.5pt}}
\newcommand{\fp}{\underset{\varepsilon \to 0}{\textup{f.p.}}}
\newcommand{\scalProd}[2]{\left(#1|#2\right)}
\newcommand{\toDo}[1]{{\color{red}#1}}
\newcommand{\bs}[1]{\boldsymbol{#1}}
\newcommand{\varInRange}[4]{(#1_{#2})_{#3 \leq #2 \leq #4}}
\newcommand{\from}{\colon}
\newcommand{\Cinf}{C^{\infty}}
\newcommand{\isdef}{\mathrel{\mathop:}=}
\newcommand{\defis}{=\mathrel{\mathop:}}

\renewcommand{\algorithmicrequire}{\textbf{Inputs:}}
\renewcommand{\algorithmicensure}{\textbf{Outputs:}}

\pgfplotsset{compat=1.13}
%opening
\title{Concis disque}
\author{Martin, François}

\begin{document}

\maketitle

\begin{abstract}
	We do the same as for the segment but this time with the disk.
\end{abstract}

\section{Laplace}

We define 
$$D = \enstq{x \in \R^3}{ x_1^2 + x_2^2 \leq  1,\quad  z = 0}$$ 
the unit disk in the plane $z = 0$ and 
$$\mathbbm{S}^2 = \enstq{x \in \R^3}{x_1^2 + x_2^2 + x_3^2 = 1}$$
the unit sphere in dimension $3$. 

\begin{Def}
	For $x \in D$, we put 
	\[\omega(x) = \sqrt{1 - \rho^2}\]
	where $\rho^2 = x_1^2 + x_2^2$. 
	
	We define the single layer operator $\mathcal{S}$ by 
	\begin{equation}
		\mathcal{S \varphi}(x) = \int_{D} \frac{1}{4\pi \norm{x - y}}\varphi(y) dD(y)
		\label{defS}
	\end{equation}
	and $S \isdef \gamma \mathcal{S}$, where $\norm{x} = \sqrt{x_1^2 + x_2^2 + x_3^2}$ denotes the Euclidean norm of $x$ and $dD$ is the surface measure on $D$. The hypersingular operator is
	\begin{equation}
		N \varphi(x) \isdef \textup{f.p.} \int \frac{\partial^2}{\partial n_x n_y}\frac{1}{\norm{x - y}} \varphi(y) dD(y) 
		\label{defN}
	\end{equation}
	We let $S_\omega = S \frac{1}{\omega}$ and $N_\omega = N \omega$.
\end{Def}

Let us review some properties of the spherical harmonics $Y_{l}^m$. 

\begin{Def}
	For $l \in \N$ and $- l \leq m \leq l$, the function $Y_l^m(\theta,\varphi)$ is defined by 
	\[Y_{l}^m(\theta,\varphi) = \gamma_l^m P_{lm}(\cos\theta) e^{im\varphi}\,,\]
	where $P_{lm}(x)$ is the so-called "Associated Legendre" polynomial which is the solution of the differential equation 
	\[(1 - x^2)\frac{d^2}{dx^2}P_{lm}(x) - 2 x \frac{d}{dx} P_{lm}(x) + \left[l(l+1) - \frac{m^2}{1 - x^2} \right] P_{lm} (x) = 0.\]
	
	The spherical harmonics are eigenfunctions of the Laplace-Beltrami operator on $\mathbb{S}^2$ which takes the following form on the sphere: 
	\[\Delta_{\mathbbm{S}^2} u(\theta,\varphi) = \frac{1}{\sin\theta} \partial_\theta \left(\sin\theta \frac{\partial}{\partial_{\theta}} u\right) - \frac{1}{\sin^2 \theta} \frac{\partial^2 u}{\partial \varphi^2}\,.\]
	With this definition, one has
	\[-\Delta_{\mathbbm{S}^2} Y_{l}^m = l(l+1) Y_{l}^m \] 
\end{Def}

\begin{Prop}
	The single-layer operator and the hypersingular operator $S_{\mathbbm{S^2}}$ and $N_{\mathbbm{S^2}}$ defined as in \cref{defS,defN} replacing the domain of integration by $\mathbbm{S}^2$, satisfy
	\[S_{\mathbbm{S^2}} Y_{l}^m = \frac{1}{2l+1} Y_l^m\]
	\[N_{\mathbbm{S^2}} Y_{l}^m = -\frac{l(l+1)}{2l + 1} Y_l^m\]
\end{Prop}
This proposition illustrates the fact that $S_{\mathbbm{S^2}}N_{\mathbbm{S^2}}$ is an order $0$ operator on the unit sphere, with a spectrum concentrated towards $\frac{1}{4}$. Moreover, both operators have the property that they map smooth functions to smooth functions bijectively. This fails to be the case for the operators $S$ and $N$ on the disk. However, once they are appropriately weighted, we have explicit eigenfunctions and eigenvalues. Let us introduce the "projected spherical harmonics" PSH:
\begin{Def}
	We define the PSH $y_{l}^m$ by 
	\[y_{lm}(\rho,\varphi) = \eta_{l}^m e^{im \varphi} P_{lm}(\sqrt{1 - \rho^2}) \propto Y_{lm}(\pi/2 - \arccos(\rho),\varphi)\]
	where $\rho,\varphi$ is a system of cylindrical coordinates on $D$. The constant $\eta_{l}^m$ is defined by 
	\[\eta_{l}^m = (-1)^m \sqrt{\frac{(2l+1)}{4\pi}\frac{(l-m)!}{(l+m)!}}\,.\]
	If $l_1$ and $m_1$ and $l_2$ $m_2$ have the same parity, there holds 
	\[\int_{D} \frac{y_{l_1}^{m_1}y_{l_2}^{m_2}}{\omega} = \delta_{l_1 = l_2}\delta_{m_1 = m_2}\,.\]
\end{Def}

\begin{Def}
	To reinforce the connection between this work and our previous work in 2d, we denote 
	\[T_{l}^m(x) = y_{l}^m(x,\varphi)\] 
	\[U_{l}^m(x) = \frac{y_{l+1}^m(x,\varphi)}{\sqrt{1 - x^2}}\]
	where $l+m$ is even.
	Both $T_{l}^m$ and $U_{l}^m$ are polynomials of order $l$ in $\rho$. 
\end{Def}
\begin{Prop}
	The functions $T_{l}^m$ form an orthogonal basis of $L^2_\frac{1}{\omega}(D)$ and the functions $U_{l}^m$ an orthogonal basis of $L^2_\omega(D)$. 
\end{Prop}
\begin{Prop}
	Let $l,m \in \N$ with $l + m$ even. Then
	\[S_\omega T_{l}^m = \frac{1}{2\lambda_{l}^m} T_{l}^m \]
	and 
	\[N_\omega U_{l}^m = \frac{\lambda_{l+1}^m}{2} U_{l}^m\]
	where
	\[\lambda_{l}^m = 2\frac{\Gamma\left(\frac{l+m+2}{2}\right) \Gamma\left(\frac{l-m+2}{2}\right)}{\Gamma\left(\frac{l+m+1}{2}\right) \Gamma\left(\frac{l-m+1}{2}\right)}\]
\end{Prop}

\begin{Def}
	We define the "angular moments" $\mathcal{L}_+$ and $\mathcal{L}_-$ on the disk by the following formulas for smooth functions
	\[\mathcal{L}_\pm u(\rho,\varphi) \isdef e^{\pm i \varphi}\left(\pm \frac{\partial u}{\partial\rho} + i \frac{1}{\rho}\frac{\partial u}{\partial \varphi}\right)\]
\end{Def}

One has

\[\nabla_D = \frac{1}{2}\begin{pmatrix}\mathcal{L}_+ - \mathcal{L}- \\ -i(\mathcal{L}_+ + \mathcal{L}_-)  \end{pmatrix}\]

therefore, using $\Delta_D = \nabla_D \cdot \nabla_D$, 
\[\Delta_D = -\frac{1}{2}(\mathcal{L}_+ \mathcal{L}_{-} + \mathcal{L}_{-}\mathcal{L}_+)\]
\toDo{Ici je pense qu'il y a une erreur dans la thèse de Pedro, les deux termes ne sont pas égaux (p. 62)}
Later on, we drop the subscript $D$. 
\begin{Prop}
	Let $l,m$ such that $l+m$ is even. Then 
	\[\mathcal{L}_{\pm} T_{l,m} = \sqrt{(l\pm m)(l\pm m+1)}U_{l-1}^{m \pm 1}\]
	while
	\[\omega \mathcal{L}_\pm \omega U_{l}^m = \sqrt{(l \pm m)(l\pm m +1)} T_{l+1}^{m \pm 1}\]
\end{Prop}
\begin{proof}
	Those identities are easily deduced from the formulas for $\mathcal{L}_{+} y_{l}^m$ and $\mathcal{L}_- y_{l}^m$ established in \cite{ramaciotti2016theoretical} and summarized in \cite{hiptmair2017closed}.
\end{proof}
One can deduce the following identities

\begin{Cor}
	\[\omega \mathcal{L}_{+} \omega \mathcal{L}_{-} T_{l}^m = (l(l+1) - m^2 + m)T_{l,m} \]
	\[\omega \mathcal{L}_{-} \omega \mathcal{L}_{+} T_{l}^m = (l(l+1) - m^2 - m)T_{l,m} \]
	while
	\[\mathcal{L}_{+} \omega \mathcal{L}_{-}\omega U_{l}^m = ((l+1)(l+2) - m^2 + m)U_{l,m} \]
	\[\omega \mathcal{L}_{-} \omega \mathcal{L}_{+} U_{l}^m = ((l+1)(l+2) - m^2 - m)U_{l,m} \]
\end{Cor}

We further introduce two other operators, $(\omega \nabla \omega) \cdot \nabla $ and $\nabla \cdot (\omega \nabla \omega)$ which satisfy 
\[(\omega \nabla \omega) \cdot \nabla = \frac{1}{2}\left(\omega \mathcal{L}_{+} \omega \mathcal{L}_{-} + \omega \mathcal{L}_{-} \omega \mathcal{L}_{+}\right)\]
and
\[\nabla \cdot (\omega \nabla \omega)  = \frac{1}{2}\left( \mathcal{L}_{+} \omega \mathcal{L}_{-}\omega + \mathcal{L}_{-} \omega \mathcal{L}_{+}\omega \right)\]

\begin{Cor}
	One has, for $l+m$ even,
	\[(\omega \nabla \omega) \cdot \nabla T_{l}^m = (l(l+1) - m^2)T_{l,m}\,,\]
	and
	\[\nabla \cdot (\omega \nabla \omega) U_{l}^m = ((l+1)(l+2) - m^2)U_{l,m}\,.\]
\end{Cor}

\begin{Prop}
	Let $l \in \N$ and $-l \leq m \leq l$, such that $l$ and $m$ are not both zero. Then
	\begin{equation}
		1 \leq \frac{4 \left(\lambda_{l}^m\right)^2}{l(l+1) - m^2} \leq 2
		\label{PropinegalitesEq}
	\end{equation}
\end{Prop}
\begin{proof}
 	By Gauschi's inequality \cite{gautschi1959some}, there holds for all $x > 0$ and $s \in (0,1)$, 
 	\[x^{1-s} \leq \frac{\Gamma(x + 1)}{\Gamma(x + s)} \leq (x+1)^{1-s}\]
 	This inequality, applied with $s = \frac{1}{2}$ and $x = \frac{l \pm m}{2}$ yields 
 	\[\frac{3}{4} \leq \frac{4 \left(\lambda_{l}^m\right)^2}{l(l+1) - m^2} \leq 4\,.\]
 	Note that for $s = \frac{1}{2}$, both sides of Gautstchi's inequality remain valid for $x = 0$, so the cases $m = \pm l$ are permitted. To find the constants $1$ and $2$ instead of $\frac{3}{4}$ and $4$, one can use the following improvement of Gautschi's inequality \cite{kershaw1983some}:
 	\[\left(x + \frac{s}{2}\right)^{1-s} < \frac{\Gamma(x + 1)}{\Gamma(x + s)} < \left(x - \frac{1}{2} + \sqrt{s + \frac{1}{4}}\right)^{1 - s}\,.\]
 	Again, this inequality remains valid in the case $x = 0$ for $s = \frac{1}{2}$. Taking again $x = \frac{ l \pm m }{2}$, multiplying the squares of the resulting inequalities, and multiplying by $\frac{4}{l(l+1) - m^2}$, one finds
 	\begin{equation}
 		\frac{(l+ \frac{1}{2})^2 - m^2}{l(l+1) - m^2} < \frac{4 \left(\lambda_{l}^m\right)^2}{l(l+1) - m^2} < \frac{(l+ \frac{\sqrt{3} - 1}{2})^2 - m^2}{l(l+1) - m^2}
 		\label{inegaliteCle}
 	\end{equation}
 	Some manipulations on those	 inequalities yields
 	\[1 + \frac{1}{4(l(l+1) - m^2)} <  \frac{4 \left(\lambda_{l}^m\right)^2}{l(l+1) - m^2} < 1 + \frac{(2\sqrt{3} - 3)l}{l(l+1) - m^2} + \frac{4 - 2\sqrt{3}}{l(l+1) - m^2}\]
 	The left side of the latter inequality immediately implies the left inequality of \eqref{PropinegalitesEq}. For the right side, we obtain the result after using $l(l+1) - m^2 \geq l$ for the denominator of the second term, and $l(l+1) - m^2 \geq 1$ for the third term.  
\end{proof}
As it can be seen in the proof, the lower bound is sharp, since for $m = 0$ and in the limit $l \to \infty$, the two sides of \eqref{inegaliteCle} converge to $1$. For the upper bound, it seems, based on numerical evidence, that the highest value of the quantity being estimated is reached for $l = 1$ and $m =\pm 1$. In this case, one has $\frac{4 \left(\lambda_{l}^m\right)^2}{l(l+1) - m^2} = \frac{16}{\pi^2} \approx 1.6$. 

\section{Helmholtz}

Much like in the 2-dimensional case, the following commutation holds.
\begin{The}
	There holds the commutations
	\[\left[(\omega \nabla \omega)\cdot \nabla - k^2 \omega^2\right]  S_{k,\omega} =S_{k,\omega} \left[(\omega \nabla \omega)\cdot \nabla - k^2 \omega^2\right]\]
	and 
	\[\left[\nabla \cdot (\omega \nabla \omega)  - k^2 \omega^2\right] N_{k,\omega} = N_{k,\omega} \left[\nabla \cdot (\omega \nabla \omega)  - k^2 \omega^2\right] \]
\end{The}
\begin{proof}
	+
\end{proof}


\bibliographystyle{plain}
\bibliography{../Biblio/biblio}

\end{document}
