\documentclass[main]{subfiles}

\chapter{Elements of Semidefinite Programming}
\label{chap:lmisdp}


A linear matrix inequality (LMI for short) refers to a constraint of the form
\begin{align}
\label{eq:lmi}
A_0 + \sum_{k=1}^d x_k A_k \succeq 0 \,,
\end{align}
where $x \in \R^d$ is the  variable, $(A_k)_{1\leq k\leq d}$ are given symmetric $n\times n$
matrices and $\preceq$ is the L\"owner order.
In other words, given a symmetric matrix $A(x_1, \dots, x_d)$ whose entries depend
in an affine way on $x \in \R^d$, the constraint \enquote{$A(x_1, \dots, x_d)$ is positive semidefinite} is an LMI. 

Several LMIs can be combined into a single LMI, since
\begin{align*}
A(x) \succeq 0 \wedge B(x) \succeq 0 \iff 
\begin{psmallmatrix}
A(x) & 0 \\ 0 & B(x)
\end{psmallmatrix} \succeq 0 \,.
\end{align*}
A constraint of the form $X \succeq A_0$ is an LMI in the variable $X \in \Sn$, since  
\begin{align*}
X \succeq A_0 \iff \sum_{i \leq j} x_{ij} E^s_{i,j} \succeq A_0
\end{align*}
 in the variables $X = (x_{ij})_{1 \leq i \leq j \leq n}$, where $E^s_{i,j}$ denotes the matrix with zeroes everywhere except for a $1$ in the $(i,j)$-th and $(j,i)$-th entry.

Finally, the combination of the LMIs $A(x) \preceq 0$ and $A(x) \succeq 0$ allows one to consider equality constraints $A(x) = 0$.

The problem of minimizing a convex function in the variable $x$ that satisfies the LMI in \Cref{eq:lmi} is called a semidefinite program (SDP).
We refer to~\cite{boyd} for introductive background on these programs.


Semidefinite programs can be solved in ``polynomial time'' in 
the following approximate sense (semidefinite feasibility is not known to
be polynomial time in the Turing model of computation). 
Given an accuracy parameter $\varepsilon>0$, one can obtain,
in particular by interior point methods (the most efficient in practice),
a $\varepsilon$-approximate solution of a SDP in a number of arithmetic
operations 
which is polynomial in $n$, $d$, $\log\varepsilon$, and $\log (R/r)$, assuming
that the set $\mathcal{F}$ of vectors
which satisfy~\eqref{eq:lmi} is such that $B(a,r)\subset \mathcal{F}\subset
B(a,R)$ for some point $a\in \R^n$, where $B(a,r)$
denotes the Euclidean ball of center $a$ and radius $r$,
see~\cite{deklerk_vallentin}. 
%% In the present approach, $\log (R/r)$ will remain small unless
%% we deal with nearly flat ellipsoids. Hence, the present
%% semidefinite programs will be effectively solvable in polynomial time
%% with a prescribed accuracy.
We warn the reader, however, that the exponent of the polynomial
is relatively high.
 (see Section~\ref{sec:alt_app} for details).
Hence, it is essential for scalability purposes to limit as far
as possible the growth of the dimension $n$ and of the number
of variables $d$, which is one of our main goals in this thesis.
