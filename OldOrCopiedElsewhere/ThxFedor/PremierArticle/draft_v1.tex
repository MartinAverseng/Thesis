
\documentclass[]{article}

\usepackage{adjustbox}
\usepackage{algorithm}
\usepackage{algorithmic}
\usepackage{amsmath}
\usepackage{amssymb}
\usepackage{amsthm}
\usepackage{amsfonts}
\usepackage{afterpage}
\usepackage{blindtext}
\usepackage[font=footnotesize,labelfont=bf]{caption}
\usepackage{hyperref}
\usepackage[english]{babel}
\usepackage{bbm}
\usepackage{bigints}
\usepackage{bm}
\usepackage{cite}
\usepackage{color}
\usepackage{float}
\usepackage[left=2cm,right=2cm,top=2cm,bottom=2cm]{geometry}
\usepackage{graphicx}
\usepackage[utf8]{inputenc}
\usepackage{mathtools}
\usepackage{mdframed}
\usepackage{pgfplots} 
\usepackage{subfigure}
\usepackage{stmaryrd}
\usepackage{textcomp}
\usepackage{tikz}
\usepackage{url}
\renewcommand{\proofname}{Proof}
\theoremstyle{plain}
\newtheorem{monTheoNumrote}{Théorème}[section] % Environnement numéroté en fonction de la section
\newtheorem*{monTheoNonNumerote}{Théorème}  % Environnement non numéroté
\newtheorem{The}{Theorem}[section]
\newtheorem*{The*}{Theorem}
\newtheorem{Prop}{Proposition}[section]
\newtheorem*{Prop*}{Proposition} 
\newtheorem{Cor}{Corollary}[section]
\newtheorem*{Cor*}{Corollary}
\newtheorem{Conj}{Conjecture}[section]
\newtheorem{Lem}{Lemma}[section]
\renewcommand{\qed}{\unskip\nobreak\quad\qedsymbol}%
\numberwithin{equation}{section} % Numérote les équations section.numéro.
\theoremstyle{definition}
\newtheorem{Def}{Definition}[section]
\newtheorem{Rem}{Remark}[section]
\newtheorem*{Rem*}{Remark}
\newtheorem*{Lem*}{Lemma}
\newtheorem{Que}{Question}
\newcommand{\enstq}[2]{\left\{#1\mathrel{}\middle|\mathrel{}#2\right\}}
\newcommand{\Lp}[2]{L^#1(#2)}
\newcommand{\Sob}[3]{W^{#1,#2}(#3)}
\newcommand{\Rd}[0]{\mathbb{R}^d}
\newcommand{\RN}[0]{\mathbb{R}^N}
\newcommand{\Rn}[0]{\mathbb{R}^n}
\newcommand{\norm}[1]{\left\|#1\right\|}
\newcommand{\sinc}[0]{\textup{sinc}}
\newcommand{\functionDef}[5]{\begin{array}{lllll}
#1 & : & #2 & \longrightarrow & #3 \\
 & & #4 & \longmapsto &\displaystyle #5 \\
\end{array}}
\newcommand{\Theautorefname}{Theorem}
\newcommand{\Propautorefname}{Proposition}
\newcommand{\Corautorefname}{Corollary}
\newcommand{\Lemautorefname}{Lemma}
\newcommand{\Defautorefname}{Definition}
\newcommand{\N}{\mathbb{N}}
\newcommand{\Z}{\mathbb{Z}}
\newcommand{\D}{\mathbb{D}}
\newcommand{\R}{\mathbb{R}}
\newcommand{\A}{\mathcal{A}_{a,b}}
\newcommand{\Crad}{C^\infty_{c,rad}(B)}
\newcommand{\Lrad}{L^2_{rad}(B)}
\newcommand{\Lradab}{L^2_{rad}(\mathcal{A}_{a,b})}
\newcommand{\duality}[2]{\left\langle #1,#2\right\rangle}
\newcommand{\Hrad}{H^1_{rad}(B)}
\newcommand{\Hzrad}{H^1_{0,rad}(B)}
\newcommand{\rmin}{\delta_{\min}}
\newcommand{\rmax}{\delta_{\max}}
\newcommand{\corr}{\gamma}
\newcommand{\question}[1]{\begin{Que} \ 
#1
\end{Que}}
\newcommand{\abs}[1]{\left\lvert #1 \right\rvert}
\newcommand{\CL}[2]{\textup{CL}\left(\enstq{#1}{#2}\right)}
\newcommand{\Script}[1]{`\texttt{#1}`}
\newcommand{\espace}{\text{ }\qquad} 
\newcommand{\loc}{\text{loc}}
\newcommand{\SL}{\textup{SL}\hspace{1.5pt}}
\newcommand{\DL}{\textup{DL}\hspace{1.5pt}}
\newcommand{\fp}{\underset{\varepsilon \to 0}{\textup{f.p.}}}
\newcommand{\scalProd}[2]{\left(#1|#2\right)}
\newcommand{\toDo}[1]{{\color{red}#1}}
\newcommand{\bs}[1]{\boldsymbol{#1}}
\newcommand{\varInRange}[4]{(#1_{#2})_{#3 \leq #2 \leq #4}}
\newcommand{\from}{\colon}
\newcommand{\Cinf}{C^{\infty}}
\newcommand{\isdef}{\mathrel{\mathop:}=}
\newcommand{\defis}{=\mathrel{\mathop:}}

\renewcommand{\algorithmicrequire}{\textbf{Inputs:}}
\renewcommand{\algorithmicensure}{\textbf{Outputs:}}

\pgfplotsset{compat=1.13}

\author{François Alouges, Martin Averseng}
\title{Integral equations on open-curves : a new preconditioner}
\begin{document}
\maketitle

\abstract{In this paper, we analyze preconditioners for the integral equations arising in the resolution of acoustic scattering by an open arc in 2D in the Galerkin setting.}



\section{Preliminaries}
\toDo{Globalement, il faut recopier les définitions de Bruno.} 
Let $\Gamma$ a smooth open simple curve in $\R^2$, and $u_D$ and $u_N$ two smooth functions on $\Gamma$. We consider the following boundary value problems, namely the Dirichlet problem (D):  
\begin{equation*}
\left \{
\begin{aligned}
-\Delta u - k^2 u &= 0, && \text{ in } \R^2 \setminus \Gamma \\
u &= u_D, && \text{ on } \Gamma
\end{aligned} \right.
\end{equation*}
and the Neumann problem (N): 
\begin{equation*}
\left \{
\begin{aligned}
-\Delta u - k^2 u &= 0, && \text{ in } \R^2 \setminus \Gamma \\
\dfrac{\partial u}{\partial n} &= u_N. && \text{ on } \Gamma
\end{aligned} \right.
\end{equation*} 
These problems can be solved using integral equations. Let $G$ the Green's function defined by 
\begin{equation}
\left \{
\begin{aligned}
G(z) &= -\dfrac{1}{2\pi} \ln \abs{z}, && \text{ if } k= 0,\\
G(z) &= \frac{i}{4}H_0(k|z|), && \text{ if } k > 0.
\end{aligned} \right.
\end{equation} 
We consider the single-layer potential defined for $x \notin \Gamma$ by 
\begin{equation}
	\text{SL}\lambda(x) = \int_{\Gamma}G(x-y)\lambda(y)d\sigma(y)
\end{equation}
where $\sigma$ is the arc measure on $\Gamma$.. Denoting by $\gamma$ the trace operator on $\Gamma$ and $S = \gamma \text{SL}$, it is well-known that the solution $u$ of (D) is given by 
\[ u = \text{SL} \lambda \]
if $\lambda$ is a solution of the integral equation 
\begin{equation}
	S \lambda = u_D.
	\label{Slambda}
\end{equation} 
The solution $\lambda$ to the former problem is unique and well-defined. However, because of the edges of $\Gamma$, it is not smooth, leading in poor performance of numerical methods based on the discretization of $\lambda$ itself. It is known that there exists a smooth function $\varphi$ such that $ \lambda = \frac{\varphi}{\omega(x)}$ whith 
\[\omega(x) = \dfrac{1}{\sqrt{d(x,\partial \Gamma)}}\]
This is why, in \cite{bruno2012second}, a weighted operator $S_{\omega}$ is introduced, defined by 
\[S_{\omega} \varphi \isdef S \left(\dfrac{\varphi}{\omega}\right).\]
This time, $S_\omega$ sends smooth functions on smooth functions, leading to improved convergence in numerical methods. 
Symmetrically, if we let 
\begin{equation}
\text{DL}\nu(x) = \int_{\Gamma}\dfrac{\partial}{\partial n_y} G(x-y)\nu(y)d\sigma(y)
\end{equation}
the solution to problem (N) is obtained as 
\[u = \text{DL}\nu\] where $\nu$ is the solution of the integral equation
\begin{equation}
N \nu = u_N,
	\label{Nmu}
\end{equation}
and $N$ is the so-called hypersingular operator defined by 
\[N\nu = \lim_{z \to 0^+}\dfrac{\partial}{\partial z}\text{DL}\nu(x + z n_x).\]
Similarly, if $u_N$ is smooth, there exists a smooth function $\psi$ such that 
\[\nu = \psi\omega,\]
thus the corresponding weighted hyersingular operator is defined by 
\[N_{\omega}\psi \isdef N\left(\psi \omega\right)\]
In \cite{bruno2012second}, it is shown that the operators $S_{\omega}$ and $N_{\omega}$ are inverse modulo a compact operator, justifying that they are good mutual preconditioners in the process of solving \eqref{Slambda} and \eqref{Nmu} numerically. 
Here study a new preconditioning technique based on a weighted version of the Laplace operator: for any function $u$ defined on $\Gamma$
\[\Delta_{\omega}u \isdef \omega\left(\omega u'\right)'\]
where the derivative is taken along the curvilinear abscissa. We analyze preconditioners given by that $S_\omega (\Delta_\omega - k^2 \omega^2)$ for equation \eqref{Slambda} and $N_{\omega}$.

\section{Functional framework}

\begin{Def}
	Definition of the modified Sobolev spaces $H^s_{\omega}$ as complement of polynomials for the norm 
	\[ \norm{u}_s^2 = \sum_{n=0}^{+\infty} (1+n^2)^{\frac{s}{2}}\abs{{u}_n}^2\]
	where $\hat{u}_n = \int_{-1}^1 \dfrac{uT_n}{\omega}$ for all reak $s$.  
\end{Def}
\begin{The}
	For the bilinear form $(u,v)\mapsto\int_{-1}^{1}\frac{uv}{\omega}$, $H^s_{\omega}$ is the dual of $H^{-s}_{\omega}$. We have compact injections from $H^s_{\omega} \to H^t_{\omega}$ when $s < t$. 
\end{The}
\begin{The}
	The space of (smooth functions) polynomials is dense in $H^s_\omega$ for all $s$. 
\end{The}
\begin{The}
	For all $s \in \N$, (we note $s=n$ in this case), we have 
	\[ \norm{u}_n = \int_{-1}^{1} \dfrac{\abs{\left(\omega dx\right)^n u}^2}{\omega}dx\] 
	For $n=1$ in particular, one has 
	\[ \norm{u}_{1} = \int_{-1}^{1} \omega \abs{u'}^2\]
\end{The}
\begin{The}
	Eventuellement un résultat d'interpolation ? C'est facile à faire avec les suites, on peut en déduire le résultat sous forme "locale" qu'on avait tant cherché. Peut-être qu'on peut même en arriver à une interpolation autre part que sur le segment (pour un arc ouvert) par changement de variable. 
\end{The}
(par définition)

\begin{The}
	On peut définir de manière implicite les espaces de Sobolev pour $s = \pm \frac{1}{2}$. Par exemple 
	
	\[\norm{u}_{\frac{1}{2}} = \int_{\Gamma\times \Gamma} \ln\abs{x-y}\omega(x)\omega(y) u'(x)u'(y)dxdy\]
\end{The}
Dans ce cas il faudrait montrer que les deux manières d'étendre la définition se raccordent bien (et ça permettrait de démontrer les résultats d'interpolation requis). On aurait également un vision Fourier discrète pour les courbes fermées. 


\section{Revisiting the litterature with these definitions}


Symm's operator. 
\begin{Def}
	Maillage adapté pour ce poids. 
\end{Def}
\begin{The}
	The Galerkine approximations converge at a given order of convergence. 
\end{The}
\begin{The}
	We have
	\[S_{\omega}\Delta_\omega S_\omega = I\]
	and they commute
\end{The}
\begin{The}
	$S_\omega\sqrt{-\Delta_\omega} = I$. 
\end{The}
\subsection{Numerical evidence}

\begin{Rem}
	Si le maillage est pas adapté pas de convergence. 
\end{Rem}
\begin{Rem}
	Link between Darbas and Antoine, and Bruno Litner. 
\end{Rem}


\subsection{Study on the segment with $k \neq 0$.}

We write $H_0(z) = \frac{-1}{2\pi} \ln|z| J_0(z) + R(z)$ where $R$ is an even entire function. 

\begin{Prop}
	The functions $r \mapsto \frac{J_0(r) - 1}{r^2}$ and $r \mapsto \frac{J_0'(r)}{r}$ are bounded on $\mathbb{R}$. 
	\begin{proof}
		We have for all $r \in \mathbb{R}$
		\[\frac{J_0(r)-1}{r^2} = -\sum_{n=0}^{+\infty} \frac{(-1)^n}{(n-1)!^2}\left(\frac{r}{2}\right)^{2n}\]
		which is easily shown to be absolutely convergent series for all $r >0$. 
		A similar argument gives the other result. 
	\end{proof}	
\end{Prop}
Pas sûr que ce soit utile.
\begin{Prop}
	For any $k$, we define $ k_1 : r \mapsto \left(J_0(kr) - 1 \right) \ln(|r|)$ defined on $\mathbb{R}$. Then the function $-\Delta_{\omega} k_1$ is bounded on $\mathbb{R}$. 
	\begin{proof}
		We can write $-\Delta_{\omega}K_1 = (r^2-1) \partial_{rr}K_1 + r \partial_r K_1(r)$, yielding 
		\[-\Delta_{\omega}K_1 = \ln(|r|) \left(kJ_0'(kr) + k^2(r^2 - 1)J_0''(kr)\right) + (r^2-1) \left(2\frac{kJ_0'(kr)}{|r|} - \frac{J_0(kr) - 1}{r^2}\right) + J_0(kr) - 1.\]
	\end{proof}
\end{Prop}
\begin{Cor} The function 
	\[k_1 : f \mapsto \int_{-1}^{1}\frac{k_1(k|x-y|)}{\omega(y)}f(y)dy\]
	Is continuous from $H^s_{\omega} \to H^{s+2}_{\omega}$  
\end{Cor}

\begin{The}
	An operator of the form 
	\begin{equation}
		Kf = \int_{-1}^{1} \frac{k(x,y)f(y)}{\omega(y)}
	\end{equation}
	with $k \in C^{\infty}(-1,1)$ maps $H^s_{\omega}$ to $H^{+\infty}_{\omega}$ for all $s$. 
\end{The}

\subsection{Non-flat arc, non-zero frequency}


\begin{Def}
	Transported sobolev. We can use the implicit def for $n$ integer or half integer. 
\end{Def}

\begin{Prop}
	The two possible definitions are compatible ?
\end{Prop}

We consider a smooth non-intersecting curve $\Gamma$ in $\mathbb{R}^2$ and a smooth parametrization $\textbf{r}: [-1,1] \to \Gamma$. We choose $\textbf{r}$ such that $\norm{\frac{dr}{dt}} = 1$. Indeed we can assume the curve has unit length by proper rescaling. Indeed, if $u$ is solution of the Helmholtz equation outside $\Omega$ with some boundary conditions on $\Gamma$ (Dirichlet or Neuman) and if we define $u^\lambda = u(\lambda r, \theta)$, we find $\Delta u^{\lambda} + k^2 \lambda^2 u^{\lambda} = 0$ outside $\Omega_{\lambda} = \frac{\Omega}{\lambda}$. By choosing $\lambda$ = $\abs{\Gamma}$, the border of the new domain is of length $1$. 

Without the rescaling (but still assuming constant speed parametrization), we can write 

\[\abs{r(t) - r(t')}^2 = L^2\abs{t-t'}^2 + \frac{C(t')^2}{2}\abs{t-t'}^4\]


\begin{The}
	One has 
	\[S^k_\omega\Delta_\omega S^k_\omega = I + K\]
	Where $K$ is compact (ou $K : H^{s}_{\omega} \to H^{s+1}_{\omega}$ for all $s$). Moreover, the leading order of $K$ is 
	\[K = \omega^2k^2 n^{-2}\]
	so that 
	\[S_k\left(\Delta_\omega + k^2 \omega^2\right)S_k = I + K\]
	with this time $K : H^{s} \to H^{s+2}$.
\end{The}

\begin{The}
	The operators $S_k$ and $\Delta_\omega + k^2 \omega^2$ commute. 
\end{The}

We note $G_k(t,t')$ the kernel of the non-zero non-flat arc operator. 

\begin{Lem}
	We have the following expansion
	\[J_0(k\abs{r(t) - r(t')}) = 1 - \frac{k^2}{4}L^2\abs{t-t'}^2 + \left(\frac{k^4L^4}{64} -\frac{C(t')^2k^2}{8}\right)\abs{t-t'}^4 + (t-t')^5 F(t,t')\]
	where $F$ is a smooth bounded function. 
\end{Lem}

\begin{Lem} If $L$ is the length of the curve and $C(t')$ the curvature at a point $t'$, one has
	\begin{align*}G_k(t,t') &= -\frac{1}{2\pi} \ln|t-t'|\left(1 - \frac{k^2}{4}L^2\abs{t-t'}^2 + \left(\frac{k^4L^4}{64} -\frac{C(t')^2k^2}{8}\right)\abs{t-t'}^4 + (t-t')^5 F(t,t')\right)\\
	 &+ R(t,t')
z	 \end{align*}
	 where $R$ is in $C^{\infty}([-1,1]^2)$.
\end{Lem}

\begin{Lem}
	\begin{eqnarray*}
		\Delta_{\omega}^{t'} \left((t-t')^2\ln\abs{t-t'}\right) &=& \omega^2(t')\frac{d^2}{dt'^2}\left((t-t')^2\ln\abs{t-t'}\right) - t' \frac{d}{dt'} \left((t-t')^2\ln\abs{t-t'}\right)\\
		&=&\omega^2(t')\left(2\ln\abs{t-t'} + 4 - 1\right) - t'\left(2(t'-t)\ln\abs{t-t'} + 2(t'-t)\right)\\
		&=&2\omega^2(t)\ln\abs{t-t'} + 2(t-t')\ln\abs{t-t'}(t + 2t') + P(t,t')
	\end{eqnarray*}
	where $P$ is a polynmial in $t$ and $t'$. 
\end{Lem}
\begin{Lem}
	The second term in this decomposition is the kernel of a bounded operator from $H^s_{\omega}$ to $H^{s+2}_{\omega}$.
	\begin{proof}
		Même raisonnement (on regarde une fois de plus l'action de $\Delta_\omega$ et on conclut avec une borne $L^2$ de l'intégrale de Cauchy. )
	\end{proof}
\end{Lem}
En fait, on devrait pouvoir montrer
\begin{Lem}
	The operator $L_n$ with kernel
	\[K(t,t') = (t-t')^n\ln\abs{t-t'}\]
	is continuous $H^s \to H^{s + n + 1}$.
\end{Lem}
Le résultat suivant est vrai pour $n=1$, à vérifier...
\begin{Conj}
	For any $n>0$, the operator $C_n$ with kernel
	\[K(t,t') = \frac{1}{(t-t')^n}\]
	(where the integral is taken as a finite part) is continuous $H^s_{\omega} \to H^{s+n-1}_{\omega}$. 
\end{Conj}
\begin{Lem}
	The application $f \mapsto \omega^2 f$ is continuous in $H^s_{\omega}$ for any $s$. 
	\begin{proof}
		This is obvious as $\abs{\duality{\omega^2f}{T_n}_{\omega}} \leq  \abs{\duality{f}{T_n}_{\omega}}$.
	\end{proof}
\end{Lem}
\begin{Lem}
	The application $f \mapsto \omega^2 f'$ is continuous from $H^s_\omega$ to $H^{s-1}_\omega$.
	The proof involves the Cesaro thm. 
\end{Lem}
\begin{Lem}
	The operator $\Delta_{\omega}\omega^2 - \omega^2 \Delta_{\omega}$ is continuous from $H_{\omega}^s$ to $H_{\omega}^{s-1}$. 
	\begin{proof}
		We use the formula : $\Delta_{\omega}\omega^2 - \omega^2 \Delta_{\omega} = 4x\omega^2f' + (4x^2 - 2)f$
	\end{proof}
\end{Lem}
\begin{Lem}
	The operator $S_0\omega^2 - \omega^2S_0$ is continuous from $H^s_{\omega}$ to $H^{s+2}_\omega$.
	\begin{proof}
		Use $\omega^2 T_n = \frac{2T_n + T_{n+2} + T_{\abs{n-2}}}{4}$.
	\end{proof} 
\end{Lem}
Be careful, the map $f\mapsto f'$ is not continuous from $H^1_{\omega}$ to $L^2_{\omega}$, as can be checked with the example $f = \omega$. 

Les deux lemmes précédents : même preuve.


\begin{Rem}
	Mathieu functions pour l'analyse du segment $k\neq 0$. Dire que c'est pas connu. 
\end{Rem}

\begin{Lem}
	$S_k - S_0 = \omega(t)^2\frac{k^2}{2}L^2 S_0 \Delta_{\omega}^{-1}$ 
\end{Lem}

\section{Application to preconditioning}

In this section we propose two preconditioners for the Scattering problem, one based on the use of a method for the square root of an operator. We prove that the preconditioned systems have a conditioning that is independent of the mesh size. We show applications. 


\bibliographystyle{plain}
\IfFileExists{biblio.bib}{\bibliography{biblio}}{\bibliography{/home/martin/Thesis/Biblio/biblio}}

\end{document}
